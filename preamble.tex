\usepackage[margin=1in]{geometry}
\usepackage{csquotes}
\usepackage{fancyhdr}
\usepackage{marginnote}
\usepackage{enumitem}
\usepackage{scrextend}
\usepackage[bottom]{footmisc}
\usepackage{xr}
\usepackage{siunitx}
\usepackage[style=apa]{biblatex}
\usepackage{booktabs}
\usepackage{xparse}
\usepackage{subcaption,float}
\usepackage{tikz}
\usepackage{amsmath,amssymb,amsthm}
\usepackage{bm,physics,mathtools,empheq}
\usepackage{mhchem}
\usepackage{chemfig}
\usepackage[hidelinks]{hyperref}

\MakeOuterQuote{"}

\fancypagestyle{main}{
    \fancyhf{}
    \fancyhead[L]{\leftmark}
    \fancyhead[R]{CHEM 26300}
    \fancyfoot[R]{Labalme\ \thepage}
}
\fancypagestyle{plain}{
    \fancyhead{}
    \renewcommand{\headrulewidth}{0pt}
}

\reversemarginpar

\setitemize[3]{label={\scriptsize$\blacksquare$}}

\deffootnotemark{\textsuperscript{\textup{[}\thefootnotemark\textup{]}}}
\deffootnote[2.1em]{0em}{0em}{\textsuperscript{\thefootnote}}

\externaldocument{notes}

\DeclareSIUnit{\angstrom}{\textup{\AA}}
\DeclareSIUnit{\molar}{M}

\addbibresource{../main.bib}
\DefineBibliographyStrings{english}{bibliography={References}}

\usetikzlibrary{backgrounds,decorations.markings,decorations.text,fpu,intersections,calc}
\colorlet{blx}{blue!90!green!80}
\colorlet{bly}{blue!90!green!60}
\colorlet{blz}{blue!90!green!40}
\colorlet{blt}{blue!90!green!30}
\colorlet{rex}{red!80!black!90!orange!80}
\colorlet{rey}{red!80!black!90!orange!40}
\colorlet{ret}{red!80!black!90!orange!30}
\colorlet{grx}{green!50!black!90!yellow!80}
\colorlet{gry}{green!50!black!90!yellow!40}
\colorlet{orx}{orange!80!black!90!yellow!80}
\colorlet{pux}{blx!50!rex!50!magenta}

\DeclareMathOperator{\Prob}{Prob}
\DeclareMathOperator{\pH}{pH}
\DeclareMathOperator{\pOH}{pOH}

\setchemfig{atom sep=2em,fixed length=true,bond offset=3pt}

\newcommand{\kB}{k_\text{B}}
\newcommand{\NA}{N_\text{A}}

\newcommand{\e}[1][]{\text{e}^{#1}}
\newcommand{\prb}[1]{\left\langle{#1}\right\rangle}
\NewDocumentCommand{\Longleftrightarrows}{O{}O{}}{\tikz[baseline=-1mm]{
    \node at (0,0.4) {\scriptsize${#1}$};
    \node at (0,0.1) {$\Longrightarrow$};
    \node at (0,-0.1) {$\Longleftarrow$};
    \node at (0,-0.4) {\scriptsize${#2}$};
}}
\newcommand{\cnc}[2][]{[\ce{#2}]_\text{#1}}

\usepackage{subfiles}