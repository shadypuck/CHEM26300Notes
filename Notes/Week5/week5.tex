\documentclass[../notes.tex]{subfiles}

\pagestyle{main}
\renewcommand{\chaptermark}[1]{\markboth{\chaptername\ \thechapter\ (#1)}{}}
\setcounter{chapter}{4}

\begin{document}




\chapter{Catalysis}
\section{Midterm Review and Intro to Catalysts}
\begin{itemize}
    \item \marginnote{4/27:}Example problem 1: Steady-state approximation.
    \begin{itemize}
        \item Let
        \begin{equation*}
            \ce{A <=>[$k_a$][$k_a'$] B <=>[$k_b$][$k_b'$] C <=>[$k_c$][$k_c'$] D}
        \end{equation*}
        Suppose $\cnc{A}$ is maintained at a fixed value and the produce \ce{D} is removed from the reaction as it is formed. Find the rate at which the product is formed in terms of $\cnc{A}$.
        \item By hypothesis, we have that at all times $t$, $\cnc{A}=\cnc[0]{A}$ and $\cnc{D}=0$.
        \item The hypotheses also imply that we can apply the steady-state approximation to both \ce{B} and \ce{C}.
        \item Thus, we have that
        \begin{align*}
            \dv{\cnc{C}}{t} = 0 &= k_b\cnc{B}-k_c\cnc{C}-k_b'\cnc{C}\\
            \cnc{B} &= \frac{k_b'+k_c}{k_b}\cnc{C}
        \end{align*}
        so that
        \begin{align*}
            \dv{\cnc{B}}{t} &= k_a\cnc{A}-k_b\cnc{B}-k_a'\cnc{B}+k_b'\cnc{C}\\
            0 &= k_a\cnc{A}-k_b\cdot\frac{k_b'+k_c}{k_b}\cnc{C}-k_a'\cdot\frac{k_b'+k_c}{k_b}\cnc{C}+k_b'\cnc{C}\\
            % &= k_a\cnc{A}-(k_b'+k_c+\frac{k_a'k_b'}{k_b}+\frac{k_a'k_c}{k_b}-k_b')\cnc{C}\\
            % &= k_a\cnc{A}-(\frac{k_bk_c+k_a'k_b'+k_a'k_c}{k_b})\cnc{C}\\
            \cnc{C} &= \frac{k_ak_b}{k_bk_c+k_a'k_b'+k_a'k_c}\cnc{A}
        \end{align*}
        and therefore
        \begin{align*}
            \dv{\cnc{D}}{t} &= k_c\cnc{C}-k_c'\cdot 0\\
            &= \frac{k_ak_bk_c}{k_bk_c+k_a'k_b'+k_a'k_c}\cnc{A}
        \end{align*}
    \end{itemize}
    \item Example problem 2.
    \begin{itemize}
        \item Consider the reaction
        \begin{equation*}
            \ce{HCl + CH3CH=CH2 <=> CH3CHClCH3}
        \end{equation*}
        which proceeds by the mechanism
        \begin{enumerate}
            \item \ce{HCl + HCl <=> (HCl)2} (equilibrium constant $K_1$).
            \item \ce{HCl + CH3CH=CH2 <=> complex} (equilibrium constant $K_2$).
            \item \ce{(HCl)2 + complex <=> CH3CHClCH3 + HCl + HCl} (equilibrium constant $K_3$).
        \end{enumerate}
        \item The equilibrium constants for the two pre-equilibria are
        \begin{align*}
            K_1 &= \frac{\cnc[eq]{(HCl)2}c^\circ}{\cnc[eq]{HCl}^2}&
            K_2 &= \frac{\cnc[eq]{complex}c^\circ}{\cnc[eq]{HCl}\cnc[eq]{CH3CH=CH2}}
        \end{align*}
        \item We can divide the mass-action expression for $K_1$ by $(c^\circ)^2$ to get each concentration over $c^\circ$ within its exponent.
        \item The rate of product formation is
        \begin{align*}
            v &= \dv{\cnc{CH3CHClCH3}}{t}\\
            &= k_r\cnc{(HCl)2}\cnc{complex}\\
            &\approx k_r\cnc[eq]{(HCl)2}\cnc[eq]{complex}\\
            &= k_r\cdot\frac{K_1\cnc[eq]{HCl}^2}{c^\circ}\cdot\frac{K_2\cnc[eq]{HCl}\cnc[eq]{CH3CH=CH2}}{c^\circ}\\
            &= \frac{k_rK_1K_2}{(c^\circ)^2}\cnc[eq]{HCl}^3\cnc[eq]{CH3CH=CH2}
        \end{align*}
        \begin{itemize}
            \item There's a key assumption with the steady state and something about being able to apply the equilibrium concentration of the intermediate as the steady-state quantity.
            \item This question wants to let you know that an equilibrium constant like $K_1$ might indicate a steady-state approximation.
        \end{itemize}
    \end{itemize}
    \item Note: Mind the positive and negative signs when constructing differential rate laws!
    \item The midterm will be posted this Friday (April 29) and will be available until the following Friday (May 6). There will be a timed 2 hour period to take it.
    \item \textbf{Catalyst}: A substance that participates in the chemical reaction but is not consumed in the process.
    \begin{itemize}
        \item A catalyst affects the mechanism and activation energy of a chemical reaction.
        \item A catalyst can give rise to a reaction path with a negligible activation barrier.
        \item The exothermicity or endothermicity of the chemical reaction is not altered by the presence of a catalyst.
    \end{itemize}
    \item \textbf{Homogeneous catalysis}: Catalysis in which the catalyst is in the same phase as the reactants and products.
    \item \textbf{Heterogeneous catalysis}: Catalysis in which the catalyst is in a different phase from the reactants and products.
    \item Imagine that initially, we have the reaction
    \begin{equation*}
        \ce{A ->[$k$] products}
    \end{equation*}
    where $k$ is the observed rate constant.
    \begin{itemize}
        \item When a catalyst is introduced into solution, this mechanism continues, but we now also have the new reaction pathway
        \begin{equation*}
            \ce{A + catalyst ->[$k_\text{cat}$] products + catalyst}
        \end{equation*}
        \item If each of these competing reactions is an elementary process, then
        \begin{equation*}
            -\dv{\cnc{A}}{t} = k\cnc{A}+k_\text{cat}\cnc{A}\cnc{catalyst}
        \end{equation*}
        \item In most cases, catalysts enhance reaction rates by many orders of magnitude, and therefore only the rate law for the catalyzed reaction need be considered in analyzing experimental data.
    \end{itemize}
    \item Reviews the Nobel Prizes in 2020 and 2021 (for CRISPR and asymmetric organocatalysis, respectively).
    \item An example of homogeneous catalysis.
    \begin{itemize}
        \item Consider the reaction
        \begin{equation*}
            \ce{2Ce^4+(aq) + Tl+(aq) -> 2Ce^3+(aq) + Tl^3+(aq)}
        \end{equation*}
        \item In the absence of a catalyst,
        \begin{equation*}
            v = k\cnc{Tl+}\cnc{Ce^4+}^2
        \end{equation*}
        and the mechanism is a termolecular elementary reaction.
        \item However, with \ce{Mn^2+} as the catalyst, we have the mechanism
        \begin{align*}
            \ce{Ce^4+(aq) + Mn^2+(aq)} &\xRightarrow{k_\text{cat}}                \ce{Mn^3+(aq) + Ce^3+(aq)}\\
            \ce{Ce^4+(aq) + Mn^3+(aq)} &\xRightarrow{{\color{white}k_\text{cat}}} \ce{Mn^4+(aq) + Ce^3+(aq)}\\
            \ce{Mn^4+(aq) + Tl+(aq)}   &\xRightarrow{{\color{white}k_\text{cat}}} \ce{Mn^2+(aq) + Tl^3+(aq)}
        \end{align*}
        where the step with $k_\text{cat}$ is the rate-determining step.
        \begin{itemize}
            \item Thus, for this mechanism, we have that
            \begin{equation*}
                v = k_\text{cat}\cnc{Ce^4+}\cnc{Mn^2+}
            \end{equation*}
        \end{itemize}
        \item The overall rate law for this reaction is therefore
        \begin{equation*}
            v = k\cnc{Tl+}\cnc{Ce^4+}^2+k_\text{cat}\cnc{Ce^4+}\cnc{Mn^2+}
        \end{equation*}
    \end{itemize}
\end{itemize}



\section{Enzymatic Catalysis}
\begin{itemize}
    \item \marginnote{4/27:}Midterm questions:
    \begin{itemize}
        \item First 10 are T/F. He will test key concepts by making statements that are either true or false.
        \begin{itemize}
            \item We should expect to spend no more than 30 minutes out of our 2 hours on these.
        \end{itemize}
        \item 4 calculation problems.
        \begin{itemize}
            \item First- and second-order reactions.
            \item Collisions.
            \item A reaction mechanism problem.
        \end{itemize}
        \item Use calculators, do online searches, and use the textbook.
        \item Do not talk to your classmates.
        \item The midterm will become available Friday at noon.
    \end{itemize}
    \item \textbf{Enzyme}: A protein molecule that catalyzes a specific biochemical reaction.
    \begin{itemize}
        \item For example, hexokinase converts glucose and ATP to glucose 6-phosphate, ADP, and \ce{H+}.
    \end{itemize}
    \item \textbf{Substrate}: The reactant molecule acted upon by an enzyme.
    \item \textbf{Active site}: The region of the enzyme where the substrate reacts.
    \item \textbf{Lock-and-key model}: The active site and substrate have complementary three-dimensional structures and dock without the need for major atomic rearrangements.
    \item \textbf{Induced fit model}: Binding of the substrate induces a conformation change in the active site. The substrate fits well in the active site after the conformational change has taken place.
    \item The Michaelis-Menten mechanism is a reaction mechanism for enzyme catalysis.
    \item Intuition.
    \begin{itemize}
        \item Imagine we have a solution of enzymes and substrate molecules.
        \item Limiting factors of an enzymatically catalyzed reaction.
        \begin{itemize}
            \item The enzyme-substrate affinity.
            \item The turnover number.
        \end{itemize}
        \item If the substrate concentration is low (i.e., $\cnc[0]{S}\ll\cnc[0]{E}$) and the enzyme-substrate affinity is strong (but not so strong that the enzyme-substrate complex is energetically favorable), then we expect $v_\text{initial}\propto\cnc[0]{S}$ because we'd think that all of the substrate will immediately be absorbed and transformed.
        \item If the substrate concentration is large (i.e., $\cnc[0]{S}\gg\cnc[0]{E}$) and the enzyme-substrate affinity is strong, then we expect $v_\text{initial}\propto\cnc[0]{E}$ and, importantly, $v_\text{initial}\not\propto\cnc[0]{S}$.
    \end{itemize}
    \item Mathematical derivation.
    \begin{itemize}
        \item Experimental studies reveal that the rate law for many enzyme-catalyzed reactions has the form
        \begin{equation*}
            -\dv{\cnc{S}}{t} = \frac{k\cnc{S}}{K_m+\cnc{S}}
        \end{equation*}
        \begin{itemize}
            \item This is the final goal of the derivation.
        \end{itemize}
        \item The mechanism is
        \begin{equation*}
            \ce{S + E} \Longleftrightarrows[k_1][k_{-1}] \ce{ES}
            \Longleftrightarrows[k_2][k_{-2}] \ce{P + E}
        \end{equation*}
        \item Thus,
        \begin{gather*}
            -\dv{\cnc{S}}{t} = k_1\cnc{E}\cnc{S}-k_{-1}\cnc{ES}\\
            -\dv{\cnc{ES}}{t} = (k_2+k_{-1})\cnc{ES}-k_1\cnc{E}\cnc{S}-k_{-2}\cnc{E}\cnc{P}\\
            \dv{\cnc{P}}{t} = k_2\cnc{ES}-k_{-1}\cnc{E}\cnc{P}
        \end{gather*}
        \item Note that
        \begin{equation*}
            \cnc[0]{E} = \cnc{ES}+\cnc{E}
        \end{equation*}
        \item Plugging that equation into the rate law for the enzyme-substrate complex and applying the steady-state approximation yields
        \begin{align*}
            -\dv{\cnc{ES}}{t} = 0 &= \cnc{ES}(k_1\cnc{S}+k_{-1}+k_2+k_{-2}\cnc{P})-k_1\cnc{S}\cnc[0]{E}-k_{-2}\cnc{P}\cnc[0]{E}\\
            \cnc{ES} &= \frac{k_1\cnc{S}+k_{-2}\cnc{P}}{k_1\cnc{S}+k_{-2}\cnc{P}+k_{-1}+k_2}\cnc[0]{E}
        \end{align*}
        \item Substituting this and the original expression for $\cnc[0]{E}$ into the rate law for the substrate yields
        \begin{equation*}
            v = -\dv{\cnc{S}}{t}
            = \frac{k_1k_2\cnc{S}+k_{-1}k_{-2}\cnc{P}}{k_1\cnc{S}+k_{-2}\cnc{P}+k_{-1}+k_2}\cnc[0]{E}
        \end{equation*}
        \item If the experimental measurements of the reaction rate are taken during the time period when only a small percentage (1-3\%) of the substrate is converted to product, then
        \begin{equation*}
            \cnc{S} \approx \cnc[0]{S}
        \end{equation*}
        and
        \begin{equation*}
            \cnc{P} \approx 0
        \end{equation*}
        \item Using this approximation simplifies the above rate law to
        \begin{equation*}
            v = -\dv{\cnc{S}}{t}
            = \frac{k_1k_2\cnc[0]{S}\cnc[0]{E}}{k_1\cnc[0]{S}+k_{-1}+k_2}
            = \frac{k_2\cnc[0]{S}\cnc[0]{E}}{K_m+\cnc[0]{S}}
        \end{equation*}
        where $K_m=(k_{-1}+k_2)/k_1$ is the \textbf{Michaelis constant}.
        \begin{itemize}
            \item The Michaelis constant tells you the ratio of dissociation of the enzyme-substrate complex to the formation of the enzyme-substrate complex. In other words, it provides information on the enzyme-substrate affinity.
            \item Note that $k_{-2}$ is not present in the denominator of the Michaelis constant because for a good enzyme, $k_{-2}$ should be very small.
            \item The unit of $K_m$ should be concentration.
            \item When $K_m=\cnc[0]{S}$, $v=v_\text{max}/2$
        \end{itemize}
        \item An enzyme-catalyzed reaction is first order in the substrate at low substrate concentrations ($K_m\gg\cnc[0]{S}$) and then becomes zero order in the substrate at high substrate concentrations ($K_m\ll\cnc[0]{S}$).
        \item Thus, at low substrate concentrations, the above equation holds, but at high substrate concentrations,
        \begin{align*}
            -\dv{\cnc{S}}{t} &= k_2\cnc[0]{E}&
            v_\text{max} &= k_2\cnc[0]{E}
        \end{align*}
        resulting in the \textbf{Lineweaver-Burk plot}, canonically represented by the second of the two equivalent forms below.
        \begin{align*}
            v &= \frac{v_\text{max}}{1+K_m/\cnc[0]{S}}&
            \frac{1}{v} &= \frac{1}{v_\text{max}}+\frac{K_m}{v_\text{max}}\frac{1}{\cnc[0]{S}}
        \end{align*}
    \end{itemize}
\end{itemize}



\section{Measuring Catalytic Efficiency and Correcting Collision Theory}
\begin{itemize}
    \item \marginnote{4/29:}Everything on the midterm comes from Tian's lecture notes. Thus, he recommends we go through them before taking the midterm.
    \begin{itemize}
        \item The T/F will be 20-30\% of the grade.
        \item There will be some integration on the calculation problems, but they'll be pretty easy. All formulas that will appear have been covered in class.
        \item The midterm will cover up to Monday's class (this week).
    \end{itemize}
    \item Consider the substrate concentration $\cnc[0]{S}$ vs. the initial rate $v_0$.
    \begin{figure}[h!]
        \centering
        \begin{subfigure}[b]{0.45\linewidth}
            \centering
            \begin{tikzpicture}[scale=2]
                \small
                \draw [stealth-stealth] (0,1.1) node[above]{$v_0$} -- (0,0) -- (2.1,0) node[right]{$\cnc[0]{S}$};
                \footnotesize
                \draw (0.05,1) -- ++(-0.1,0) node[left]{$v_\text{max}$};
                \draw (0.05,0.5) -- ++(-0.1,0) node[left]{$\frac{v_\text{max}}{2}$};
                \draw (0.2,0.05) -- ++(0,-0.1) node[below]{$K_m$};
        
                \draw [dashed,very thin]
                    (0,1) -- ++(2,0)
                    (0.2,0) -- ++(0,0.5) -- ++(-0.2,0)
                ;
        
                \draw [rex,thick] plot[domain=0:2,smooth,samples=100] (\x,{\x/(0.2+\x)});
            \end{tikzpicture}
            \caption{Normal plot.}
            \label{fig:v0S0a}
        \end{subfigure}
        \begin{subfigure}[b]{0.45\linewidth}
            \centering
            \begin{tikzpicture}[scale=2]
                \small
                \draw [stealth-stealth] (0,1.1) node[above]{$\frac{1}{v_0}$} -- (0,0) -- (2.1,0) node[right]{$\frac{1}{\cnc[0]{S}}$};
                \footnotesize
                \path (0.2,0.05) -- ++(0,-0.1) node[white,below]{$K_m$};
        
                \draw [rex,thick] (0,0.25) node[circle,fill,inner sep=1.5pt]{} node[black,left]{$\left( 0,\frac{1}{v_\text{max}} \right)$} -- node[black,above=1mm]{$\text{slope}=\frac{K_m}{v_\text{max}}$} (2,0.35);
            \end{tikzpicture}
            \caption{Reciprocal plot.}
            \label{fig:v0S0b}
        \end{subfigure}
        \caption{Plotting $v_0$ vs. $\cnc[0]{S}$.}
        \label{fig:v0S0}
    \end{figure}
    \begin{itemize}
        \item Normal plot (Figure \ref{fig:v0S0a}).
        \begin{itemize}
            \item As $\cnc{S}\to\infty$, $v_0$ approaches an asymptote line defined by $v_\text{max}=k_2\cnc[0]{E}$.
            \item At the beginning (low $\cnc{S}$), the paradigm is almost linear (thus, the reaction is first order wrt. substrate concentration here).
            \item Note that we make use of the assumptions that $\cnc{P}\approx 0$ and $\cnc{S}\approx\cnc[0]{S}$.
            \item Where we have $v_\text{max}/2$ on the $v_0$-axis, we have $K_m$ (the Michaelis constant) on the $\cnc[0]{S}$ axis.
        \end{itemize}
        \item Reciprocal plot (Figure \ref{fig:v0S0b}).
        \begin{itemize}
            \item Note that a plot of $v_0$ vs. $\cnc[0]{S}$ is nonlinear but a plot of $1/v_0$ vs. $1/\cnc[0]{S}$ is linear.
            \item This reciprocal plot (the Lineweaver-Burk plot) gives $v_\text{max}$ (via the $y$-intercept) and $K_m$ via this information and the slope.
            \item Note that
            \begin{align*}
                \frac{1}{v_0} &\propto \frac{K_m}{v_\text{max}}\frac{1}{\cnc[0]{S}}\\
                &= \frac{(k_{-1}+k_2)/k_1}{k_2\cnc[0]{E}}\frac{1}{\cnc[0]{S}}\\
                &= \frac{k_{-1}+k_2}{k_1k_2}\frac{1}{\cnc[0]{E}\cnc[0]{S}}
            \end{align*}
        \end{itemize}
    \end{itemize}
    \item Evaluating the performance of a catalyst.
    \begin{equation*}
        \frac{k_{-1}+K_2}{k_1k_2} = \frac{K_m}{v_\text{max}}\cdot\cnc[0]{E}
    \end{equation*}
    \begin{itemize}
        \item $K_m$ is relevant to the enzyme-substrate affinity.
        \item $v_\text{max}$ tells us about conversion from the ES with a focus on the second elementary step.
    \end{itemize}
    \item An alternate form of the Lineweaver-Burk plot is
    \begin{equation*}
        \frac{1}{v_0} = \frac{1}{v_\text{max}}+\frac{k_2+k_{-1}}{k_1k_2}\frac{1}{\cnc[0]{E}\cnc[0]{S}}
    \end{equation*}
    \begin{itemize}
        \item Regrouping the terms, we have
        \begin{equation*}
            \frac{1}{v_0} = \frac{1}{v_\text{max}}+\frac{k_2+k_{-1}}{k_2}\frac{1}{k_1\cnc[0]{S}\cnc[0]{E}}
            = \frac{1}{v_\text{max}}+\frac{k_2+k_{-1}}{k_2}\frac{1}{v_{f1}}
        \end{equation*}
        where $v_{f1}$ is the forward reaction rate for the first elementary step.
    \end{itemize}
    \item \textbf{Turnover number}: The number of catalytic cycles that each active site undergoes per unit time. \emph{Given by}
    \begin{equation*}
        \text{TON} = \frac{v_\text{max}}{n\cnc[0]{E}}
        = \frac{k_2\cnc[0]{E}}{n\cnc[0]{E}}
        = \frac{k_2}{n}
    \end{equation*}
    \begin{itemize}
        \item Indicates how fast the \ce{ES} complex proceeds to \ce{E + P}.
        \item $k_2/n$ is the number of active sites per enzyme.
    \end{itemize}
    \item \textbf{Catalytic efficiency}: The following quantity. \emph{Given by}
    \begin{equation*}
        \frac{\text{TON}}{K_m}
    \end{equation*}
    \item Consider the reaction
    \begin{equation*}
        \ce{A(g) + B(g)} \xRightarrow{k} \ce{products}
    \end{equation*}
    \begin{itemize}
        \item The rate of the general bimolecular elementary gas-phase reaction is
        \begin{equation*}
            v = -\dv{\cnc{A}}{t} = k\cnc{A}\cnc{B}
        \end{equation*}
        \item Using the na\"{i}ve assumption that every collision between the hard spheres \ce{A} and \ce{B} yields products,
        \begin{equation*}
            v = Z_{\ce{A}\ce{B}} = \sigma_{\ce{A}\ce{B}}\prb{u_r}\rho_{\ce{A}}\rho_{\ce{B}}
        \end{equation*}
        \item Moreover,
        \begin{equation*}
            k = \sigma_{\ce{A}\ce{B}}\prb{u_r}
        \end{equation*}
        \item Unfortunately, this is not accurate. We make our first improvement to collision theory by taking into account the dependence of the reaction rate on the relative speed, or energy, of the collision. Thus, we average over all possible collision speeds.
        \begin{equation*}
            k = \int_0^\infty\dd{u_r}f(u_r)k(u_r)
            = \int_0^\infty\dd{u_r}u_rf(u_r)\sigma_r(u_r)
        \end{equation*}
        \item Since $f(u_r)$ is the distribution of relative speeds in the gas sample, we have that
        \begin{equation*}
            u_rf(u_r)\dd{u_r} = \left( \frac{\mu}{\kB T} \right)^{3/2}\left( \frac{2}{\pi} \right)^{1/2}u_r^3\e[-\mu u_r^2/2\kB T]\dd{u_r}
        \end{equation*}
        \item To compare this with the traditional Arrhenius form of $k$, we need to change the dependent variable from $u_r$ to $E$, which we can do via
        \begin{align*}
            E_r &= \frac{1}{2}\mu u_r^2&
            u_r &= \sqrt{\frac{2E_r}{\mu}}&
            \dd{u_r} &= \sqrt{\frac{1}{2\mu E_r}}\dd{E_r}&
            E_a &= \kB T^2\dv{\ln k}{T}
        \end{align*}
    \end{itemize}
\end{itemize}



\section{Office Hours (Tian)}
\begin{itemize}
    \item Can Problem 28-46 be done purely with integrated forms of the Arrhenius equation?
    \begin{itemize}
        \item We can use either case depending on how the question was asked, and often it's an instance of which would be easier to use. We should always think both ways.
    \end{itemize}
    \item Ask about the Problem 29-24 derivation from class final steps as well as molecularity.
    \begin{itemize}
        \item You have a 2 and a squared. The exponent comes from the fact that it's bimolecular. The coefficient comes from the fact that we're talking about the rate of change of that substance.
        \item In particular, if \ce{2C ->[$k_1$] D}, then
        \begin{equation*}
            v = k_1\cnc{C}^2
        \end{equation*}
        \item But recall that $v$ is the \emph{rate of reaction}, a specifically defined quantity. Indeed, we know that
        \begin{align*}
            v &= -\frac{1}{\nu_{\ce{C}}}\dv{\cnc{C}}{t}
                = \frac{1}{\nu_{\ce{D}}}\dv{\cnc{D}}{t}\\
            &= -\frac{1}{2}\dv{\cnc{C}}{t}
            = \frac{1}{1}\dv{\cnc{D}}{t}
        \end{align*}
        so that this step's contribution to $\dv*{\cnc{C}}{t}$ is
        \begin{align*}
            -\frac{1}{2}\dv{\cnc{C}}{t} &= k_1\cnc{C}^2\\
            \dv{\cnc{C}}{t} &= -2k_1\cnc{C}^2
        \end{align*}
    \end{itemize}
    \item Eyring equation and thermodynamics questions: So what we need to know and be able to work with are the Eyring equation and you said "the dimensional analysis?" I'm still unclear on what $\prb{u_\text{ac}}$ and $\nu_c$ are.
    \begin{itemize}
        \item $\nu_c$ is frequency. It's units are reciprocal seconds. It's 1 over the time it takes to cross the barrier. Length divided by time gives the rate of crossing the barrier. And rate depends on reaction coordinate.
        \item This is very hard to visualize along a typical, complicated reaction coordinate. However, if we think about an S\textsubscript{N}2 for instance, we can picture $\delta$ as some tolerance around the equatorial ligands being perfectly coplanar with the central atom, $1/\nu_c$ as the time from when the molecules begin reacting (nucleophilic attack) to when the leaving group has fully left, and thus $\prb{u_\text{ac}}$ as the average speed with which the activated complex crosses the barrier top, i.e., is formed, moved, and dissociates.
        \item The conclusion is more important than the derivation. There's a lot of assumptions that are not as relevant to the eventual result. The conclusion being that we can express the pre-exponential factor can be expressed in terms of molecular quantities from statistical mechanics.
        \item TST is imperfect; some textbooks use $P^\circ$ instead of $c^\circ$ since this is a gas-phase reaction.
        \item $\Delta^\ddagger PV=\Delta^\ddagger nRT$ hails from the ideal gas law. $\Delta^\ddagger n$ is the change in the number of molecules from the reactants to the activated complex (which is always unimolecular!). $1-\Delta n=\text{molecularity}$. We need $\Delta^\ddagger U^\circ$. To get an expression for it, we invoke enthalpy ($H$) which brings in the change in $PV$ going to the transition state, which is related to the change in the number of moles of gas!
    \end{itemize}
    \item What does the double vs. single arrow signify?
    \begin{itemize}
        \item Elementary reactions vs. reactions that might not be.
    \end{itemize}
    \item The differences between $F(u)$, $f(u_x)$, and $h(u)$.
    \begin{itemize}
        \item $f(u_x)$ is the Gaussian distribution of velocity components.
        \item $F(u)$ is the M-B distribution.
        \item $h(u_x,u_y,u_z)$ is the product $f(u_x)f(u_y)f(u_z)$ and describes the fraction of molecules not just moving linearly in one dimension or the other, but in any direction. Graphically, we'd need four dimensions, but we can visualize it either as equipotential 2-spheres or, using the color model in 3D, as intensity getting brighter with a Gaussian distribution cubed as you approach the origin.
    \end{itemize}
    \item Midterm stuff.
    \begin{itemize}
        \item The midterm will have two problems on rate laws (one of which might concern the Eyring equation), one mechanism question (how the approximations provided lead into the final rate law), and one collision theory question (the original collision theory stuff, not the more recent stuff).
        \item For each T/F problem, there will be two sentences. You need to identify all of the mistakes in the sentences. Answer if it's true or false, and explain. One point for T/F; two points for your explanation. If the sentence is true, a few words might be helpful, too.
        \item We'll need to show some decent steps to recreate the work. Clear doubt from the TA's mind.
        \item If there is a mistake in a problem, just do your best to solve it and flag it for him.
    \end{itemize}
    \item Fast equilibrium vs. steady state?
    \begin{itemize}
        \item Detailed balance only applies to elementary steps?
    \end{itemize}
\end{itemize}



\section{Chapter 29: Chemical Kinetics II --- Reaction Mechanisms}
\emph{From \textcite{bib:McQuarrieSimon}.}
\begin{itemize}
    \item \marginnote{5/1:}Methods of increasing the rate of reaction.
    \begin{enumerate}
        \item Increase the temperature.
        \begin{itemize}
            \item Drawback: Reactions in solution are constrained to the temperature range between the melting and boiling point of the solvent.
        \end{itemize}
        \item Enable the reaction to proceed by an alternate mechanism having a lower activation energy.
    \end{enumerate}
    \item \textbf{Catalyst}: A substance that participates in a chemical reaction but is not consumed in the process.
    \begin{itemize}
        \item By participating, a catalyst provides an alternate mechanism.
        \item The trick is to construct this alternate mechanism such that it has a lower activation barrier.
    \end{itemize}
    \item \textcite{bib:McQuarrieSimon} defines \textbf{homogeneous} and \textbf{heterogeneous} catalysis.
    \item Because a catalyst is not consumed or otherwise chemically altered (in a net sense), the thermodynamics of the reaction do not change when one is present.
    \item The exponential form of the Arrhenius equation implies that even small changes in activation energy can lead to substantial changes in reaction rate.
    \item Note that since a catalyzed reaction has multiple mechanisms, we say it proceeds along reaction \emph{coordinates}, plural.
    \item \textcite{bib:McQuarrieSimon} derives the rate law for a unimolecular uncatalyzed, bimolecular catalyzed reaction.
    \item \textcite{bib:McQuarrieSimon} gives the ionic homogeneous catalysis example.
    \item An example of heterogeneous catalysis.
    \begin{itemize}
        \item Consider the reaction
        \begin{equation*}
            \ce{3H2(g) + N2(g) -> 2NH3(g)}
        \end{equation*}
        \item The activation barrier is roughly \SI[per-mode=symbol]{940}{\kilo\joule\per\mole}.
        \item In the presence of an iron surface, however, the activation barrier drops over an order of magnitude to \SI[per-mode=symbol]{80}{\kilo\joule\per\mole}.
        \item The mechanism will be discussed in depth in Chapter 31, along with other heterogeneous surface-catalyzed gas-phase reactions.
    \end{itemize}
    \item An example where both types of catalysis play a role.
    \begin{itemize}
        \item Consider the destruction of ozone in the stratosphere.
        \item Naturally, this reaction occurs via
        \begin{equation*}
            \ce{O3(g) + O(g)} \Longrightarrow \ce{2O2(g)}
        \end{equation*}
        \item In the presence of chlorine atoms, however, the following mechanism becomes available.
        \begin{align*}
            \ce{O3(g) + Cl(g)} &\Longrightarrow \ce{ClO(g) + O2(g)}\\
            \ce{ClO(g) + O(g)} &\Longrightarrow \ce{O2(g) + Cl(g)}
        \end{align*}
        \begin{itemize}
            \item This is an example of homogeneous catalysis.
        \end{itemize}
        \item Over time, however, chlorine atoms get bound up in the reservoir molecules \ce{HCl(g)} and \ce{ClONO2(g)}.
        \item Nevertheless, the surface of polar stratospheric clouds catalyzes the reaction between these two molecules to liberate a molecule of diatomic chlorine gas (which can then be homolytically dissociated by sunlight to regenerate the gaseous monoatomic chlorine catalyst) and a moleucle of \ce{HNO3(g)}.
        \begin{itemize}
            \item Since the surface of the clouds are of a different phase from the gas, this is an example of heterogeneous catalysis.
        \end{itemize}
    \end{itemize}
    \item \textcite{bib:McQuarrieSimon} defines \textbf{enzymes}, \textbf{substrates}, and \textbf{active sites}.
    \item \textbf{Michaelis-Menten mechanism}: A simple mechanism that accounts for the rate law commonly observed for enzyme-catalyzed reactions. \emph{Given by}
    \begin{equation*}
        \ce{E + S} \Longleftrightarrows[k_1][k_{-1}] \ce{ES} \Longleftrightarrows[k_2][k_{-2}] \ce{E + P}
    \end{equation*}
    \begin{itemize}
        \item Proposed by Leonor Michaelis and Maude Menten in 1913.
        \item This reaction sees an initial buildup period of \ce{ES}, followed by a period during which $\cnc{ES}$ is relatively constant. Thus, we may apply the SS approximation to it.
    \end{itemize}
    \item High substrate concentrations in the Michaelis-Menten mechanism.
    \begin{itemize}
        \item At high substrate concentrations, essentially all of the enzymes are tied up with substrate, so adding more substrate doesn't do anything and the reaction is zero-order in substrate concentration.
        \item Assuming that all enzyme is tied up in substrate means that $\cnc{ES}\approx\cnc[0]{E}$ and the rate of SM consumption is essentially equal to the rate of product formation, which is just proportional to the dissociation of the reacted enzyme-substrate complex. Mathematically,
        \begin{align*}
            -\dv{\cnc{S}}{t} &= k_2\cnc[0]{E}\\
            v_\text{max} &= k_2\cnc[0]{E}
        \end{align*}
        \begin{itemize}
            \item Note that we introduce the $v_\text{max}$ terminology because it is under these conditions (excess substrate) that the reaction rate is at its maximum.
        \end{itemize}
    \end{itemize}
    \item \textbf{Turnover number}: The maximum rate divided by the concentration of enzyme active sites.
    \begin{itemize}
        \item By definition, the turnover is the maximum number of substrate molecules that can be converted into product molecules per unit time by an enzyme molecule.
        \item Note that the concentration of enzyme \emph{active sites} is not necessarily equal to $\cnc{E}$ because some enzymes have more than one active site.
    \end{itemize}
    \item For an enzyme having only one active site, the turnover number is given by $v_\text{max}/\cnc[0]{E}=k_2$.
\end{itemize}



\section{Chapter 30: Gas-Phase Reaction Dynamics}
\emph{From \textcite{bib:McQuarrieSimon}.}
\begin{itemize}
    \item \marginnote{5/18:}Goals of the chapter.
    \begin{itemize}
        \item Describe bimolecular gas-phase reactions, some of the simplest naturally occurring elementary kinetic processes.
        \item Analyze the reaction $\ce{F(g) + D2(g)}\Rightarrow\ce{DF(g) + D(g)}$, an exothermic cousin of the hydrogen exchange reaction $\ce{H_A + H_B-H_C}\Rightarrow\ce{H_A-H_B + H_C}$.
    \end{itemize}
    \item Na\"{i}ve hard-sphere collision theory.
    \begin{itemize}
        \item Consider the following general bimolecular elementary gas-phase reaction.
        \begin{equation*}
            \ce{A(g) + B(g)} \xRightarrow{k} \text{products}
        \end{equation*}
        \item The rate of reaction is given by
        \begin{equation*}
            v = -\dv{\cnc{A}}{t}
            = k\cnc{A}\cnc{B}
        \end{equation*}
        \item Recall that when we derived this equation in Chapter 29, we assumed that every collision between molecules of \ce{A} and \ce{B} is chemically active and thus related the rate of reaction to the collision frequency per unit volume $Z_{\ce{A}\ce{B}}$ via
        \begin{equation*}
            v = Z_{\ce{A}\ce{B}}
            = \sigma_{\ce{A}\ce{B}}\prb{u_r}\rho_{\ce{A}}\rho_{\ce{B}}
        \end{equation*}
        where $\sigma_{\ce{A}\ce{B}}$ is the hard-sphere collision cross section of \ce{A} and \ce{B} molecules (see Figure \ref{fig:collisionCylinder}), $\prb{u_r}$ is the average relative speed of a colliding pair of \ce{A} and \ce{B} molecules, and $\rho_{\ce{A}},\rho_{\ce{B}}$ are the respective number densities.
        \begin{itemize}
            \item Note that $\sigma_{\ce{A}\ce{B}}=\pi d_{\ce{A}\ce{B}}^2$ where $d_{\ce{A}\ce{B}}$ is the sum of the radii of the two colliding spheres.
            \item Since we are assuming that every collision is successful, we have that every one of the $Z_{\ce{A}\ce{B}}$ collisions happening every second in every cubic meter of volume consumes a reactant molecule. In other words, $Z_{\ce{A}\ce{B}}$ gives the change in molecular concentration of \ce{A} per unit time, so the statement $v=-\dv*{\cnc{A}}{t}=Z_{\ce{A}\ce{B}}$ is justified.
        \end{itemize}
        \item Identifying $\cnc{A}\sim\rho_{\ce{A}}$ and $\cnc{B}\sim\rho_{\ce{B}}$ gives the rate constant as
        \begin{equation*}
            k = \sigma_{\ce{A}\ce{B}}\prb{u_r}
        \end{equation*}
    \end{itemize}
    \item To convert $k$ from the standard SI units of \si{\cubic\meter\per\molecule\per\second} to the more conventional units of \si{\cubic\deci\meter\per\mole\per\second}, we can multiply the above expression by $(\SI{1000}{\cubic\deci\meter\per\cubic\meter})(\NA\ \si{\per\mole})$.
    \item Problems with na\"{i}ve hard-sphere collision theory.
    \begin{itemize}
        \item The calculated rate constants are often significantly larger than experimental rate constants.
        \item Since $\prb{u_r}\propto T^{1/2}$, it predicts $k\propto T^{1/2}$ instead of the experimental Arrhenius dependence of $k\propto\e[1/T]$.
    \end{itemize}
    \item Two underlying assumptions of na\"{i}ve hard-sphere collision theory.
    \begin{itemize}
        \item Each pair of reactants approaches one another with a relative speed of $\prb{u_r}$.
        \begin{itemize}
            \item Reality: Pairs of reactant molecules approach each other with a (M-B) distribution of speeds.
        \end{itemize}
        \item Every collision is chemically reactive, regardless of speed or energy.
        \begin{itemize}
            \item Reality: Since the valence electrons of the two molecules repel one another, a reaction will not occur unless the relative speed is sufficient to overcome this repulsive force.
        \end{itemize}
    \end{itemize}
    \item Developing a model of collision theory that accounts for differing speeds.
    \begin{itemize}
        \item We consider each speed $u_r$ with which molecules can collide to give rise to a different rate constant $k(u_r)$ and corresponding \textbf{reaction cross section} $\sigma_r(u_r)$. These quantities are related via
        \begin{equation*}
            k(u_r) = u_r\sigma_r(u_r)
        \end{equation*}
        \begin{itemize}
            \item It makes sense to continue using this relationship since $k=\sigma_{\ce{A}\ce{B}}\prb{u_r}$ relies on the assumption that all molecules travel at the same speed $\prb{u_r}$, and $k(u_r)=u_r\sigma_r(u_r)$ relies on the assumption that all molecules travel at the same speed $u_r$.
        \end{itemize}
        \item To calculate the observed rate constant, we must average all $k(u_r)$'s over the speed distribution. We can do this with
        \begin{equation*}
            k = \int_0^\infty\dd{u_r}f(u_r)k(u_r)
            = \int_0^\infty\dd{u_r}u_rf(u_r)\sigma_r(u_r)
        \end{equation*}
        \item As per Chapter 27,
        \begin{align*}
            u_rf(u_r)\dd{u_r} &= 4\pi\left( \frac{\mu}{2\pi\kB T} \right)^{3/2}u_r^3\e[-\mu u_r^2/2\kB T]\dd{u_r}\\
            &= \left( \frac{\mu}{\kB T} \right)^{3/2}\left( \frac{2}{\pi} \right)^{1/2}u_r^3\e[-\mu u_r^2/2\kB T]\dd{u_r}
        \end{align*}
        \item We now encounter the question of what our threshold relative molecular collision speed is. In fact, we have never discussed such a quantity. However, we have spent plenty of time developing a theory of the threshold energy $E_a$, the Arrhenius activation energy. As such, a change of variables from speed to energy is in order. In particular, since
        \begin{equation*}
            E_r = \frac{1}{2}\mu u_r^2
        \end{equation*}
        we have that
        \begin{align*}
            u_r &= \left( \frac{2E_r}{\mu} \right)^{1/2}&
            \dd{u_r} &= \left( \frac{1}{2\mu E_r} \right)^{1/2}\dd{E_r}
        \end{align*}
        \item Making these substitutions yields
        \begin{equation*}
            u_rf(u_r)\dd{u_r} = \left( \frac{2}{\kB T} \right)^{3/2}\left( \frac{1}{\mu\pi} \right)^{1/2}E_r\e[-E_r/\kB T]\dd{E_r}
        \end{equation*}
        \item We may now easily define the reaction cross section $\sigma_r(E_r)$ by the following simple model, where $E_0$ is the threshold energy.
        \begin{equation*}
            \sigma_r(E_r) =
            \begin{cases}
                0 & E_r<E_0\\
                \sigma_{\ce{A}\ce{B}} & E_r\geq E_0
            \end{cases}
        \end{equation*}
        \item Resubstituting into the original expression for $k$ gives
        \begin{align*}
            k &= \left( \frac{2}{\kB T} \right)^{3/2}\left( \frac{1}{\mu\pi} \right)^{1/2}\int_0^\infty\dd{E_r}E_r\e[-E_r/\kB T]\sigma_r(E_r)\\
            &= \left( \frac{2}{\kB T} \right)^{3/2}\left( \frac{1}{\mu\pi} \right)^{1/2}\int_{E_0}^\infty\dd{E_r}E_r\e[-E_r/\kB T]\sigma_{\ce{A}\ce{B}}\\
            &= \left( \frac{8\kB T}{\mu\pi} \right)^{1/2}\sigma_{\ce{A}\ce{B}}\e[-E_0/\kB T]\left( 1+\frac{E_0}{\kB T} \right)\\
            &= \prb{u_r}\sigma_{\ce{A}\ce{B}}\e[-E_0/\kB T]\left( 1+\frac{E_0}{\kB T} \right)
        \end{align*}
    \end{itemize}
    \item Using this model, if we reverse engineer $E_0$ from experimental data on the other parameters, we can often get an answer on the same order of magnitude as $E_a$!
\end{itemize}




\end{document}