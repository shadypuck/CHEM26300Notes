\documentclass[../notes.tex]{subfiles}

\pagestyle{main}
\renewcommand{\chaptermark}[1]{\markboth{\chaptername\ \thechapter\ (#1)}{}}
\setcounter{chapter}{4}

\begin{document}




\chapter{Catalysis}
\section{Midterm Review and Intro to Catalysts}
\begin{itemize}
    \item \marginnote{4/27:}Example problem 1: Steady-state approximation.
    \begin{itemize}
        \item Let
        \begin{equation*}
            \ce{A <=>[$k_a$][$k_a'$] B <=>[$k_b$][$k_b'$] C <=>[$k_c$][$k_c'$] D}
        \end{equation*}
        Suppose $\cnc{A}$ is maintained at a fixed value and the produce \ce{D} is removed from the reaction as it is formed. Find the rate at which the product is formed in terms of $\cnc{A}$.
        \item By hypothesis, we have that at all times $t$, $\cnc{A}=\cnc[0]{A}$ and $\cnc{D}=0$.
        \item The hypotheses also imply that we can apply the steady-state approximation to both \ce{B} and \ce{C}.
        \item Thus, we have that
        \begin{align*}
            \dv{\cnc{C}}{t} = 0 &= k_b\cnc{B}-k_c\cnc{C}-k_b'\cnc{C}\\
            \cnc{B} &= \frac{k_b'+k_c}{k_b}\cnc{C}
        \end{align*}
        so that
        \begin{align*}
            \dv{\cnc{B}}{t} &= k_a\cnc{A}-k_b\cnc{B}-k_a'\cnc{B}+k_b'\cnc{C}\\
            0 &= k_a\cnc{A}-k_b\cdot\frac{k_b'+k_c}{k_b}\cnc{C}-k_a'\cdot\frac{k_b'+k_c}{k_b}\cnc{C}+k_b'\cnc{C}\\
            % &= k_a\cnc{A}-(k_b'+k_c+\frac{k_a'k_b'}{k_b}+\frac{k_a'k_c}{k_b}-k_b')\cnc{C}\\
            % &= k_a\cnc{A}-(\frac{k_bk_c+k_a'k_b'+k_a'k_c}{k_b})\cnc{C}\\
            \cnc{C} &= \frac{k_ak_b}{k_bk_c+k_a'k_b'+k_a'k_c}\cnc{A}
        \end{align*}
        and therefore
        \begin{align*}
            \dv{\cnc{D}}{t} &= k_c\cnc{C}-k_c'\cdot 0\\
            &= \frac{k_ak_bk_c}{k_bk_c+k_a'k_b'+k_a'k_c}\cnc{A}
        \end{align*}
    \end{itemize}
    \item Example problem 2.
    \begin{itemize}
        \item Consider the reaction
        \begin{equation*}
            \ce{HCl + CH3CH=CH2 <=> CH3CHClCH3}
        \end{equation*}
        which proceeds by the mechanism
        \begin{enumerate}
            \item \ce{HCl + HCl <=> (HCl)2} (equilibrium constant $K_1$).
            \item \ce{HCl + CH3CH=CH2 <=> complex} (equilibrium constant $K_2$).
            \item \ce{(HCl)2 + complex <=> CH3CHClCH3 + HCl + HCl} (equilibrium constant $K_3$).
        \end{enumerate}
        \item The equilibrium constants for the two pre-equilibria are
        \begin{align*}
            K_1 &= \frac{\cnc[eq]{(HCl)2}c^\circ}{\cnc[eq]{HCl}^2}&
            K_2 &= \frac{\cnc[eq]{complex}c^\circ}{\cnc[eq]{HCl}\cnc[eq]{CH3CH=CH2}}
        \end{align*}
        \item We can divide the mass-action expression for $K_1$ by $(c^\circ)^2$ to get each concentration over $c^\circ$ within its exponent.
        \item The rate of product formation is
        \begin{align*}
            v &= \dv{\cnc{CH3CHClCH3}}{t}\\
            &= k_r\cnc{(HCl)2}\cnc{complex}\\
            &\approx k_r\cnc[eq]{(HCl)2}\cnc[eq]{complex}\\
            &= k_r\cdot\frac{K_1\cnc[eq]{HCl}^2}{c^\circ}\cdot\frac{K_2\cnc[eq]{HCl}\cnc[eq]{CH3CH=CH2}}{c^\circ}\\
            &= \frac{k_rK_1K_2}{(c^\circ)^2}\cnc[eq]{HCl}^3\cnc[eq]{CH3CH=CH2}
        \end{align*}
        \begin{itemize}
            \item There's a key assumption with the steady state and something about being able to apply the equilibrium concentration of the intermediate as the steady-state quantity.
            \item This question wants to let you know that an equilibrium constant like $K_1$ might indicate a steady-state approximation.
        \end{itemize}
    \end{itemize}
    \item Note: Mind the positive and negative signs when constructing differential rate laws!
    \item The midterm will be posted this Friday (April 29) and will be available until the following Friday (May 6). There will be a timed 2 hour period to take it.
    \item \textbf{Catalyst}: A substance that participates in the chemical reaction but is not consumed in the process.
    \begin{itemize}
        \item A catalyst affects the mechanism and activation energy of a chemical reaction.
        \item A catalyst can give rise to a reaction path with a negligible activation barrier.
        \item The exothermicity or endothermicity of the chemical reaction is not altered by the presence of a catalyst.
    \end{itemize}
    \item \textbf{Homogeneous catalysis}: Catalysis in which the catalyst is in the same phase as the reactants and products.
    \item \textbf{Heterogeneous catalysis}: Catalysis in which the catalyst is in a different phase from the reactants and products.
    \item Imagine that initially, we have the reaction
    \begin{equation*}
        \ce{A ->[$k$] products}
    \end{equation*}
    where $k$ is the observed rate constant.
    \begin{itemize}
        \item When a catalyst is introduced into solution, this mechanism continues, but we now also have the new reaction pathway
        \begin{equation*}
            \ce{A + catalyst ->[$k_\text{cat}$] products + catalyst}
        \end{equation*}
        \item If each of these competing reactions is an elementary process, then
        \begin{equation*}
            -\dv{\cnc{A}}{t} = k\cnc{A}+k_\text{cat}\cnc{A}\cnc{catalyst}
        \end{equation*}
        \item In most cases, catalysts enhance reaction rates by many orders of magnitude, and therefore only the rate law for the catalyzed reaction need be considered in analyzing experimental data.
    \end{itemize}
    \item Reviews the Nobel Prizes in 2020 and 2021 (for CRISPR and asymmetric organocatalysis, respectively).
    \item An example of homogeneous catalysis.
    \begin{itemize}
        \item Consider the reaction
        \begin{equation*}
            \ce{2Ce^4+(aq) + Tl+(aq) -> 2Ce^3+(aq) + Tl^3+(aq)}
        \end{equation*}
        \item In the absence of a catalyst,
        \begin{equation*}
            v = k\cnc{Tl+}\cnc{Ce^4+}^2
        \end{equation*}
        and the mechanism is a termolecular elementary reaction.
        \item However, with \ce{Mn^2+} as the catalyst, we have the mechanism
        \begin{align*}
            \ce{Ce^4+(aq) + Mn^2+(aq)} &\xRightarrow{k_\text{cat}}                \ce{Mn^3+(aq) + Ce^3+(aq)}\\
            \ce{Ce^4+(aq) + Mn^3+(aq)} &\xRightarrow{{\color{white}k_\text{cat}}} \ce{Mn^4+(aq) + Ce^3+(aq)}\\
            \ce{Mn^4+(aq) + Tl+(aq)}   &\xRightarrow{{\color{white}k_\text{cat}}} \ce{Mn^2+(aq) + Tl^3+(aq)}
        \end{align*}
        where the step with $k_\text{cat}$ is the rate-determining step.
        \begin{itemize}
            \item Thus, for this mechanism, we have that
            \begin{equation*}
                v = k_\text{cat}\cnc{Ce^4+}\cnc{Mn^2+}
            \end{equation*}
        \end{itemize}
        \item The overall rate law for this reaction is therefore
        \begin{equation*}
            v = k\cnc{Tl+}\cnc{Ce^4+}^2+k_\text{cat}\cnc{Ce^4+}\cnc{Mn^2+}
        \end{equation*}
    \end{itemize}
\end{itemize}



\section{Enzymatic Catalysis}
\begin{itemize}
    \item \marginnote{4/27:}Midterm questions:
    \begin{itemize}
        \item First 10 are T/F. He will test key concepts by making statements that are either true or false.
        \begin{itemize}
            \item We should expect to spend no more than 30 minutes out of our 2 hours on these.
        \end{itemize}
        \item 4 calculation problems.
        \begin{itemize}
            \item First- and second-order reactions.
            \item Collisions.
            \item A reaction mechanism problem.
        \end{itemize}
        \item Use calculators, do online searches, and use the textbook.
        \item Do not talk to your classmates.
        \item The midterm will become available Friday at noon.
    \end{itemize}
    \item Enzymes are protein molecules that catalyze specific biochemical reactions.
    \begin{itemize}
        \item For example, hexokinase converts glucose and ATP to glucose 6-phosphate, ADP, and \ce{H+}.
    \end{itemize}
    \item \textbf{Substrate}: The reactant molecule acted upon by an enzyme.
    \item \textbf{Active site}: The region of the enzyme where the substrate reacts.
    \item \textbf{Lock-and-key model}: The active site and substrate have complementary three-dimensional structures and dock without the need for major atomic rearrangements.
    \item \textbf{Induced fit model}: Binding of the substrate induces a conformation change in the active site. The substrate fits well in the active site after the conformational change has taken place.
    \item The Michaelis-Menten Mechanism is a reaction mechanism for enzyme catalysis.
    \item Intuition.
    \begin{itemize}
        \item Imagine we have a solution of enzymes and substrate molecules.
        \item Limiting factors of an enzymatically catalyzed reaction.
        \begin{itemize}
            \item The enzyme-substrate affinity.
            \item The turnover number.
        \end{itemize}
        \item If the substrate concentration is low (i.e., $\cnc[0]{S}\ll\cnc[0]{E}$) and the enzyme-substrate affinity is strong (but not so strong that the enzyme-substrate complex is energetically favorable), then we expect $v_\text{initial}\propto\cnc[0]{S}$ because we'd think that all of the substrate will immediately be absorbed and transformed.
        \item If the substrate concentration is large (i.e., $\cnc[0]{S}\gg\cnc[0]{E}$) and the enzyme-substrate affinity is strong, then we expect $v_\text{initial}\propto\cnc[0]{E}$ and, importantly, $v_\text{initial}\not\propto\cnc[0]{S}$.
    \end{itemize}
    \item Mathematical derivation.
    \begin{itemize}
        \item Experimental studies reveal that the rate law for many enzyme-catalyzed reactions has the form
        \begin{equation*}
            -\dv{\cnc{S}}{t} = \frac{k\cnc{S}}{K_m+\cnc{S}}
        \end{equation*}
        \begin{itemize}
            \item This is the final goal of the derivation.
        \end{itemize}
        \item The mechanism is
        \begin{equation*}
            \ce{S + E} \Longleftrightarrows[k_1][k_{-1}] \ce{ES}
            \Longleftrightarrows[k_2][k_{-2}] \ce{P + E}
        \end{equation*}
        \item Thus,
        \begin{gather*}
            -\dv{\cnc{S}}{t} = k_1\cnc{E}\cnc{S}-k_{-1}\cnc{ES}\\
            -\dv{\cnc{ES}}{t} = (k_2+k_{-1})\cnc{ES}-k_1\cnc{E}\cnc{S}-k_{-2}\cnc{E}\cnc{P}\\
            \dv{\cnc{P}}{t} = k_2\cnc{ES}-k_{-1}\cnc{E}\cnc{P}
        \end{gather*}
        \item Note that
        \begin{equation*}
            \cnc[0]{E} = \cnc{ES}+\cnc{E}
        \end{equation*}
        \item Plugging that equation into the rate law for the enzyme-substrate complex and applying the steady-state approximation yields
        \begin{align*}
            -\dv{\cnc{ES}}{t} = 0 &= \cnc{ES}(k_1\cnc{S}+k_{-1}+k_2+k_{-1}\cnc{P})-k_1\cnc{S}\cnc[0]{E}-k_2\cnc{P}\cnc[0]{P}\\
            \cnc{ES} &= \frac{k_1\cnc{S}+k_{-1}\cnc{P}}{k_1\cnc{S}+k_{-2}\cnc{P}+k_{-1}+k_2}\cnc[0]{E}
        \end{align*}
        \item Substituting this and the original expression for $\cnc[0]{E}$ into the rate law for the substrate yields
        \begin{equation*}
            v = -\dv{\cnc{S}}{t}
            = \frac{k_1k_2\cnc{S}+k_{-1}k_{-2}\cnc{P}}{k_1\cnc{S}+k_{-2}\cnc{P}+k_{-1}+k_2}\cnc[0]{E}
        \end{equation*}
        \item If the experimental measurements of the reaction rate are taken during the time period when only a small percentage (1-3\%) of the substrate is converted to product, then
        \begin{equation*}
            \cnc{S} \approx \cnc[0]{S}
        \end{equation*}
        and
        \begin{equation*}
            \cnc{P} \approx 0
        \end{equation*}
        \item Using this approximation simplifies the above rate law to
        \begin{equation*}
            v = -\dv{\cnc{S}}{t}
            = \frac{k_1k_2\cnc[0]{S}\cnc[0]{E}}{k_1\cnc[0]{S}+k_{-1}+k_2}
            = \frac{k_2\cnc[0]{S}\cnc[0]{E}}{K_m+\cnc[0]{S}}
        \end{equation*}
        where $K_m=(k_{-1}+k_2)/k_1$ is the \textbf{Michaelis constant}.
        \begin{itemize}
            \item The Michaelis constant tells you the ratio of dissociation of the enzyme-substrate complex to the formation of the enzyme-substrate complex. In other words, it provides information on the enzyme-substrate affinity.
            \item Note that $k_{-2}$ is not present in the denominator of the Michaelis constant because for a good enzyme, $k_{-2}$ should be very small.
            \item The unit of $K_m$ should be concentration.
            \item When $K_m=\cnc[0]{S}$, $v=v_\text{max}/2$
        \end{itemize}
        \item An enzyme-catalyzed reaction is first order in the substrate at low substrate concentrations ($K\gg\cnc[0]{S}$) and then becomes zero order in the substrate at high substrate concentrations ($K\ll\cnc[0]{S}$).
        \item Thus, at low substrate concentrations, the above equation holds, but at high substrate concentrations,
        \begin{align*}
            -\dv{\cnc{S}}{t} &= k_2\cnc[0]{E}&
            v_\text{max} &= k_2\cnc[0]{E}
        \end{align*}
        resulting in the \textbf{Lineweaver-Burk plot}, canonically represented by the second of the two equivalent forms below.
        \begin{align*}
            v &= \frac{v_\text{max}}{1+K_m/\cnc[0]{S}}&
            \frac{1}{v} &= \frac{1}{v_\text{max}}+\frac{K_m}{v_\text{max}}\frac{1}{\cnc[0]{S}}
        \end{align*}
    \end{itemize}
\end{itemize}



\section{Measuring Catalytic Efficiency and Correcting Collision Theory}
\begin{itemize}
    \item \marginnote{4/29:}Everything on the midterm comes from Tian's lecture notes. Thus, he recommends we go through them before taking the midterm.
    \begin{itemize}
        \item The T/F will be 20-30\% of the grade.
        \item There will be some integration on the calculation problems, but they'll be pretty easy. All formulas that will appear have been covered in class.
        \item The midterm will cover up to Monday's class (this week).
    \end{itemize}
    \item Consider the substrate concentration $\cnc[0]{S}$ vs. the initial rate $v_0$.
    \begin{figure}[h!]
        \centering
        \begin{subfigure}[b]{0.45\linewidth}
            \centering
            \begin{tikzpicture}[scale=2]
                \small
                \draw [stealth-stealth] (0,1.1) node[above]{$v_0$} -- (0,0) -- (2.1,0) node[right]{$\cnc[0]{S}$};
                \footnotesize
                \draw (0.05,1) -- ++(-0.1,0) node[left]{$v_\text{max}$};
                \draw (0.05,0.5) -- ++(-0.1,0) node[left]{$\frac{v_\text{max}}{2}$};
                \draw (0.2,0.05) -- ++(0,-0.1) node[below]{$K_m$};
        
                \draw [dashed,very thin]
                    (0,1) -- ++(2,0)
                    (0.2,0) -- ++(0,0.5) -- ++(-0.2,0)
                ;
        
                \draw [rex,thick] plot[domain=0:2,smooth,samples=100] (\x,{\x/(0.2+\x)});
            \end{tikzpicture}
            \caption{Normal plot.}
            \label{fig:v0S0a}
        \end{subfigure}
        \begin{subfigure}[b]{0.45\linewidth}
            \centering
            \begin{tikzpicture}[scale=2]
                \small
                \draw [stealth-stealth] (0,1.1) node[above]{$\frac{1}{v_0}$} -- (0,0) -- (2.1,0) node[right]{$\frac{1}{\cnc[0]{S}}$};
                \footnotesize
                \path (0.2,0.05) -- ++(0,-0.1) node[white,below]{$K_m$};
        
                \draw [rex,thick] (0,0.25) node[circle,fill,inner sep=1.5pt]{} node[black,left]{$\left( 0,\frac{1}{v_\text{max}} \right)$} -- node[black,above=1mm]{$\text{slope}=\frac{K_m}{v_\text{max}}$} (2,0.35);
            \end{tikzpicture}
            \caption{Reciprocal plot.}
            \label{fig:v0S0b}
        \end{subfigure}
        \caption{Plotting $v_0$ vs. $\cnc[0]{S}$.}
        \label{fig:v0S0}
    \end{figure}
    \begin{itemize}
        \item Normal plot (Figure \ref{fig:v0S0a}).
        \begin{itemize}
            \item As $\cnc{S}\to\infty$, $v_0$ approaches an asymptote line defined by $v_\text{max}=k_2\cnc[0]{E}$.
            \item At the beginning (low $\cnc{S}$), the paradigm is almost linear (thus, the reaction is first order wrt. substrate concentration here).
            \item Note that we make use of the assumptions that $\cnc{P}\approx 0$ and $\cnc{S}\approx\cnc[0]{S}$.
            \item Where we have $v_\text{max}/2$ on the $v_0$-axis, we have $K_m$ (the Michaelis constant) on the $\cnc[0]{S}$ axis.
        \end{itemize}
        \item Reciprocal plot (Figure \ref{fig:v0S0b}).
        \begin{itemize}
            \item Note that a plot of $v_0$ vs. $\cnc[0]{S}$ is nonlinear but a plot of $1/v_0$ vs. $1/\cnc[0]{S}$ is linear.
            \item This reciprocal plot (the Lineweaver-Burk plot) gives $v_\text{max}$ (via the $y$-intercept) and $K_m$ via this information and the slope.
            \item Note that
            \begin{align*}
                \frac{1}{v_0} &\propto \frac{K_m}{v_\text{max}}\frac{1}{\cnc[0]{S}}\\
                &= \frac{(k_{-1}+k_2)/k_1}{k_2\cnc[0]{E}}\frac{1}{\cnc[0]{S}}
                &= \frac{k_{-1}+k_2}{k_1k_2}\frac{1}{\cnc[0]{E}\cnc[0]{S}}
            \end{align*}
        \end{itemize}
    \end{itemize}
    \item Evaluating the performance of a catalyst.
    \begin{equation*}
        \frac{k_{-1}+K_2}{k_1k_2} = \frac{K_m}{v_\text{max}}\cdot\cnc[0]{E}
    \end{equation*}
    \begin{itemize}
        \item $K_m$ is relevant to the enzyme-substrate affinity.
        \item $v_\text{max}$ tells us about conversion from the ES with a focus on the second elementary step.
    \end{itemize}
    \item An alternate form of the Lineweaver-Burk plot is
    \begin{equation*}
        \frac{1}{v_0} = \frac{1}{v_\text{max}}+\frac{k_2+k_{-1}}{k_1k_2}\frac{1}{\cnc[0]{E}\cnc[0]{S}}
    \end{equation*}
    \begin{itemize}
        \item Regrouping the terms, we have
        \begin{equation*}
            \frac{1}{v_0} = \frac{1}{v_\text{max}}+\frac{k_2+k_{-1}}{k_2}\frac{1}{k_1\cnc[0]{S}\cnc[0]{E}}
            = \frac{1}{v_\text{max}}+\frac{k_2+k_{-1}}{k_2}\frac{1}{v_{f1}}
        \end{equation*}
        where $v_{f1}$ is the forward reaction rate for the first elementary step.
    \end{itemize}
    \item \textbf{Turnover number}: The number of catalytic cycles that each active site undergoes per unit time. \emph{Given by}
    \begin{equation*}
        \text{TON} = \frac{v_\text{max}}{n\cnc[0]{E}}
        = \frac{k_2\cnc[0]{E}}{n\cnc[0]{E}}
        = \frac{k_2}{n}
    \end{equation*}
    \begin{itemize}
        \item Indicates how fast the \ce{ES} complex proceeds to \ce{E + P}.
        \item $k_2/n$ is the number of active sites per enzyme.
    \end{itemize}
    \item \textbf{Catalytic efficiency}: The following quantity. \emph{Given by}
    \begin{equation*}
        \frac{\text{TON}}{K_m}
    \end{equation*}
    \item Consider the reaction
    \begin{equation*}
        \ce{A(g) + B(g)} \xRightarrow{k} \ce{products}
    \end{equation*}
    \begin{itemize}
        \item The rate of the general bimolecular elementary gas-phase reaction is
        \begin{equation*}
            v = -\dv{\cnc{A}}{t} = k\cnc{A}\cnc{B}
        \end{equation*}
        \item Using the na\"{i}ve assumption that every collision between the hard spheres \ce{A} and \ce{B} yields products,
        \begin{equation*}
            v = Z_{\ce{A}\ce{B}} = \sigma_{\ce{A}\ce{B}}\prb{u_r}\rho_{\ce{A}}\rho_{\ce{B}}
        \end{equation*}
        \item Moreover,
        \begin{equation*}
            k = \sigma_{\ce{A}\ce{B}}\prb{u_r}
        \end{equation*}
        \item Unfortunately, this is not accurate. We make our first improvement to collision theory by taking into account the dependence of the reaction rate on the relative speed, or energy, of the collision. Thus, we average over all possible collision speeds.
        \begin{equation*}
            k = \int_0^\infty\dd{u_r}f(u_r)k(u_r)
            = \int_0^\infty\dd{u_r}u_rf(u_r)\sigma_r(u_r)
        \end{equation*}
        \item Since $f(u_r)$ is the distribution of relative speeds in the gas sample, we have that
        \begin{equation*}
            u_rf(u_r)\dd{u_r} = \left( \frac{\mu}{\kB T} \right)^{3/2}\left( \frac{2}{\pi} \right)^{1/2}u_r^3\e[-\mu u_r^2/2\kB T]\dd{u_r}
        \end{equation*}
        \item To compare this with the traditional Arrhenius form of $k$, we need to change the dependent variable from $u_r$ to $E$, which we can do via
        \begin{align*}
            E_r &= \frac{1}{2}\mu u_r^2&
            u_r &= \sqrt{\frac{2E_r}{\mu}}&
            \dd{u_r} &= \sqrt{\frac{1}{2\mu E_r}}\dd{E_r}&
            E_a &= \kB T^2\dv{\ln k}{T}
        \end{align*}
    \end{itemize}
\end{itemize}




\end{document}