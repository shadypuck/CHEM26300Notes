\documentclass[../notes.tex]{subfiles}

\pagestyle{main}
\renewcommand{\chaptermark}[1]{\markboth{\chaptername\ \thechapter\ (#1)}{}}
\setcounter{chapter}{6}

\begin{document}




\chapter{Gas-Phase Product Molecule Analysis and Intro to Lattices}
\section{Directional Scattering of the Product Molecule}
\begin{itemize}
    \item \marginnote{5/9:}The velocity and angular distribution of the products of a reactive collision.
    \begin{itemize}
        \item We have that
        \begin{align*}
            E_\text{trans}'+E_\text{vib}' &= E_\text{trans}+E_\text{vib}-[D_e(\ce{D2})-D_e(\ce{DF})]\\
            &= \SI{7.62}{\kilo\joule\per\mole}+\SI{17.9}{\kilo\joule\per\mole}+\SI{140}{\kilo\joule\per\mole}\\
            &= \SI{166}{\kilo\joule\per\mole}
        \end{align*}
        \item Additionally, we know that
        \begin{equation*}
            E_\text{trans}'+E_\text{vib}' = \frac{1}{2}\mu'u_r'^2+(\SI{34.8}{\kilo\joule\per\mole})\left( v+\frac{1}{2} \right)
            = \SI{166}{\kilo\joule\per\mole}
        \end{equation*}
        \item The relationship between the vibrational quantum number, the relative speed of the products, and the speed of \ce{DF} relative to the center of mass has been tabulated.
    \end{itemize}
    \item A contour map of the angular and speed distributions for the product molecule.
    \begin{itemize}
        \item The contour plot.
        \begin{itemize}
            \item The center of mass is fixed at the origin.
            \item The dashed circles correspond to the maximum relative speeds a \ce{DF} molecule can have for the indicated vibrational state.
            \item The product molecules preferentially scatter back in the direction of the incident fluorine atom, a scattering angle of $\theta=\ang{180}$.
            \item The arrows at the bottom of the figure show the direction with which each reactant molecule approaches the other.
        \end{itemize}
        \item Another picture is provided, illustrating the atom-molecule reaction \ce{F + D2} in which $\theta=\ang{0}$ and $\theta=\ang{180}$.
        \item The influence of rotation.
        \begin{itemize}
            \item Large numbers of product molecules have speeds between the dashed circles.
            \item The dash circles correspond to the case where there is internal energy only in the vibrational states of the molecule, in which case the rotational energy corresponding to these circles is $E_\text{rot}=0$ with $J=0$.
            \item If \ce{DF} is produced in an excited rotational state, we would expect to observe a speed that has a value intermediate between two fo the dashed circles.
        \end{itemize}
    \end{itemize}
    \item Not all gas-phase chemical reactions are rebound reactions.
    \begin{itemize}
        \item Consider the reaction
        \begin{equation*}
            \ce{K(g) + I2(g) -> KI(g) + I(g)}
        \end{equation*}
        \begin{itemize}
            \item The product diatomic molecule in this case (\ce{KI}) is preferentially scattered in the forward direction along the direction of the incident \ce{K} atom.
        \end{itemize}
        \item Consider the reaction
        \begin{equation*}
            \ce{O(g) + Br2(g)} \Longrightarrow \ce{BrO(g) + Br(g)}
        \end{equation*}
        \begin{itemize}
            \item The product molecule \ce{BrO} is forward and back scattered with equal intensity.
        \end{itemize}
        \item Both of these observations can be read off of the contour maps of the two reactions.
    \end{itemize}
\end{itemize}



\section{Potential Energy Surfaces}
\begin{itemize}
    \item \marginnote{5/11:}The velocity and angular distribution of the products of a reactive collision.
    \begin{figure}[h!]
        \centering
        \begin{tikzpicture}
            \footnotesize
            \draw [dashed] (-5.5,0) -- (-0.18,0) (5.5,0) -- (0.18,0);
    
            \draw
                (-4,0) ellipse (7mm and 1.5cm) node[above=2cm,align=center]{Approaching\\reactant}
                (4,0) ellipse (1.05cm and 2.25cm) node[above=2.75cm,align=center]{Leaving\\product}
            ;
            \draw
                (-4,0) ++(145:1.5) ++(0.65,0) coordinate(A) -- (-4,0)
                (-4,0) ++(20:1.5) ++(-0.75,0) coordinate (A') -- (-4,0)
                ($(-4,0)!0.3!(A)$) arc[start angle=150,end angle=18,x radius=2.1mm,y radius=4.5mm]
                ($(-4,0)!0.25!(A')$) -- ++(0.25,0) -- (-3.75,0)
            ;
            \node at (-4,0.65) {$\phi$};
            \draw
                (4,0) ++(145:2.25) ++(0.98,0) coordinate(C) -- (4,0)
                (4,0) ++(20:2.25) ++(-1.13,0) coordinate (C') -- (4,0)
                ($(4,0)!0.2!(C)$) arc[start angle=150,end angle=18,x radius=2.1mm,y radius=4.5mm]
                ($(4,0)!0.17!(C')$) -- ++(0.25,0) -- (4.25,0)
            ;
            \node at (4,0.65) {$\phi$};
    
            \draw [semithick,-latex] (A) -- node[above]{$\mathbf{u}_r$} (A -| 0,0) -- node[above]{$\mathbf{u}_r'$} ($(A -| 0,0)!0.84!(C)$);
            \draw (A -| -0.45,0) arc[start angle=180,end angle=8,radius=4.5mm];
            \node [fill=white,inner sep=1.5pt] at ([yshift=4.5mm]A -| 0,0) {$\theta$};
    
            \draw [<->,shorten <=1pt,shorten >=1pt] (-2,0 |- A) -- node[right]{$b$} (-2,0);
    
            \begin{scope}[on background layer]
                \fill [ball color=rex] (0,0) circle (1cm) node{\ce{B}};
                \fill [ball color=blx] (A) circle (5mm) node[left]{\ce{A}};
                \fill [ball color=blx] (C) circle (5mm);
            \end{scope}
        \end{tikzpicture}
        \caption{Velocity and angular distributions of the products.}
        \label{fig:velocityAngleDist}
    \end{figure}
    \begin{itemize}
        \item For a fixed value of the impact parameter $b$, the reactants and products take on all possible angles $\phi$ with equal probability, thereby forming a cone around the relative velocity vector $\mathbf{u}_r$.
        \item The angle $\theta$, however, depends on the dynamics of the reaction and must be determined experimentally.
    \end{itemize}
    \item The potential energy of a polyatomic molecule depends on more than one variable.
    \begin{itemize}
        \item \ce{D2}.
        \begin{itemize}
            \item Consider the potential energy curve of \ce{D2}. The zero of energy is defined to be that of the two separated atoms. The minimum of the potential energy curve corresponds to the equilibrium bond length of the \ce{D2} molecule.
        \end{itemize}
        \item \ce{H2O}.
        \begin{itemize}
            \item The potential energy of a water molecule is a function of the three parameters $r_{\ce{O-H_A}}$, $r_{\ce{O-H_B}}$, and $\alpha$ (two bond lengths and the interbond angle). In an equation,
            \begin{equation*}
                V = V(r_{\ce{O-H_A}},r_{\ce{O-H_B}},\alpha)
            \end{equation*}
            \item A plot of the complete potential energy surface of a water molecule therefore requires four axes.
        \end{itemize}
    \end{itemize}
    \item The potential energy of a chemical reaction depends on more than one variable.
    \begin{figure}[H]
        \centering
        \begin{tikzpicture}
            \footnotesize
            \begin{scope}[xshift=-5cm]
                \node (DA) at (0,0)   {\ce{D_A}};
                \node (DB) at (1.3,0) {\ce{D_B}};
                \node (F)  at (180:2) {\ce{F}};
    
                \draw [semithick,dashed] (F)  -- node[above]{$\beta=\ang{180}$} (DA);
                \draw [semithick] (DA) -- (DB);
            \end{scope}
            \begin{scope}[xshift=0cm]
                \node (DA) at (0,0)   {\ce{D_A}};
                \node (DB) at (1.3,0) {\ce{D_B}};
                \node (F)  at (135:2) {\ce{F}};
    
                \draw [semithick,dashed] (F)  -- node[above right]{$\beta=\ang{180}$} (DA);
                \draw [semithick] (DA) -- (DB);
            \end{scope}
            \begin{scope}[xshift=5cm]
                \node (DA) at (0,0)   {\ce{D_A}};
                \node (DB) at (1.3,0) {\ce{D_B}};
                \node (F)  at (90:2) {\ce{F}};
    
                \draw [semithick,dashed] (F)  -- node[right]{$\beta=\ang{180}$} (DA);
                \draw [semithick] (DA) -- (DB);
            \end{scope}
        \end{tikzpicture}
        \caption{Angle of attack in \ce{F + D2}.}
        \label{fig:FD2attackAngle}
    \end{figure}
    \begin{itemize}
        \item Consider, once again, the reaction
        \begin{equation*}
            \ce{F(g) + D_AD_B(g)} \Longrightarrow \ce{D_AF(g) + D_B(g)}
        \end{equation*}
        \item When the reactants are at infinite separation, there are no attractive or repulsive forces between the fluorine atom and the \ce{D2} molecule, so the potential energy surface for the reaction is the same as that for an isolated \ce{D2} molecule.
        \item Likewise, when the products are at infinite separation, the potential-energy surface for the reaction is the same as the for the isolated \ce{DF} molecule.
        \item As the reaction occurs, however, the distance between the fluroine atom and \ce{D_A} decreases and the distance between \ce{D_A} and \ce{D_B} increases, and the potential energy depends on both distances.
        \item The potential energy also depends on the angle at which the fluorine atom approaches the \ce{D2} molecule.
    \end{itemize}
    \item As with the angular and speed distributions, we can draw an energy contour map for the reaction.
    \begin{itemize}
        \item The zero of energy is defined as the infinitely separated reactants. Point B is the location of the transition state of the reaction.
    \end{itemize}
    \item Tian does a brief summary of Chapter 30.
\end{itemize}




\end{document}