\documentclass[../notes.tex]{subfiles}

\pagestyle{main}
\renewcommand{\chaptermark}[1]{\markboth{\chaptername\ \thechapter\ (#1)}{}}
\setcounter{chapter}{28}

\begin{document}




\chapter{Reaction Mechanisms}
\section{TST and Overview of Mechanisms}
\begin{itemize}
    \item \marginnote{4/18:}Overview of key concepts.
    \begin{itemize}
        \item Reaction mechanisms can involve more than one elementary step.
        \item Reactions can be sequential (single- or multi-step).
        \item To establish a mechanism, we use several techniques, approaches, assumptions, and approximations.
        \item Establish rate determining steps: The rate law and rate constants associated with these steps tend to dominate the kinetics of the overall reaction.
        \item Invoke the steady-state approximation to help solve the complicated mathematics of reaction kinetics.
        \item Enzyme kinetics, Michaelis-Menten mechanism involves an SS approximation.
    \end{itemize}
    \item Oftentimes, reactions are of the form
    \begin{align*}
        \ce{E + S <=>[$k_1$][$k_{-1}$] ->[$k_r$] P + E}
    \end{align*}
    \begin{itemize}
        \item Note that this form is very much analogous to the form analyzed in TST.
    \end{itemize}
    \item \textbf{Elementary reaction}: A reaction that does not involve the formation of a reaction intermediate; the products must be formed directly from the reactants.
    \begin{itemize}
        \item Denoted by the double arrow.
        \item An elementary reaction can still be reversible.
    \end{itemize}
    \item \textbf{Molecularity} (of an elementary reaction): The number of reactant molecules involved in the chemical reaction.
    \item \textbf{Unimolecular} (reaction): An elementary reaction with molecularity one. \emph{General form}
    \begin{equation*}
        \ce{A} \Longrightarrow \text{products}
    \end{equation*}
    \emph{Rate law}
    \begin{equation*}
        v = k[\ce{A}]
    \end{equation*}
    \item \textbf{Bimolecular} (reaction): An elementary reaction with molecularity two. \emph{General form}
    \begin{equation*}
        \ce{A}+\ce{B} \Longrightarrow \text{products}
    \end{equation*}
    \emph{Rate law}
    \begin{equation*}
        v = k[\ce{A}][\ce{B}]
    \end{equation*}
    \item \textbf{Termolecular} (reaction): An elementary reaction with molecularity three. \emph{General form}
    \begin{equation*}
        \ce{A}+\ce{B}+\ce{C} \Longrightarrow \text{products}
    \end{equation*}
    \emph{Rate law}
    \begin{equation*}
        v = k[\ce{A}][\ce{B}][\ce{C}]
    \end{equation*}
    \item No elementary reaction with molecularity greater than three is known, and the overwhelming majority of elementary reactions are bimolecular.
    \item When a complex reaction is at equilibrium, the rate of the forward process is equal to the rate of the reverse process for each and every step of the reaction mechanism.
    \begin{itemize}
        \item We denote a reversible elementary reaction as follows.
        \begin{equation*}
            \ce{A + B} \Longleftrightarrows[k_1][k_{-1}] \ce{C + D}
        \end{equation*}
        \item A reversible elementary reaction signifies that the reaction occurs in both the forward and reverse directions to a significant extent and that the reaction in each direction is an elementary reaction.
        \item The rate laws are
        \begin{align*}
            v_1 &= k_1[\ce{A}][\ce{B}]&
            v_{-1} &= k_{-1}[\ce{C}][\ce{D}]
        \end{align*}
        \item At equilibrium,
        \begin{align*}
            k_1[\ce{A}]_\text{eq}[\ce{B}]_\text{eq} &= k_{-1}[\ce{C}]_\text{eq}[\ce{D}]_\text{eq}\\
            \frac{k_1}{k_{-1}} &= \frac{[\ce{C}]_\text{eq}[\ce{D}]_\text{eq}}{[\ce{A}]_\text{eq}[\ce{B}]_\text{eq}} = K_c
        \end{align*}
    \end{itemize}
    \item \textbf{Principle of detailed balance}: The following relationship, which holds for all reversible elementary reactions. \emph{Given by}
    \begin{equation*}
        K_c = \frac{k_1}{k_{-1}}
    \end{equation*}
\end{itemize}



\section{The Two-Step Consecutive Reaction Mechanism}
\begin{itemize}
    \item \marginnote{4/20:}Consider the general complex reaction
    \begin{equation*}
        \ce{A ->[$k_\text{obs}$] P}
    \end{equation*}
    \begin{itemize}
        \item Suppose that the reaction occurs by the two step mechanism
        \begin{align*}
            \ce{A} &\stackrel{k_1}{\Longrightarrow} \ce{I}&
            \ce{I} &\stackrel{k_2}{\Longrightarrow} \ce{P}
        \end{align*}
        \item Because each step of this mechanism is an elementary reaction, the rate laws for each species are
        \begin{align*}
            \dv{[\ce{A}]}{t} &= -k_1[\ce{A}]&
            \dv{[\ce{I}]}{t} &= k_1[\ce{A}]-k_2[\ce{I}]&
            \dv{[\ce{P}]}{t} &= k_2[\ce{I}]
        \end{align*}
        \item Thus, assuming that the initial concentrations at time $t=0$ are $[\ce{A}]=[\ce{A}]_0$ and $[\ce{I}]_0=[\ce{P}]_0=0$, we have that
        \begin{gather*}
            [\ce{A}] = [\ce{A}]_0\e[-k_1t]\\
            [\ce{I}] = \frac{k_1[\ce{A}]_0}{k_2-k_1}(\e[-k_1t]-\e[-k_2t])\\
            [\ce{P}] = [\ce{A}]_0-[\ce{A}]-[\ce{I}]
                = [\ce{A}]_0\left\{ 1+\frac{1}{k_1-k_2}(k_2\e[-k_1t]-k_1\e[-k_2t]) \right\}
        \end{gather*}
    \end{itemize}
    \item Distinguishing the two-step consecutive reaction mechanism unambiguously from the one-step reaction.
    \begin{itemize}
        \item For a single step reaction,
        \begin{equation*}
            [\ce{P}] = [\ce{A}]_0(1-\e[-k_1t])
        \end{equation*}
        \item The two-step consecutive reaction mechanism has the following alternate form.
        \begin{equation*}
            [\ce{P}] = [\ce{A}]_0\left\{ 1+\frac{1}{k_1-k_2}(k_2\e[-k_1t]-k_1\e[-k_2t]) \right\}
        \end{equation*}
        \item However, if $k_2\gg k_1$, then
        \begin{align*}
            [\ce{P}] &= [\ce{A}]_0\left\{ 1+\frac{1}{k_1-k_2}(k_2\e[-k_1t]-k_1\e[-k_2t]) \right\}\\
            &\approx [\ce{A}]_0\left\{ 1+\frac{1}{-k_2}k_2\e[-k_1t] \right\}\\
            &= [\ce{A}]_0(1-\e[-k_1t])
        \end{align*}
        \item If $k_1\gg k_2$, he reaction reduces to
        \begin{equation*}
            [\ce{P}] \approx [\ce{A}]_0(1-\e[-k_2t])
        \end{equation*}
        \item Thus, the only ambiguous situation is $k_2\gg k_1$.
    \end{itemize}
    \item The steady-state approximation simplifies rate expressions.
    \begin{itemize}
        \item We assume that $\dv*{[\ce{I}]}{t}=0$, where \ce{I} is a reaction intermediate.
        \item Given the above differential equation for $\dv*{[\ce{I}]}{t}$, making the above assumption yields
        \begin{equation*}
            [\ce{I}]_\text{SS} = \frac{k_1[\ce{A}]}{k_2}
        \end{equation*}
        \item It follows that
        \begin{equation*}
            [\ce{I}]_\text{SS} = \frac{k_1}{k_2}[\ce{A}]_0\e[-k_1t]
        \end{equation*}
        \item Thus,
        \begin{equation*}
            \dv{[\ce{I}]_\text{SS}}{t} = \frac{-k_1^2}{k_2}[\ce{A}]_0\e[-k_1t]
        \end{equation*}
        \item We get $k_2\gg k_1^2[\ce{A}]_0$ and $[\ce{P}]=[\ce{A}]_0(1-\e[-k_1t])$.
    \end{itemize}
    \item Example: Decomposition of ozone.
    \begin{equation*}
        \ce{2O3(g) -> 3O2(g)}
    \end{equation*}
    \begin{itemize}
        \item The reaction mechanism is
        \begin{align*}
            \ce{M(g) + O3(g)} &\Longleftrightarrows[k_1][k_{-1}] \ce{O2(g) + O(g) + M(g)}\\
            \ce{O(g) + O3(g)} &\xRightarrow{k_2} \ce{2O2(g)}
        \end{align*}
        where \ce{M} is a molecule that can exchange energy with the reacting ozone molecule through a collision, but \ce{M} itself does not react.
        \item The rate equations for \ce{O3(g)} and \ce{O(g)} are
        \begin{gather*}
            \dv{[\ce{O3}]}{t} = -k_1[\ce{O3}][\ce{M}]+k_{-1}[\ce{O2}][\ce{O}][\ce{M}]-k_2[\ce{O}][\ce{O3}]\\
            \dv{[\ce{O}]}{t} = k_1[\ce{O3}][\ce{M}]-k_{-1}[\ce{O2}][\ce{O}][\ce{M}]-k_2[\ce{O}][\ce{O3}]
        \end{gather*}
        \item Invoking the steady-state approximation for the intermediate \ce{O} yields
        \begin{equation*}
            [\ce{O}] = \frac{k_1[\ce{O3}][\ce{M}]}{k_{-1}[\ce{O2}][\ce{M}]+k_2[\ce{O3}]}
        \end{equation*}
        \item Substituting this result into the rate equation for \ce{O3} gives
        \begin{equation*}
            \dv{[\ce{O3}]}{t} = -\frac{2k_1k_2[\ce{O3}]^2[\ce{M}]}{k_{-1}[\ce{O2}][\ce{M}]+k_2[\ce{O3}]}
        \end{equation*}
    \end{itemize}
\end{itemize}




\end{document}