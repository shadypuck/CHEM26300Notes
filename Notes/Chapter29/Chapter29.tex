\documentclass[../notes.tex]{subfiles}

\pagestyle{main}
\renewcommand{\chaptermark}[1]{\markboth{\chaptername\ \thechapter\ (#1)}{}}
\setcounter{chapter}{28}

\begin{document}




\chapter{Reaction Mechanisms}
\section{TST and Overview of Mechanisms}
\begin{itemize}
    \item \marginnote{4/18:}Overview of key concepts.
    \begin{itemize}
        \item Reaction mechanisms can involve more than one elementary step.
        \item Reactions can be sequential (single- or multi-step).
        \item To establish a mechanism, we use several techniques, approaches, assumptions, and approximations.
        \item Establish rate determining steps: The rate law and rate constants associated with these steps tend to dominate the kinetics of the overall reaction.
        \item Invoke the steady-state approximation to help solve the complicated mathematics of reaction kinetics.
        \item Enzyme kinetics, Michaelis-Menten mechanism involves an SS approximation.
    \end{itemize}
    \item Oftentimes, reactions are of the form
    \begin{align*}
        \ce{E + S <=>[$k_1$][$k_{-1}$] ->[$k_r$] P + E}
    \end{align*}
    \begin{itemize}
        \item Note that this form is very much analogous to the form analyzed in TST.
    \end{itemize}
    \item \textbf{Elementary reaction}: A reaction that does not involve the formation of a reaction intermediate; the products must be formed directly from the reactants.
    \begin{itemize}
        \item Denoted by the double arrow.
        \item An elementary reaction can still be reversible.
    \end{itemize}
    \item \textbf{Molecularity} (of an elementary reaction): The number of reactant molecules involved in the chemical reaction.
    \item \textbf{Unimolecular} (reaction): An elementary reaction with molecularity one. \emph{General form}
    \begin{equation*}
        \ce{A} \Longrightarrow \text{products}
    \end{equation*}
    \emph{Rate law}
    \begin{equation*}
        v = k[\ce{A}]
    \end{equation*}
    \item \textbf{Bimolecular} (reaction): An elementary reaction with molecularity two. \emph{General form}
    \begin{equation*}
        \ce{A}+\ce{B} \Longrightarrow \text{products}
    \end{equation*}
    \emph{Rate law}
    \begin{equation*}
        v = k[\ce{A}][\ce{B}]
    \end{equation*}
    \item \textbf{Termolecular} (reaction): An elementary reaction with molecularity three. \emph{General form}
    \begin{equation*}
        \ce{A}+\ce{B}+\ce{C} \Longrightarrow \text{products}
    \end{equation*}
    \emph{Rate law}
    \begin{equation*}
        v = k[\ce{A}][\ce{B}][\ce{C}]
    \end{equation*}
    \item No elementary reaction with molecularity greater than three is known, and the overwhelming majority of elementary reactions are bimolecular.
    \item When a complex reaction is at equilibrium, the rate of the forward process is equal to the rate of the reverse process for each and every step of the reaction mechanism.
    \begin{itemize}
        \item We denote a reversible elementary reaction as follows.
        \begin{equation*}
            \ce{A + B} \Longleftrightarrows[k_1][k_{-1}] \ce{C + D}
        \end{equation*}
        \item A reversible elementary reaction signifies that the reaction occurs in both the forward and reverse directions to a significant extent and that the reaction in each direction is an elementary reaction.
        \item The rate laws are
        \begin{align*}
            v_1 &= k_1[\ce{A}][\ce{B}]&
            v_{-1} &= k_{-1}[\ce{C}][\ce{D}]
        \end{align*}
        \item At equilibrium,
        \begin{align*}
            k_1[\ce{A}]_\text{eq}[\ce{B}]_\text{eq} &= k_{-1}[\ce{C}]_\text{eq}[\ce{D}]_\text{eq}\\
            \frac{k_1}{k_{-1}} &= \frac{[\ce{C}]_\text{eq}[\ce{D}]_\text{eq}}{[\ce{A}]_\text{eq}[\ce{B}]_\text{eq}} = K_c
        \end{align*}
    \end{itemize}
    \item \textbf{Principle of detailed balance}: The following relationship, which holds for all reversible elementary reactions. \emph{Given by}
    \begin{equation*}
        K_c = \frac{k_1}{k_{-1}}
    \end{equation*}
\end{itemize}



\section{The Two-Step Consecutive Reaction Mechanism}
\begin{itemize}
    \item \marginnote{4/20:}Consider the general complex reaction
    \begin{equation*}
        \ce{A ->[$k_\text{obs}$] P}
    \end{equation*}
    \begin{itemize}
        \item Suppose that the reaction occurs by the two step mechanism
        \begin{align*}
            \ce{A} &\stackrel{k_1}{\Longrightarrow} \ce{I}&
            \ce{I} &\stackrel{k_2}{\Longrightarrow} \ce{P}
        \end{align*}
        \item Because each step of this mechanism is an elementary reaction, the rate laws for each species are
        \begin{align*}
            \dv{[\ce{A}]}{t} &= -k_1[\ce{A}]&
            \dv{[\ce{I}]}{t} &= k_1[\ce{A}]-k_2[\ce{I}]&
            \dv{[\ce{P}]}{t} &= k_2[\ce{I}]
        \end{align*}
        \item Thus, assuming that the initial concentrations at time $t=0$ are $[\ce{A}]=[\ce{A}]_0$ and $[\ce{I}]_0=[\ce{P}]_0=0$, we have that
        \begin{gather*}
            [\ce{A}] = [\ce{A}]_0\e[-k_1t]\\
            [\ce{I}] = \frac{k_1[\ce{A}]_0}{k_2-k_1}(\e[-k_1t]-\e[-k_2t])\\
            [\ce{P}] = [\ce{A}]_0-[\ce{A}]-[\ce{I}]
                = [\ce{A}]_0\left\{ 1+\frac{1}{k_1-k_2}(k_2\e[-k_1t]-k_1\e[-k_2t]) \right\}
        \end{gather*}
    \end{itemize}
    \item Distinguishing the two-step consecutive reaction mechanism unambiguously from the one-step reaction.
    \begin{itemize}
        \item For a single step reaction,
        \begin{equation*}
            [\ce{P}] = [\ce{A}]_0(1-\e[-k_1t])
        \end{equation*}
        \item The two-step consecutive reaction mechanism has the following alternate form.
        \begin{equation*}
            [\ce{P}] = [\ce{A}]_0\left\{ 1+\frac{1}{k_1-k_2}(k_2\e[-k_1t]-k_1\e[-k_2t]) \right\}
        \end{equation*}
        \item However, if $k_2\gg k_1$, then
        \begin{align*}
            [\ce{P}] &= [\ce{A}]_0\left\{ 1+\frac{1}{k_1-k_2}(k_2\e[-k_1t]-k_1\e[-k_2t]) \right\}\\
            &\approx [\ce{A}]_0\left\{ 1+\frac{1}{-k_2}k_2\e[-k_1t] \right\}\\
            &= [\ce{A}]_0(1-\e[-k_1t])
        \end{align*}
        \item If $k_1\gg k_2$, he reaction reduces to
        \begin{equation*}
            [\ce{P}] \approx [\ce{A}]_0(1-\e[-k_2t])
        \end{equation*}
        \item Thus, the only ambiguous situation is $k_2\gg k_1$.
    \end{itemize}
    \item The steady-state approximation simplifies rate expressions.
    \begin{itemize}
        \item We assume that $\dv*{[\ce{I}]}{t}=0$, where \ce{I} is a reaction intermediate.
        \item Given the above differential equation for $\dv*{[\ce{I}]}{t}$, making the above assumption yields
        \begin{equation*}
            [\ce{I}]_\text{SS} = \frac{k_1[\ce{A}]}{k_2}
        \end{equation*}
        \item It follows that
        \begin{equation*}
            [\ce{I}]_\text{SS} = \frac{k_1}{k_2}[\ce{A}]_0\e[-k_1t]
        \end{equation*}
        \item Thus,
        \begin{equation*}
            \dv{[\ce{I}]_\text{SS}}{t} = \frac{-k_1^2}{k_2}[\ce{A}]_0\e[-k_1t]
        \end{equation*}
        \item We get $k_2\gg k_1^2[\ce{A}]_0$ and $[\ce{P}]=[\ce{A}]_0(1-\e[-k_1t])$.
    \end{itemize}
    \item Example: Decomposition of ozone.
    \begin{equation*}
        \ce{2O3(g) -> 3O2(g)}
    \end{equation*}
    \begin{itemize}
        \item The reaction mechanism is
        \begin{align*}
            \ce{M(g) + O3(g)} &\Longleftrightarrows[k_1][k_{-1}] \ce{O2(g) + O(g) + M(g)}\\
            \ce{O(g) + O3(g)} &\xRightarrow{k_2} \ce{2O2(g)}
        \end{align*}
        where \ce{M} is a molecule that can exchange energy with the reacting ozone molecule through a collision, but \ce{M} itself does not react.
        \item The rate equations for \ce{O3(g)} and \ce{O(g)} are
        \begin{gather*}
            \dv{[\ce{O3}]}{t} = -k_1[\ce{O3}][\ce{M}]+k_{-1}[\ce{O2}][\ce{O}][\ce{M}]-k_2[\ce{O}][\ce{O3}]\\
            \dv{[\ce{O}]}{t} = k_1[\ce{O3}][\ce{M}]-k_{-1}[\ce{O2}][\ce{O}][\ce{M}]-k_2[\ce{O}][\ce{O3}]
        \end{gather*}
        \item Invoking the steady-state approximation for the intermediate \ce{O} yields
        \begin{equation*}
            [\ce{O}] = \frac{k_1[\ce{O3}][\ce{M}]}{k_{-1}[\ce{O2}][\ce{M}]+k_2[\ce{O3}]}
        \end{equation*}
        \item Substituting this result into the rate equation for \ce{O3} gives
        \begin{equation*}
            \dv{[\ce{O3}]}{t} = -\frac{2k_1k_2[\ce{O3}]^2[\ce{M}]}{k_{-1}[\ce{O2}][\ce{M}]+k_2[\ce{O3}]}
        \end{equation*}
    \end{itemize}
\end{itemize}



\section{Complex Reactions}
\begin{itemize}
    \item \marginnote{4/22:}Expect the midterm to be 2 hours in length, available all next week, and to incorporate largely HW-like questions but also some open-ended, design-an-experiment questions. Completely open note.
    \item The rate law for a complex reaction does not imply a unique mechanism.
    \begin{itemize}
        \item Consider the reaction
        \begin{equation*}
            \ce{2NO(g) + O2(g) ->[$k_\text{obs}$] 2NO2(g)}
        \end{equation*}
        \item The rate law is
        \begin{equation*}
            \frac{1}{2}\dv{[\ce{NO2}]}{t} = k_\text{obs}[\ce{NO}]^2[\ce{O2}]
        \end{equation*}
        \item Experimental studies confirm that the reaction is not an elementary reaction, but we can propose multiple mechanisms that would both yield the same rate law. Here are two examples.
        \begin{itemize}
            \item Mechanism 1.
            \begin{align*}
                \ce{NO(g) + O2(g)} &\Longleftrightarrows[k_1][k_{-1}] \ce{NO3(g)}\\
                \ce{NO3(g) + NO(g)} &\ce{->[$k_2$]} \ce{2 NO2(g)}
            \end{align*}
            \item Mechanism 2.
            \begin{align*}
                \ce{2NO(g)} &\Longleftrightarrows[k_1][k_{-1}] \ce{N2O2(g)}\\
                \ce{N2O2(g) + O2(g)} &\ce{->[$k_2$]} \ce{2 NO2(g)}
            \end{align*}
        \end{itemize}
        \item One experiment to design is to capture or otherwise detect the intermediate species.
        \item Through such an experiment, we can verify Mechanism 2.
    \end{itemize}
    \item The Lindemann Mechanism explains how unimolecular reactions occur.
    \begin{itemize}
        \item Consider the reaction
        \begin{equation*}
            \ce{CH3NC(g) ->[$k_\text{obs}$] CH3CN(g)}
        \end{equation*}
        \item The following rate law is only correct at $[\ce{CH3NC}]$.
        \begin{equation*}
            \dv{[\ce{CH3NC}]}{t} = -k_\text{obs}[\ce{CH3NC}]
        \end{equation*}
        \item At low $[\ce{CH3NC}]$, we have
        \begin{equation*}
            \dv{[\ce{CH3NC}]}{t} = -k_\text{obs}[\ce{CH3NC}]^2
        \end{equation*}
        which is not the rate law for a unimolecular reaction.
        \item The Lindemann mechanism for unimolecular reactions of the form \ce{A(g) -> B(g)} is
        \begin{align*}
            \ce{A(g) + M(g)} &\Longleftrightarrows[k_1][k_{-1}] \ce{A(g)^* + M(g)}\\
            \ce{A(g)^*} &\ce{->[$k_2$]} \ce{B(g)}
        \end{align*}
        \item The symbol \ce{A(g)^*} represents an energized reactant molecule. \ce{M(g)} is the collision partner.
        \item By the steady-state approximation, we have that
        \begin{align*}
            \dv{[\ce{A^*}]}{t} = 0 &= k_1[\ce{A}][\ce{M}]-k_{-1}[\ce{A^*}][\ce{M}]-k_2[\ce{A^*}]\\
            [\ce{A^*}] &= \frac{k_1[\ce{M}][\ce{A}]}{k_2+k_{-1}[\ce{M}]}
        \end{align*}
        so that
        \begin{align*}
            \dv{[\ce{B}]}{t} &= k_2[\ce{A^*}]\\
            -\dv{[\ce{A}]}{t} = \dv{[\ce{B}]}{t} &= \underbrace{\frac{k_2k_1[\ce{M}]}{k_2+k_{-1}[\ce{M}]}}_{k_\text{obs}}[\ce{A}]
        \end{align*}
        \item At high $[\ce{M}]$, we have that $k_{-1}[\ce{M}][\ce{A^*}]\gg k_2[\ce{A^*}]$, or $k_{-1}[\ce{M}]\gg k_2$. Thus,
        \begin{equation*}
            k_\text{obs} = \frac{k_1k_2}{k_{-1}}
        \end{equation*}
        \item At low $[\ce{M}]$, we have that $k_2\gg k_{-1}[\ce{M}]$ so that
        \begin{align*}
            \dv{[\ce{B}]}{t} &= k_1[\ce{M}][\ce{A}]\\
            &= k_1[\ce{A}]^2
        \end{align*}
        \item This mechanism was proposed by the British chemists J. A. Christiansen in 1921 and F. A. Lindemann in 1922. Their work underlies the current theory of unimolecular reaction rates.
    \end{itemize}
    \item Some reaction mechanisms involve chain reactions.
    \begin{itemize}
        \item Chain reactions involve amplification.
        \item For example, \ce{H2(g) + Br2(g) <=> 2HBr(g)} follows the ensuing mechanism.
        \begin{itemize}
            \item Initiation.
            \begin{equation*}
                \ce{Br2 + M(g) ->[k_1] 2Br(g) + M(g)}
            \end{equation*}
            \item Propagation.
            \begin{align*}
                \ce{Br(g) + H2(g)} &\ce{->[k_2]} \ce{HBr(g) + H(g)}\\
                \ce{H(g) + Br2(g)} &\ce{->[k_3]} \ce{HBr(g) + Br(g)}
            \end{align*}
            \item Inhibition.
            \begin{align*}
                \ce{HBr(g) + H(g)} &\ce{->[k_{-2}]} \ce{Br(g) + H2(g)}\\
                \ce{HBr(g) + Br(g)} &\ce{->[k_{-3}]} \ce{H(g) + Br2(g)}
            \end{align*}
            \item Termination.
            \begin{equation*}
                \ce{2Br + M(g) ->[k_{-1}] Br2(g) + M(g)}
            \end{equation*}
        \end{itemize}
        \item The fifth step can be ignored.
        \item Notice that the inhibition and termination reactions are the reverse reactions of the propagation and initiation reaction(s), respectively.
        \begin{itemize}
            \item Termination does not need to be the reverse of initiation, though. Termination just kills any reactive species.
            \item Inhibition is the reverse of propagation, though.
        \end{itemize}
        \item When you want to design a chain reaction species, make sure you have a reactive species (like bromine) for the initiation step. Notice, for instance, that hydrogen does not initiate.
        \item This leads to the experimentally determined rate law
        \begin{equation*}
            \frac{1}{2}\dv{[\ce{HBr}]}{t} = \frac{k[\ce{H2}][\ce{Br2}]^{1/2}}{1+k'[\ce{HBr}][\ce{Br2}]^{-1}}
        \end{equation*}
        \item Deriving said rate law.
        \begin{itemize}
            \item We have that
            \begin{align*}
                \dv{[\ce{HBr}]}{t} &= k_2[\ce{Br}][\ce{H2}]-k_{-2}[\ce{HBr}][\ce{H}]+k_3[\ce{H}][\ce{Br2}]\\
                \dv{[\ce{H}]}{t} &= k_2[\ce{Br}][\ce{H2}]-k_{-2}[\ce{HBr}][\ce{H}]-k_3[\ce{H}][\ce{Br2}]\\
                \dv{[\ce{Br}]}{t} &= 2k_1[\ce{Br2}][\ce{M}]-k_{-1}[\ce{Br}]^2[\ce{M}]-k_2[\ce{Br}][\ce{H2}]+k_{-2}[\ce{HBr}][\ce{H}]+k_3[\ce{H}][\ce{Br2}]
            \end{align*}
            \item We can apply the SS approximation to the second and third equations above, which both describe intermediate species.
            \begin{align*}
                0 &= k_2[\ce{Br}][\ce{H2}]-k_{-2}[\ce{HBr}][\ce{H}]-k_3[\ce{H}][\ce{Br2}]\\
                0 &= 2k_1[\ce{Br2}][\ce{M}]-k_{-1}[\ce{Br}]^2[\ce{M}]-k_2[\ce{Br}][\ce{H2}]+k_{-2}[\ce{HBr}][\ce{H}]+k_3[\ce{H}][\ce{Br2}]
            \end{align*}
            \item Solving the two equations above for $[\ce{H}]$ and $[\ce{Br}]$, respectively, is made substantially easier by noting that the negative of the first expression appears in its entirety in the second expression. Thus, we may simply substitute the former into the latter and solve to find an expression for $[\ce{Br}]$.
            \begin{align*}
                0 &= 2k_1[\ce{Br2}][\ce{M}]-k_{-1}[\ce{Br}]^2[\ce{M}]-0\\
                [\ce{Br}] &= \left( \frac{k_1}{k_{-1}} \right)^{1/2}[\ce{Br2}]^{1/2}\\
                [\ce{Br}] &= (K_{c,1})^{1/2}[\ce{Br2}]^{1/2}
            \end{align*}
            \item Resubstituting yields an expression for $[\ce{H}]$.
            \begin{align*}
                0 &= k_2[\ce{Br}][\ce{H2}]-k_{-2}[\ce{HBr}][\ce{H}]-k_3[\ce{H}][\ce{Br2}]\\
                0 &= k_2(K_{c,1})^{1/2}[\ce{Br2}]^{1/2}[\ce{H2}]-(k_{-2}[\ce{HBr}]+k_3[\ce{Br2}])[\ce{H}]\\
                [\ce{H}] &= \frac{k_2(K_{c,1})^{1/2}[\ce{Br2}]^{1/2}[\ce{H2}]}{k_{-2}[\ce{HBr}]+k_3[\ce{Br2}]}
            \end{align*}
            \item Substituting these two expressions back into the original differential equation for $[\ce{HBr}]$ yields
            \begin{align*}
                \dv{[\ce{HBr}]}{t} ={}& k_2[\ce{Br}][\ce{H2}]-k_{-2}[\ce{HBr}][\ce{H}]+k_3[\ce{H}][\ce{Br2}]\\
                \begin{split}
                    ={}& k_2(K_{c,1})^{1/2}[\ce{Br2}]^{1/2}[\ce{H2}]-k_{-2}[\ce{HBr}]\cdot\frac{k_2(K_{c,1})^{1/2}[\ce{Br2}]^{1/2}[\ce{H2}]}{k_{-2}[\ce{HBr}]+k_3[\ce{Br2}]}\\
                    &+k_3\cdot\frac{k_2(K_{c,1})^{1/2}[\ce{Br2}]^{1/2}[\ce{H2}]}{k_{-2}[\ce{HBr}]+k_3[\ce{Br2}]}\cdot[\ce{Br2}]
                \end{split}\\
                \begin{split}
                    ={}& k_2(K_{c,1})^{1/2}[\ce{Br2}]^{1/2}[\ce{H2}]-\frac{k_2k_{-2}(K_{c,1})^{1/2}[\ce{HBr}][\ce{Br2}]^{1/2}[\ce{H2}]}{k_{-2}[\ce{HBr}]+k_3[\ce{Br2}]}\\
                    &+\frac{k_2k_3(K_{c,1})^{1/2}[\ce{Br2}]^{3/2}[\ce{H2}]}{k_{-2}[\ce{HBr}]+k_3[\ce{Br2}]}
                \end{split}\\
                ={}& k_2(K_{c,1})^{1/2}[\ce{Br2}]^{1/2}[\ce{H2}]\left( 1-\frac{k_{-2}[\ce{HBr}]}{k_{-2}[\ce{HBr}]+k_3[\ce{Br2}]}+\frac{k_3[\ce{Br2}]}{k_{-2}[\ce{HBr}]+k_3[\ce{Br2}]} \right)\\
                ={}& k_2(K_{c,1})^{1/2}[\ce{Br2}]^{1/2}[\ce{H2}]\left( \frac{k_{-2}[\ce{HBr}]+k_3[\ce{Br2}]}{k_{-2}[\ce{HBr}]+k_3[\ce{Br2}]}-\frac{k_{-2}[\ce{HBr}]-k_3[\ce{Br2}]}{k_{-2}[\ce{HBr}]+k_3[\ce{Br2}]} \right)\\
                ={}& k_2(K_{c,1})^{1/2}[\ce{Br2}]^{1/2}[\ce{H2}]\cdot\frac{2k_3[\ce{Br2}]}{k_{-2}[\ce{HBr}]+k_3[\ce{Br2}]}\\
                \frac{1}{2}\dv{[\ce{HBr}]}{t} &= k_2(K_{c,1})^{1/2}[\ce{Br2}]^{1/2}[\ce{H2}]\cdot\frac{1}{(k_{-2}/k_3)[\ce{HBr}][\ce{Br2}]^{-1}+1}\\
                ={}& \frac{k_2(K_{c,1})^{1/2}[\ce{H2}][\ce{Br2}]^{1/2}}{1+(k_{-2}/k_3)[\ce{HBr}][\ce{Br2}]^{-1}}\\
                ={}& \frac{k[\ce{H2}][\ce{Br2}]^{1/2}}{1+k'[\ce{HBr}][\ce{Br2}]^{-1}}
            \end{align*}
            where we have substituted $k=k_2(K_{c,1})^{1/2}$ and $k'=k_{-2}/k_3$ in the last expression.
        \end{itemize}
    \end{itemize}
    \item Problem 29-24.
    \begin{itemize}
        \item The reaction
        \begin{equation*}
            \ce{CH3CHO(g) ->[$k_\text{obs}$] CH4(g) + CO(g)}
        \end{equation*}
        proceeds by the mechanism
        \begin{align*}
            \ce{CH3CHO(g)} &\ce{->[$k_1$]} \ce{CH3(g) + CHO(g)}\\
            \ce{CH3(g) + CH3CHO(g)} &\ce{->[$k_2$]} \ce{CH4(g) + CH3CO(g)}\\
            \ce{CH3CO(g)} &\ce{->[$k_3$]} \ce{CH3(g) + CO(g)}\\
            \ce{2CH3(g)} &\ce{->[$k_4$]} \ce{C2H6(g)}
        \end{align*}
        \item The initiation step is the first equation, the propagation steps are the second and third equations, and the termination step is the fourth equation.
        \item We can write the rate laws
        \begin{align*}
            \dv{[\ce{CH4}]}{t} &= k_2[\ce{CH3}][\ce{CH3CHO}]\\
            \dv{[\ce{CH3}]}{t} &= k_1[\ce{CH3CHO}]-k_2[\ce{CH3}][\ce{CH3CHO}]+k_3[\ce{CH3CO}]-2k_4[\ce{CH3}]\\
            \dv{[\ce{CH3CO}]}{t} &= k_2[\ce{CH3}][\ce{CH3CHO}]-k_3[\ce{CH3CO}]
        \end{align*}
        \item Applying the SS approximation to the last second and third equations yields (respectively)
        \begin{align*}
            [\ce{CH3}] &= \frac{k_1[\ce{CH3CHO}]+k_3[\ce{CH3CO}]}{k_2[\ce{CH3CHO}]+2k_4}&
            [\ce{CH3CO}] &= \frac{k_2}{k_3}[\ce{CH3}][\ce{CH3CHO}]
        \end{align*}
        \item Substituting the right equation above into the left equation above and solving for $[\ce{CH3}]$ yields an expression for $[\ce{CH3}]$ purely in terms of $[\ce{CH3CHO}]$.
        \begin{align*}
            [\ce{CH3}] &= \frac{k_1[\ce{CH3CHO}]+k_2[\ce{CH3}][\ce{CH3CHO}]}{k_2[\ce{CH3CHO}]+2k_4}\\
            k_2[\ce{CH3}][\ce{CH3CHO}]+2k_4[\ce{CH3}] &= k_1[\ce{CH3CHO}]+k_2[\ce{CH3}][\ce{CH3CHO}]\\
            2k_4[\ce{CH3}] &= k_1[\ce{CH3CHO}]\\
            [\ce{CH3}] &= \frac{k_1}{2k_4}[\ce{CH3CHO}]
        \end{align*}
        \item The final result is
        \begin{align*}
            \dv{[\ce{CH4}]}{t} &= k_2\left( \frac{k_1}{2k_4}[\ce{CH3CHO}] \right)[\ce{CH3CHO}]\\
            &= k_2\left( \frac{k_1}{2k_4} \right)^{1/2}[\ce{CH3CHO}]^{3/2}
        \end{align*}
        \begin{itemize}
            \item What's the issue here?
        \end{itemize}
    \end{itemize}
\end{itemize}



\section{Midterm Review and Intro to Catalysts}
\begin{itemize}
    \item \marginnote{4/27:}Example problem 1: Steady-state approximation.
    \begin{itemize}
        \item Let
        \begin{equation*}
            \ce{A <=>[$k_a$][$k_a'$] B <=>[$k_b$][$k_b'$] C <=>[$k_c$][$k_c'$] D}
        \end{equation*}
        Suppose $\cnc{A}$ is maintained at a fixed value and the produce \ce{D} is removed from the reaction as it is formed. Find the rate at which the product is formed in terms of $\cnc{A}$.
        \item By hypothesis, we have that at all times $t$, $\cnc{A}=\cnc[0]{A}$ and $\cnc{D}=0$.
        \item The hypotheses also imply that we can apply the steady-state approximation to both \ce{B} and \ce{C}.
        \item Thus, we have that
        \begin{align*}
            \dv{\cnc{C}}{t} = 0 &= k_b\cnc{B}-k_c\cnc{C}-k_b'\cnc{C}\\
            \cnc{B} &= \frac{k_b'+k_c}{k_b}\cnc{C}
        \end{align*}
        so that
        \begin{align*}
            \dv{\cnc{B}}{t} &= k_a\cnc{A}-k_b\cnc{B}-k_a'\cnc{B}+k_b'\cnc{C}\\
            0 &= k_a\cnc{A}-k_b\cdot\frac{k_b'+k_c}{k_b}\cnc{C}-k_a'\cdot\frac{k_b'+k_c}{k_b}\cnc{C}+k_b'\cnc{C}\\
            % &= k_a\cnc{A}-(k_b'+k_c+\frac{k_a'k_b'}{k_b}+\frac{k_a'k_c}{k_b}-k_b')\cnc{C}\\
            % &= k_a\cnc{A}-(\frac{k_bk_c+k_a'k_b'+k_a'k_c}{k_b})\cnc{C}\\
            \cnc{C} &= \frac{k_ak_b}{k_bk_c+k_a'k_b'+k_a'k_c}\cnc{A}
        \end{align*}
        and therefore
        \begin{align*}
            \dv{\cnc{D}}{t} &= k_c\cnc{C}-k_c'\cdot 0\\
            &= \frac{k_ak_bk_c}{k_bk_c+k_a'k_b'+k_a'k_c}\cnc{A}
        \end{align*}
    \end{itemize}
    \item Example problem 2.
    \begin{itemize}
        \item Consider the reaction
        \begin{equation*}
            \ce{HCl + CH3CH=CH2 <=> CH3CHClCH3}
        \end{equation*}
        which proceeds by the mechanism
        \begin{enumerate}
            \item \ce{HCl + HCl <=> (HCl)2} (equilibrium constant $K_1$).
            \item \ce{HCl + CH3CH=CH2 <=> complex} (equilibrium constant $K_2$).
            \item \ce{(HCl)2 + complex <=> CH3CHClCH3 + HCl + HCl} (equilibrium constant $K_3$).
        \end{enumerate}
        \item The equilibrium constants for the two pre-equilibria are
        \begin{align*}
            K_1 &= \frac{\cnc[eq]{(HCl)2}c^\circ}{\cnc[eq]{HCl}^2}&
            K_2 &= \frac{\cnc[eq]{complex}c^\circ}{\cnc[eq]{HCl}\cnc[eq]{CH3CH=CH2}}
        \end{align*}
        \item We can divide the mass-action expression for $K_1$ by $(c^\circ)^2$ to get each concentration over $c^\circ$ within its exponent.
        \item The rate of product formation is
        \begin{align*}
            v &= \dv{\cnc{CH3CHClCH3}}{t}\\
            &= k_r\cnc{(HCl)2}\cnc{complex}\\
            &\approx k_r\cnc[eq]{(HCl)2}\cnc[eq]{complex}\\
            &= k_r\cdot\frac{K_1\cnc[eq]{HCl}^2}{c^\circ}\cdot\frac{K_2\cnc[eq]{HCl}\cnc[eq]{CH3CH=CH2}}{c^\circ}\\
            &= \frac{k_rK_1K_2}{(c^\circ)^2}\cnc[eq]{HCl}^3\cnc[eq]{CH3CH=CH2}
        \end{align*}
        \begin{itemize}
            \item There's a key assumption with the steady state and something about being able to apply the equilibrium concentration of the intermediate as the steady-state quantity.
            \item This question wants to let you know that an equilibrium constant like $K_1$ might indicate a steady-state approximation.
        \end{itemize}
    \end{itemize}
    \item Note: Mind the positive and negative signs when constructing differential rate laws!
    \item The midterm will be posted this Friday (April 29) and will be available until the following Friday (May 6). There will be a timed 2 hour period to take it.
    \item \textbf{Catalyst}: A substance that participates in the chemical reaction but is not consumed in the process.
    \begin{itemize}
        \item A catalyst affects the mechanism and activation energy of a chemical reaction.
        \item A catalyst can give rise to a reaction path with a negligible activation barrier.
        \item The exothermicity or endothermicity of the chemical reaction is not altered by the presence of a catalyst.
    \end{itemize}
    \item \textbf{Homogeneous catalysis}: Catalysis in which the catalyst is in the same phase as the reactants and products.
    \item \textbf{Heterogeneous catalysis}: Catalysis in which the catalyst is in a different phase from the reactants and products.
    \item Imagine that initially, we have the reaction
    \begin{equation*}
        \ce{A ->[$k$] products}
    \end{equation*}
    where $k$ is the observed rate constant.
    \begin{itemize}
        \item When a catalyst is introduced into solution, this mechanism continues, but we now also have the new reaction pathway
        \begin{equation*}
            \ce{A + catalyst ->[$k_\text{cat}$] products + catalyst}
        \end{equation*}
        \item If each of these competing reactions is an elementary process, then
        \begin{equation*}
            -\dv{\cnc{A}}{t} = k\cnc{A}+k_\text{cat}\cnc{A}\cnc{catalyst}
        \end{equation*}
        \item In most cases, catalysts enhance reaction rates by many orders of magnitude, and therefore only the rate law for the catalyzed reaction need be considered in analyzing experimental data.
    \end{itemize}
    \item Reviews the Nobel Prizes in 2020 and 2021 (for CRISPR and asymmetric organocatalysis, respectively).
    \item An example of homogeneous catalysis.
    \begin{itemize}
        \item Consider the reaction
        \begin{equation*}
            \ce{2Ce^4+(aq) + Tl+(aq) -> 2Ce^3+(aq) + Tl^3+(aq)}
        \end{equation*}
        \item In the absence of a catalyst,
        \begin{equation*}
            v = k\cnc{Tl+}\cnc{Ce^4+}^2
        \end{equation*}
        and the mechanism is a termolecular elementary reaction.
        \item However, with \ce{Mn^2+} as the catalyst, we have the mechanism
        \begin{align*}
            \ce{Ce^4+(aq) + Mn^2+(aq)} &\xRightarrow{k_\text{cat}}                \ce{Mn^3+(aq) + Ce^3+(aq)}\\
            \ce{Ce^4+(aq) + Mn^3+(aq)} &\xRightarrow{{\color{white}k_\text{cat}}} \ce{Mn^4+(aq) + Ce^3+(aq)}\\
            \ce{Mn^4+(aq) + Tl+(aq)}   &\xRightarrow{{\color{white}k_\text{cat}}} \ce{Mn^2+(aq) + Tl^3+(aq)}
        \end{align*}
        where the step with $k_\text{cat}$ is the rate-determining step.
        \begin{itemize}
            \item Thus, for this mechanism, we have that
            \begin{equation*}
                v = k_\text{cat}\cnc{Ce^4+}\cnc{Mn^2+}
            \end{equation*}
        \end{itemize}
        \item The overall rate law for this reaction is therefore
        \begin{equation*}
            v = k\cnc{Tl+}\cnc{Ce^4+}^2+k_\text{cat}\cnc{Ce^4+}\cnc{Mn^2+}
        \end{equation*}
    \end{itemize}
\end{itemize}




\end{document}