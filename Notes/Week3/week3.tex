\documentclass[../notes.tex]{subfiles}

\pagestyle{main}
\renewcommand{\chaptermark}[1]{\markboth{\chaptername\ \thechapter\ (#1)}{}}
\setcounter{chapter}{2}

\begin{document}




\chapter{Rate Laws}
\section{Integrated Rate Laws}
\begin{itemize}
    \item \marginnote{4/11:}First order reactions have exponential integrated rate laws.
    \begin{itemize}
        \item Suppose \ce{A + B -> products}.
        \item Suppose the reaction is first order in \ce{A}.
        \item If the concentration of \ce{A} is $[\ce{A}]_0$ at $t=0$ and $[\ce{A}]$ at time $t$, then
        \begin{align*}
            v(t) = -\dv{[\ce{A}]}{t} &= k[\ce{A}]\\
            \int_{[\ce{A}]_0}^{[\ce{A}]}\frac{\dd{[\ce{A}]}}{[\ce{A}]} &= -\int_0^tk\dd{t}\\
            \ln\frac{[\ce{A}]}{[\ce{A}]_0} &= -kt\\
            [\ce{A}] &= [\ce{A}]_0\e[-kt]
        \end{align*}
        is the integrated rate law.
        \item Goes over both the concentration plot and the linear logarithmic plot.
    \end{itemize}
    \item The half-life of a first-order reaction is independent of the initial amount of reactant.
    \begin{itemize}
        \item The half-life is found from the point
        \begin{equation*}
            [\ce{A}(t_{1/2})] = \frac{[\ce{A}(0)]}{2}
            = \frac{[\ce{A}]_0}{2}
        \end{equation*}
        \item We have
        \begin{align*}
            \ln\frac{1}{2} &= -kt_{1/2}\\
            t_{1/2} &= \frac{\ln 2}{k} \approx \frac{0.693}{k}
        \end{align*}
        \item Notice that the above equation does not depend on $[\ce{A}]$ or $[\ce{B}]$!
    \end{itemize}
    \item Second order reactions have inverse concentration integrated rate laws.
    \begin{itemize}
        \item Suppose \ce{A + B -> products}, as before, and that the reaction is second order in \ce{A}.
        \item Then
        \begin{align*}
            -\dv{[\ce{A}]}{t} &= k[\ce{A}]^2\\
            \int_{[\ce{A}]_0}^{[\ce{A}]}-\frac{\dd{[\ce{A}]}}{[\ce{A}]^2} &= \int_0^tk\dd{t}\\
            \frac{1}{[\ce{A}]} &= \frac{1}{[\ce{A}]_0}+kt
        \end{align*}
        is the integrated rate law.
    \end{itemize}
    \item The half-life of a second-order reaction is dependent on the initial amount of reaction.
    \begin{itemize}
        \item We have that
        \begin{align*}
            \frac{1}{[\ce{A}]_0/2} &= \frac{1}{[\ce{A}]_0}+kt_{1/2}\\
            \frac{1}{[\ce{A}]_0} &= kt_{1/2}\\
            t_{1/2} &= \frac{1}{k[\ce{A}]_0}
        \end{align*}
    \end{itemize}
    \item If a reaction is $n^\text{th}$-order in a reactant for $n\geq 2$, then the integrated rate law is given by
    \begin{align*}
        -\dv{[\ce{A}]}{t} &= k[\ce{A}]^n\\
        \int_{[\ce{A}]_0}^{[\ce{A}]}-\frac{\dd{[\ce{A}]}}{[\ce{A}]^n} &= \int_0^tk\dd{t}\\
        \frac{1}{n-1}\left( \frac{1}{[\ce{A}]^{n-1}}-\frac{1}{[\ce{A}]_0^{n-1}} \right) &= kt
    \end{align*}
    \begin{itemize}
        \item The associated half life is
        \begin{align*}
            \frac{1}{n-1}\left( \frac{1}{([\ce{A}]_0/2)^{n-1}}-\frac{1}{[\ce{A}]_0^{n-1}} \right) &= kt_{1/2}\\
            \frac{1}{n-1}\cdot\frac{2^{n-1}-1}{[\ce{A}]_0^{n-1}} &= kt_{1/2}\\
            t_{1/2} &= \frac{2^{n-1}-1}{k(n-1)[\ce{A}]_0^{n-1}}
        \end{align*}
    \end{itemize}
    \item Second order reactions that are first order in each reactant.
    \begin{itemize}
        \item We have that
        \begin{align*}
            -\dv{[\ce{A}]}{t} = -\dv{[\ce{B}]}{t} &= k[\ce{A}][\ce{B}]\\
            kt &= \frac{1}{[\ce{A}]_0-[\ce{B}]_0}\ln\frac{[\ce{A}][\ce{B}]_0}{[\ce{B}][\ce{A}]_0}
        \end{align*}
        \item The actual determination is more complicated (there is a textbook problem that walks us through the derivation, though).
        \item When $[\ce{A}]_0=[\ce{B}]_0$, the integrated rate law simplifies to the second-order integrated rate laws in $[\ce{A}]$ and $[\ce{B}]$.
        \begin{itemize}
            \item In this limited case, the half-life is that of the second-order integrated rate law, too, i.e., $t_{1/2}=1/k[\ce{A}]_0$.
        \end{itemize}
    \end{itemize}
    \item The reaction paths and mechanism for parallel reactions.
    \begin{itemize}
        \item Suppose \ce{A} can become both \ce{B} and \ce{C} with respective rate constants $k_B$ and $k_C$.
        \item Then
        \begin{align*}
            \dv{[\ce{A}]}{t} &= -k_B[\ce{A}]-k_C[\ce{A}] = -(k_B+k_C)[\ce{A}]&
            \dv{[\ce{B}]}{t} &= k_B[\ce{A}]&
            \dv{[\ce{C}]}{t} &= k_C[\ce{A}]
        \end{align*}
        \item The integrated rate laws here are
        \begin{align*}
            [\ce{A}] &= [\ce{A}]_0\e[-(k_B+k_C)t]&
            [\ce{B}] &= \frac{k_B}{k_B+k_C}[\ce{A}]_0\left( 1-\e[-(k_B+k_C)t] \right)&
            [\ce{C}] &= \frac{k_C}{k_B+k_C}[\ce{A}]_0\left( 1-\e[-(k_B+k_C)t] \right)
        \end{align*}
        \item The ratio of product concentrations is
        \begin{equation*}
            \frac{[\ce{B}]}{[\ce{C}]} = \frac{k_B}{k_C}
        \end{equation*}
        \item The yield $\Phi_i$ is the probability that a given product $i$ will be formed from the decay of the reactant
        \begin{align*}
            \Phi_i &= \frac{k_i}{\sum_nk_n}&
            \sum_i\Phi_i &= 1
        \end{align*}
    \end{itemize}
    \item Example: If we have parallel reactions satisfying $k_B=2k_C$, then
    \begin{equation*}
        \Phi_C = \frac{k_C}{k_B+k_C}
        = \frac{k_C}{2k_C+k_C}
        = \frac{1}{3}
    \end{equation*}
\end{itemize}



\section{Office Hours (Tian)}
\begin{itemize}
    \item Why does the reduced mass work in the collision frequency derivation?
    \begin{itemize}
        \item We need to start with a simpler case, or the problem will be really hard; thus, we begin by assuming that the particles all are static.
        \item We use the reduced mass to consider the relative speed $u_r$ of the particles with respect to the moving particle as our reference frame. So all the others are the relative speeds to our particle. But this necessitates using the relative mass of the particles to our particle (which is the reduced mass).
    \end{itemize}
\end{itemize}



\section{Reversible Reactions}
\begin{itemize}
    \item \marginnote{4/13:}Let
    \begin{equation*}
        \ce{A <=>[$k_1$][$k_{-1}$] B}
    \end{equation*}
    be a reversible reaction, where $k_1$ is the rate constant for the forward reaction and $k_{-1}$ is the rate constant for the reverse reaction.
    \begin{itemize}
        \item In this case, we have an equilibrium constant expression
        \begin{equation*}
            K_c = \frac{[\ce{B}]_\text{eq}}{[\ce{A}]_\text{eq}}
        \end{equation*}
        \item Additionally, the kinetic conditions for equilibrium are
        \begin{equation*}
            -\dv{[\ce{A}]}{t} = \dv{[\ce{B}]}{t} = 0
        \end{equation*}
        \item If the reaction is first order in both $[\ce{A}]$ and $[\ce{B}]$, then
        \begin{equation*}
            -\dv{[\ce{A}]}{t} = k_1[\ce{A}]-k_{-1}[\ce{B}]
        \end{equation*}
        \item If $[\ce{A}]=[\ce{A}]_0$ at $t=0$, then $[\ce{B}]=[\ce{A}]_0-[\ce{A}]$ and
        \begin{equation*}
            -\dv{[\ce{A}]}{t} = (k_1+k_{-1})[\ce{A}]-k_{-1}[\ce{A}]_0
        \end{equation*}
        \begin{itemize}
            \item Note that $[\ce{B}]=[\ce{A}]_0-[\ce{A}]$ iff there is no initial concentration of \ce{B}, the initial equation was balanced (i.e., each unit of \ce{A} forms one unit of \ce{B}), and there is not another component \ce{C} into which \ce{A} decomposes.
        \end{itemize}
        \item Integrating yields
        \begin{equation*}
            [\ce{A}] = ([\ce{A}]_0-[\ce{A}]_\text{eq})\e[-(k_1+k_{-1})t]+[\ce{A}]_\text{eq}
        \end{equation*}
        \begin{itemize}
            \item Note that this equation reduces to the irreversible first order equation as $k_{-1}\to 0$ and hence $[\ce{A}]_\text{eq}\to 0$ as well.
            \item Similarly, if only the reverse reaction takes place (and we have no initial concentration of \ce{B}), then $[\ce{A}]=[\ce{A}]_\text{eq}$ and the above equation reduces to exactly that statement, as desired.
        \end{itemize}
        \item Since
        \begin{equation*}
            \ln([\ce{A}]-[\ce{A}]_\text{eq}) = \ln([\ce{A}]_0-[\ce{A}]_\text{eq})-(k_1+k_{-1})t
        \end{equation*}
        we have a straight line that allows us to determine the sum $k_1+k_{-1}$. However, we cannot determine each term individually from the above.
        \item One way that we can is by noting that at equilibrium, $\dv*{[\ce{A}]}{t}=0$, so the differential rate law reduces to
        \begin{equation*}
            k_1[\ce{A}]_\text{eq} = k_{-1}[\ce{B}]_\text{eq}
        \end{equation*}
        \item Another way we can resolve each term individually is by noting that
        \begin{equation*}
            \frac{k_1}{k_{-1}} = \frac{[\ce{B}]_\text{eq}}{[\ce{A}]_\text{eq}}
            = K_c
        \end{equation*}
    \end{itemize}
    \item \textbf{Stopped flow method}: Fast mixing of reactants.
    \begin{itemize}
        \item The limit is about 1 microsecond time resolution (mixing rate).
        \item Lots of issues?
    \end{itemize}
    \item \textbf{Pump-probe method}: An optical/IR method that ranges from femtoseconds to nanoseconds.
    \begin{itemize}
        \item Nobel Prize (1999) to Zewail "for his studies of the transition states of chemical reactions using femtosecond spectroscopy."
    \end{itemize}
    \item \textbf{Perturbation-Relaxation method}: You perturb a thermodynamic variable (e.g., $T$, $P$, $\pH$, etc.) and then follow the kinetics of relaxation of the system to equilibrium.
    \begin{figure}[H]
        \centering
        \begin{tikzpicture}[scale=1.1]
            \small
            \draw (0,3.5) -- (0,0) -- node[below=4mm]{$t$} (5,0);
    
            \footnotesize
            \draw [very thin,dashed]
                (1.5,0) -- ++(0,2.5)
                (0,1) -- ++(5,0)
                (2.2,2.5) -- ++(0,-1.15)
            ;
            \node [below] at (1.5,0) {0};
            \node [left]  at (0,1)   {$[\ce{B}]_{2,\text{eq}}$};
            \node [left]  at (0,2.5) {$[\ce{B}]_{1,\text{eq}}$};
    
            \draw [blx,thick] (0,2.5)
                -- node[black,above,align=center,text width=1.5cm]{Initial equilibrium state} ++(1.5,0)
                % plot[domain=1.5:3.5,smooth] (\x,{2.5+1.53*(e^(-1*(2*(\x-1.5)))-1)})
                to[out=-71.9,in=180,looseness=1.2] (3.5,1)
                -- node[black,below,align=center,text width=1.5cm]{Final equilibrium state} ++(1.5,0)
            ;
    
            \scriptsize
            \begin{scope}[line width=0.3pt,<->,shorten <=1pt,shorten >=1pt]
                \draw (1,1) -- node[fill=white,inner sep=1.5pt]{$\Delta[\ce{B}]_0$} ++(0,1.5);
                \draw [shorten >=2pt] (1.85,1) -- ++(0,0.7);
                \node at (1.85,0.5) {$\Delta[\ce{B}]$}
                    edge[out=90,in=180,very thin,->,in looseness=2] (1.85,1.35)
                ;
                \draw (1.5,2.5) -- ++(0.7,0);
                \node [align=center] at (3.8,2.5) {\scriptsize Relaxation time\\$\tau=1/(k_1+k_{-1})$}
                    edge[out=180,in=90,very thin,->,in looseness=2] (1.85,2.5)
                ;
                \node at (3.5,3) {\scriptsize Temperature jump}
                    edge[out=180,in=90,very thin,->,in looseness=1.2] (1.5,2.5)
                ;
            \end{scope}
        \end{tikzpicture}
        \caption{Relaxation methods to determine rate constants.}
        \label{fig:relaxation}
    \end{figure}
    \begin{itemize}
        \item Nobel Prize (1967) to Porter, Norrish, and Eigen "for their studies of extremely fast chemical reactions, effected by disturbing the equilibrium by means of very short pulses of energy."
        \item Example: Consider water autoionization. Here, we'd perturb $\pH$ and $T$.
        \item Our initial point is the first equilibrium condition; out final point is the second equilibrium condition (i.e., that with the perturbed variables).
        \item We should have
        \begin{align*}
            [\ce{A}] &= [\ce{A}]_{2,\text{eq}}+\Delta[\ce{A}]&
            [\ce{B}] &= [\ce{B}]_{2,\text{eq}}+\Delta[\ce{B}]
        \end{align*}
        so that
        \begin{equation*}
            \dv{\Delta[\ce{B}]}{t} = k_1[\ce{A}]_{2,\text{eq}}+k_1\Delta[\ce{A}]-k_{-1}[\ce{B}]_{2,\text{eq}}-k_{-1}\Delta[\ce{B}]
        \end{equation*}
        \item The sum of the concentrations is constant, so $\Delta([\ce{A}]+[\ce{B}])=\Delta[\ce{A}]+\Delta[\ce{B}]=0$.
        \item Additionally, detailed balance is satisfied.
        \begin{equation*}
            k_1[\ce{A}]_{2,\text{eq}} = k_{-1}[\ce{B}]_{2,\text{eq}}
        \end{equation*}
        \item As a result,
        \begin{equation*}
            \dv{\Delta[\ce{B}]}{t} = -(k_1+k_{-1})\Delta[\ce{B}]
        \end{equation*}
        \item Integrating yields
        \begin{equation*}
            \Delta[\ce{B}]_0 = [\ce{B}]_{1,\text{eq}}-[\ce{B}]_{2,\text{eq}}
            = \Delta[\ce{B}]_0\e[-t/\tau]
        \end{equation*}
        where $\tau$ is the \textbf{relaxation time}.
        \item It follows that
        \begin{equation*}
            \Delta[\ce{B}] = \Delta[\ce{B}]_0\e[-(k_1+k_{-1})t]
        \end{equation*}
        \item Some textbooks use different notation; we should know this, too.
        \begin{itemize}
            \item They denote by $\xi$ or $\xi_0$ the difference between $[\ce{A}]$ (the initial equilibrium's concentration) and $[\ce{A}]_\text{eq}$ (the final equilibrium's concentration).
            \item They also use $k_A,k_B$ for the initial equilibrium \ce{A <=>[$k_A$][$k_B$] B} and $k_A^+,k_B^+$ for the final equilibrium \ce{A <=>[$k_A^+$][$k_B^+$] B}.
        \end{itemize}
    \end{itemize}
    \item \textbf{Relaxation time}: The following quantity. \emph{Denoted by} $\bm{\tau}$. \emph{Given by}
    \begin{equation*}
        \tau = \frac{1}{k_1+k_{-1}}
    \end{equation*}
    \item We'll start with water dissociation next lecture.
\end{itemize}



\section{Water Dissociation, Temperature Dependence, and TST}
\begin{itemize}
    \item \marginnote{4/15:}\textbf{T-jump}: A temperature perturbation.
    \item Relaxation methods and water dissociation.
    \begin{itemize}
        \item Consider the equilibrium
        \begin{equation*}
            \ce{H2O <=>[$k_f$][$k_r$] H+ + OH-}
        \end{equation*}
        \item The differential rate laws are
        \begin{align*}
            \dv{[\ce{H2O}]}{t} &= -k_f[\ce{H2O}]+k_r[\ce{H+}][\ce{OH-}]&
            \dv{[\ce{H+}]}{t} &= k_f[\ce{H2O}]-k_r[\ce{H+}][\ce{OH-}]
        \end{align*}
        \item After the T-jump, the system relaxes to a new equilibrium
        \begin{equation*}
            K_c = \frac{k_f^+}{k_r^+}
            = \frac{[\ce{H+}]_\text{eq}[\ce{OH-}]_\text{eq}}{[\ce{H2O}]_\text{eq}}
        \end{equation*}
        \item It follows that
        \begin{align*}
            \dv{\xi}{t} &= -k_f^+[\ce{H2O}]+k_r^+[\ce{H+}][\ce{OH-}]\\
            &= -k_f^+\xi-k_r^+\xi([\ce{H+}]_\text{eq}+[\ce{OH-}]_\text{eq})+\text{O}(\xi^2)
        \end{align*}
        \begin{itemize}
            \item Note that we get from the first line to the second by substituting $[\ce{H+}]=[\ce{H+}]_\text{eq}-\xi$ and $[\ce{OH-}]=[\ce{OH-}]_\text{eq}-\xi$ and expanding.
        \end{itemize}
        \item The associated relaxation time is
        \begin{equation*}
            \frac{1}{\tau} = k_f^++k_r^+([\ce{H+}]_\text{eq}+[\ce{OH-}]_\text{eq})
        \end{equation*}
        \begin{itemize}
            \item Note that this implies that this relaxation time can be measured experimentally.
        \end{itemize}
    \end{itemize}
    \item Rates of reaction depend on temperature.
    \item The empirical temperature dependence of the rate constant $k$ is given by
    \begin{equation*}
        \dv{\ln k}{T} = \frac{E_a}{RT^2}
    \end{equation*}
    \begin{itemize}
        \item If the activation energy is independent of temperature, then
        \begin{align*}
            \ln k &= \ln A-\frac{E_a}{RT}\\
            k &= A\e[-E_a/RT]
        \end{align*}
        i.e., we get the Arrhenius equation.
        \item If we obtain two rate constants at two temperatures, we can get
        \begin{equation*}
            \ln\frac{k_1}{k_2} = \frac{E_a}{R}\left( \frac{1}{T_2}-\frac{1}{T_1} \right)
        \end{equation*}
        \item Note that plots of $k$ vs. $1/T$ can be nonlinear if the prefactor or "encounter frequency" is temperature-dependent, i.e., if we have
        \begin{equation*}
            k = aT^m\e[-E'/RT]
        \end{equation*}
        where $a, E'$, and $m$ are temperature-independent constants.
    \end{itemize}
    \item Using Transition State Theory (TST) to estimate rate constants.
    \begin{itemize}
        \item Let the following be a chemical reaction and its rate law.
        \begin{align*}
            \ce{A + B} &\ce{->} \ce{P}&
            \dv{[\ce{P}]}{t} = k[\ce{A}][\ce{B}]
        \end{align*}
        \item Suppose that the reaction proceeds by way of a special intermediate species, the activated complex.
        \begin{equation*}
            \ce{A + B <=> AB${}^\ddagger$ -> P}
        \end{equation*}
        \item We know that
        \begin{equation*}
            K_c^\ddagger = \frac{[\ce{AB}^\ddagger]/c^\circ}{[\ce{A}]/c^\circ[\ce{B}]/c^\circ}
            = \frac{[\ce{AB}^\ddagger]c^\circ}{[\ce{A}][\ce{B}]}
        \end{equation*}
        where $c^\circ$ is the standard-state concentration.
        \item Write the equilibrium constant expression in terms of the partition functions $q_{\ce{A}}$, $q_{\ce{B}}$, and $q^\ddagger$ for \ce{A}, \ce{B}, and $\ce{AB}^\ddagger$.
        \begin{equation*}
            K_c^\ddagger = \frac{(q^\ddagger/V)c^\circ}{(q_{\ce{A}}/V)(q_{\ce{B}}/V)}
        \end{equation*}
        \item If $\nu_c$ is the frequency of crossing the barrier top, then
        \begin{equation*}
            \dv{[\ce{P}]}{t} = k[\ce{A}][\ce{B}]
            = \nu_c[\ce{AB}^\ddagger]
            = \nu_c\frac{[\ce{A}][\ce{B}]K_c^\ddagger}{c^\circ}
        \end{equation*}
        \item Thus, we can relate
        \begin{equation*}
            k = \frac{\nu_cK_c^\circ}{c^\circ}
        \end{equation*}
    \end{itemize}
    \item 2 hour midterm at the end of this month (April) taken at home.
\end{itemize}



\section{Office Hours (Tian)}
\begin{itemize}
    \item Stopped flow method?
    \begin{itemize}
        \item Two syringes have substances that get mixed and then become a homogeneous mixture where they start to do all of the interesting chemistry. Before the substances enter the chamber, though, they pass by a detector that monitors the concentration of the initial species. Concentration is measured after good mixing.
        \item It is called \emph{stopped} flow because we want to fix the initial concentration of \ce{A} and \ce{B}. Inject them, let them mix, stop the flow, measure the concentration, and then let the chemistry proceed.
        \item Only used if mixing is much faster than reaction.
        \item This is the experimental set-up for the method of initial rates or the method of exhaustion.
        \item Caveats/issues: Approximating $\Delta t$ as $\dd{t}$.
    \end{itemize}
    \item TST diagram lines?
    \begin{itemize}
        \item The quantized states lines refer to the energy levels of the reactants and products summarized by the partition function.
        \item The reactants reach the activated complex just at some higher quantized energy state!
    \end{itemize}
    \item Physical interpretation of $\tau$ beyond the time it takes the initial reactants to reach $1/\e$ of their initial concentration.
    \begin{itemize}
        \item You wanna see how fast the transition/relaxation would be, and $\tau$ is just a measure of how fast the transition goes.
        \item Also relates to $k_1$ and $k_{-1}$.
        \item Think in terms of adaptability (biological systems). Relation to how fast you can adapt to things like new temperature changes. We want to adapt to environmental changes as fast as possible.
        \item A measure of adaptability, response time, and smart materials that labs are developing to respond to changes very quickly. Also instrumentation response time (which you want to be very fast).
        \item Sometimes, you don't want to adapt to changes too quickly (such as cold-blooded animals).
    \end{itemize}
    \item Importance of Chapter 24 (or 26, depending on edition)?
    \begin{itemize}
        \item Good to know general stuff/big picture ideas as a prerequisite.
        \item Don't worry about specific things tho.
    \end{itemize}
    \item Pump-probe method?
    \begin{itemize}
        \item No further discussion of it in this chapter; Tian might talk about it more in later chapters, tho.
        \item Mostly for intra-molecular reactions, like accessing excited states and seeing how they decay.
        \item Optical pumping (form IChem I, PSet 8) is one way to do a pump-probe experiment.
    \end{itemize}
    \item Parallel reactions?
    \begin{itemize}
        \item Behave much the same kinetically as others; only difference is there is a yield.
    \end{itemize}
\end{itemize}



\section{Chapter 28: Chemical Kinetics I --- Rate Laws}
\emph{From \textcite{bib:McQuarrieSimon}.}
\begin{itemize}
    \item \marginnote{4/17:}\textcite{bib:McQuarrieSimon} derives the first-order integrated rate law.
    \item To determine the rate constant of a first-order reaction from concentration vs. time data, plot the log of the concentration vs. time and perform a linear regression.
    \item \textbf{Half-life}: The length of time required for half of the reactant to disappear. \emph{Denoted by} $\bm{t_{1/2}}$.
    \item \textcite{bib:McQuarrieSimon} derives the half-life of a first-order reaction.
    \item "A particular rate law does not provide any information on the magnitude of the rate constant" \parencite[1057]{bib:McQuarrieSimon}.
    \item \textcite{bib:McQuarrieSimon} derives the second-order integrated rate law.
    \item Similarly to first-order reactions, a plot of $1/[\ce{A}]$ vs. time will yield a straight line with slope $k$ and intercept $1/[\ce{A}]_0$.
    \item To determine if a reaction is first- or second-order, we can make plots of $\ln[\ce{A}]$ and $1/[\ce{A}]$ vs. $t$ and see which one is a straight line.
    \item \textcite{bib:McQuarrieSimon} derives the half-life of a second-order reaction.
    \item \textcite{bib:McQuarrieSimon} gives the integrated rate law for a reaction that is first-order in each reactant, and second-order overall.
    \item \textbf{Reversible} (reaction): A reaction that occurs in both directions.
    \item Consider the reversible reaction
    \begin{equation*}
        \ce{A <=>[$k_1$][$k_{-1}$] B}
    \end{equation*}
    \item The kinetic condition for the above reversible reaction to be at equilibrium is
    \begin{equation*}
        -\dv{[\ce{A}]}{t} = \dv{[\ce{B}]}{t} = 0
    \end{equation*}
    \item \textbf{Dynamic equilibrium}: An equilibrium in which individual molecules of reactants and products continue interconverting but in such a way that there is no \emph{net} change in either concentration.
    \item The rate law for the above reversible reaction is
    \begin{equation*}
        -\dv{[\ce{A}]}{t} = k_1[\ce{A}]-k_{-1}[\ce{B}]
    \end{equation*}
    \begin{itemize}
        \item The first term accounts for the rate at which \ce{A} reacts to form \ce{B}.
        \item The second term accounts for the rate at which \ce{B} reacts to form \ce{A}.
        \item "The difference in sign of these two terms reflects that the forward reaction depletes the concentration of \ce{A} and the back reaction increases the concentration of \ce{A} with time" \parencite[1063]{bib:McQuarrieSimon}.
    \end{itemize}
    \item \textcite{bib:McQuarrieSimon} derives the integrated rate law corresponding to the above differential rate law and the case that $[\ce{B}]=0$.
    \item \textbf{Relaxation method}: A method of determining the rate law for a chemical reaction with half-life shorter than the mixing time, involving perturbing a chemical system at one equilibrium to a state that will require a new equilibrium by suddenly changing a condition.
    \begin{itemize}
        \item Examples of conditions that can be changed are temperature, pressure, $\pH$, and $\pOH$.
    \end{itemize}
    \item \textbf{Temperature-jump relaxation technique}: A relaxation method in which the temperature of the equilibrium reaction mixture is suddenly changed at constant pressure.
    \begin{itemize}
        \item The change in temperature causes the chemical system to relax to a new equilibrium state that corresponds to the new temperature.
        \item The rate constants for the forward and reverse reactions are related to the time required for the system to relax to its new equilibrium state.
    \end{itemize}
    \item "Experimentally, the temperature of a solution can be increased by about \SI{5}{\kelvin} in one microsecond by discharging a high-voltage capacitor through the reaction solution" \parencite[1067]{bib:McQuarrieSimon}.
    \begin{itemize}
        \item Since equilibrium constants depend exponentially on the inverse of the temperature ($\ln K_P=-\Delta_rG^\circ/RT$), such a perturbation can cause a large change in equilibrium conditions.
    \end{itemize}
    \item As per the Van't Hoff equation, the equilibrium concentration of \ce{B} increases following the temperature jump if $\Delta_rH^\circ$ is positive and decreases if $\Delta_rH^\circ$ is negative.
    \begin{itemize}
        \item If $\Delta_rH^\circ=0$, then a temperature-jump relaxation experiment will not yield any useful data.
    \end{itemize}
    \item Temperature-jump relaxation technique rate law derivation.
    \begin{itemize}
        \item Suppose the initial temperature is $T_1$ and we increase the temperature to $T_2$. Suppose furthermore that $\Delta_rH^\circ<0$ so that the concentration of \ce{B} decreases following the perturbation.
        \item Let $[\ce{A}]_{1,\text{eq}},[\ce{B}]_{1,\text{eq}}$ be the equilibrium concentrations of \ce{A} and \ce{B}, respectively, at $T_1$. Let $[\ce{A}]_{2,\text{eq}},[\ce{B}]_{2,\text{eq}}$ be the equilibrium concentrations of \ce{A} and \ce{B}, respectively, at $T_2$. Let $k_1,k_{-1}$ be the rate constants of the forward and reverse reactions, respectively, at $T_2$. Let $[\ce{A}],[\ce{B}]$ be the concentrations of \ce{A} and \ce{B}, respectively, at some time $T$ after the temperature jump. Let $\Delta[\ce{A}],\Delta[\ce{B}]$ be the differences in the concentrations of \ce{A} and \ce{B}, respectively, from equilibrium at time $t$.
        \item As before, we begin from the fact that
        \begin{equation*}
            \dv{[\ce{B}]}{t} = k_1[\ce{A}]-k_{-1}[\ce{B}]
        \end{equation*}
        \begin{itemize}
            \item Notice how the sign convention follows from our hypothesis that $\Delta_rH^\circ<0$.
        \end{itemize}
        \item It follows by substituting
        \begin{align*}
            \Delta[\ce{A}] &= [\ce{A}]-[\ce{A}]_{2,\text{eq}}&
            \Delta[\ce{B}] &= [\ce{B}]-[\ce{B}]_{2,\text{eq}}
        \end{align*}
        into the above that
        \begin{align*}
            \dv{t}([\ce{B}]_{2,\text{eq}}+\Delta[\ce{B}]) &= k_1([\ce{A}]_{2,\text{eq}}+\Delta[\ce{A}])-k_{-1}([\ce{B}]_{2,\text{eq}}+\Delta[\ce{B}])\\
            \dv{\Delta[\ce{B}]}{t} &= k_1[\ce{A}]_{2,\text{eq}}+k_1\Delta[\ce{A}]-k_{-1}[\ce{B}]_{2,\text{eq}}-k_{-1}\Delta[\ce{B}]
        \end{align*}
        \item The fact that $k_1[\ce{A}]_{2,\text{eq}}=k_{-1}[\ce{B}]_{2,\text{eq}}$ gives us
        \begin{equation*}
            \dv{\Delta[\ce{B}]}{t} = k_1\Delta[\ce{A}]-k_{-1}\Delta[\ce{B}]
        \end{equation*}
        \item The fact that $\Delta[\ce{A}]+\Delta[\ce{B}]=0$ gives us
        \begin{equation*}
            \dv{\Delta[\ce{B}]}{t} = -(k_1+k_{-1})\Delta[\ce{B}]
        \end{equation*}
        \item If $\Delta[\ce{B}]_0=[\ce{B}]_{1,\text{eq}}-[\ce{B}]_{2,\text{eq}}$, then it follows by integration that
        \begin{align*}
            \int_{\Delta[\ce{B}]_0}^{\Delta[\ce{B}]}\frac{\dd{\Delta[\ce{B}]}}{\Delta[\ce{B}]} &= \int_0^t-(k_1+k_{-1})\dd{t}\\
            \Delta[\ce{B}] &= \Delta[\ce{B}]_0\e[-(k_1+k_{-1})t]
        \end{align*}
    \end{itemize}
    \item \textbf{Relaxation time}: The reciprocal of the sum of the forward and reverse rate constants. \emph{Denoted by} $\bm{\tau}$. \emph{Units} \textbf{s}. \emph{Given by}
    \begin{equation*}
        \tau = \frac{1}{k_1+k_{-1}}
    \end{equation*}
    \begin{itemize}
        \item A measure of how long it takes for $\Delta[\ce{B}]$ to decay to $1/\e$ of its initial value.
    \end{itemize}
    \item Temperature-jump relaxation technique rate law derivation for
    \begin{equation*}
        \ce{A + B <=>[$k_1$][$k_{-1}$] P}
    \end{equation*}
    \begin{itemize}
        \item We have that
        \begin{equation*}
            \dv{[\ce{P}]}{t} = k_1[\ce{A}][\ce{B}]-k_{-1}[\ce{P}]
        \end{equation*}
        \item If we define $\Delta[\ce{P}]$ and the related relevant terms as in the above derivation, then we get
        \begin{align*}
            \dv{t}([\ce{P}]_{2,\text{eq}}+\Delta[\ce{P}]) &= k_1([\ce{A}]_{2,\text{eq}}+\Delta[\ce{A}])([\ce{B}]_{2,\text{eq}}+\Delta[\ce{B}])-k_{-1}([\ce{P}]_{2,\text{eq}}+\Delta[\ce{P}])\\
            \dv{\Delta[\ce{P}]}{t} &= k_1([\ce{A}]_{2,\text{eq}}[\ce{B}]_{2,\text{eq}}+[\ce{A}]_{2,\text{eq}}\Delta[\ce{B}]+\Delta[\ce{A}][\ce{B}]_{2,\text{eq}}+\Delta[\ce{A}]\Delta[\ce{B}])-k_{-1}([\ce{P}]_{2,\text{eq}}+\Delta[\ce{P}])\\
            &= k_1([\ce{A}]_{2,\text{eq}}\Delta[\ce{B}]+\Delta[\ce{A}][\ce{B}]_{2,\text{eq}}+\Delta[\ce{A}]\Delta[\ce{B}])-k_{-1}\Delta[\ce{P}]\\
            &= k_1(-[\ce{A}]_{2,\text{eq}}\Delta[\ce{P}]-\Delta[\ce{P}][\ce{B}]_{2,\text{eq}}+\Delta[\ce{P}]^2)-k_{-1}\Delta[\ce{P}]\\
            &= -[k_1([\ce{A}]_{2,\text{eq}}+[\ce{B}]_{2,\text{eq}})+k_{-1}]\Delta[\ce{P}]+\text{O}(\Delta[\ce{P}]^2)\\
            \Delta[\ce{P}] &\approx \Delta[\ce{P}]_0\e[-t/\tau]
        \end{align*}
        where
        \begin{equation*}
            \tau = \frac{1}{k_1([\ce{A}]_{2,\text{eq}}+[\ce{B}]_{2,\text{eq}})+k_{-1}}
        \end{equation*}
    \end{itemize}
    \item Water dissociation, as per
    \begin{equation*}
        \ce{H+(aq) + OH-(aq) <=>[$k_1$][$k_{-1}$] H2O(l)}
    \end{equation*}
    \begin{itemize}
        \item Time-dependent conductivity measurements following a temperature jump in water paired with the known equilibrium constant and the above derivation revealed a relaxation time that corresponds to one of the fastest second-order rate constants ever measured.
    \end{itemize}
    \item Common temperature dependencies of chemical reactions.
    \begin{itemize}
        \item As the temperature increases, the rate of reaction increases exponentially.
        \item The temperature dependence is exponential until a threshold temperature, and then increases extremely rapidly (i.e., the substance becomes explosive).
        \item The temperature dependence increases up until a threshold temperature, and then falls off rapidly (i.e., an enzyme-controlled reaction where the enzyme denatures at a certain temperature).
    \end{itemize}
    \item In the first case (the most common), the temperature dependence of the rate constant is described approximately by the empirical equation
    \begin{equation*}
        \dv{\ln k}{T} = \frac{E_a}{RT^2}
    \end{equation*}
    \item Integrating yields the \textbf{Arrhenius equation}.
    \item \textbf{Arrhenius equation}: The following equation. \emph{Given by}
    \begin{equation*}
        k = A\e[-E_a/RT]
    \end{equation*}
    \item \textbf{Pre-exponential factor}: The constant $A$ in the Arrhenius equation. \emph{Denoted by} $\bm{A}$.
    \item \textbf{Activation energy}: The constant $E_a$ in the Arrhenius equation. \emph{Denoted by} $\bm{E_a}$.
    \item The magnitude of the temperature effect on reaction rates is much too large to be explained in terms of only a change in the translational energy of the reactions. Thus, a chemical reaction require more than just a collision between reactants.
    \item \textbf{Reaction coordinate}: The unit along which a chemical reaction proceeds from reactant to product.
    \begin{itemize}
        \item "Generally multidimensional, representing the bond lengths and bond angles associated with the chemical process" in question \parencite[1073]{bib:McQuarrieSimon}.
        \item A simple example, though, is that the \ce{I-I} bond length serves as the reaction coordinate for the thermal dissociation of \ce{I2}.
    \end{itemize}
    \item The Arrhenius equation is imperfect, and many reactions obey equations of the form
    \begin{equation*}
        k = aT^m\e[-E'/RT]
    \end{equation*}
    where $a$, $E'$, and $m$ are temperature-independent constants.
    \begin{itemize}
        \item Note that
        \begin{align*}
            E_a &= E'+mRT&
            A &= aT^m\e[m]
        \end{align*}
    \end{itemize}
\end{itemize}




\end{document}