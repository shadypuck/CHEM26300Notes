\documentclass[../notes.tex]{subfiles}

\pagestyle{main}
\renewcommand{\chaptermark}[1]{\markboth{\chaptername\ \thechapter\ (#1)}{}}
\setcounter{chapter}{3}

\begin{document}




\chapter{Reaction Mechanisms}
\section{TST and Overview of Mechanisms}
\begin{itemize}
    \item \marginnote{4/18:}More TST theory.
    \begin{itemize}
        \item We can intuitively rationalize concentration as $q/V$ by thinking of the partition function in some way giving some information about the number of particles. When we divide this by $V$, we thus get \emph{some} information about concentration.
        \item $\delta$ is the finite window in which the activated complex is loosely defined to exist.
        \item $\nu_c$ is the frequency of crossing the $\delta$ region in the energy diagram.
        \item We substitute $[\ce{AB}^\ddagger]=[\ce{A}][\ce{B}]K_c^\ddagger/c^\circ$ from the equilibrium constant expression.
        \item The translational partition function plays a role. Note that $m^\ddagger$ is the mass of the activated complex.
        \begin{equation*}
            q_\text{trans} = \frac{\sqrt{2\pi m^\ddagger\kB T}}{h}\delta
        \end{equation*}
        \begin{itemize}
            \item Dimensional analysis (as an analysis of units to help appreciate the term): $\kB T$ is energy ($E\propto mv^2/2$) and $m^\ddagger$ is mass, so their product is momentum squared ($p^2\propto mv^2$).
            \item Thus, the top of the expression gives momentum, and $h/\delta$ also gives momentum.
            \item Remember that the momentum is rather like that of the particle in a box.
        \end{itemize}
        \item We have that
        \begin{equation*}
            q^\ddagger = q_\text{trans}^\ddagger\cdot q_\text{int}^\ddagger
        \end{equation*}
        where $q^\ddagger$ is the partition function for the whole species, which we can split.
        \item This permits
        \begin{align*}
            K_c^\ddagger &= \frac{(q^\ddagger/V)c^\circ}{(q_{\ce{A}}/V)(q_{\ce{B}}/V)}\\
            &= \frac{\sqrt{2\pi m^\ddagger\kB T}}{h}\delta\frac{(q_\text{int}^\ddagger/V)c^\circ}{(q_{\ce{A}}/V)(q_{\ce{B}}/V)}
        \end{align*}
        and the following
        \begin{equation*}
            k = \nu_c\frac{\sqrt{2\pi m^\ddagger\kB T}}{hc^\circ}\delta\frac{(q_\text{int}^\ddagger/V)c^\circ}{(q_{\ce{A}}/V)(q_{\ce{B}}/V)}
        \end{equation*}
        \item We now define the speed of the activated complex crossing the barrier top (gas collision). This is $\prb{u_\text{ac}}$, where $\prb{u_\text{ac}}=\nu_c\delta$ (the frequency $x$ distance).
        \item We have that
        \begin{equation*}
            \prb{u_\text{ac}} = \int_0^\infty uf(u)\dd{u}
            = \sqrt{\frac{m^\ddagger}{2\pi\kB T}}\int_0^\infty u\e[-m^\ddagger u^2/2\kB T]\dd{u}
            = \sqrt{\frac{\kB T}{2\pi m^\ddagger}}
        \end{equation*}
        and the following
        \begin{equation*}
            k = \frac{\kB T}{hc^\circ}\frac{(q_\text{int}^\ddagger/V)c^\circ}{(q_{\ce{A}}/V)(q_{\ce{B}}/V)}
            = \frac{\kB T}{hc^\circ}K^\ddagger
        \end{equation*}
        \begin{itemize}
            \item We start at zero instead of $-\infty$ because we're only considering the forward direction of the reaction. If we did $-\infty$, we'd be accounting for the reverse reaction as well.
        \end{itemize}
        \item Introducing $\Delta^\ddagger G^\circ=-RT\ln K^\ddagger=\Delta^\ddagger H^\circ-T\Delta^\ddagger S^\circ$ yields the \textbf{Eyring equation}
        \begin{equation*}
            k(T) = \frac{\kB T}{hc^\circ}\e[-\Delta^\ddagger G^\circ/RT]
            = \frac{\kB T}{hc^\circ}\e[\Delta^\ddagger S^\circ/R]\e[-\Delta^\ddagger H^\circ/RT]
        \end{equation*}
        \item We can relate the Eyring equation to the Arrhenius equation by differentiating the logarithm.
        \begin{align*}
            \dv{\ln k}{T} &= \frac{1}{T}+\dv{\ln K^\ddagger}{T}&
            \dv{\ln K_c}{T} &= \frac{\Delta U^\circ}{RT^2}
        \end{align*}
        gives
        \begin{equation*}
            \dv{\ln k}{T} = \frac{1}{T}+\frac{\Delta^\ddagger U^\circ}{RT^2}
        \end{equation*}
        \item This all serves to relate thermodynamics and kinetics.
        \item Additionally, since $\Delta^\ddagger H^\circ=\Delta^\ddagger U^\circ+\Delta^\ddagger PV=\Delta^\ddagger U^\circ+\Delta^\ddagger nRT$, we have that
        \begin{equation*}
            \dv{\ln k}{T} = \frac{\Delta^\ddagger H^\circ+2RT}{RT^2}
        \end{equation*}
        \begin{itemize}
            \item Note that $\Delta n=-1$ for bimolecular gas phase reactions.
            \item $\Delta(PV)^\ddagger$ is the difference in the number of moles of gaseous products and reactants.
            \item $\Delta n^\ddagger=0$: Unimolecular.
            \item $\Delta n^\ddagger=-1$: Bimolecular.
            \item $\Delta n^\ddagger=-2$: Trimolecular.
        \end{itemize}
        \item Comparing with the Arrhenius $\dv*{\ln k}{T}=E_a/RT^2$ gives
        \begin{equation*}
            E_a = \Delta^\ddagger H^\circ+2RT
        \end{equation*}
        \item Substituting into the Eyring equation yields
        \begin{equation*}
            k(T) = \frac{\e[2]\kB T}{hc^\circ}\e[\Delta^\ddagger S^\circ/R]\e[-E_a/RT]
        \end{equation*}
        \item Gas, uni: $E_a=\Delta H^\ddagger+RT$, $A=\e\kB T/h\cdot\e[\Delta S^\ddagger/R]$.
        \item Gas, bi: $E_a=\Delta H^\ddagger+2RT$, $A=\e[2]\kB T/hc^\circ\cdot\e[\Delta S^\ddagger/R]$.
        \item Gas, tri: $E_a=\Delta H^\ddagger+3RT$, $A=\e[3]\kB T/h(c^\circ)^2\cdot\e[\Delta S^\ddagger/R]$.
    \end{itemize}
    \item Important things to memorize from TST.
    \begin{itemize}
        \item The Eyring equation.
        \item Don't worry about the partition function mathematics, but understand the dimensional analysis.
        \begin{itemize}
            \item Know that we collapse two $\kB T$ terms together; one from $q_\text{trans}$ and one from $\prb{u_\text{ac}}$.
        \end{itemize}
    \end{itemize}
    \item Consider cyclohexane conformations.
    \begin{itemize}
        \item We have that
        \begin{align*}
            \Delta^\ddagger H^\circ &= \SI{31.38}{\kilo\joule\per\mole}&
            \Delta^\ddagger S^\circ &= \SI{16.74}{\joule\per\mole\per\kelvin}&
            T &= \SI{325}{\kelvin}
        \end{align*}
        \item We want to know $\Delta^\ddagger G^\circ$.
        \item But by the definition of the Gibbs energy,
        \begin{equation*}
            \Delta^\ddagger G^\circ = \Delta^\ddagger H^\circ-T\Delta^\ddagger S^\circ
            = \SI{25.94}{\kilo\joule\per\mole}
        \end{equation*}
        \item It follows that
        \begin{equation*}
            k = \frac{\kB T}{h}\e[-\Delta^\ddagger G^\circ/RT]
            = \SI{4.59e8}{\per\second}
        \end{equation*}
    \end{itemize}
    \item Overview of key mechanism concepts.
    \begin{itemize}
        \item Reaction mechanisms can involve more than one elementary step.
        \item Reactions can be sequential (single- or multi-step).
        \item To establish a mechanism, we use several techniques, approaches, assumptions, and approximations.
        \item Establish rate determining steps: The rate law and rate constants associated with these steps tend to dominate the kinetics of the overall reaction.
        \item Invoke the steady-state approximation to help solve the complicated mathematics of reaction kinetics.
        \item Enzyme kinetics, Michaelis-Menten mechanism involves an SS approximation.
    \end{itemize}
    \item Oftentimes, reactions are of the form
    \begin{align*}
        \ce{E + S <=>[$k_1$][$k_{-1}$] ->[$k_r$] P + E}
    \end{align*}
    \begin{itemize}
        \item Note that this form is very much analogous to the form analyzed in TST.
    \end{itemize}
    \item \textbf{Elementary reaction}: A reaction that does not involve the formation of a reaction intermediate; the products must be formed directly from the reactants.
    \begin{itemize}
        \item Denoted by the double arrow.
        \item An elementary reaction can still be reversible.
    \end{itemize}
    \item \textbf{Molecularity} (of an elementary reaction): The number of reactant molecules involved in the chemical reaction.
    \item \textbf{Unimolecular} (reaction): An elementary reaction with molecularity one. \emph{General form}
    \begin{equation*}
        \ce{A} \Longrightarrow \text{products}
    \end{equation*}
    \emph{Rate law}
    \begin{equation*}
        v = k[\ce{A}]
    \end{equation*}
    \item \textbf{Bimolecular} (reaction): An elementary reaction with molecularity two. \emph{General form}
    \begin{equation*}
        \ce{A}+\ce{B} \Longrightarrow \text{products}
    \end{equation*}
    \emph{Rate law}
    \begin{equation*}
        v = k[\ce{A}][\ce{B}]
    \end{equation*}
    \item \textbf{Termolecular} (reaction): An elementary reaction with molecularity three. \emph{General form}
    \begin{equation*}
        \ce{A}+\ce{B}+\ce{C} \Longrightarrow \text{products}
    \end{equation*}
    \emph{Rate law}
    \begin{equation*}
        v = k[\ce{A}][\ce{B}][\ce{C}]
    \end{equation*}
    \item No elementary reaction with molecularity greater than three is known, and the overwhelming majority of elementary reactions are bimolecular.
    \item When a complex reaction is at equilibrium, the rate of the forward process is equal to the rate of the reverse process for each and every step of the reaction mechanism.
    \begin{itemize}
        \item We denote a reversible elementary reaction as follows.
        \begin{equation*}
            \ce{A + B} \Longleftrightarrows[k_1][k_{-1}] \ce{C + D}
        \end{equation*}
        \item A reversible elementary reaction signifies that the reaction occurs in both the forward and reverse directions to a significant extent and that the reaction in each direction is an elementary reaction.
        \item The rate laws are
        \begin{align*}
            v_1 &= k_1[\ce{A}][\ce{B}]&
            v_{-1} &= k_{-1}[\ce{C}][\ce{D}]
        \end{align*}
        \item At equilibrium,
        \begin{align*}
            k_1[\ce{A}]_\text{eq}[\ce{B}]_\text{eq} &= k_{-1}[\ce{C}]_\text{eq}[\ce{D}]_\text{eq}\\
            \frac{k_1}{k_{-1}} &= \frac{[\ce{C}]_\text{eq}[\ce{D}]_\text{eq}}{[\ce{A}]_\text{eq}[\ce{B}]_\text{eq}} = K_c
        \end{align*}
    \end{itemize}
    \item \textbf{Principle of detailed balance}: The following relationship, which holds for all reversible elementary reactions. \emph{Given by}
    \begin{equation*}
        K_c = \frac{k_1}{k_{-1}}
    \end{equation*}
\end{itemize}



\section{The Two-Step Consecutive Reaction Mechanism}
\begin{itemize}
    \item \marginnote{4/20:}Consider the general complex reaction
    \begin{equation*}
        \ce{A ->[$k_\text{obs}$] P}
    \end{equation*}
    \begin{itemize}
        \item Suppose that the reaction occurs by the two step mechanism
        \begin{align*}
            \ce{A} &\stackrel{k_1}{\Longrightarrow} \ce{I}&
            \ce{I} &\stackrel{k_2}{\Longrightarrow} \ce{P}
        \end{align*}
        \item Because each step of this mechanism is an elementary reaction, the rate laws for each species are
        \begin{align*}
            \dv{[\ce{A}]}{t} &= -k_1[\ce{A}]&
            \dv{[\ce{I}]}{t} &= k_1[\ce{A}]-k_2[\ce{I}]&
            \dv{[\ce{P}]}{t} &= k_2[\ce{I}]
        \end{align*}
        \item Thus, assuming that the initial concentrations at time $t=0$ are $[\ce{A}]=[\ce{A}]_0$ and $[\ce{I}]_0=[\ce{P}]_0=0$, we have that
        \begin{gather*}
            [\ce{A}] = [\ce{A}]_0\e[-k_1t]\\
            [\ce{I}] = \frac{k_1[\ce{A}]_0}{k_2-k_1}(\e[-k_1t]-\e[-k_2t])\\
            [\ce{P}] = [\ce{A}]_0-[\ce{A}]-[\ce{I}]
                = [\ce{A}]_0\left\{ 1+\frac{1}{k_1-k_2}(k_2\e[-k_1t]-k_1\e[-k_2t]) \right\}
        \end{gather*}
    \end{itemize}
    \item Distinguishing the two-step consecutive reaction mechanism unambiguously from the one-step reaction.
    \begin{itemize}
        \item For a single step reaction,
        \begin{equation*}
            [\ce{P}] = [\ce{A}]_0(1-\e[-k_1t])
        \end{equation*}
        \item The two-step consecutive reaction mechanism has the following alternate form.
        \begin{equation*}
            [\ce{P}] = [\ce{A}]_0\left\{ 1+\frac{1}{k_1-k_2}(k_2\e[-k_1t]-k_1\e[-k_2t]) \right\}
        \end{equation*}
        \item However, if $k_2\gg k_1$, then
        \begin{align*}
            [\ce{P}] &= [\ce{A}]_0\left\{ 1+\frac{1}{k_1-k_2}(k_2\e[-k_1t]-k_1\e[-k_2t]) \right\}\\
            &\approx [\ce{A}]_0\left\{ 1+\frac{1}{-k_2}k_2\e[-k_1t] \right\}\\
            &= [\ce{A}]_0(1-\e[-k_1t])
        \end{align*}
        \item If $k_1\gg k_2$, he reaction reduces to
        \begin{equation*}
            [\ce{P}] \approx [\ce{A}]_0(1-\e[-k_2t])
        \end{equation*}
        \item Thus, the only ambiguous situation is $k_2\gg k_1$.
    \end{itemize}
    \item The steady-state approximation simplifies rate expressions.
    \begin{itemize}
        \item We assume that $\dv*{[\ce{I}]}{t}=0$, where \ce{I} is a reaction intermediate.
        \item Given the above differential equation for $\dv*{[\ce{I}]}{t}$, making the above assumption yields
        \begin{equation*}
            [\ce{I}]_\text{SS} = \frac{k_1[\ce{A}]}{k_2}
        \end{equation*}
        \item It follows that
        \begin{equation*}
            [\ce{I}]_\text{SS} = \frac{k_1}{k_2}[\ce{A}]_0\e[-k_1t]
        \end{equation*}
        \item Thus,
        \begin{equation*}
            \dv{[\ce{I}]_\text{SS}}{t} = \frac{-k_1^2}{k_2}[\ce{A}]_0\e[-k_1t]
        \end{equation*}
        \item We get $k_2\gg k_1^2[\ce{A}]_0$ and $[\ce{P}]=[\ce{A}]_0(1-\e[-k_1t])$.
    \end{itemize}
    \item Example: Decomposition of ozone.
    \begin{equation*}
        \ce{2O3(g) -> 3O2(g)}
    \end{equation*}
    \begin{itemize}
        \item The reaction mechanism is
        \begin{align*}
            \ce{M(g) + O3(g)} &\Longleftrightarrows[k_1][k_{-1}] \ce{O2(g) + O(g) + M(g)}\\
            \ce{O(g) + O3(g)} &\xRightarrow{k_2} \ce{2O2(g)}
        \end{align*}
        where \ce{M} is a molecule that can exchange energy with the reacting ozone molecule through a collision, but \ce{M} itself does not react.
        \item The rate equations for \ce{O3(g)} and \ce{O(g)} are
        \begin{gather*}
            \dv{[\ce{O3}]}{t} = -k_1[\ce{O3}][\ce{M}]+k_{-1}[\ce{O2}][\ce{O}][\ce{M}]-k_2[\ce{O}][\ce{O3}]\\
            \dv{[\ce{O}]}{t} = k_1[\ce{O3}][\ce{M}]-k_{-1}[\ce{O2}][\ce{O}][\ce{M}]-k_2[\ce{O}][\ce{O3}]
        \end{gather*}
        \item Invoking the steady-state approximation for the intermediate \ce{O} yields
        \begin{equation*}
            [\ce{O}] = \frac{k_1[\ce{O3}][\ce{M}]}{k_{-1}[\ce{O2}][\ce{M}]+k_2[\ce{O3}]}
        \end{equation*}
        \item Substituting this result into the rate equation for \ce{O3} gives
        \begin{equation*}
            \dv{[\ce{O3}]}{t} = -\frac{2k_1k_2[\ce{O3}]^2[\ce{M}]}{k_{-1}[\ce{O2}][\ce{M}]+k_2[\ce{O3}]}
        \end{equation*}
    \end{itemize}
\end{itemize}



\section{Complex Reactions}
\begin{itemize}
    \item \marginnote{4/22:}Expect the midterm to be 2 hours in length, available all next week, and to incorporate largely HW-like questions but also some open-ended, design-an-experiment questions. Completely open note.
    \item The rate law for a complex reaction does not imply a unique mechanism.
    \begin{itemize}
        \item Consider the reaction
        \begin{equation*}
            \ce{2NO(g) + O2(g) ->[$k_\text{obs}$] 2NO2(g)}
        \end{equation*}
        \item The rate law is
        \begin{equation*}
            \frac{1}{2}\dv{[\ce{NO2}]}{t} = k_\text{obs}[\ce{NO}]^2[\ce{O2}]
        \end{equation*}
        \item Experimental studies confirm that the reaction is not an elementary reaction, but we can propose multiple mechanisms that would both yield the same rate law. Here are two examples.
        \begin{itemize}
            \item Mechanism 1.
            \begin{align*}
                \ce{NO(g) + O2(g)} &\Longleftrightarrows[k_1][k_{-1}] \ce{NO3(g)}\\
                \ce{NO3(g) + NO(g)} &\ce{->[$k_2$]} \ce{2 NO2(g)}
            \end{align*}
            \item Mechanism 2.
            \begin{align*}
                \ce{2NO(g)} &\Longleftrightarrows[k_1][k_{-1}] \ce{N2O2(g)}\\
                \ce{N2O2(g) + O2(g)} &\ce{->[$k_2$]} \ce{2 NO2(g)}
            \end{align*}
        \end{itemize}
        \item One experiment to design is to capture or otherwise detect the intermediate species.
        \item Through such an experiment, we can verify Mechanism 2.
    \end{itemize}
    \item The Lindemann Mechanism explains how unimolecular reactions occur.
    \begin{itemize}
        \item Consider the reaction
        \begin{equation*}
            \ce{CH3NC(g) ->[$k_\text{obs}$] CH3CN(g)}
        \end{equation*}
        \item The following rate law is only correct at $[\ce{CH3NC}]$.
        \begin{equation*}
            \dv{[\ce{CH3NC}]}{t} = -k_\text{obs}[\ce{CH3NC}]
        \end{equation*}
        \item At low $[\ce{CH3NC}]$, we have
        \begin{equation*}
            \dv{[\ce{CH3NC}]}{t} = -k_\text{obs}[\ce{CH3NC}]^2
        \end{equation*}
        which is not the rate law for a unimolecular reaction.
        \item The Lindemann mechanism for unimolecular reactions of the form \ce{A(g) -> B(g)} is
        \begin{align*}
            \ce{A(g) + M(g)} &\Longleftrightarrows[k_1][k_{-1}] \ce{A(g)^* + M(g)}\\
            \ce{A(g)^*} &\ce{->[$k_2$]} \ce{B(g)}
        \end{align*}
        \item The symbol \ce{A(g)^*} represents an energized reactant molecule. \ce{M(g)} is the collision partner.
        \item By the steady-state approximation, we have that
        \begin{align*}
            \dv{[\ce{A^*}]}{t} = 0 &= k_1[\ce{A}][\ce{M}]-k_{-1}[\ce{A^*}][\ce{M}]-k_2[\ce{A^*}]\\
            [\ce{A^*}] &= \frac{k_1[\ce{M}][\ce{A}]}{k_2+k_{-1}[\ce{M}]}
        \end{align*}
        so that
        \begin{align*}
            \dv{[\ce{B}]}{t} &= k_2[\ce{A^*}]\\
            -\dv{[\ce{A}]}{t} = \dv{[\ce{B}]}{t} &= \underbrace{\frac{k_2k_1[\ce{M}]}{k_2+k_{-1}[\ce{M}]}}_{k_\text{obs}}[\ce{A}]
        \end{align*}
        \item At high $[\ce{M}]$, we have that $k_{-1}[\ce{M}][\ce{A^*}]\gg k_2[\ce{A^*}]$, or $k_{-1}[\ce{M}]\gg k_2$. Thus,
        \begin{equation*}
            k_\text{obs} = \frac{k_1k_2}{k_{-1}}
        \end{equation*}
        \item At low $[\ce{M}]$, we have that $k_2\gg k_{-1}[\ce{M}]$ so that
        \begin{align*}
            \dv{[\ce{B}]}{t} &= k_1[\ce{M}][\ce{A}]\\
            &= k_1[\ce{A}]^2
        \end{align*}
        \item This mechanism was proposed by the British chemists J. A. Christiansen in 1921 and F. A. Lindemann in 1922. Their work underlies the current theory of unimolecular reaction rates.
    \end{itemize}
    \item Some reaction mechanisms involve chain reactions.
    \begin{itemize}
        \item Chain reactions involve amplification.
        \item For example, \ce{H2(g) + Br2(g) <=> 2HBr(g)} follows the ensuing mechanism.
        \begin{itemize}
            \item Initiation.
            \begin{equation*}
                \ce{Br2 + M(g) ->[k_1] 2Br(g) + M(g)}
            \end{equation*}
            \item Propagation.
            \begin{align*}
                \ce{Br(g) + H2(g)} &\ce{->[k_2]} \ce{HBr(g) + H(g)}\\
                \ce{H(g) + Br2(g)} &\ce{->[k_3]} \ce{HBr(g) + Br(g)}
            \end{align*}
            \item Inhibition.
            \begin{align*}
                \ce{HBr(g) + H(g)} &\ce{->[k_{-2}]} \ce{Br(g) + H2(g)}\\
                \ce{HBr(g) + Br(g)} &\ce{->[k_{-3}]} \ce{H(g) + Br2(g)}
            \end{align*}
            \item Termination.
            \begin{equation*}
                \ce{2Br + M(g) ->[k_{-1}] Br2(g) + M(g)}
            \end{equation*}
        \end{itemize}
        \item The fifth step can be ignored.
        \item Notice that the inhibition and termination reactions are the reverse reactions of the propagation and initiation reaction(s), respectively.
        \begin{itemize}
            \item Termination does not need to be the reverse of initiation, though. Termination just kills any reactive species.
            \item Inhibition is the reverse of propagation, though.
        \end{itemize}
        \item When you want to design a chain reaction species, make sure you have a reactive species (like bromine) for the initiation step. Notice, for instance, that hydrogen does not initiate.
        \item This leads to the experimentally determined rate law
        \begin{equation*}
            \frac{1}{2}\dv{[\ce{HBr}]}{t} = \frac{k[\ce{H2}][\ce{Br2}]^{1/2}}{1+k'[\ce{HBr}][\ce{Br2}]^{-1}}
        \end{equation*}
        \item Deriving said rate law.
        \begin{itemize}
            \item We have that
            \begin{align*}
                \dv{[\ce{HBr}]}{t} &= k_2[\ce{Br}][\ce{H2}]-k_{-2}[\ce{HBr}][\ce{H}]+k_3[\ce{H}][\ce{Br2}]\\
                \dv{[\ce{H}]}{t} &= k_2[\ce{Br}][\ce{H2}]-k_{-2}[\ce{HBr}][\ce{H}]-k_3[\ce{H}][\ce{Br2}]\\
                \dv{[\ce{Br}]}{t} &= 2k_1[\ce{Br2}][\ce{M}]-k_{-1}[\ce{Br}]^2[\ce{M}]-k_2[\ce{Br}][\ce{H2}]+k_{-2}[\ce{HBr}][\ce{H}]+k_3[\ce{H}][\ce{Br2}]
            \end{align*}
            \item We can apply the SS approximation to the second and third equations above, which both describe intermediate species.
            \begin{align*}
                0 &= k_2[\ce{Br}][\ce{H2}]-k_{-2}[\ce{HBr}][\ce{H}]-k_3[\ce{H}][\ce{Br2}]\\
                0 &= 2k_1[\ce{Br2}][\ce{M}]-k_{-1}[\ce{Br}]^2[\ce{M}]-k_2[\ce{Br}][\ce{H2}]+k_{-2}[\ce{HBr}][\ce{H}]+k_3[\ce{H}][\ce{Br2}]
            \end{align*}
            \item Solving the two equations above for $[\ce{H}]$ and $[\ce{Br}]$, respectively, is made substantially easier by noting that the negative of the first expression appears in its entirety in the second expression. Thus, we may simply substitute the former into the latter and solve to find an expression for $[\ce{Br}]$.
            \begin{align*}
                0 &= 2k_1[\ce{Br2}][\ce{M}]-k_{-1}[\ce{Br}]^2[\ce{M}]-0\\
                [\ce{Br}] &= \left( \frac{k_1}{k_{-1}} \right)^{1/2}[\ce{Br2}]^{1/2}\\
                [\ce{Br}] &= (K_{c,1})^{1/2}[\ce{Br2}]^{1/2}
            \end{align*}
            \item Resubstituting yields an expression for $[\ce{H}]$.
            \begin{align*}
                0 &= k_2[\ce{Br}][\ce{H2}]-k_{-2}[\ce{HBr}][\ce{H}]-k_3[\ce{H}][\ce{Br2}]\\
                0 &= k_2(K_{c,1})^{1/2}[\ce{Br2}]^{1/2}[\ce{H2}]-(k_{-2}[\ce{HBr}]+k_3[\ce{Br2}])[\ce{H}]\\
                [\ce{H}] &= \frac{k_2(K_{c,1})^{1/2}[\ce{Br2}]^{1/2}[\ce{H2}]}{k_{-2}[\ce{HBr}]+k_3[\ce{Br2}]}
            \end{align*}
            \item Substituting these two expressions back into the original differential equation for $[\ce{HBr}]$ yields
            \begin{align*}
                \dv{[\ce{HBr}]}{t} ={}& k_2[\ce{Br}][\ce{H2}]-k_{-2}[\ce{HBr}][\ce{H}]+k_3[\ce{H}][\ce{Br2}]\\
                \begin{split}
                    ={}& k_2(K_{c,1})^{1/2}[\ce{Br2}]^{1/2}[\ce{H2}]-k_{-2}[\ce{HBr}]\cdot\frac{k_2(K_{c,1})^{1/2}[\ce{Br2}]^{1/2}[\ce{H2}]}{k_{-2}[\ce{HBr}]+k_3[\ce{Br2}]}\\
                    &+k_3\cdot\frac{k_2(K_{c,1})^{1/2}[\ce{Br2}]^{1/2}[\ce{H2}]}{k_{-2}[\ce{HBr}]+k_3[\ce{Br2}]}\cdot[\ce{Br2}]
                \end{split}\\
                \begin{split}
                    ={}& k_2(K_{c,1})^{1/2}[\ce{Br2}]^{1/2}[\ce{H2}]-\frac{k_2k_{-2}(K_{c,1})^{1/2}[\ce{HBr}][\ce{Br2}]^{1/2}[\ce{H2}]}{k_{-2}[\ce{HBr}]+k_3[\ce{Br2}]}\\
                    &+\frac{k_2k_3(K_{c,1})^{1/2}[\ce{Br2}]^{3/2}[\ce{H2}]}{k_{-2}[\ce{HBr}]+k_3[\ce{Br2}]}
                \end{split}\\
                ={}& k_2(K_{c,1})^{1/2}[\ce{Br2}]^{1/2}[\ce{H2}]\left( 1-\frac{k_{-2}[\ce{HBr}]}{k_{-2}[\ce{HBr}]+k_3[\ce{Br2}]}+\frac{k_3[\ce{Br2}]}{k_{-2}[\ce{HBr}]+k_3[\ce{Br2}]} \right)\\
                ={}& k_2(K_{c,1})^{1/2}[\ce{Br2}]^{1/2}[\ce{H2}]\left( \frac{k_{-2}[\ce{HBr}]+k_3[\ce{Br2}]}{k_{-2}[\ce{HBr}]+k_3[\ce{Br2}]}-\frac{k_{-2}[\ce{HBr}]-k_3[\ce{Br2}]}{k_{-2}[\ce{HBr}]+k_3[\ce{Br2}]} \right)\\
                ={}& k_2(K_{c,1})^{1/2}[\ce{Br2}]^{1/2}[\ce{H2}]\cdot\frac{2k_3[\ce{Br2}]}{k_{-2}[\ce{HBr}]+k_3[\ce{Br2}]}\\
                \frac{1}{2}\dv{[\ce{HBr}]}{t} &= k_2(K_{c,1})^{1/2}[\ce{Br2}]^{1/2}[\ce{H2}]\cdot\frac{1}{(k_{-2}/k_3)[\ce{HBr}][\ce{Br2}]^{-1}+1}\\
                ={}& \frac{k_2(K_{c,1})^{1/2}[\ce{H2}][\ce{Br2}]^{1/2}}{1+(k_{-2}/k_3)[\ce{HBr}][\ce{Br2}]^{-1}}\\
                ={}& \frac{k[\ce{H2}][\ce{Br2}]^{1/2}}{1+k'[\ce{HBr}][\ce{Br2}]^{-1}}
            \end{align*}
            where we have substituted $k=k_2(K_{c,1})^{1/2}$ and $k'=k_{-2}/k_3$ in the last expression.
        \end{itemize}
    \end{itemize}
    \item Problem 29-24.
    \begin{itemize}
        \item The reaction
        \begin{equation*}
            \ce{CH3CHO(g) ->[$k_\text{obs}$] CH4(g) + CO(g)}
        \end{equation*}
        proceeds by the mechanism
        \begin{align*}
            \ce{CH3CHO(g)} &\ce{->[$k_1$]} \ce{CH3(g) + CHO(g)}\\
            \ce{CH3(g) + CH3CHO(g)} &\ce{->[$k_2$]} \ce{CH4(g) + CH3CO(g)}\\
            \ce{CH3CO(g)} &\ce{->[$k_3$]} \ce{CH3(g) + CO(g)}\\
            \ce{2CH3(g)} &\ce{->[$k_4$]} \ce{C2H6(g)}
        \end{align*}
        \item The initiation step is the first equation, the propagation steps are the second and third equations, and the termination step is the fourth equation.
        \item We can write the rate laws
        \begin{align*}
            \dv{[\ce{CH4}]}{t} &= k_2[\ce{CH3}][\ce{CH3CHO}]\\
            \dv{[\ce{CH3}]}{t} &= k_1[\ce{CH3CHO}]-k_2[\ce{CH3}][\ce{CH3CHO}]+k_3[\ce{CH3CO}]-2k_4[\ce{CH3}]\\
            \dv{[\ce{CH3CO}]}{t} &= k_2[\ce{CH3}][\ce{CH3CHO}]-k_3[\ce{CH3CO}]
        \end{align*}
        \item Applying the SS approximation to the last second and third equations yields (respectively)
        \begin{align*}
            [\ce{CH3}] &= \frac{k_1[\ce{CH3CHO}]+k_3[\ce{CH3CO}]}{k_2[\ce{CH3CHO}]+2k_4}&
            [\ce{CH3CO}] &= \frac{k_2}{k_3}[\ce{CH3}][\ce{CH3CHO}]
        \end{align*}
        \item Substituting the right equation above into the left equation above and solving for $[\ce{CH3}]$ yields an expression for $[\ce{CH3}]$ purely in terms of $[\ce{CH3CHO}]$.
        \begin{align*}
            [\ce{CH3}] &= \frac{k_1[\ce{CH3CHO}]+k_2[\ce{CH3}][\ce{CH3CHO}]}{k_2[\ce{CH3CHO}]+2k_4}\\
            k_2[\ce{CH3}][\ce{CH3CHO}]+2k_4[\ce{CH3}] &= k_1[\ce{CH3CHO}]+k_2[\ce{CH3}][\ce{CH3CHO}]\\
            2k_4[\ce{CH3}] &= k_1[\ce{CH3CHO}]\\
            [\ce{CH3}] &= \frac{k_1}{2k_4}[\ce{CH3CHO}]
        \end{align*}
        \item The final result is
        \begin{align*}
            \dv{[\ce{CH4}]}{t} &= k_2\left( \frac{k_1}{2k_4}[\ce{CH3CHO}] \right)[\ce{CH3CHO}]\\
            &= k_2\left( \frac{k_1}{2k_4} \right)^{1/2}[\ce{CH3CHO}]^{3/2}
        \end{align*}
        \begin{itemize}
            \item What's the issue here?
        \end{itemize}
    \end{itemize}
\end{itemize}



\section{Chapter 26: Chemical Equilibrium}
\emph{From \textcite{bib:McQuarrieSimon}.}
\begin{itemize}
    \item \marginnote{4/28:}"Thermodynamics enables us to predict with confidence the equilibrium pressures or concentrations of reaction mixtures" \parencite[963]{bib:McQuarrieSimon}.
    \item Goal of this chapter: Derive a relationship between the standard Gibbs energy change and the equilibrium constant.
    \item We begin by considering the following general gas-phase reaction.
    \begin{equation*}
        \ce{\nu_{\ce{A}} A(g) + \nu_{\ce{B}} B(g) <=> \nu_{\ce{Y}} Y(g) + \nu_{\ce{Z}} Z(g)}
    \end{equation*}
    \item \textbf{Extent of reaction}: A measure of how far along its reaction coordinate a chemical reaction is. \emph{Denoted by} $\bm{\xi}$. \emph{Units} \textbf{mol}. \emph{Given by}
    \begin{align*}
        n_{\ce{A}} &= n_{\ce{A}0}-\nu_{\ce{A}}\xi&
            n_{\ce{Y}} &= n_{\ce{Y}0}+\nu_{\ce{Y}}\xi\\
        n_{\ce{B}} &= n_{\ce{B}0}-\nu_{\ce{B}}\xi&
            n_{\ce{Z}} &= n_{\ce{Z}0}+\nu_{\ce{Z}}\xi
    \end{align*}
    where $n_j$ is the number of moles for each species at the time during the reaction corresponding to extent of reaction $\xi$ and $n_{j0}$ is the initial number of moles for each species.
    \begin{itemize}
        \item As the reaction proceeds, $\xi$ varies from zero to some maximum value.
        \item An example of the units of $\xi$: If $n_{\ce{A}0}$ equals $\nu_{\ce{A}}$ moles and $n_{\ce{B}0}$ equals $\nu_{\ce{B}}$ moles, then $\xi$ varies from zero moles to one mole over the course of the reaction.
    \end{itemize}
    \item Relating the change in Gibbs energy to the change in extent of reaction.
    \begin{itemize}
        \item The Gibbs energy for this multicomponent system depends on $T$, $P$, and $n_j$ for the two reactants and the two products. Thus,
        \begin{align*}
            \begin{split}
                \dd{G} ={}& \left( \pdv{G}{T} \right)_{P,n_j}\dd{T}+\left( \pdv{G}{P} \right)_{T,n_j}\dd{P}+\left( \pdv{G}{n_{\ce{A}}} \right)_{T,P,n_{j\neq\ce{A}}}\dd{n_{\ce{A}}}\\
                &+\left( \pdv{G}{n_{\ce{B}}} \right)_{T,P,n_{j\neq\ce{B}}}\dd{n_{\ce{B}}}+\left( \pdv{G}{n_{\ce{Y}}} \right)_{T,P,n_{j\neq\ce{Y}}}\dd{n_{\ce{Y}}}+\left( \pdv{G}{n_{\ce{Z}}} \right)_{T,P,n_{j\neq\ce{Z}}}\dd{n_{\ce{Z}}}
            \end{split}\\
            ={}& -S\dd{T}+V\dd{P}+\mu_{\ce{A}}\dd{n_{\ce{A}}}+\mu_{\ce{B}}\dd{n_{\ce{B}}}+\mu_{\ce{Y}}\dd{n_{\ce{Y}}}+\mu_{\ce{Z}}\dd{n_{\ce{Z}}}
        \end{align*}
        \item Taking $T,P$ to be constant simplifies the above to
        \begin{equation*}
            \dd{G} = \mu_{\ce{A}}\dd{n_{\ce{A}}}+\mu_{\ce{B}}\dd{n_{\ce{B}}}+\mu_{\ce{Y}}\dd{n_{\ce{Y}}}+\mu_{\ce{Z}}\dd{n_{\ce{Z}}}
        \end{equation*}
        \item Differentiating the equations used to define the extent of reaction yields
        \begin{align*}
            \dd{n_{\ce{A}}} &= -\nu_{\ce{A}}\dd{\xi}&
                \dd{n_{\ce{Y}}} &= \nu_{\ce{Y}}\dd{\xi}\\
            \dd{n_{\ce{B}}} &= -\nu_{\ce{B}}\dd{\xi}&
                \dd{n_{\ce{Z}}} &= \nu_{\ce{Z}}\dd{\xi}
        \end{align*}
        so that
        \begin{align*}
            \dd{G} &= -\nu_{\ce{A}}\mu_{\ce{A}}\dd{\xi}-\nu_{\ce{B}}\mu_{\ce{B}}\dd{\xi}+\nu_{\ce{Y}}\mu_{\ce{Y}}\dd{\xi}+\nu_{\ce{Z}}\mu_{\ce{Z}}\dd{\xi}\\
            &= (\nu_{\ce{Y}}\mu_{\ce{Y}}+\nu_{\ce{Z}}\mu_{\ce{Z}}-\nu_{\ce{A}}\mu_{\ce{A}}-\nu_{\ce{B}}\mu_{\ce{B}})\dd{\xi}\\
            \left( \pdv{G}{\xi} \right)_{T,P} &= \nu_{\ce{Y}}\mu_{\ce{Y}}+\nu_{\ce{Z}}\mu_{\ce{Z}}-\nu_{\ce{A}}\mu_{\ce{A}}-\nu_{\ce{B}}\mu_{\ce{B}}
        \end{align*}
    \end{itemize}
    \item $\bm{\Delta_rG}$: The change in Gibbs energy when the extent of reaction changes by one mole. \emph{Units} \textbf{J\,mol\textsuperscript{$\bm{-1}$}}. \emph{Given by}
    \begin{equation*}
        \Delta_rG = \left( \pdv{G}{\xi} \right)_{T,P}
        = \nu_{\ce{Y}}\mu_{\ce{Y}}+\nu_{\ce{Z}}\mu_{\ce{Z}}-\nu_{\ce{A}}\mu_{\ce{A}}-\nu_{\ce{B}}\mu_{\ce{B}}
    \end{equation*}
    \item Relating standard and nonstandard states.
    \begin{itemize}
        \item Let all partial pressures be sufficiently low to assume ideality.
        \item Then substituting the equation $\mu_j(T,P)=\mu_j^\circ(T)+RT\ln(P_j/P^\circ)$ yields
        \begin{align*}
            \Delta_rG &= \nu_{\ce{Y}}\mu_{\ce{Y}}^\circ(T)+\nu_{\ce{Z}}\mu_{\ce{Z}}^\circ(T)-\nu_{\ce{A}}\mu_{\ce{A}}^\circ(T)-\nu_{\ce{B}}\mu_{\ce{B}}^\circ(T)+RT\left( \nu_{\ce{Y}}\ln\frac{P_{\ce{Y}}}{P^\circ}+\nu_{\ce{Z}}\ln\frac{P_{\ce{Z}}}{P^\circ}-\nu_{\ce{A}}\ln\frac{P_{\ce{A}}}{P^\circ}-\nu_{\ce{B}}\ln\frac{P_{\ce{B}}}{P^\circ} \right)\\
            &= \underbrace{\nu_{\ce{Y}}\mu_{\ce{Y}}^\circ(T)+\nu_{\ce{Z}}\mu_{\ce{Z}}^\circ(T)-\nu_{\ce{A}}\mu_{\ce{A}}^\circ(T)-\nu_{\ce{B}}\mu_{\ce{B}}^\circ(T)\vphantom{\frac{(P_{\ce{Y}}/P^\circ)^{\nu_{\ce{Y}}}(P_{\ce{Z}}/P^\circ)^{\nu_{\ce{Z}}}}{(P_{\ce{A}}/P^\circ)^{\nu_{\ce{A}}}(P_{\ce{B}}/P^\circ)^{\nu_{\ce{B}}}}}}_{\Delta_rG^\circ}+RT\ln\underbrace{\frac{(P_{\ce{Y}}/P^\circ)^{\nu_{\ce{Y}}}(P_{\ce{Z}}/P^\circ)^{\nu_{\ce{Z}}}}{(P_{\ce{A}}/P^\circ)^{\nu_{\ce{A}}}(P_{\ce{B}}/P^\circ)^{\nu_{\ce{B}}}}}_Q
        \end{align*}
    \end{itemize}
    \item $\bm{\Delta_rG^\circ(T)}$: The change in standard Gibbs energy for the reaction between unmixed reactants in their standard states at temperature $T$ and a pressure of one bar to form unmixed products in their standard states at the same temperature $T$ and pressure of one par.
    \begin{itemize}
        \item Note that since $P^\circ$ is taken to be \SI{1}{\bar}, the $P^\circ$'s are usually dropped in the definition of $Q$. However, they must be remembered in the sense that they make $Q$ unitless whether shown or not.
    \end{itemize}
    \item \textbf{Equilibrium}: The position of the reaction system at which the Gibbs energy is a minimum with respect to any displacement.
    \begin{itemize}
        \item Mathematically, we have that at equilibrium, $\Delta_rG=0$.
        \item It follows that at equilibrium,
        \begin{align*}
            0 &= \Delta_rG^\circ(T)+RT\ln\left( \frac{P_{\ce{Y}}^{\nu_{\ce{Y}}}P_{\ce{Z}}^{\nu_{\ce{Z}}}}{P_{\ce{A}}^{\nu_{\ce{A}}}P_{\ce{B}}^{\nu_{\ce{B}}}} \right)_\text{eq}\\
            \Delta_rG^\circ(T) &= -RT\ln\left( \frac{P_{\ce{Y}}^{\nu_{\ce{Y}}}P_{\ce{Z}}^{\nu_{\ce{Z}}}}{P_{\ce{A}}^{\nu_{\ce{A}}}P_{\ce{B}}^{\nu_{\ce{B}}}} \right)_\text{eq}
        \end{align*}
    \end{itemize}
    \item \textbf{Equilibrium constant}: A constant describing the relative pressures of reactants to products that will result in the reaction system achieving equilibrium. \emph{Denoted by} $\bm{K_P(T)}$. \emph{Given by}
    \begin{equation*}
        K_P(T) = \left( \frac{P_{\ce{Y}}^{\nu_{\ce{Y}}}P_{\ce{Z}}^{\nu_{\ce{Z}}}}{P_{\ce{A}}^{\nu_{\ce{A}}}P_{\ce{B}}^{\nu_{\ce{B}}}} \right)_\text{eq}
    \end{equation*}
    \begin{itemize}
        \item The value of the equilibrium constant depends on how we write the chemical equation for the reaction at hand. For instance, the equilibrium constant expression for \ce{3H2 + N2 <=> 2NH3} is the square of the equilibrium constant expression for \ce{\frac{3}{2}H2 + \frac{1}{2}N2 <=> NH3}.
    \end{itemize}
    \item An equilibrium constant is a function of temperature only.
    \begin{itemize}
        \item This is because in deriving the equilibrium constant expression, we set $\Delta_rG(T,P)=0$, and pressure appears nowhere else in the equation $\Delta_rG^\circ(T)+RT\ln Q$.
        \item In particular, the ratio that defines $Q$ must remain constant at different initial pressures of reactants and products.
    \end{itemize}
    \item Thinking through some equilibrium concepts in an example.
    \begin{itemize}
        \item Consider the reaction
        \begin{equation*}
            \ce{PCl5(g) <=> PCl3(g) + Cl2(g)}
        \end{equation*}
        \item The equilibrium constant expression is
        \begin{equation*}
            K_P(T) = \frac{P_{\ce{PCl3}}P_{\ce{Cl2}}}{P_{\ce{PCl5}}}
        \end{equation*}
        \item If we initially have one mole of \ce{PCl5} and no \ce{PCl3} or \ce{Cl2}, then when the reaction occurs to an extent $\xi$, there will be $1-\xi$ moles \ce{PCl5} and $\xi$ moles \ce{PCl3} and \ce{Cl2}. This leads to an overall $1+\xi$ moles of gas.
        \item It follows that if $\xi_\text{eq}$ is the extent of reaction at equilibrium, then the partial pressures of each gas at equilibrium are given by
        \begin{align*}
            P_{\ce{PCl3}} = P_{\ce{Cl2}} &= \frac{\xi_\text{eq}}{1+\xi_\text{eq}}P&
            P_{\ce{PCl5}} &= \frac{1-\xi_\text{eq}}{1+\xi_\text{eq}}P
        \end{align*}
        where $P$ is the total pressure.
        \item Thus, the equilibrium constant expression is
        \begin{equation*}
            K_P(T) = \frac{\xi_\text{eq}^2}{1-\xi_\text{eq}^2}P
        \end{equation*}
        \item While the above expression sure makes it seem like $K_P(T)$ depends on $P$, we know by the above that it can't. Thus, it must be the position of the equilibrium along the extent of reaction that changes as $P$ changes.
        \item In particular, as per \textbf{Le Ch\^{a}telier's principle}, we can note that for $K_P>1$, increasing pressure favors lesser extents of reaction (i.e., favors the reactants). This should make intuitive sense harkening back to AP Chemistry since it stands to reason that pressure increases would favor shifting the equilibrium to have fewer moles of gas. Now, however, we have a quantitative rule for how changes in pressure will affect the equilibrium.
    \end{itemize}
    \item \textbf{Le Ch\^{a}telier's principle}: If a chemical reaction at equilibrium is subjected to a change in conditions that displaces it from equilibrium, then the reaction adjusts toward a new equilibrium state.
    \item Note that we can also express the equilibrium constant in terms of concentration via the relationship $P=cRT$ where $c=n/V$ is the concentration.
    \begin{equation*}
        K_P = \frac{c_{\ce{Y}}^{\nu_{\ce{Y}}}c_{\ce{Z}}^{\nu_{\ce{Z}}}}{c_{\ce{A}}^{\nu_{\ce{A}}}c_{\ce{B}}^{\nu_{\ce{B}}}}\left( \frac{RT}{P^\circ} \right)^{\nu_{\ce{Y}}+\nu_{\ce{Z}}-\nu_{\ce{A}}-\nu_{\ce{B}}}
    \end{equation*}
    \item $\bm{c^\circ}$: The standard concentration. \emph{Given by}
    \begin{equation*}
        c^\circ = \SI[per-mode=fraction,fraction-function=\tfrac]{1}{\mole\per\liter}
    \end{equation*}
    \item The standard concentration enables the following definitions.
    \begin{equation*}
        K_P = \underbrace{\frac{(c_{\ce{Y}}/c^\circ)^{\nu_{\ce{Y}}}(c_{\ce{Z}}/c^\circ)^{\nu_{\ce{Z}}}}{(c_{\ce{A}}/c^\circ)^{\nu_{\ce{A}}}(c_{\ce{B}}/c^\circ)^{\nu_{\ce{B}}}}}_{K_c}\left( \frac{c^\circ RT}{P^\circ} \right)^{\nu_{\ce{Y}}+\nu_{\ce{Z}}-\nu_{\ce{A}}-\nu_{\ce{B}}}
    \end{equation*}
    \begin{itemize}
        \item Note that both $K_c$ and the term by which it is multiplied above are unitless.
        \item As such, we have to be careful what units we use for $R$. In the standard situation of $P^\circ=\SI{1}{\bar}$ and $c^\circ=\SI{1}{\mole\per\liter}$, we must use $R=\SI{0.083145}{\liter\bar\per\mole\per\kelvin}$.
    \end{itemize}
    \item By combining the two equations below (both given in derivations above), we can obtain a relation between the equilibrium constant $K_p$ and the standard molar Gibbs energies (i.e., chemical potentials) of the relevant substances.
    \begin{align*}
        \Delta_rG^\circ(T) &= -RT\ln K_P\\
        \Delta_rG^\circ(T) &= \nu_{\ce{Y}}\mu_{\ce{Y}}^\circ(T)+\nu_{\ce{Z}}\mu_{\ce{Z}}^\circ(T)-\nu_{\ce{A}}\mu_{\ce{A}}^\circ(T)-\nu_{\ce{B}}\mu_{\ce{B}}^\circ(T)
    \end{align*}
    \begin{itemize}
        \item Note that $\mu_j^\circ(T)=\Delta_fG^\circ[j]$ if we choose appropriate standard states.
        \item Thus, we can use tables of standard molar Gibbs energies of formation to calculate equilibrium constants.
    \end{itemize}
    \item Example: Deriving a function for the Gibbs energy of a reaction in terms of the extent of reaction.
    \begin{itemize}
        \item Consider the following reaction, which occurs at \SI{298.15}{\kelvin}.
        \begin{equation*}
            \ce{N2O4(g) <=> 2NO2(g)}
        \end{equation*}
        \item Let the initial conditions be one mole of \ce{N2O4} and no \ce{NO2}. Then
        \begin{align*}
            G(\xi) &= n_{\ce{N2O4}}\overline{G}_{\ce{N2O4}}+n_{\ce{NO2}}\overline{G}_{\ce{NO2}}\\
            &= (1-\xi)\overline{G}_{\ce{N2O4}}+2\xi\overline{G}_{\ce{NO2}}\\
            &= (1-\xi)G_{\ce{N2O4}}^\circ+2\xi G_{\ce{NO2}}^\circ+(1-\xi)RT\ln P_{\ce{N2O4}}+2\xi RT\ln P_{\ce{NO2}}
        \end{align*}
        \item Let the reaction be carried out at a constant total pressure of one bar. This assumption combined with the fact that the total number of moles in the reaction mixture is $(1-\xi)+2\xi=1+\xi$ reveals that
        \begin{align*}
            P_{\ce{N2O4}} &= x_{\ce{N2O4}}P_\text{total}
                = \frac{1-\xi}{1+\xi}\cdot 1
                = \frac{1-\xi}{1+\xi}&
            P_{\ce{NO2}} &= x_{\ce{NO2}}P_\text{total}
                = \frac{2\xi}{1+\xi}\cdot 1
                = \frac{2\xi}{1+\xi}
        \end{align*}
        so that
        \begin{equation*}
            G(\xi) = (1-\xi)G_{\ce{N2O4}}^\circ+2\xi G_{\ce{NO2}}^\circ+(1-\xi)RT\ln\frac{1-\xi}{1+\xi}+2\xi RT\ln\frac{2\xi}{1+\xi}
        \end{equation*}
        \item Choosing appropriate standard states, we can obtain the final form
        \begin{equation*}
            G(\xi) = (1-\xi)\Delta_fG_{\ce{N2O4}}^\circ+2\xi\Delta_fG_{\ce{NO2}}^\circ+(1-\xi)RT\ln\frac{1-\xi}{1+\xi}+2\xi RT\ln\frac{2\xi}{1+\xi}
        \end{equation*}
        \item Plugging in
        \begin{align*}
            \Delta_fG_{\ce{N2O4}}^\circ &= \SI{97.787}{\kilo\joule\per\mole}&
            \Delta_fG_{\ce{NO2}}^\circ &= \SI{51.258}{\kilo\joule\per\mole}
        \end{align*}
        we can determine that the minimum of the curve occurs at $\xi_\text{eq}=\SI{0.1892}{\mole}$.
        \item Thus,
        \begin{equation*}
            K_P = \frac{P_{\ce{NO2}}^2}{P_{\ce{N2O4}}}
            = \frac{[2\xi_\text{eq}/(1+\xi_\text{eq})]^2}{(1-\xi_\text{eq})/(1+\xi_\text{eq})}
            = \frac{4\xi_\text{eq}^2}{1-\xi_\text{eq}^2}
            = 0.148
        \end{equation*}
        \begin{itemize}
            \item Note that this value compares exactly with the one obtained via the $-RT\ln K_P=\sum\nu_j\mu_j^\circ(T)$ method.
        \end{itemize}
        \item Note that differentiating our final form for $G(\xi)$ wrt. $\xi$ yields $\Delta_rG=\Delta_rG^\circ+RT\ln K_P$, as expected.
    \end{itemize}
    \item \textbf{Reaction quotient}: A quantity describing the relative amounts of reactants and products present in the reaction system at a given instant in time. \emph{Denoted by} $\bm{Q_P}$. \emph{Given by}
    \begin{equation*}
        Q_P = \frac{P_{\ce{Y}}^{\nu_{\ce{Y}}}P_{\ce{Z}}^{\nu_{\ce{Z}}}}{P_{\ce{A}}^{\nu_{\ce{A}}}P_{\ce{B}}^{\nu_{\ce{B}}}}
    \end{equation*}
    \item We have that
    \begin{align*}
        \Delta_rG &= \Delta_rG^\circ+RT\ln Q_P\\
        &= -RT\ln K_P+RT\ln Q_P\\
        &= RT\ln\frac{Q_P}{K_P}
    \end{align*}
    \begin{itemize}
        \item Thus, the ratio of the reaction quotient to the equilibrium constant determines the direction in which a reaction will proceed.
    \end{itemize}
    \item The sign of $\Delta_rG$ and not that of $\Delta_rG^\circ$ determines the direction of reaction spontaneity.
    \begin{itemize}
        \item Indeed, the sign of $\Delta_rG^\circ$ determines the direction of reaction spontaneity only when all substances are mixed at one bar partial pressures (i.e., when $\Delta_rG=\Delta_rG^\circ$).
    \end{itemize}
    \item "The fact that a process will occur spontaneously does not imply that it will necessarily occur at a detectable rate" \parencite[977]{bib:McQuarrieSimon}.
    \begin{itemize}
        \item For example, the negative $\Delta_rG^\circ$ of water at one bar and \SI{25}{\celsius} tells us that the reaction \ce{2H2 + O2 <=> 2H2O} will occur spontaneously. However, experimental evidence reveals that a spark or catalyst is needed to convert hydrogen and oxygen to water; once said activation energy is introduced, the reaction proceeds explosively.
        \item Indeed, "the `no' of thermodynamics is emphatic. If thermodynamics says that a certain process will not occur spontaneously, it will not occur. The `yes' of thermodynamics, on the other hand, is actually a `maybe'" \parencite[977]{bib:McQuarrieSimon}.
    \end{itemize}
    \item \marginnote{4/29:}Deriving a relationship between $K_P$ and $T$.
    \begin{itemize}
        \item Recall the Gibbs-Helmholtz equation
        \begin{equation*}
            \left( \pdv{\Delta G^\circ/T}{T} \right)_P = -\frac{\Delta H^\circ}{T^2}
        \end{equation*}
        \item Substituting $\Delta G^\circ(T)=-RT\ln K_P(T)$ yields the \textbf{Van't Hoff equation}.
        \begin{equation*}
            \left( \pdv{\ln K_P(T)}{T} \right)_P = \dv{\ln K_P}{T}
            = \frac{\Delta_rH^\circ}{RT^2}
        \end{equation*}
        \item Qualitatively, the above equation tells us that if $\Delta_rH^\circ>0$ (i.e., if the reaction is endothermic), then $K_P(T)$ increases with temperature, as expected since more available energy should drive an endothermic reaction, and vice versa if $\Delta_rH^\circ<0$.
        \begin{itemize}
            \item This is another example of Le Ch\^{a}telier's principle.
        \end{itemize}
        \item Quantitatively, the above equation can be integrated to give
        \begin{equation*}
            \ln\frac{K_P(T_2)}{K_P(T_1)} = \int_{T_1}^{T_2}\frac{\Delta_rH^\circ(T)}{RT^2}\dd{T}
        \end{equation*}
        \item If the temperature range or the magnitude change of $\Delta_rH^\circ(T)$ is sufficiently small, we may take $\Delta_rH^\circ$ to be constant and write
        \begin{equation*}
            \ln\frac{K_P(T_2)}{K_P(T_1)} = -\frac{\Delta_rH^\circ(T)}{R}\left( \frac{1}{T_2}-\frac{1}{T_1} \right)
        \end{equation*}
        \begin{itemize}
            \item One implication of this equation is that over a sufficiently small temperature range, a plot of $\ln K$ vs. $1/T$ is linear with slope $-\Delta_rH^\circ/R$.
        \end{itemize}
        \item If the temperature range is not sufficiently small, we still have options.
        \item For example, recall that
        \begin{equation*}
            \Delta_rH^\circ(T_2) = \Delta_rH^\circ(T_1)+\int_{T_1}^{T_2}\Delta C_P^\circ(T)\dd{T}
        \end{equation*}
        where $\Delta C_P^\circ$ is the difference between the heat capacities of the products and reactants.
        \item Alternatively, we may present experimental heat capacity data as a polynomial in temperature of the form
        \begin{equation*}
            \Delta_rH^\circ(T) = \alpha+\beta T+\gamma T^2+\delta T^3+\cdots
        \end{equation*}
        so that
        \begin{equation*}
            \ln K_P(T) = -\frac{\alpha}{RT}+\frac{\beta}{R}\ln T+\frac{\gamma}{R}T+\frac{\delta}{2R}T^2+A
        \end{equation*}
        where $A$ is a constant of integration.
        \begin{itemize}
            \item It follows from this equation that in reality, a plot of $\ln K_P$ vs. $1/T$ is not linear but has a slight curvature.
        \end{itemize}
        \item More generally, we may always take
        \begin{equation*}
            \ln K_P(T) = \ln K_P(T_1)+\int_{T_1}^T\frac{\Delta_rH^\circ(T')}{RT'^2}\dd{T}
        \end{equation*}
        regardless of how $\Delta_rH^\circ$ varies with temperature.
    \end{itemize}
    \item \textbf{Van't Hoff equation}: An ordinary differential equation describing the temperature dependence of the equilibrium constant. \emph{Given by}
    \begin{equation*}
        \dv{\ln K_P}{T} = \frac{\Delta_rH^\circ}{RT^2}
    \end{equation*}
    \item A note on the similarity in form between the integrated Van't Hoff equation at constant $\Delta_rH^\circ$ and the Clausius-Clapeyron equation: "These equations are essentially the same because the vaporization of a liquid can be represented by the `chemical equation' \ce{X(l) <=> X(g)}" \parencite[980]{bib:McQuarrieSimon}.
    \item \marginnote{4/30:}Calculating equilibrium constants in terms of partition functions.
    \begin{itemize}
        \item Consider the general homogeneous gas-phase reaction
        \begin{equation*}
            \ce{\nu_{\ce{A}} A(g) + \nu_{\ce{B}} B(g) <=> \nu_{\ce{Y}} Y(g) + \nu_{\ce{Z}} Z(g)}
        \end{equation*}
        in a reaction vessel at fixed volume and temperature.
        \item It follows that
        \begin{equation*}
            \dd{A} = \mu_{\ce{A}}\dd{n_{\ce{A}}}+\mu_{\ce{B}}\dd{n_{\ce{B}}}+\mu_{\ce{Y}}\dd{n_{\ce{Y}}}+\mu_{\ce{Z}}\dd{n_{\ce{Z}}}
        \end{equation*}
        \item As before, this equation gives the condition for chemical equilibrium as
        \begin{equation*}
            \nu_{\ce{Y}}\mu_{\ce{Y}}+\nu_{\ce{Z}}\mu_{\ce{Z}}-\nu_{\ce{A}}\mu_{\ce{A}}-\nu_{\ce{B}}\mu_{\ce{B}} = 0
        \end{equation*}
        \item We now express the chemical potentials above in terms of partition functions.
        \item Since the species are independent in an ideal gas, we have that
        \begin{align*}
            Q(N_{\ce{A}},N_{\ce{B}},N_{\ce{Y}},N_{\ce{Z}},V,T) &= Q(N_{\ce{A}},V,T)Q(N_{\ce{B}},V,T)Q(N_{\ce{Y}},V,T)Q(N_{\ce{Z}},V,T)\\
            &= \frac{q_{\ce{A}}(V,T)^{N_{\ce{A}}}}{N_{\ce{A}}!}\frac{q_{\ce{B}}(V,T)^{N_{\ce{B}}}}{N_{\ce{B}}!}\frac{q_{\ce{Y}}(V,T)^{N_{\ce{Y}}}}{N_{\ce{Y}}!}\frac{q_{\ce{Z}}(V,T)^{N_{\ce{Z}}}}{N_{\ce{Z}}!}
        \end{align*}
        \item Thus, for example,
        \begin{equation*}
            \mu_{\ce{A}} = -RT\left( \pdv{\ln Q(N_{\ce{A}},N_{\ce{B}},N_{\ce{Y}},N_{\ce{Z}},V,T)}{N_{\ce{A}}} \right)_{N_j,V,T}
            = -RT\ln\frac{q_{\ce{A}}(V,T)}{N_{\ce{A}}}
        \end{equation*}
        where we have used Stirling's approximation for $N_{\ce{A}}!$.
        \item Substituting the above equation and its variations for \ce{B}, \ce{Y}, and \ce{Z} into the equilibrium conditions yields
        \begin{align*}
            0 &= \nu_{\ce{Y}}\left( -RT\ln\frac{q_{\ce{Y}}}{N_{\ce{Y}}} \right)+\nu_{\ce{Z}}\left( -RT\ln\frac{q_{\ce{Z}}}{N_{\ce{Z}}} \right)-\nu_{\ce{A}}\left( -RT\ln\frac{q_{\ce{A}}}{N_{\ce{A}}} \right)-\nu_{\ce{B}}\left( -RT\ln\frac{q_{\ce{B}}}{N_{\ce{B}}} \right)\\
            % &= \nu_{\ce{Y}}\ln\frac{q_{\ce{Y}}}{N_{\ce{Y}}}+\nu_{\ce{Z}}\ln\frac{q_{\ce{Z}}}{N_{\ce{Z}}}-\nu_{\ce{A}}\ln\frac{q_{\ce{A}}}{N_{\ce{A}}}-\nu_{\ce{B}}\ln\frac{q_{\ce{B}}}{N_{\ce{B}}}\\
            &= \ln\frac{q_{\ce{Y}}^{\nu_{\ce{Y}}}}{N_{\ce{Y}}^{\nu_{\ce{Y}}}}+\ln\frac{q_{\ce{Z}}^{\nu_{\ce{Z}}}}{N_{\ce{Z}}^{\nu_{\ce{Z}}}}-\ln\frac{q_{\ce{A}}^{\nu_{\ce{A}}}}{N_{\ce{A}}^{\nu_{\ce{A}}}}-\ln\frac{q_{\ce{B}}^{\nu_{\ce{B}}}}{N_{\ce{B}}^{\nu_{\ce{B}}}}\\
            % &= \ln\frac{\frac{q_{\ce{Y}}^{\nu_{\ce{Y}}}q_{\ce{Z}}^{\nu_{\ce{Z}}}}{q_{\ce{A}}^{\nu_{\ce{A}}}q_{\ce{B}}^{\nu_{\ce{B}}}}}{\frac{N_{\ce{Y}}^{\nu_{\ce{Y}}}N_{\ce{Z}}^{\nu_{\ce{Z}}}}{N_{\ce{A}}^{\nu_{\ce{A}}}N_{\ce{B}}^{\nu_{\ce{B}}}}}\\
            \e[0] &= \frac{\frac{q_{\ce{Y}}^{\nu_{\ce{Y}}}q_{\ce{Z}}^{\nu_{\ce{Z}}}}{q_{\ce{A}}^{\nu_{\ce{A}}}q_{\ce{B}}^{\nu_{\ce{B}}}}}{\frac{N_{\ce{Y}}^{\nu_{\ce{Y}}}N_{\ce{Z}}^{\nu_{\ce{Z}}}}{N_{\ce{A}}^{\nu_{\ce{A}}}N_{\ce{B}}^{\nu_{\ce{B}}}}}\\
            \frac{N_{\ce{Y}}^{\nu_{\ce{Y}}}N_{\ce{Z}}^{\nu_{\ce{Z}}}}{N_{\ce{A}}^{\nu_{\ce{A}}}N_{\ce{B}}^{\nu_{\ce{B}}}} &= \frac{q_{\ce{Y}}^{\nu_{\ce{Y}}}q_{\ce{Z}}^{\nu_{\ce{Z}}}}{q_{\ce{A}}^{\nu_{\ce{A}}}q_{\ce{B}}^{\nu_{\ce{B}}}}
        \end{align*}
        \item Now $N_j/V=\rho_j$ is concentration, so dividing every term in the left expression above by $V^{\nu_j}$ gives $K_c$.
        \begin{equation*}
            K_c(T) = \frac{\rho_{\ce{Y}}^{\nu_{\ce{Y}}}\rho_{\ce{Z}}^{\nu_{\ce{Z}}}}{\rho_{\ce{A}}^{\nu_{\ce{A}}}\rho_{\ce{B}}^{\nu_{\ce{B}}}}
            = \frac{(q_{\ce{Y}}/V)^{\nu_{\ce{Y}}}(q_{\ce{Z}}/V)^{\nu_{\ce{Z}}}}{(q_{\ce{A}}/V)^{\nu_{\ce{A}}}(q_{\ce{B}}/V)^{\nu_{\ce{B}}}}
        \end{equation*}
        \begin{itemize}
            \item Recall that $q_j/V$ is a function of temperature only. Thus, the above of definition does give $K_c(T)$ in terms of the partition functions and as a function of temperature only, as it should.
        \end{itemize}
        \item Lastly, recall that we can use the definition of $K_P$ in terms of $K_c$ to calculate $K_P$ in terms of partition functions.
    \end{itemize}
    \item \textcite{bib:McQuarrieSimon} goes through two examples of using the above result to calculate the equilibrium constant from molecular parameters.
    \begin{itemize}
        \item Using the rigid-rotator harmonic oscillator approximation (which we may recall is the basis of all partition functions we've derived thus far) gives results in good (but not great) agreement with experiment.
    \end{itemize}
    \item \marginnote{5/1:}We can achieve better agreement with experimental data using more laborious calculations.
    \item Alternatively, we can turn to tabulated data, such as the \textbf{JANAF tables}.
    \item \textbf{JANAF tables}: The \underline{j}oint, \underline{a}rmy, \underline{n}avy, \underline{a}ir \underline{f}orce tables \parencite{bib:JANAFtables}, which are one of the most extensive tabulations of the thermochemical properties of substances.
    \begin{itemize}
        \item The JANAF tables use as a reference point for relative data the standard molar enthalpies at \SI{298.15}{\kelvin}.
    \end{itemize}
    \item \textcite{bib:McQuarrieSimon} works through several examples of how to use the JANAF tables and manipulate the thermodynamic equations we've derived thus far to best suit them.
    \item \textcite{bib:McQuarrieSimon} discusses equilibrium constants for real gases and solutions.
\end{itemize}


\subsection*{Problems}
\begin{enumerate}[label={\textbf{26-\arabic*.}},ref={26-\arabic*},leftmargin=3.5em]
    \setcounter{enumi}{13}
    \item \label{prb:26-14}Show that
    \begin{equation*}
        \dv{\ln K_c}{T} = \frac{\Delta_rU^\circ}{RT^2}
    \end{equation*}
    for reactions involving ideal gases.
    \begin{proof}[Answer]
        Starting from the Van't Hoff equation, we have that
        \begin{align*}
            \dv{\ln K_P}{T} &= \frac{\Delta_rH^\circ}{RT^2}\\
            \dv{T}\ln(K_c\left( \frac{c^\circ RT}{P^\circ} \right)^{\nu_{\ce{Y}}+\nu_{\ce{Z}}-\nu_{\ce{A}}-\nu_{\ce{B}}}) &= \frac{\Delta_rU^\circ+\Delta_r(PV)}{RT^2}\\
            \dv{\ln K_c}{T}+\dv{T}\{[(\nu_{\ce{Y}}+\nu_{\ce{Z}})-(\nu_{\ce{A}}+\nu_{\ce{B}})]\ln T\}+\dv{T}(\frac{c^\circ R}{P^\circ}) &= \frac{\Delta_rU^\circ+\Delta_r(nRT)}{RT^2}\\
            \dv{\ln K_c}{T}+\frac{n_f-n_i}{T}+0 &= \frac{\Delta_rU^\circ}{RT^2}+\frac{\Delta_rn}{T}\\
            \dv{\ln K_c}{T} &= \frac{\Delta_rU^\circ}{RT^2}
        \end{align*}
        as desired.
    \end{proof}
\end{enumerate}



\section{Chapter 28: Chemical Kinetics I --- Rate Laws}
\emph{From \textcite{bib:McQuarrieSimon}.}
\begin{itemize}
    \item \textbf{Transition state}: The transient species in the vicinity of the top of the activation barrier to reaction. \emph{Also known as} \textbf{activated complex}. \emph{Denoted by} $\textbf{AB}^{\bm{\ddagger}}$.
    \item \textbf{Transition-state theory}: A theory focusing on the transition states that can be used to estimate reaction rate constants. \emph{Also known as} \textbf{activated-complex theory}.
    \begin{itemize}
        \item Developed in the 1930s by Henry Eyring.
    \end{itemize}
    \item We will now develop the fundamental postulates of transition state theory and use them to express the rate constant, activation energy, and Arrhenius pre-exponential factor of the reaction
    \begin{equation*}
        \ce{A + B -> P}
    \end{equation*}
    where \ce{P} represents one or more products in terms of thermodynamic quantities including partition functions as well as more fundamental variables and constants.
    \begin{figure}[h!]
        \centering
        \begin{tikzpicture}[
            every node/.style=black
        ]
            \small
            \draw [stealth-stealth] (0,4) -- node[rotate=90,above]{Energy} (0,0) -- node[below]{Reaction coordinate} (6,0);
            
            \footnotesize
            \draw [blx,thick,name path=E] (0.7,1.5)
                to[out=-70,in=180] (1.3,1) node[below]{\ce{A + B}}
                to[out=0,in=180,out looseness=0.8,in looseness=0.7] (3,3.5) node[below=2mm]{\ce{AB${}^\ddagger$}}
                to[out=0,in=180,out looseness=0.7,in looseness=0.8] (4.7,0.6) node[below]{Products}
                to[out=0,in=-110] (5.3,1.1)
            ;
            \foreach \y in {1.1,1.2,1.3,1.4} {
                \path [name path=p] (0.5,\y) -- ++(2,0);
                \draw [grx,semithick,name intersections={of=E and p}] (intersection-1) -- (intersection-2);
            }
            \foreach \y in {0.7,0.8,0.9,1.0} {
                \path [name path=p] (3.5,\y) -- ++(2,0);
                \draw [grx,semithick,name intersections={of=E and p}] (intersection-1) -- (intersection-2);
            }
    
            \draw [very thin,|-|] (2.8,3.8) -- node[above]{$\delta$} ++(0.4,0);
        \end{tikzpicture}
        \caption{Transition state theory energy diagram.}
        \label{fig:TST}
    \end{figure}
    \begin{itemize}
        \item The first assumption of transition state theory is that the activated complex is in equilibrium with the reactants as per
        \begin{equation*}
            \ce{A + B <=> AB${}^\ddagger$ -> P}
        \end{equation*}
        \item It follows from the definition of the concentration equilibrium constant that the transition-state equilibrium is defined by
        \begin{equation*}
            K_c^\ddagger = \frac{\cnc{AB^\ddagger}/c^\circ}{\cnc{A}/c^\circ\cnc{B}/c^\circ}
            = \frac{\cnc{AB^\ddagger}c^\circ}{\cnc{A}\cnc{B}}
        \end{equation*}
        \item We can substitute partition functions into the above expression as follows
        \begin{equation*}
            K_c^\ddagger = \frac{(q^\ddagger/V)c^\circ}{(q_{\ce{A}}/V)(q_{\ce{B}}/V)}
        \end{equation*}
        where $q_{\ce{A}}$, $q_{\ce{B}}$, and $q^\ddagger$ are the partition functions of \ce{A}, \ce{B}, and $\ce{AB^\ddagger}$, respectively.
        \item The second assumption of transition state theory is that the activated complex is stable throughout a small region of width $\delta$ centered at the top of the energy barrier (see Figure \ref{fig:TST}).
        \item It follows from this assumption that we can define the rate of product formation in terms of the concentration of the preceding intermediate (the activated complex), as we have been throughout this chapter. This may be done as follows.
        \begin{equation*}
            \dv{\cnc{P}}{t} = \nu_c\cnc{AB^\ddagger}
        \end{equation*}
        \begin{itemize}
            \item $\bm{\nu_c}$ functions as a type of rate constant.
            \item The form of this expression implies that "the motion of the reacting system over the barrier top is a one-dimensional translational motion" \parencite[1166]{bib:McQuarrieSimon}.
        \end{itemize}
        \item We also have from the original chemical equation that
        \begin{equation*}
            \dv{\cnc{P}}{t} = k\cnc{A}\cnc{B}
        \end{equation*}
        \item Thus, solving the expression defining $K_c^\ddagger$ for $\cnc{AB^\ddagger}$ and substituting into the above, we have that
        \begin{equation*}
            \dv{\cnc{P}}{t} = \nu_c\cnc{AB^\ddagger}
            = \nu_c\frac{\cnc{A}\cnc{B}K_c^\ddagger}{c^\circ}
            = \underbrace{\frac{\nu_cK_c^\ddagger}{c^\circ}}_k\cnc{A}\cnc{B}
        \end{equation*}
        \begin{itemize}
            \item This expression gives $k$ in units of \si{\per\molar\per\second}.
        \end{itemize}
        \item Since the rate law in terms of $\cnc{AB^\ddagger}$ implies 1D translational motion and the 1D translational partition function is
        \begin{equation*}
            q_\text{trans} = \frac{\sqrt{2\pi m^\ddagger\kB T}}{h}\delta
        \end{equation*}
        we have that $q^\ddagger=q_\text{trans}q_\text{int}^\ddagger$.
        \item Thus,
        \begin{align*}
            k &= \frac{\nu_c}{c^\circ}K_c^\ddagger\\
            &= \frac{\nu_c}{c^\circ}\frac{(q^\ddagger/V)c^\circ}{(q_{\ce{A}}/V)(q_{\ce{B}}/V)}\\
            &= \nu_c\frac{\sqrt{2\pi m^\ddagger\kB T}}{hc^\circ}\delta\frac{(q_\text{int}^\ddagger/V)c^\circ}{(q_{\ce{A}}/V)(q_{\ce{B}}/V)}
        \end{align*}
        \item This equation is looking pretty good, but it still contains $\nu_c$ and $\delta$, both tricky quantities to define and determine. Their product $\bm{\prb{u_\textbf{ac}}}$, however, has a much nicer interpretation.
        \item Consequently, we have that
        \begin{align*}
            k &= \sqrt{\frac{\kB T}{2\pi m^\ddagger}}\frac{\sqrt{2\pi m^\ddagger\kB T}}{hc^\circ}\frac{(q_\text{int}^\ddagger/V)c^\circ}{(q_{\ce{A}}/V)(q_{\ce{B}}/V)}\\
            &= \frac{\kB T}{hc^\circ}K^\ddagger
        \end{align*}
        \item Since we can relate $\bm{K^\ddagger}$ and the \textbf{standard Gibbs energy of activation} by
        \begin{equation*}
            \Delta^\ddagger G^\circ = -RT\ln K^\ddagger
        \end{equation*}
        we have that
        \begin{equation*}
            k(T) = \frac{\kB T}{hc^\circ}\e[-\Delta^\ddagger G^\circ/RT]
        \end{equation*}
        \item Since we can express the standard Gibbs energy of activation in terms of the \textbf{standard enthalpy of activation} and the \textbf{standard entropy of activation} via
        \begin{equation*}
            \Delta^\ddagger G^\circ = \Delta^\ddagger H^\circ-T\Delta^\ddagger S^\circ
        \end{equation*}
        we have that
        \begin{equation*}
            k(T) = \frac{\kB T}{hc^\circ}\e[\Delta^\ddagger S^\circ/R]\e[-\Delta^\ddagger H^\circ/RT]
        \end{equation*}
        \item We now look to use the above equation as a launch point to relate the activation energy and Arrhenius pre-exponential factor to molecular quantities.
        % At first glance, it might seem like this easy task: Surely we could simply take $E_a=\Delta^\ddagger G^\circ$ and $A=\kB T/hc^\circ$. However, under this definition, $A$ would not be a constant but a function of temperature. As such, we must look to combine the two occurrences of temperature in the above equation. To do this, we use the differential form of the Arrhenius equation.
        \item Recall that the differential Arrhenius equation is
        \begin{equation*}
            \dv{\ln k}{T} = \frac{E_a}{RT^2}
        \end{equation*}
        \item It follows using a previous form of $k(T)$ that
        \begin{equation*}
            \dv{\ln k}{T} = \dv{T}\ln(\frac{\kB}{hc^\circ})+\dv{\ln T}{T}+\dv{\ln K^\ddagger}{T}
            = \frac{1}{T}+\dv{\ln K^\ddagger}{T}
        \end{equation*}
        \item Invoking the result of Problem \ref{prb:26-14} yields
        \begin{equation*}
            \dv{\ln k}{T} = \frac{1}{T}+\frac{\Delta^\ddagger U^\circ}{RT^2}
        \end{equation*}
        \item Additionally, since
        \begin{align*}
            \Delta^\ddagger H^\circ &= \Delta^\ddagger U^\circ+\Delta^\ddagger PV
                = \Delta^\ddagger U^\circ+RT\Delta^\ddagger n
                = \Delta^\ddagger U^\circ-RT\\
            \Delta^\ddagger H^\circ+RT &= \Delta^\ddagger U^\circ
        \end{align*}
        we have that
        \begin{align*}
            \dv{\ln k}{T} &= \frac{RT}{RT^2}+\frac{\Delta^\ddagger H^\circ+RT}{RT^2}\\
            &= \frac{\Delta^\ddagger H^\circ+2RT}{RT^2}
        \end{align*}
        \item Therefore, we have by direct comparison with the Arrhenius equation that
        \begin{equation*}
            E_a = \Delta^\ddagger H^\circ+2RT
        \end{equation*}
        \item Substituting this result into the form of $k(T)$ containing $\Delta^\ddagger H^\circ$ yields
        \begin{align*}
            k(T) &= \frac{\kB T}{hc^\circ}\e[\Delta^\ddagger S^\circ/R]\e[-\Delta^\ddagger H^\circ/RT]\\
            &= \frac{\kB T}{hc^\circ}\e[\Delta^\ddagger S^\circ/R]\e[-E_a/RT]\e[2RT/RT]\\
            &= \frac{\e[2]\kB T}{hc^\circ}\e[\Delta^\ddagger S^\circ/R]\e[-E_a/RT]
        \end{align*}
        \item Once again, we have by direct comparison with the Arrhenius equation that
        \begin{equation*}
            A = \frac{\e[2]\kB T}{hc^\circ}\e[\Delta^\ddagger S^\circ/R]
        \end{equation*}
    \end{itemize}
    \item $\bm{\nu_c}$: The frequency with which the activated complex crosses over the barrier top.
    \item $\bm{m^\ddagger}$: The mass of the activated complex.
    \item $\bm{q_\textbf{int}^\ddagger}$: The partition function accounting for all the remaining degrees of freedom of the activated complex.
    \item $\bm{\prb{u_\textbf{ac}}}$: The average speed with which the activated complex crosses the barrier. \emph{Given by}
    \begin{equation*}
        \prb{u_\text{ac}} = \nu_c\delta
        = \sqrt{\frac{\kB T}{2\pi m^\ddagger}}
    \end{equation*}
    \begin{itemize}
        \item The latter expression is derived using the one-dimensional Maxwell-Boltzmann distribution (a 1D molecular velocity component Gaussian distribution), which applies since "we have assumed that the reactants and activated complex are in equilibrium" \parencite[1167]{bib:McQuarrieSimon}.
        \begin{align*}
            \prb{u_\text{ac}} &= \int_0^\infty uf(u)\dd{u}\\
            &= \sqrt{\frac{m^\ddagger}{2\pi\kB T}}\int_0^\infty u\e[-m^\ddagger u^2/2\kB T]\dd{u}\\
            &= \sqrt{\frac{\kB T}{2\pi m^\ddagger}}
        \end{align*}
        \begin{itemize}
            \item We integrate over only the positive values because we are only interested in the particles traveling over the activation barrier in the forward direction.
        \end{itemize}
    \end{itemize}
    \item $\bm{K^\ddagger}$: The "equilibrium constant" for the formation of the transition state from the reactants, but the motion along the reaction coordinate excluded in $q_\text{int}^\ddagger$. \emph{Given by}
    \begin{equation*}
        K^\ddagger = \frac{(q_\text{int}^\ddagger/V)c^\circ}{(q_{\ce{A}}/V)(q_{\ce{B}}/V)}
    \end{equation*}
    \item \textbf{Standard Gibbs energy of activation}: The change in Gibbs energy in going from the reactants at a concentration $c^\circ$ to the transition state at a concentration $c^\circ$. \emph{Denoted by} $\bm{\Delta^\ddagger G^\circ}$.
    \item \textbf{Standard enthalpy of activation}: The change in enthalpy in going from the reactants at a concentration $c^\circ$ to the transition state at a concentration $c^\circ$. \emph{Denoted by} $\bm{\Delta^\ddagger H^\circ}$.
    \item \textbf{Standard entropy of activation}: The change in entropy in going from the reactants at a concentration $c^\circ$ to the transition state at a concentration $c^\circ$. \emph{Denoted by} $\bm{\Delta^\ddagger S^\circ}$.
    \begin{itemize}
        \item Values of $\Delta^\ddagger S^\circ$ give information about the relative structures of the activated complex and the reactants.
        \item For example, if the activated complex is less ordered than the reactants, then $\Delta^\ddagger S^\circ=+$, and vice versa if the activated complex is more ordered than the reactants.
    \end{itemize}
    \item $\bm{\Delta^\ddagger n}$: The change in the number of molecules from the reactants to the transition state.
    \begin{itemize}
        \item $\Delta^\ddagger n=0$ for a unimolecular reaction (here both the reactants and transition state consist of one molecule).
        \item $\Delta^\ddagger n=-1$ for a bimolecular reaction (here the reactants consist of two molecules while the transition state consists of one molecule).
        \item $\Delta^\ddagger n=-2$ for a termolecular reaction (here the reactants consist of three molecules while the transition state consists of one molecule).
    \end{itemize}
\end{itemize}




\end{document}