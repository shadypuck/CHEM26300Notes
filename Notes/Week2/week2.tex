\documentclass[../notes.tex]{subfiles}

\pagestyle{main}
\renewcommand{\chaptermark}[1]{\markboth{\chaptername\ \thechapter\ (#1)}{}}
\stepcounter{chapter}

\begin{document}




\chapter{Intermolecular Dynamics and Intro to Rate Laws}
\section{Mean Free Path}
\begin{itemize}
    \item \marginnote{4/4:}The midterm will have some computational problems; the final will be nearly entirely conceptual.
    \item Reviews the conclusions of the derivation associated with Figure \ref{fig:collisionFrequency}.
    \item The Maxwell-Boltzmann Distribution has been verified experimentally.
    \begin{itemize}
        \item A furnace with a very small hole that allowed a beam of atoms (such as potassium) to emerge into an evacuated chamber. The beam bassed through a pair of collimating slits and then through a velocity-selector.
        \item In the second method, clocks the time it takes for molecules to travel a fixed distance. A very short pulse of molecules leaves the chopper and then spread out in space as they travel toward the detector.
        \item Either way, we observe very good agreement with the M-B distribution.
    \end{itemize}
    \item \textbf{Mean free path}: The average distance a molecule travels between collisions.
    \item \textbf{Collision cylinder}: The cylinder of radius $d$ that encapsulates the trajectory of a particle of diameter $d$.
    \begin{figure}[h!]
        \centering
        \begin{tikzpicture}[scale=1.3]
            \footnotesize
            \draw (0,1) -- ++(5,0) arc[start angle=90,end angle=-90,x radius=4mm,y radius=1cm] -- ++(-5,0) arc[start angle=270,end angle=90,x radius=4mm,y radius=1cm];
    
            \fill [ball color=orx] circle (5mm);
            \fill [ball color=orx] (1.6,1.6) circle (5mm);
            \fill [ball color=orx] (1.8,-0.9) circle (5mm);
            \fill [ball color=orx] (2.6,0.8) circle (5mm);
            \fill [ball color=orx] (3,-1.1) circle (5mm);
            \fill [ball color=orx] (4,-0.3) circle (5mm);
    
            \draw [-latex] (1.6,0) -- node[pos=0.4,left]{$>d$} ++(0,1.6) node[above left=4mm]{Miss};
            \draw [-latex] (1.8,0) -- node[pos=0.25,right]{$<d$} ++(0,-0.9) node[below=6mm]{Just hit};
            \draw [-latex] (2.6,0) -- node[pos=0.25,right]{$<d$} ++(0,0.8);
            \draw [-latex] (3,0) -- node[pos=0.3,left]{$>d$} ++(0,-1.1) node[below=6mm]{Just miss};
            \draw [-latex] (4,0) -- node[pos=0.7,left]{$<d$} ++(0,-0.3) node[above right=4mm]{Hit};
    
            \node at (-1,-1.2) {Area $\sigma$}
                edge [out=90,in=180,-latex,out looseness=0.8] (0,-0.7)
            ;
            \draw [dash pattern=on 2mm off 2pt on 2pt off 2pt] (-1,0) -- (6,0);
            \draw [latex-latex,semithick] (0,1) -- node[pos=0.35,right=-1pt]{$d$} (0,0) -- (1,0);
            \draw (0,1) arc[start angle=90,end angle=-90,x radius=4mm,y radius=1cm];
        \end{tikzpicture}
        \caption{Collision cylinder.}
        \label{fig:collisionCylinder}
    \end{figure}
    \begin{itemize}
        \item Particles whose center of mass lies within the collision cylinder collide with the original particle, and vice versa for particles whose center of mass lies outside the collision cylinder.
        \item Hard-sphere collision cross section $\pi d^2$ denoted by $\sigma$.
    \end{itemize}
    \item Collision frequency in terms of cylinder parameters.
    \begin{itemize}
        \item The number of collision in the time interval $\dd{t}$ is
        \begin{equation*}
            \dd{N_\text{coll}} = \rho\sigma\prb{u}\dd{t}
        \end{equation*}
        where $\rho=N/V$.
        \item The collision frequency $z_{\ce{A}}$ is
        \begin{equation*}
            z_{\ce{A}} = \dv{N_\text{coll}}{t}
            = \rho\sigma\prb{u}
            = \rho\sigma\sqrt{\frac{8\kB T}{\pi m}}
        \end{equation*}
        \item Treat the motion of two bodies of masses $m_1,m_2$ moving with respect to each other by the motion of one body with a reduced mass $\mu=m_1m_2/(m_1+m_2)$ moving with respect to the other one being fixed.
        \item If the masses of the two colliding molecules are the same, then $\mu=m/2$.
        \item Remember that $\prb{u_r}=\sqrt{2}\prb{u}$.
        \item Thus, the correct expression for $z_{\ce{A}}$ is
        \begin{equation*}
            z_{\ce{A}} = \rho\sigma\prb{u_r}
            = \sqrt{2}\rho\sigma\prb{u}
        \end{equation*}
    \end{itemize}
    \item The mean free path is temperature- and pressure-dependent.
    \begin{itemize}
        \item The average distance traveled between collisions is given by
        \begin{equation*}
            l = \frac{\prb{u}}{z_{\ce{A}}}
            = \frac{\prb{u}}{\sqrt{2}\rho\sigma\prb{u}}
            = \frac{1}{\sqrt{2}\rho\sigma}
        \end{equation*}
        \item If we replace $\rho=P\NA/RT$ by its ideal gas value, then we have
        \begin{equation*}
            l = \frac{RT}{\sqrt{2}\NA\sigma P}
            = \frac{\kB T}{\sqrt{2}\sigma\rho}
        \end{equation*}
        \item Now $\kB T$ has units of thermal energy, and we know from physics that $E=F\cdot l$ (energy is force times distance). Thus, $F\propto\sigma\rho$ by the above since $E=\kB T$ and $l=l$.
    \end{itemize}
    \item The probability of a molecular collision.
    \begin{itemize}
        \item The probability that one molecule will suffer a collision is $\sigma\rho dx$.
        \begin{itemize}
            \item This should make intuitive sense as $\sigma$ is the area inside which a molecule must be to collide with some particle, $\rho$ is the density (related to the number of particles likely to be within that area), and $\dd{x}$ tells us over how much space we're moving.
            \item $\sigma\dd{x}$ is a volume.
        \end{itemize}
        \item Let $n(x)$ be the number of molecules that travel a distance $x$ without a collision.
        \item Then the number of molecules that undergo a collision between $x,x+\dd{x}$ is
        \begin{align*}
            n(x)-n(x+\dd{x}) &= \sigma\rho n(x)\dd{x}\\
            \frac{n(x+\dd{x})-n(x)}{\dd{x}} &= -\sigma\rho n(x)\\
            \dv{n}{x} &= -\sigma\rho n
        \end{align*}
    \end{itemize}
\end{itemize}



\section{Collision Frequency and Gas-Phase Reaction Rate}
\begin{itemize}
    \item \marginnote{4/6:}Submit homework in paper next Monday.
    \item Picking up with the probability of a molecular collision from last time.
    \begin{itemize}
        \item Solving the differential equation gives
        \begin{equation*}
            n(x) = n_0\e[-\sigma\rho x]
            = n_0\e[-x/l]
        \end{equation*}
        where $l$ is the mean free path.
        \begin{itemize}
            \item Note that the $\sqrt{2}$ arises from treating every other molecule as static, so we don't need it in this case?
        \end{itemize}
        \item The number of molecules that collide in the interval $x,x+\dd{x}$ is $n(x)-n(x+\dd{x})$.
        \item The probability that one of the initial $n_0$ molecules will collide in this interval is
        \begin{equation*}
            p(x)\dd{x} = \frac{n(x)-n(x+\dd{x})}{n_0}
            = -\frac{1}{n_0}\dv{n}{x}\dd{x}
            = \frac{1}{l}\e[-x/l]\dd{x}
        \end{equation*}
        \item Discussion of Figure 27.12.
        \begin{itemize}
            \item Figure 27.12 does not graph the above equation.
            \item Rather, it graphs the accumulated (integrated) probability from 0 to $x$. We call this function $P(x)$.
            \begin{equation*}
                P(x) = \int_0^xp(x')\dd{x'}
            \end{equation*}
        \end{itemize}
    \end{itemize}
    \item Collision frequency of one particular molecule per unit volume.
    \begin{itemize}
        \item $z_{\ce{A}}$ is the collision frequency of one particular molecule.
        \item $Z_{\ce{A}\ce{A}}$ is the total collision frequency per unit volume.
        \item We have
        \begin{equation*}
            Z_{\ce{A}\ce{A}} = \frac{1}{2}\rho z_{\ce{A}}
        \end{equation*}
        \begin{itemize}
            \item Multiplying by the number density should make intuitive sense.
            \item We divide by two to avoid counting a collision between a pair of similar molecules as two distinct collisions.
        \end{itemize}
        \item It follows that
        \begin{equation*}
            Z_{\ce{A}\ce{A}} = \frac{1}{2}\sigma\prb{u_r}\rho^2
            = \frac{\sigma\prb{u}\rho^2}{\sqrt{2}}
        \end{equation*}
        \item In a gas consisting of two types of molecules, say \ce{A} and \ce{B}, then the collision frequency per unit volume is
        \begin{equation*}
            Z_{\ce{A}\ce{B}} = \sigma_{\ce{A}\ce{B}}\prb{u_r}\rho_{\ce{A}}\rho_{\ce{B}}
        \end{equation*}
        where
        \begin{align*}
            \sigma_{\ce{A}\ce{B}} &= \pi\left( \frac{d_{\ce{A}}+d_{\ce{B}}}{2} \right)^2&
            \prb{u_r} &= \sqrt{\frac{8\kB T}{\pi\mu}}&
            \mu &= \frac{m_{\ce{A}}m_{\ce{B}}}{m_{\ce{A}}+m_{\ce{B}}}
        \end{align*}
        \begin{itemize}
            \item There is no $1/2$ coefficient here because there are also \ce{A}\ce{A} and \ce{B}\ce{B} collisions.
            \item Indeed, $Z_{\ce{A}\ce{B}}$ is not the \emph{total} collision frequency but just the collision frequency of \ce{A}\ce{B} collisions.
        \end{itemize}
    \end{itemize}
    \item The rate of a gas-phase chemical reaction depends on the rate of collisions.
    \begin{itemize}
        \item The rate of collisions is not just the total frequency of collisions.
        \item The relative energy of the two colliding molecules exceeds a certain critical value. This does not show up directly in the equation for $Z_{\ce{A}\ce{B}}$.
        \item The number of collisions per unit time per unit are with the wall by molecules whose speeds are in the range $u,u+\dd{u}$ and whose direction lies within the solid angle $\sin\theta\dd{\theta}\dd{\phi}$ is approximately $u^3\e[-mu^2/2\kB T]$.
        \item We can account for the fact that the molecules collide with each other rather than with a stationary wall by replacing $m$ with the reduced mass $\mu=m_{\ce{A}}m_{\ce{B}}/(m_{\ce{A}}+m_{\ce{B}})$.
        \item The collision frequency per unit volume between molecules \ce{A} and \ce{B} in which they collide with a relative speed between $u,u+\dd{u}$.
        \begin{itemize}
            \item We have that $\dd{Z_{\ce{A}\ce{B}}}\propto u_r^3\e[-\mu u_r^2/2\kB T]\dd{u_r}$. Thus, if \ce{A} is a proportionality constant, then
            \begin{equation*}
                \dd{Z_{\ce{A}\ce{B}}} = Au_r^3\e[-\mu u_r^2/2\kB T]\dd{u_r}
            \end{equation*}
            \item It follows since $Z_{\ce{A}\ce{B}}=\sigma_{\ce{A}\ce{B}}\prb{u_r}\rho_{\ce{A}}\rho_{\ce{B}}$ and $\prb{u_r}=\sqrt{8\kB T/\pi\mu}$ that
            \begin{align*}
                \sigma_{\ce{A}\ce{B}}\rho_{\ce{A}}\rho_{\ce{B}}\sqrt{\frac{8\kB T}{\pi\mu}} &= A\int_0^\infty u_r^3\e[-\mu u_r^2/2\kB T]\dd{u_r}\\
                &= 2A\left( \frac{\kB T}{\mu} \right)^2\\
                A &= \sigma_{\ce{A}\ce{B}}\rho_{\ce{A}}\rho_{\ce{B}}\sqrt{\left( \frac{\mu}{\kB T} \right)^3\cdot\frac{2}{\pi}}
            \end{align*}
            \item Thus, we know that
            \begin{equation*}
                \dd{Z_{\ce{A}\ce{B}}} = \sigma_{\ce{A}\ce{B}}\rho_{\ce{A}}\rho_{\ce{B}}\sqrt{\left( \frac{\mu}{\kB T} \right)^3\cdot\frac{2}{\pi}}\e[-\mu u_r^2/2\kB T]u_r^3\dd{u_r}
            \end{equation*}
        \end{itemize}
        \item Integrating the above from the certain critical value to infinity yields the desired rate.
    \end{itemize}
    \item Key information from this chapter.
    \begin{itemize}
        \item Pressure from a molecular approach.
        \item The distribution for speed components and the speed are different.
        \item The speeds $u_\text{mp}$, $\prb{u}$, and $u_\text{rms}$.
        \item The frequency of collisions per molecule and the total frequency of collisions per volume.
        \item Rate of gas phase reactions.
    \end{itemize}
\end{itemize}



\section{Rate Law Definitions and Methods of Determination}
\begin{itemize}
    \item \marginnote{4/8:}Consider a general chemical equation
    \begin{equation*}
        \ce{\nu_{\ce{A}}A + \nu_{\ce{B}}B -> \nu_{\ce{Y}}Y + \nu_{\ce{Z}}Z}
    \end{equation*}
    \item The extent of the reaction via the progress variable $\xi$ is
    \begin{align*}
        n_{\ce{A}}(t) &= n_{\ce{A}}(0)-\nu_{\ce{A}}\xi(t)&
        n_{\ce{Y}}(t) &= n_{\ce{Y}}(0)+\nu_{\ce{Y}}\xi(t)
    \end{align*}
    \item The rate of change (moles/second) is
    \begin{align*}
        \dv{n_{\ce{A}}}{t} &= -\nu_{\ce{A}}\dv{\xi}{t}&
        \dv{n_{\ce{Y}}}{t} &= \nu_{\ce{Y}}\dv{\xi}{t}
    \end{align*}
    \item Deriving the rate of reaction for a gas-based chemical reaction.
    \begin{itemize}
        \item Time-dependent concentration changes
        \begin{align*}
            \frac{1}{V}\dv{n_{\ce{A}}}{t} &= \dv{[\ce{A}]}{t} = -\frac{\nu_{\ce{A}}}{V}\dv{\xi}{t}&
            \frac{1}{V}\dv{n_{\ce{Y}}}{t} &= \dv{[\ce{Y}]}{t} = -\frac{\nu_{\ce{Y}}}{V}\dv{\xi}{t}
        \end{align*}
        \item The rate (or speed) of reaction, also known as the differential rate law, is
        \begin{equation*}
            v(t) = -\frac{1}{\nu_{\ce{A}}}\dv{[\ce{A}]}{t}
            = -\frac{1}{\nu_{\ce{B}}}\dv{[\ce{B}]}{t}
            = \frac{1}{\nu_{\ce{Y}}}\dv{[\ce{Y}]}{t}
            = \frac{1}{\nu_{\ce{Z}}}\dv{[\ce{Z}]}{t}
            = \frac{1}{V}\dv{\xi}{t}
        \end{equation*}
        \item All terms are positive.
        \item Rate laws with a constant $k$ are of the form
        \begin{equation*}
            v(t) = k[\ce{A}]^{m_{\ce{A}}}[\ce{B}]^{m_{\ce{B}}}
        \end{equation*}
        \item The exponents are known as \textbf{orders}.
        \item The overall order reaction is $\sum m_i$.
        \item The orders and overall order of the reaction depends on the fundamental reaction steps and the reaction mechanism.
    \end{itemize}
    \item For example, for the reaction \ce{2NO_{(g)} + O2_{(g)} -> 2NO2_{(g)}}, we have
    \begin{equation*}
        v(t) = -\frac{1}{2}\dv{[\ce{NO}]}{t}
        = -\dv{[\ce{O2}]}{t}
        = -\frac{1}{2}\dv{[\ce{NO2}]}{t}
    \end{equation*}
    \begin{itemize}
        \item It follows that $v(t)=k[\ce{NO}]^2[\ce{O2}]$.
        \item This is a rare elementary reaction that proceeds with the kinetics illustrated by the equation.
    \end{itemize}
    \item Rate laws must be determined by experiment.
    \begin{itemize}
        \item Multi-step reactions may have more complex rate law expressions.
        \item Oftentimes, $1/2$ exponents indicate more complicated mechanisms.
        \item For example, even an equation as simple looking as \ce{H2 + Br2 -> 2HBr} has rate law
        \begin{equation*}
            v(t) = \frac{k'[\ce{H2}][\ce{Br2}]^{1/2}}{1+k''[\ce{HBr}][\ce{Br2}]^{-1}}
        \end{equation*}
    \end{itemize}
    \item Determining rate laws.
    \begin{itemize}
        \item Method of isolation.
        \begin{itemize}
            \item Put in a large initial excess of \ce{A} so that it's concentration doesn't change that much; essentially incorporates $[\ce{A}]^{m_{\ce{A}}}$ into $k$ for determination of the order of \ce{B}.
            \item We can then do the same thing the other way around.
        \end{itemize}
        \item Method of initial rates.
        \begin{itemize}
            \item We approximate
            \begin{equation*}
                v = -\frac{\dd{[\ce{A}]}}{\nu_{\ce{A}}\dd{t}}
                \approx -\frac{\Delta[\ce{A}]}{\nu_{\ce{A}}\Delta t}
                = k[\ce{A}]^{m_{\ce{A}}}[\ce{B}]^{m_{\ce{B}}}
            \end{equation*}
            \item Consider two different initial values of $[\ce{B}]$, which we'll call $[\ce{B}]_1,[\ce{B}]_2$. Then
            \begin{align*}
                v_1 &= -\frac{1}{\nu_{\ce{A}}}\left( \frac{\Delta[\ce{A}]}{\Delta t} \right)_1 = k[\ce{A}]_0^{m_{\ce{A}}}[\ce{B}]_1^{m_{\ce{B}}}&
                v_2 &= -\frac{1}{\nu_{\ce{A}}}\left( \frac{\Delta[\ce{A}]}{\Delta t} \right)_2 = k[\ce{A}]_0^{m_{\ce{A}}}[\ce{B}]_2^{m_{\ce{B}}}
            \end{align*}
            \item Take the logarithm and solve for $m_{\ce{B}}$.
            \begin{equation*}
                m_{\ce{B}} = \frac{\ln(v_1/v_2)}{\ln([\ce{B}]_1/[\ce{B}]_2)}
            \end{equation*}
        \end{itemize}
    \end{itemize}
    \item Does an example problem.
\end{itemize}



\section{Chapter 27: The Kinetic Theory of Gases}
\emph{From \textcite{bib:McQuarrieSimon}.}
\begin{itemize}
    \item \marginnote{4/10:}\textbf{Mean free path}: The average distance that a molecule travels between collisions. \emph{Denoted by} $\bm{l}$. \emph{Given by}
    \begin{equation*}
        l = \frac{1}{\sqrt{2}\rho\sigma}
    \end{equation*}
    \begin{itemize}
        \item Naturally, the average distance that a molecule travels between collisions is equal to how far it travels per unit time (the average speed) divided by the number of collisions per unit time (the collision frequency). Thus, $l=\prb{u}/z_{\ce{A}}$, which is how the above is derived.
    \end{itemize}
    \item Substituting $\rho=N/V=n\NA/V=P\NA/RT$ yields
    \begin{equation*}
        l = \frac{RT}{\sqrt{2}\NA\sigma P}
    \end{equation*}
    \item Example: At room temperature and one bar, the mean free path of nitrogen is about 200 times the effective diameter of a nitrogen molecule.
    \item An alternate physical interpretation of the probability of a collision.
    \begin{itemize}
        \item Consider a "collision cylinder" with collision cross section of unit area. Let the thickness of this "cylinder" be $\dd{x}$. It follows that the volume of the "collision cylinder" is $1\cdot\dd{x}=\dd{x}$.
        \item Consequently, the number of molecules having center within the collision cylinder is equal to the number density times the volume, or $\rho\dd{x}$.
        \item Thus, if each molecule has target area $\sigma$, then the total target area presented by these molecules (neglecting overlap) is $\sigma\rho\dd{x}$.
        \item Therefore, since the probability of a collision can be thought of as the ration of the total target area to the total area (which we have defined to be unity), the probability of a collision is $\sigma\rho\dd{x}$.
        \begin{itemize}
            \item Note that this squares with the definition of the probability of a collision as $\rho\sigma\prb{u}\dd{t}$ with $\dd{x}=\prb{u}\dd{t}$, as we'd expect.
        \end{itemize}
    \end{itemize}
    \item As we can see, the probability of a collision increases with increasing distance traveled $\dd{x}$.
    \item If $n_0$ molecules are emitted from the origin traveling in the $x$-direction with equal velocity in a volume of unmoving molecules, let $n(x)$ be the number of molecules that travel a distance $x$ without collision.
    \begin{itemize}
        \item It follows that the number of molecules that undergo a collision between $x,x+\dd{x}$ is $n(x)\sigma\rho\dd{x}$.
        \item Furthermore, said number is naturally equal to $n(x)-n(x+\dd{x})$.
        \item Thus, we have that
        \begin{align*}
            n(x)-n(x+\dd{x}) &= \sigma\rho n(x)\dd{x}\\
            \frac{n(x+\dd{x})-n(x)}{\dd{x}} &= -\sigma\rho n(x)\\
            \dv{n}{x} &= -\sigma\rho n\\
            \int_{n_0}^n\frac{\dd{n}}{n} &= -\sigma\rho\int_0^x\dd{x}\\
            \ln(n/n_0) &= -\sigma\rho x\\
            n(x) &= n_0\e[-\sigma\rho x]
        \end{align*}
        where no factor of $\sqrt{2}$ appears because of the assumption that the other molecules do not move.
    \end{itemize}
    \item Therefore, the probability $p(x)\dd{x}$ that one of the initial $n_0$ molecules will collide in the interval $x,x+\dd{x}$ is
    \begin{align*}
        p(x)\dd{x} &= \frac{n(x)-n(x+\dd{x})}{n_0}\\
        &= -\frac{1}{n_0}\dv{n}{x}\dd{x}\\
        &= \frac{1}{l}\e[-x/l]\dd{x}
    \end{align*}
    \begin{itemize}
        \item The above equation is normalized and has $\prb{x}=l$, as expected.
    \end{itemize}
    \item The distance after which half of the molecules will have been scattered from a beam of initially $n_0$ molecules is $l\cdot\ln 2$, i.e., about 70\% of the mean free path.
    \item \textbf{Total collision frequency per unit volume} (for like molecules): The following quantity. \emph{Denoted by} $Z_{\ce{A}\ce{A}}$. \emph{Given by}
    \begin{equation*}
        Z_{\ce{A}\ce{A}} = \frac{1}{2}\rho z_{\ce{A}}
        = \frac{1}{2}\sigma\prb{u_r}\rho^2
        = \frac{\sigma\prb{u}\rho^2}{\sqrt{2}}
    \end{equation*}
    \begin{itemize}
        \item Derived by multiplying the collision frequency for \emph{one} molecule $z_{\ce{A}}$ by the number of molecules per unit volume $\rho$, and dividing by 2 in order to avoid counting a collision between a pair of similar molecules as two distinct collisions.
    \end{itemize}
    \item \textbf{Total collision frequency per unit volume} (for dislike molecules): The following quantity. \emph{Denoted by} $Z_{\ce{A}\ce{B}}$. \emph{Given by}
    \begin{equation*}
        Z_{\ce{A}\ce{B}} = \sigma_{\ce{A}\ce{B}}\prb{u_r}\rho_{\ce{A}}\rho_{\ce{B}}
    \end{equation*}
    \item The discussion of the rate of gas-phase chemical reactions is nearly identical to that given in class.
\end{itemize}



\section{Chapter 28: Chemical Kinetics I --- Rate Laws}
\emph{From \textcite{bib:McQuarrieSimon}.}
\begin{itemize}
    \item \marginnote{4/15:}Whereas \textcite{bib:McQuarrieSimon} developed Quantum Mechanics from a set of simple postulates and Thermodynamics from the three laws, "the field of chemical kinetics has not yet matured to a point where a set of unifying principles has been identified" \parencite[1047]{bib:McQuarrieSimon}.
    \begin{itemize}
        \item There are many current theoretical models of kinetics, each of which has its merits and drawbacks.
        \item Thus, right now, it is necessary to familiarize ourselves with numerous disparate ideas, as is common in developing fields of inquiry.
    \end{itemize}
    \item \textbf{Rate law}: A differential equation describing the time-dependence of the reactant and product concentrations during a chemical reaction.
    \item Consider the general chemical reaction described by
    \begin{equation*}
        \ce{$\nu_{\text{A}}$A + $\nu_{\text{B}}$B -> $\nu_{\text{Y}}$Y + $\nu_{\text{Z}}$Z}
    \end{equation*}
    \item Since
    \begin{align*}
        n_{\ce{A}}(t) &= n_{\ce{A}}(0)-\nu_{\ce{A}}\xi(t)&
        n_{\ce{B}}(t) &= n_{\ce{B}}(0)-\nu_{\ce{B}}\xi(t)&
        n_{\ce{Y}}(t) &= n_{\ce{Y}}(0)+\nu_{\ce{Y}}\xi(t)&
        n_{\ce{Z}}(t) &= n_{\ce{Z}}(0)+\nu_{\ce{Z}}\xi(t)
    \end{align*}
    we can describe the time-dependent change in the number of moles of each substance by taking a derivative with respect to $t$, as follows.
    \begin{align*}
        \dv{n_{\ce{A}}}{t} &= -\nu_{\ce{A}}\dv{\xi}{t}&
        \dv{n_{\ce{B}}}{t} &= -\nu_{\ce{B}}\dv{\xi}{t}&
        \dv{n_{\ce{Y}}}{t} &= \nu_{\ce{Y}}\dv{\xi}{t}&
        \dv{n_{\ce{Z}}}{t} &= \nu_{\ce{Z}}\dv{\xi}{t}
    \end{align*}
    \item Since most experimental techniques measure concentration, it is convenient to divide the above equations by the total volume $V$ on both sides to yield the following.
    \begin{align*}
        \dv{[\ce{A}]}{t} &= -\frac{\nu_{\ce{A}}}{V}\dv{\xi}{t}&
        \dv{[\ce{B}]}{t} &= -\frac{\nu_{\ce{B}}}{V}\dv{\xi}{t}&
        \dv{[\ce{Y}]}{t} &= \frac{\nu_{\ce{Y}}}{V}\dv{\xi}{t}&
        \dv{[\ce{Z}]}{t} &= \frac{\nu_{\ce{Z}}}{V}\dv{\xi}{t}
    \end{align*}
    \item While each individual quantity above has its purpose, it is useful to define an overall \textbf{rate of reaction}.
    \item \textbf{Rate of reaction}: The following quantity. \emph{Denoted by} $\bm{v(t)}$. \emph{Given by}
    \begin{align*}
        v(t) &= \frac{1}{V}\dv{\xi}{t}\\
        &= -\frac{1}{\nu_{\ce{A}}}\dv{[\ce{A}]}{t}
            = -\frac{1}{\nu_{\ce{B}}}\dv{[\ce{B}]}{t}
            = \frac{1}{\nu_{\ce{Y}}}\dv{[\ce{Y}]}{t}
            = \frac{1}{\nu_{\ce{Z}}}\dv{[\ce{Z}]}{t}
    \end{align*}
    \begin{itemize}
        \item Note that the rate of reaction is always positive (as long as the reaction proceeds only in the forward direction).
    \end{itemize}
    \item \textbf{Rate law}: The relationship between $v(t)$ and the concentrations of the various reactants. \emph{General form}
    \begin{equation*}
        v(t) = k[\ce{A}]^{m_{\ce{A}}}[\ce{B}]^{m_{\ce{B}}}\cdots
    \end{equation*}
    \begin{itemize}
        \item Some reactions (such as the \ce{H2 + Br2 -> 2HBR} example from class) do not have conventional rate laws.
    \end{itemize}
    \item \textbf{Rate constant}: The proportionality constant between the rate of reaction and the function of the concentrations of the chemical species involved in a rate law. \emph{Denoted by} $\bm{k}$.
    \begin{itemize}
        \item The units of the rate constant depend on the form of the rate law.
    \end{itemize}
    \item \textbf{Order} (of a reactant \ce{A}): The power to which the concentration of a reactant is raised in a rate law. \emph{Denoted by} $\bm{m_{\ce{A}}}$.
    \item \textbf{Overall order} (of a chemical reaction that obeys a general-form rate law): The sum of the orders of the reactants.
    \item We now discuss common methods for the experimental determination of a rate law.
    \item \textbf{Method of isolation}: The following procedure, which as described will determine $m_{\ce{B}}$ for a chemical reaction of the form introduced at the beginning of this section but can easily be adapted to determine $m_{\ce{A}}$ or be generalized to higher-order situations.
    \begin{enumerate}
        \item Introduce a large excess concentration of \ce{A} into the initial reaction mixture. This excess will guarantee that $[\ce{A}]$ remains essentially constant over the course of the reaction.
        \item Combine $[\ce{A}]^{m_{\ce{A}}}$ and $k$ into a new "rate constant" $k'$, reducing the rate law to the form
        \begin{equation*}
            v = k'[\ce{B}]^{m_{\ce{B}}}
        \end{equation*}
        \item Determine $m_{\ce{B}}$ by measuring $v$ as a function of $[\ce{B}]$.
    \end{enumerate}
    \item Sometimes it is not possible to have one reactant or the other in excess.
    \begin{itemize}
        \item As such, we need an alternate way to measure the rate.
        \item We cannot directly measure $\dv*{[\ce{A}]}{t}$, but we can measure $\Delta[\ce{A}]/\Delta t$ for small $\Delta t$ and approximate these measurements as $\dv*{[\ce{A}]}{t}$.
        \item This forms the basis for the \textbf{method of initial rates}.
    \end{itemize}
    \item \textbf{Method of initial rates}: The following procedure, which as described will determine $m_{\ce{B}}$ for a chemical reaction of the form introduced at the beginning of this section but can easily be adapted to determine $m_{\ce{A}}$ or be generalized to higher-order situations.
    \begin{enumerate}
        \item Take two different measurements of the initial rate (from $t=0$ to $t=t$). Let the initial concentration of \ce{A}, $[\ce{A}]_0$, be the same for each. However, for one, use $[\ce{B}]_1$ for initial concentration of \ce{B}, and for the other, use $[\ce{B}]_2$.
        \item Arranging everything into equations, we thus have
        \begin{align*}
            v_1 &= -\frac{1}{\nu_{\ce{A}}}\left( \frac{\Delta[\ce{A}]}{\Delta t} \right)_1
            = k[\ce{A}]_0^{m_{\ce{A}}}[\ce{B}]_1^{m_{\ce{B}}}&
            v_2 &= -\frac{1}{\nu_{\ce{A}}}\left( \frac{\Delta[\ce{A}]}{\Delta t} \right)_2
            = k[\ce{A}]_0^{m_{\ce{A}}}[\ce{B}]_2^{m_{\ce{B}}}
        \end{align*}
        where we have used the subscripts 1 and 2 to denote the results of the different experiments and their corresponding initial concentrations of \ce{B}.
        \item We may now solve for $m_{\ce{B}}$ by dividing the two equations, taking logarithms, and rearranging to the following.
        \begin{equation*}
            m_{\ce{B}} = \frac{\ln(v_1/v_2)}{\ln([\ce{B}]_1/[\ce{B}]_2)}
        \end{equation*}
    \end{enumerate}
    \item Both the method of isolation and the method of initial rates rely on the assumption that the reactants can be mixed, and then we can measure the rates.
    \begin{itemize}
        \item However, for some very quick reactions, the time required to mix the reactants is long compared with the reaction itself.
        \item For these cases, we need \textbf{relaxation methods}.
    \end{itemize}
\end{itemize}




\end{document}