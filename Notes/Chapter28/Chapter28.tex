\documentclass[../notes.tex]{subfiles}

\pagestyle{main}
\renewcommand{\chaptermark}[1]{\markboth{\chaptername\ \thechapter\ (#1)}{}}
\setcounter{chapter}{27}

\begin{document}




\chapter{Rate Laws}
\section{Definitions and Methods of Determination}
\begin{itemize}
    \item \marginnote{4/8:}Consider a general chemical equation
    \begin{equation*}
        \ce{\nu_AA + \nu_BB -> \nu_YY + \nu_ZZ}
    \end{equation*}
    \item The extent of the reaction via the progress variable $\xi$ is
    \begin{align*}
        n_A(t) &= n_A(0)-\nu_A\xi(t)&
        n_Y(t) &= n_Y(0)+\nu_Y\xi(t)
    \end{align*}
    \item The rate of change (moles/second) is
    \begin{align*}
        \dv{n_A}{t} &= -\nu_A\dv{\xi}{t}&
        \dv{n_Y}{t} &= \nu_Y\dv{\xi}{t}
    \end{align*}
    \item Deriving the rate of reaction for a gas-based chemical reaction.
    \begin{itemize}
        \item Time-dependent concentration changes
        \begin{align*}
            \frac{1}{V}\dv{n_A}{t} &= \dv{[A]}{t} = -\frac{\nu_A}{V}\dv{\xi}{t}&
            \frac{1}{V}\dv{n_Y}{t} &= \dv{[Y]}{t} = -\frac{\nu_Y}{V}\dv{\xi}{t}
        \end{align*}
        \item The rate (or speed) of reaction, also known as the differential rate law, is
        \begin{equation*}
            v(t) = -\frac{1}{\nu_A}\dv{[A]}{t}
            = -\frac{1}{\nu_B}\dv{[B]}{t}
            = \frac{1}{\nu_Y}\dv{[Y]}{t}
            = \frac{1}{\nu_Z}\dv{[Z]}{t}
            = \frac{1}{V}\dv{\xi}{t}
        \end{equation*}
        \item All terms are positive.
        \item Rate laws with a constant $k$ are of the form
        \begin{equation*}
            v(t) = k[A]^{m_A}[B]^{m_B}
        \end{equation*}
        \item The exponents are known as \textbf{orders}.
        \item The overall order reaction is $\sum m_i$.
        \item The orders and overall order of the reaction depends on the fundamental reaction steps and the reaction mechanism.
    \end{itemize}
    \item For example, for the reaction \ce{2NO_{(g)} + O2_{(g)} -> 2NO2_{(g)}}, we have
    \begin{equation*}
        v(t) = -\frac{1}{2}\dv{[\ce{NO}]}{t}
        = -\dv{[\ce{O2}]}{t}
        = -\frac{1}{2}\dv{[\ce{NO2}]}{t}
    \end{equation*}
    \begin{itemize}
        \item It follows that $v(t)=k[\ce{NO}]^2[\ce{O2}]$.
        \item This is a rare elementary reaction that proceeds with the kinetics illustrated by the equation.
    \end{itemize}
    \item Rate laws must be determined by experiment.
    \begin{itemize}
        \item Multi-step reactions may have more complex rate law expressions.
        \item Oftentimes, $1/2$ exponents indicate more complicated mechanisms.
        \item For example, even an equation as simple looking as \ce{H2 + Br2 -> 2HBr} has rate law
        \begin{equation*}
            v(t) = \frac{k'[\ce{H2}][\ce{Br2}]^{1/2}}{1+k''[\ce{HBr}][\ce{Br2}]^{-1}}
        \end{equation*}
    \end{itemize}
    \item Determining rate laws.
    \begin{itemize}
        \item Method of isolation.
        \begin{itemize}
            \item Put in a large initial excess of $A$ so that it's concentration doesn't change that much; essentially incorporates $[A]^{m_A}$ into $k$ for determination of the order of $B$.
            \item We can then do the same thing the other way around.
        \end{itemize}
        \item Method of initial rates.
        \begin{itemize}
            \item We approximate
            \begin{equation*}
                v = -\frac{\dd{[A]}}{\nu_A\dd{t}}
                \approx -\frac{\Delta[A]}{\nu_A\Delta t}
                = k[A]^{m_A}[B]^{m_B}
            \end{equation*}
            \item Consider two different initial values of $[B]$, which we'll call $[B_1],[B_2]$. Then
            \begin{align*}
                v_1 &= -\frac{1}{\nu_A}\left( \frac{\Delta[A]}{\Delta t} \right)_1 = k[A]_0^{m_A}[B]_1^{m_B}&
                v_2 &= -\frac{1}{\nu_A}\left( \frac{\Delta[A]}{\Delta t} \right)_2 = k[A]_0^{m_A}[B]_2^{m_B}
            \end{align*}
            \item Take the logarithm and solve for $m_B$.
            \begin{equation*}
                m_B = \frac{\ln(v_1/v_2)}{\ln([B]_1/[B]_2)}
            \end{equation*}
        \end{itemize}
    \end{itemize}
    \item Does an example problem.
\end{itemize}



\section{Integrated Rate Laws}
\begin{itemize}
    \item \marginnote{4/11:}First order reactions have exponential integrated rate laws.
    \begin{itemize}
        \item Suppose \ce{A + B -> products}.
        \item Suppose the reaction is first order in \ce{A}.
        \item If the concentration of \ce{A} is $[\ce{A}]_0$ at $t=0$ and $[\ce{A}]$ at time $t$, then
        \begin{align*}
            v(t) = -\dv{[\ce{A}]}{t} &= k[\ce{A}]\\
            \int_{[\ce{A}]_0}^{[\ce{A}]}\frac{\dd{[\ce{A}]}}{[\ce{A}]} &= -\int_0^tk\dd{t}\\
            \ln\frac{[\ce{A}]}{[\ce{A}]_0} &= -kt\\
            [\ce{A}] &= [\ce{A}]_0\e[-kt]
        \end{align*}
        is the integrated rate law.
        \item Goes over both the concentration plot and the linear logarithmic plot.
    \end{itemize}
    \item The half-life of a first-order reaction is independent of the initial amount of reactant.
    \begin{itemize}
        \item The half-life is found from the point
        \begin{equation*}
            [\ce{A}(t_{1/2})] = \frac{[\ce{A}(0)]}{2}
            = \frac{[\ce{A}]_0}{2}
        \end{equation*}
        \item We have
        \begin{align*}
            \ln\frac{1}{2} &= -kt_{1/2}\\
            t_{1/2} &= \frac{\ln 2}{k} \approx \frac{0.693}{k}
        \end{align*}
        \item Notice that the above equation does not depend on $[\ce{A}]$ or $[\ce{B}]$!
    \end{itemize}
    \item Second order reactions have inverse concentration integrated rate laws.
    \begin{itemize}
        \item Suppose \ce{A + B -> products}, as before, and that the reaction is second order in \ce{A}.
        \item Then
        \begin{align*}
            -\dv{[\ce{A}]}{t} &= k[\ce{A}]^2\\
            \int_{[\ce{A}]_0}^{[\ce{A}]}-\frac{\dd{[\ce{A}]}}{[\ce{A}]^2} &= \int_0^tk\dd{t}\\
            \frac{1}{[\ce{A}]} &= \frac{1}{[\ce{A}]_0}+kt
        \end{align*}
        is the integrated rate law.
    \end{itemize}
    \item The half-life of a second-order reaction is dependent on the initial amount of reaction.
    \begin{itemize}
        \item We have that
        \begin{align*}
            \frac{1}{[\ce{A}]_0/2} &= \frac{1}{[\ce{A}]_0}+kt_{1/2}\\
            \frac{1}{[\ce{A}]_0} &= kt_{1/2}\\
            t_{1/2} &= \frac{1}{k[\ce{A}]_0}
        \end{align*}
    \end{itemize}
    \item If a reaction is $n^\text{th}$-order in a reactant for $n\geq 2$, then the integrated rate law is given by
    \begin{align*}
        -\dv{[\ce{A}]}{t} &= k[\ce{A}]^n\\
        \int_{[\ce{A}]_0}^{[\ce{A}]}-\frac{\dd{[\ce{A}]}}{[\ce{A}]^n} &= \int_0^tk\dd{t}\\
        \frac{1}{n-1}\left( \frac{1}{[\ce{A}]^{n-1}}-\frac{1}{[\ce{A}]_0^{n-1}} \right) &= kt
    \end{align*}
    \begin{itemize}
        \item The associated half life is
        \begin{align*}
            \frac{1}{n-1}\left( \frac{1}{([\ce{A}]_0/2)^{n-1}}-\frac{1}{[\ce{A}]_0^{n-1}} \right) &= kt_{1/2}\\
            \frac{1}{n-1}\cdot\frac{2^{n-1}-1}{[\ce{A}]_0^{n-1}} &= kt_{1/2}\\
            t_{1/2} &= \frac{2^{n-1}-1}{k(n-1)[\ce{A}]_0^{n-1}}
        \end{align*}
    \end{itemize}
    \item Second order reactions that are first order in each reactant.
    \begin{itemize}
        \item We have that
        \begin{align*}
            -\dv{[\ce{A}]}{t} = -\dv{[\ce{B}]}{t} &= k[\ce{A}][\ce{B}]\\
            kt &= \frac{1}{[\ce{A}]_0-[\ce{B}]_0}\ln\frac{[\ce{A}][\ce{B}]_0}{[\ce{B}][\ce{A}]_0}
        \end{align*}
        \item The actual determination is more complicated (there is a textbook problem that walks us through the derivation, though).
        \item When $[\ce{A}]_0=[\ce{B}]_0$, the integrated rate law simplifies to the second-order integrated rate laws in $[\ce{A}]$ and $[\ce{B}]$.
        \begin{itemize}
            \item In this limited case, the half-life is that of the second-order integrated rate law, too, i.e., $t_{1/2}=1/k[\ce{A}]_0$.
        \end{itemize}
    \end{itemize}
    \item The reaction paths and mechanism for parallel reactions.
    \begin{itemize}
        \item Suppose \ce{A} can become both \ce{B} and \ce{C} with respective rate constants $k_B$ and $k_C$.
        \item Then
        \begin{align*}
            \dv{[\ce{A}]}{t} &= -k_B[\ce{A}]-k_C[\ce{A}] = -(k_B+k_C)[\ce{A}]&
            \dv{[\ce{B}]}{t} &= k_B[\ce{A}]&
            \dv{[\ce{C}]}{t} &= k_C[\ce{A}]
        \end{align*}
        \item The integrated rate laws here are
        \begin{align*}
            [\ce{A}] &= [\ce{A}]_0\e[-(k_B+k_C)t]&
            [\ce{B}] &= \frac{k_B}{k_B+k_C}[\ce{A}]_0\left( 1-\e[-(k_B+k_C)t] \right)&
            [\ce{C}] &= \frac{k_C}{k_B+k_C}[\ce{A}]_0\left( 1-\e[-(k_B+k_C)t] \right)
        \end{align*}
        \item The ratio of product concentrations is
        \begin{equation*}
            \frac{[\ce{B}]}{[\ce{C}]} = \frac{k_B}{k_C}
        \end{equation*}
        \item The yield $\Phi_i$ is the probability that a given product $i$ will be formed from the decay of the reactant
        \begin{align*}
            \Phi_i &= \frac{k_i}{\sum_nk_n}&
            \sum_i\Phi_i &= 1
        \end{align*}
    \end{itemize}
    \item Example: If we have parallel reactions satisfying $k_B=2k_C$, then
    \begin{equation*}
        \Phi_C = \frac{k_C}{k_B+k_C}
        = \frac{k_C}{2k_C+k_C}
        = \frac{1}{3}
    \end{equation*}
\end{itemize}



\section{Office Hours (Tian)}
\begin{itemize}
    \item Why does the reduced mass work in the collision frequency derivation?
    \begin{itemize}
        \item We need to start with a simpler case, or the problem will be really hard; thus, we begin by assuming that the particles all are static.
        \item We use the reduced mass to consider the relative speed $u_r$ of the particles with respect to the moving particle as our reference frame. So all the others are the relative speeds to our particle. But this necessitates using the relative mass of the particles to our particle (which is the reduced mass).
    \end{itemize}
\end{itemize}



\section{Reversible Reactions}
\begin{itemize}
    \item \marginnote{4/13:}Let
    \begin{equation*}
        \ce{A <=>[$k_1$][$k_{-1}$] B}
    \end{equation*}
    be a reversible reaction, where $k_1$ is the rate constant for the forward reaction and $k_{-1}$ is the rate constant for the reverse reaction.
    \begin{itemize}
        \item In this case, we have an equilibrium constant expression
        \begin{equation*}
            K_c = \frac{[\ce{B}]_\text{eq}}{[\ce{A}]_\text{eq}}
        \end{equation*}
        \item Additionally, the kinetic conditions for equilibrium are
        \begin{equation*}
            -\dv{[\ce{A}]}{t} = \dv{[\ce{B}]}{t} = 0
        \end{equation*}
        \item If the reaction is first order in both $[\ce{A}]$ and $[\ce{B}]$, then
        \begin{equation*}
            -\dv{[\ce{A}]}{t} = k_1[\ce{A}]-k_{-1}[\ce{B}]
        \end{equation*}
        \item If $[\ce{A}]=[\ce{A}]_0$ at $t=0$, then $[\ce{B}]=[\ce{A}]_0-[\ce{A}]$ and
        \begin{equation*}
            -\dv{[\ce{A}]}{t} = (k_1+k_{-1})[\ce{A}]-k_{-1}[\ce{A}]_0
        \end{equation*}
        \begin{itemize}
            \item Note that $[\ce{B}]=[\ce{A}]_0-[\ce{A}]$ iff there is no initial concentration of \ce{B}, the initial equation was balanced (i.e., each unit of \ce{A} forms one unit of \ce{B}), and there is not another component \ce{C} into which \ce{A} decomposes.
        \end{itemize}
        \item Integrating yields
        \begin{equation*}
            [\ce{A}] = ([\ce{A}]_0-[\ce{A}]_\text{eq})\e[-(k_1+k_{-1})t]+[\ce{A}]_\text{eq}
        \end{equation*}
        \begin{itemize}
            \item Note that this equation reduces to the irreversible first order equation as $k_{-1}\to 0$ and hence $[\ce{A}]_\text{eq}\to 0$ as well.
            \item Similarly, if only the reverse reaction takes place (and we have no initial concentration of \ce{B}), then $[\ce{A}]=[\ce{A}]_\text{eq}$ and the above equation reduces to exactly that statement, as desired.
        \end{itemize}
        \item Since
        \begin{equation*}
            \ln([\ce{A}]-[\ce{A}]_\text{eq}) = \ln([\ce{A}]_0-[\ce{A}]_\text{eq})-(k_1+k_{-1})t
        \end{equation*}
        we have a straight line that allows us to determine the sum $k_1+k_{-1}$. However, we cannot determine each term individually from the above.
        \item One way that we can is by noting that at equilibrium, $\dv*{[\ce{A}]}{t}=0$, so the differential rate law reduces to
        \begin{equation*}
            k_1[\ce{A}]_\text{eq} = k_{-1}[\ce{B}]_\text{eq}
        \end{equation*}
        \item Another way we can resolve each term individually is by noting that
        \begin{equation*}
            \frac{k_1}{k_{-1}} = \frac{[\ce{B}]_\text{eq}}{[\ce{A}]_\text{eq}}
            = K_c
        \end{equation*}
    \end{itemize}
    \item \textbf{Stopped flow method}: Fast mixing of reactants.
    \begin{itemize}
        \item The limit is about 1 microsecond time resolution (mixing rate).
        \item Lots of issues?
    \end{itemize}
    \item \textbf{Pump-probe method}: An optical/IR method that ranges from femtoseconds to nanoseconds.
    \begin{itemize}
        \item Nobel Prize (1999) to Zewail "for his studies of the transition states of chemical reactions using femtosecond spectroscopy."
    \end{itemize}
    \item \textbf{Perturbation-Relaxation method}: You perturb a thermodynamic variable (e.g., $T$, $P$, $\pH$, etc.) and then follow the kinetics of relaxation of the system to equilibrium.
    \begin{figure}[H]
        \centering
        \begin{tikzpicture}[scale=1.1]
            \small
            \draw (0,3.5) -- (0,0) -- node[below=4mm]{$t$} (5,0);
    
            \footnotesize
            \draw [very thin,dashed]
                (1.5,0) -- ++(0,2.5)
                (0,1) -- ++(5,0)
                (2.2,2.5) -- ++(0,-1.15)
            ;
            \node [below] at (1.5,0) {0};
            \node [left]  at (0,1)   {$[\ce{B}]_{2,\text{eq}}$};
            \node [left]  at (0,2.5) {$[\ce{B}]_{1,\text{eq}}$};
    
            \draw [blx,thick] (0,2.5)
                -- node[black,above,align=center,text width=1.5cm]{Initial equilibrium state} ++(1.5,0)
                % plot[domain=1.5:3.5,smooth] (\x,{2.5+1.53*(e^(-1*(2*(\x-1.5)))-1)})
                to[out=-71.9,in=180,looseness=1.2] (3.5,1)
                -- node[black,below,align=center,text width=1.5cm]{Final equilibrium state} ++(1.5,0)
            ;
    
            \scriptsize
            \begin{scope}[line width=0.3pt,<->,shorten <=1pt,shorten >=1pt]
                \draw (1,1) -- node[fill=white,inner sep=1.5pt]{$\Delta[\ce{B}]_0$} ++(0,1.5);
                \draw [shorten >=2pt] (1.85,1) -- ++(0,0.7);
                \node at (1.85,0.5) {$\Delta[\ce{B}]$}
                    edge[out=90,in=180,very thin,->,in looseness=2] (1.85,1.35)
                ;
                \draw (1.5,2.5) -- ++(0.7,0);
                \node [align=center] at (3.8,2.5) {\scriptsize Relaxation time\\$\tau=1/(k_1+k_{-1})$}
                    edge[out=180,in=90,very thin,->,in looseness=2] (1.85,2.5)
                ;
                \node at (3.5,3) {\scriptsize Temperature jump}
                    edge[out=180,in=90,very thin,->,in looseness=1.2] (1.5,2.5)
                ;
            \end{scope}
        \end{tikzpicture}
        \caption{Relaxation methods to determine rate constants.}
        \label{fig:relaxation}
    \end{figure}
    \begin{itemize}
        \item Nobel Prize (1967) to Porter, Norrish, and Eigen "for their studies of extremely fast chemical reactions, effected by disturbing the equilibrium by means of very short pulses of energy."
        \item Example: Consider water autoionization. Here, we'd perturb $\pH$ and $T$.
        \item Our initial point is the first equilibrium condition; out final point is the second equilibrium condition (i.e., that with the perturbed variables).
        \item We should have
        \begin{align*}
            [\ce{A}] &= [\ce{A}]_{2,\text{eq}}+\Delta[\ce{A}]&
            [\ce{B}] &= [\ce{B}]_{2,\text{eq}}+\Delta[\ce{B}]
        \end{align*}
        so that
        \begin{equation*}
            \dv{\Delta[\ce{B}]}{t} = k_1[\ce{A}]_{2,\text{eq}}+k_1\Delta[\ce{A}]-k_{-1}[\ce{B}]_{2,\text{eq}}-k_{-1}\Delta[\ce{B}]
        \end{equation*}
        \item The sum of the concentrations is constant, so $\Delta([\ce{A}]+[\ce{B}])=\Delta[\ce{A}]+\Delta[\ce{B}]=0$.
        \item Additionally, detailed balance is satisfied.
        \begin{equation*}
            k_1[\ce{A}]_{2,\text{eq}} = k_{-1}[\ce{B}]_{2,\text{eq}}
        \end{equation*}
        \item As a result,
        \begin{equation*}
            \dv{\Delta[\ce{B}]}{t} = -(k_1+k_{-1})\Delta[\ce{B}]
        \end{equation*}
        \item Integrating yields
        \begin{equation*}
            \Delta[\ce{B}]_0 = [\ce{B}]_{1,\text{eq}}-[\ce{B}]_{2,\text{eq}}
            = \Delta[\ce{B}]_0\e[-t/\tau]
        \end{equation*}
        where $\tau$ is the \textbf{relaxation time}.
        \item It follows that
        \begin{equation*}
            \Delta[\ce{B}] = \Delta[\ce{B}]_0\e[-(k_1+k_{-1})t]
        \end{equation*}
        \item Some textbooks use different notation; we should know this, too.
        \begin{itemize}
            \item They denote by $\xi$ or $\xi_0$ the difference between $[\ce{A}]$ (the initial equilibrium's concentration) and $[\ce{A}]_{eq}$ (the final equilibrium's concentration).
            \item They also use $k_A,k_B$ for the initial equilibrium \ce{A <=>[$k_A$][$k_B$] B} and $k_A^+,k_B^+$ for the final equilibrium \ce{A <=>[$k_A^+$][$k_B^+$] B}.
        \end{itemize}
    \end{itemize}
    \item \textbf{Relaxation time}: The following quantity. \emph{Denoted by} $\bm{\tau}$. \emph{Given by}
    \begin{equation*}
        \tau = \frac{1}{k_1+k_{-1}}
    \end{equation*}
    \item We'll start with water dissociation next lecture.
\end{itemize}



\section{Water Dissociation, Temperature Dependence, and TST}
\begin{itemize}
    \item \marginnote{4/15:}\textbf{T-jump}: A temperature perturbation.
    \item Relaxation methods and water dissociation.
    \begin{itemize}
        \item Consider the equilibrium
        \begin{equation*}
            \ce{H2O <=>[$k_f$][$k_r$] H+ + OH-}
        \end{equation*}
        \item The differential rate laws are
        \begin{align*}
            \dv{[\ce{H2O}]}{t} &= -k_f[\ce{H2O}]+k_r[\ce{H+}][\ce{OH-}]&
            \dv{[\ce{H+}]}{t} &= k_f[\ce{H2O}]-k_r[\ce{H+}][\ce{OH-}]
        \end{align*}
        \item After the T-jump, the system relaxes to a new equilibrium
        \begin{equation*}
            K_c = \frac{k_f^+}{k_r^+}
            = \frac{[\ce{H+}]_{eq}[\ce{OH-}]_{eq}}{[\ce{H2O}]_{eq}}
        \end{equation*}
        \item It follows that
        \begin{align*}
            \dv{\xi}{t} &= -k_f^+[\ce{H2O}]+k_r^+[\ce{H+}][\ce{OH-}]\\
            &= -k_f^+\xi-k_r^+\xi([\ce{H+}]_{eq}+[\ce{OH-}]_{eq})+\text{O}(\xi^2)
        \end{align*}
        \begin{itemize}
            \item Note that we get from the first line to the second by substituting $[\ce{H+}]=[\ce{H+}]_{eq}-\xi$ and $[\ce{OH-}]=[\ce{OH-}]_{eq}-\xi$ and expanding.
        \end{itemize}
        \item The associated relaxation time is
        \begin{equation*}
            \frac{1}{\tau} = k_f^++k_r^+([\ce{H+}]_{eq}+[\ce{OH-}]_{eq})
        \end{equation*}
        \begin{itemize}
            \item Note that this implies that this relaxation time can be measured experimentally.
        \end{itemize}
    \end{itemize}
    \item Rates of reaction depend on temperature.
    \item The empirical temperature dependence of the rate constant $k$ is given by
    \begin{equation*}
        \dv{\ln k}{T} = \frac{E_a}{RT^2}
    \end{equation*}
    \begin{itemize}
        \item If the activation energy is independent of temperature, then
        \begin{align*}
            \ln k &= \ln A-\frac{E_a}{RT}\\
            k &= A\e[-E_a/RT]
        \end{align*}
        i.e., we get the Arrhenius equation.
        \item If we obtain two rate constants at two temperatures, we can get
        \begin{equation*}
            \ln\frac{k_1}{k_2} = \frac{E_a}{R}\left( \frac{1}{T_2}-\frac{1}{T_1} \right)
        \end{equation*}
        \item Note that plots of $k$ vs. $1/T$ can be nonlinear if the prefactor or "encounter frequency" is temperature-dependent, i.e., if we have
        \begin{equation*}
            k = aT^m\e[-E'/RT]
        \end{equation*}
        where $a, E'$, and $m$ are temperature-independent constants.
    \end{itemize}
    \item Using Transition State Theory (TST) to estimate rate constants.
    \begin{itemize}
        \item Let the following be a chemical reaction and its rate law.
        \begin{align*}
            \ce{A + B} &\ce{->} \ce{P}&
            \dv{[\ce{P}]}{t} = k[\ce{A}][\ce{B}]
        \end{align*}
        \item Suppose that the reaction proceeds by way of a special intermediate species, the activated complex.
        \begin{equation*}
            \ce{A + B <=> AB${}^\ddagger$ -> P}
        \end{equation*}
        \item We know that
        \begin{equation*}
            K_c^\ddagger = \frac{[\ce{AB}^\ddagger]/c^\circ}{[\ce{A}]/c^\circ[\ce{B}]/c^\circ}
            = \frac{[\ce{AB}^\ddagger]c^\circ}{[\ce{A}][\ce{B}]}
        \end{equation*}
        where $c^\circ$ is the standard-state concentration.
        \item Write the equilibrium constant expression in terms of the partition functions $q_{\ce{A}}$, $q_{\ce{B}}$, and $q^\ddagger$ for \ce{A}, \ce{B}, and $\ce{AB}^\ddagger$.
        \begin{equation*}
            K_c^\ddagger = \frac{(q^\ddagger/V)c^\circ}{(q_{\ce{A}}/V)(q_{\ce{B}}/V)}
        \end{equation*}
        \item If $\nu_c$ is the frequency of crossing the barrier top, then
        \begin{equation*}
            \dv{[\ce{P}]}{t} = \nu_c[\ce{AB}^\ddagger]
        \end{equation*}
        \item Thus, we can relate
        \begin{equation*}
            k = \frac{\nu_cK_c^\circ}{c^\circ}
        \end{equation*}
    \end{itemize}
    \item 2 hour midterm at the end of this month (April) taken at home.
\end{itemize}



\section{Office Hours (Tian)}
\begin{itemize}
    \item Stopped flow method?
    \begin{itemize}
        \item Two syringes have substances that get mixed and then become a homogeneous mixture where they start to do all of the interesting chemistry. Before the substances enter the chamber, though, they pass by a detector that monitors the concentration of the initial species. Concentration is measured after good mixing.
        \item It is called \emph{stopped} flow because we want to fix the initial concentration of \ce{A} and \ce{B}. Inject them, let them mix, stop the flow, measure the concentration, and then let the chemistry proceed.
        \item Only used if mixing is much faster than reaction.
        \item This is the experimental set-up for the method of initial rates or the method of exhaustion.
        \item Caveats/issues: Approximating $\Delta t$ as $\dd{t}$.
    \end{itemize}
    \item TST diagram lines?
    \begin{itemize}
        \item The quantized states lines refer to the energy levels of the reactants and products summarized by the partition function.
        \item The reactants reach the activated complex just at some higher quantized energy state!
    \end{itemize}
    \item Physical interpretation of $\tau$ beyond the time it takes the initial reactants to reach $1/\e$ of their initial concentration.
    \begin{itemize}
        \item You wanna see how fast the transition/relaxation would be, and $\tau$ is just a measure of how fast the transition goes.
        \item Also relates to $k_1$ and $k_{-1}$.
        \item Think in terms of adaptability (biological systems). Relation to how fast you can adapt to things like new temperature changes. We want to adapt to environmental changes as fast as possible.
        \item A measure of adaptability, response time, and smart materials that labs are developing to respond to changes very quickly. Also instrumentation response time (which you want to be very fast).
        \item Sometimes, you don't want to adapt to changes too quickly (such as cold-blooded animals).
    \end{itemize}
    \item Importance of Chapter 24 (or 26, depending on edition)?
    \begin{itemize}
        \item Good to know general stuff/big picture ideas as a prerequisite.
        \item Don't worry about specific things tho.
    \end{itemize}
    \item Pump-probe method?
    \begin{itemize}
        \item No further discussion of it in this chapter; Tian might talk about it more in later chapters, tho.
        \item Mostly for intra-molecular reactions, like accessing excited states and seeing how they decay.
        \item Optical pumping (form IChem I, PSet 8) is one way to do a pump-probe experiment.
    \end{itemize}
    \item Parallel reactions?
    \begin{itemize}
        \item Behave much the same kinetically as others; only difference is there is a yield.
    \end{itemize}
\end{itemize}



\section{Chapter 28: Chemical Kinetics I --- Rate Laws}
\begin{itemize}
    \item Whereas \textcite{bib:McQuarrieSimon} developed Quantum Mechanics from a set of simple postulates and Thermodynamics from the three laws, "the field of chemical kinetics has not yet matured to a point where a set of unifying principles has been identified" \parencite[1047]{bib:McQuarrieSimon}.
    \begin{itemize}
        \item There are many current theoretical models of kinetics, each of which has its merits and drawbacks.
        \item Thus, right now, it is necessary to familiarize ourselves with numerous disparate ideas, as is common in developing fields of inquiry.
    \end{itemize}
    \item \textbf{Rate law}: A differential equation describing the time-dependence of the reactant and product concentrations during a chemical reaction.
    \item Consider the general chemical reaction described by
    \begin{equation*}
        \ce{$\nu_{\text{A}}$A + $\nu_{\text{B}}$B -> $\nu_{\text{Y}}$Y + $\nu_{\text{Z}}$Z}
    \end{equation*}
    \item Since
    \begin{align*}
        n_{\ce{A}}(t) &= n_{\ce{A}}(0)-\nu_{\ce{A}}\xi(t)&
        n_{\ce{B}}(t) &= n_{\ce{B}}(0)-\nu_{\ce{B}}\xi(t)&
        n_{\ce{Y}}(t) &= n_{\ce{Y}}(0)+\nu_{\ce{Y}}\xi(t)&
        n_{\ce{Z}}(t) &= n_{\ce{Z}}(0)+\nu_{\ce{Z}}\xi(t)
    \end{align*}
    we can describe the time-dependent change in the number of moles of each substance by taking a derivative with respect to $t$, as follows.
    \begin{align*}
        \dv{n_{\ce{A}}}{t} &= -\nu_{\ce{A}}\dv{\xi}{t}&
        \dv{n_{\ce{B}}}{t} &= -\nu_{\ce{B}}\dv{\xi}{t}&
        \dv{n_{\ce{Y}}}{t} &= \nu_{\ce{Y}}\dv{\xi}{t}&
        \dv{n_{\ce{Z}}}{t} &= \nu_{\ce{Z}}\dv{\xi}{t}
    \end{align*}
    \item Since most experimental techniques measure concentration, it is convenient to divide the above equations by the total volume $V$ on both sides to yield the following.
    \begin{align*}
        \dv{[\ce{A}]}{t} &= -\frac{\nu_{\ce{A}}}{V}\dv{\xi}{t}&
        \dv{[\ce{B}]}{t} &= -\frac{\nu_{\ce{B}}}{V}\dv{\xi}{t}&
        \dv{[\ce{Y}]}{t} &= \frac{\nu_{\ce{Y}}}{V}\dv{\xi}{t}&
        \dv{[\ce{Z}]}{t} &= \frac{\nu_{\ce{Z}}}{V}\dv{\xi}{t}
    \end{align*}
    \item While each individual quantity above has its purpose, it is useful to define an overall \textbf{rate of reaction}.
    \item \textbf{Rate of reaction}: The following quantity. \emph{Denoted by} $\bm{v(t)}$. \emph{Given by}
    \begin{align*}
        v(t) &= \frac{1}{V}\dv{\xi}{t}\\
        &= -\frac{1}{\nu_A}\dv{[A]}{t}
            = -\frac{1}{\nu_B}\dv{[B]}{t}
            = \frac{1}{\nu_Y}\dv{[Y]}{t}
            = \frac{1}{\nu_Z}\dv{[Z]}{t}
    \end{align*}
    \begin{itemize}
        \item Note that the rate of reaction is always positive (as long as the reaction proceeds only in the forward direction).
    \end{itemize}
    \item \textbf{Rate law}: The relationship between $v(t)$ and the concentrations of the various reactants. \emph{General form}
    \begin{equation*}
        v(t) = k[\ce{A}]^{m_{\ce{A}}}[\ce{B}]^{m_{\ce{B}}}\cdots
    \end{equation*}
    \begin{itemize}
        \item Some reactions (such as the \ce{H2 + Br2 -> 2HBR} example from class) do not have conventional rate laws.
    \end{itemize}
    \item \textbf{Rate constant}: The proportionality constant between the rate of reaction and the function of the concentrations of the chemical species involved in a rate law. \emph{Denoted by} $\bm{k}$.
    \begin{itemize}
        \item The units of the rate constant depend on the form of the rate law.
    \end{itemize}
    \item \textbf{Order} (of a reactant \ce{A}): The power to which the concentration of a reactant is raised in a rate law. \emph{Denoted by} $\bm{m_{\ce{A}}}$.
    \item \textbf{Overall order} (of a chemical reaction that obeys a general-form rate law): The sum of the orders of the reactants.
    \item We now discuss common methods for the experimental determination of a rate law.
    \item \textbf{Method of isolation}: The following procedure, which as described will determine $m_{\ce{B}}$ for a chemical reaction of the form introduced at the beginning of this section but can easily be adapted to determine $m_{\ce{A}}$ or be generalized to higher-order situations.
    \begin{enumerate}
        \item Introduce a large excess concentration of \ce{A} into the initial reaction mixture. This excess will guarantee that $[\ce{A}]$ remains essentially constant over the course of the reaction.
        \item Combine $[\ce{A}]^{m_{\ce{A}}}$ and $k$ into a new "rate constant" $k'$, reducing the rate law to the form
        \begin{equation*}
            v = k'[\ce{B}]^{m_{\ce{B}}}
        \end{equation*}
        \item Determine $m_{\ce{B}}$ by measuring $v$ as a function of $[\ce{B}]$.
    \end{enumerate}
    \item Sometimes it is not possible to have one reactant or the other in excess.
    \begin{itemize}
        \item As such, we need an alternate way to measure the rate.
        \item We cannot directly measure $\dv*{[\ce{A}]}{t}$, but we can measure $\Delta[\ce{A}]/\Delta t$ for small $\Delta t$ and approximate these measurements as $\dv*{[\ce{A}]}{t}$.
        \item This forms the basis for the \textbf{method of initial rates}.
    \end{itemize}
    \item \textbf{Method of initial rates}: The following procedure, which as described will determine $m_{\ce{B}}$ for a chemical reaction of the form introduced at the beginning of this section but can easily be adapted to determine $m_{\ce{A}}$ or be generalized to higher-order situations.
    \begin{enumerate}
        \item Take two different measurements of the initial rate (from $t=0$ to $t=t$). Let the initial concentration of \ce{A}, $[\ce{A}]_0$, be the same for each. However, for one, use $[\ce{B}]_1$ for initial concentration of \ce{B}, and for the other, use $[\ce{B}]_2$.
        \item Arranging everything into equations, we thus have
        \begin{align*}
            v_1 &= -\frac{1}{\nu_{\ce{A}}}\left( \frac{\Delta[\ce{A}]}{\Delta t} \right)_1
            = k[\ce{A}]_0^{m_{\ce{A}}}[\ce{B}]_1^{m_{\ce{B}}}&
            v_2 &= -\frac{1}{\nu_{\ce{A}}}\left( \frac{\Delta[\ce{A}]}{\Delta t} \right)_2
            = k[\ce{A}]_0^{m_{\ce{A}}}[\ce{B}]_2^{m_{\ce{B}}}
        \end{align*}
        where we have used the subscripts 1 and 2 to denote the results of the different experiments and their corresponding initial concentrations of \ce{B}.
        \item We may now solve for $m_{\ce{B}}$ by dividing the two equations, taking logarithms, and rearranging to the following.
        \begin{equation*}
            m_{\ce{B}} = \frac{\ln(v_1/v_2)}{\ln([\ce{B}]_1/[\ce{B}]_2)}
        \end{equation*}
    \end{enumerate}
    \item Both the method of isolation and the method of initial rates rely on the assumption that the reactants can be mixed, and then we can measure the rates.
    \begin{itemize}
        \item However, for some very quick reactions, the time required to mix the reactants is long compared with the reaction itself.
        \item For these cases, we need \textbf{relaxation methods}.
    \end{itemize}
\end{itemize}




\end{document}