\documentclass[../notes.tex]{subfiles}

\pagestyle{main}
\renewcommand{\chaptermark}[1]{\markboth{\chaptername\ \thechapter\ (#1)}{}}
\setcounter{chapter}{27}

\begin{document}




\chapter{Rate Laws}
\section{Definitions and Methods of Determination}
\begin{itemize}
    \item \marginnote{4/8:}Consider a general chemical equation
    \begin{equation*}
        \ce{\nu_AA + \nu_BB -> \nu_YY + \nu_ZZ}
    \end{equation*}
    \item The extent of the reaction via the progress variable $\xi$ is
    \begin{align*}
        n_A(t) &= n_A(0)-\nu_A\xi(t)&
        n_Y(t) &= n_Y(0)+\nu_Y\xi(t)
    \end{align*}
    \item The rate of change (moles/second) is
    \begin{align*}
        \dv{n_A}{t} &= -\nu_A\dv{\xi}{t}&
        \dv{n_Y}{t} &= \nu_Y\dv{\xi}{t}
    \end{align*}
    \item Deriving the rate of reaction for a gas-based chemical reaction.
    \begin{itemize}
        \item Time-dependent concentration changes
        \begin{align*}
            \frac{1}{V}\dv{n_A}{t} &= \dv{[A]}{t} = -\frac{\nu_A}{V}\dv{\xi}{t}&
            \frac{1}{V}\dv{n_Y}{t} &= \dv{[Y]}{t} = -\frac{\nu_Y}{V}\dv{\xi}{t}
        \end{align*}
        \item The rate (or speed) of reaction, also known as the differential rate law, is
        \begin{equation*}
            v(t) = -\frac{1}{\nu_A}\dv{[A]}{t}
            = -\frac{1}{\nu_B}\dv{[B]}{t}
            = \frac{1}{\nu_Y}\dv{[Y]}{t}
            = \frac{1}{\nu_Z}\dv{[Z]}{t}
            = \frac{1}{V}\dv{\xi}{t}
        \end{equation*}
        \item All terms are positive.
        \item Rate laws with a constant $k$ are of the form
        \begin{equation*}
            v(t) = k[A]^{m_A}[B]^{m_B}
        \end{equation*}
        \item The exponents are known as \textbf{orders}.
        \item The overall order reaction is $\sum m_i$.
        \item The orders and overall order of the reaction depends on the fundamental reaction steps and the reaction mechanism.
    \end{itemize}
    \item For example, for the reaction \ce{2NO_{(g)} + O2_{(g)} -> 2NO2_{(g)}}, we have
    \begin{equation*}
        v(t) = -\frac{1}{2}\dv{[\ce{NO}]}{t}
        = -\dv{[\ce{O2}]}{t}
        = -\frac{1}{2}\dv{[\ce{NO2}]}{t}
    \end{equation*}
    \begin{itemize}
        \item It follows that $v(t)=k[\ce{NO}]^2[\ce{O2}]$.
        \item This is a rare elementary reaction that proceeds with the kinetics illustrated by the equation.
    \end{itemize}
    \item Rate laws must be determined by experiment.
    \begin{itemize}
        \item Multi-step reactions may have more complex rate law expressions.
        \item Oftentimes, $1/2$ exponents indicate more complicated mechanisms.
        \item For example, even an equation as simple looking as \ce{H2 + Br2 -> 2HBr} has rate law
        \begin{equation*}
            v(t) = \frac{k'[\ce{H2}][\ce{Br2}]^{1/2}}{1+k''[\ce{HBr}][\ce{Br2}]^{-1}}
        \end{equation*}
    \end{itemize}
    \item Determining rate laws.
    \begin{itemize}
        \item Method of isolation.
        \begin{itemize}
            \item Put in a large initial excess of $A$ so that it's concentration doesn't change that much; essentially incorporates $[A]^{m_A}$ into $k$ for determination of the order of $B$.
            \item We can then do the same thing the other way around.
        \end{itemize}
        \item Method of initial rates.
        \begin{itemize}
            \item We approximate
            \begin{equation*}
                v = -\frac{\dd{[A]}}{\nu_A\dd{t}}
                \approx -\frac{\Delta[A]}{\nu_A\Delta t}
                = k[A]^{m_A}[B]^{m_B}
            \end{equation*}
            \item Consider two different initial values of $[B]$, which we'll call $[B_1],[B_2]$. Then
            \begin{align*}
                v_1 &= -\frac{1}{\nu_A}\left( \frac{\Delta[A]}{\Delta t} \right)_1 = k[A]_0^{m_A}[B]_1^{m_B}&
                v_2 &= -\frac{1}{\nu_A}\left( \frac{\Delta[A]}{\Delta t} \right)_2 = k[A]_0^{m_A}[B]_2^{m_B}
            \end{align*}
            \item Take the logarithm and solve for $m_B$.
            \begin{equation*}
                m_B = \frac{\ln(v_1/v_2)}{\ln([B]_1/[B]_2)}
            \end{equation*}
        \end{itemize}
    \end{itemize}
    \item Does an example problem.
\end{itemize}



\section{Integrated Rate Laws}
\begin{itemize}
    \item \marginnote{4/11:}First order reactions have exponential integrated rate laws.
    \begin{itemize}
        \item Suppose \ce{A + B -> products}.
        \item Suppose the reaction is first order in \ce{A}.
        \item If the concentration of \ce{A} is $[\ce{A}]_0$ at $t=0$ and $[\ce{A}]$ at time $t$, then
        \begin{align*}
            v(t) = -\dv{[\ce{A}]}{t} &= k[\ce{A}]\\
            \int_{[\ce{A}]_0}^{[\ce{A}]}\frac{\dd{[\ce{A}]}}{[\ce{A}]} &= -\int_0^tk\dd{t}\\
            \ln\frac{[\ce{A}]}{[\ce{A}]_0} &= -kt\\
            [\ce{A}] &= [\ce{A}]_0\e[-kt]
        \end{align*}
        is the integrated rate law.
        \item Goes over both the concentration plot and the linear logarithmic plot.
    \end{itemize}
    \item The half-life of a first-order reaction is independent of the initial amount of reactant.
    \begin{itemize}
        \item The half-life is found from the point
        \begin{equation*}
            [\ce{A}(t_{1/2})] = \frac{[\ce{A}(0)]}{2}
            = \frac{[\ce{A}]_0}{2}
        \end{equation*}
        \item We have
        \begin{align*}
            \ln\frac{1}{2} &= -kt_{1/2}\\
            t_{1/2} &= \frac{\ln 2}{k} \approx \frac{0.693}{k}
        \end{align*}
        \item Notice that the above equation does not depend on $[\ce{A}]$ or $[\ce{B}]$!
    \end{itemize}
    \item Second order reactions have inverse concentration integrated rate laws.
    \begin{itemize}
        \item Suppose \ce{A + B -> products}, as before, and that the reaction is second order in \ce{A}.
        \item Then
        \begin{align*}
            -\dv{[\ce{A}]}{t} &= k[\ce{A}]^2\\
            \int_{[\ce{A}]_0}^{[\ce{A}]}-\frac{\dd{[\ce{A}]}}{[\ce{A}]^2} &= \int_0^tk\dd{t}\\
            \frac{1}{[\ce{A}]} &= \frac{1}{[\ce{A}]_0}+kt
        \end{align*}
        is the integrated rate law.
    \end{itemize}
    \item The half-life of a second-order reaction is dependent on the initial amount of reaction.
    \begin{itemize}
        \item We have that
        \begin{align*}
            \frac{1}{[\ce{A}]_0/2} &= \frac{1}{[\ce{A}]_0}+kt_{1/2}\\
            \frac{1}{[\ce{A}]_0} &= kt_{1/2}\\
            t_{1/2} &= \frac{1}{k[\ce{A}]_0}
        \end{align*}
    \end{itemize}
    \item If a reaction is $n^\text{th}$-order in a reactant for $n\geq 2$, then the integrated rate law is given by
    \begin{align*}
        -\dv{[\ce{A}]}{t} &= k[\ce{A}]^n\\
        \int_{[\ce{A}]_0}^{[\ce{A}]}-\frac{\dd{[\ce{A}]}}{[\ce{A}]^n} &= \int_0^tk\dd{t}\\
        \frac{1}{n-1}\left( \frac{1}{[\ce{A}]^{n-1}}-\frac{1}{[\ce{A}]_0^{n-1}} \right) &= kt
    \end{align*}
    \begin{itemize}
        \item The associated half life is
        \begin{align*}
            \frac{1}{n-1}\left( \frac{1}{([\ce{A}]_0/2)^{n-1}}-\frac{1}{[\ce{A}]_0^{n-1}} \right) &= kt_{1/2}\\
            \frac{1}{n-1}\cdot\frac{2^{n-1}-1}{[\ce{A}]_0^{n-1}} &= kt_{1/2}\\
            t_{1/2} &= \frac{2^{n-1}-1}{k(n-1)[\ce{A}]_0^{n-1}}
        \end{align*}
    \end{itemize}
    \item Second order reactions that are first order in each reactant.
    \begin{itemize}
        \item We have that
        \begin{align*}
            -\dv{[\ce{A}]}{t} = -\dv{[\ce{B}]}{t} &= k[\ce{A}][\ce{B}]\\
            kt &= \frac{1}{[\ce{A}]_0-[\ce{B}]_0}\ln\frac{[\ce{A}][\ce{B}]_0}{[\ce{B}][\ce{A}]_0}
        \end{align*}
        \item The actual determination is more complicated (there is a textbook problem that walks us through the derivation, though).
        \item When $[\ce{A}]_0=[\ce{B}]_0$, the integrated rate law simplifies to the second-order integrated rate laws in $[\ce{A}]$ and $[\ce{B}]$.
        \begin{itemize}
            \item In this limited case, the half-life is that of the second-order integrated rate law, too, i.e., $t_{1/2}=1/k[\ce{A}]_0$.
        \end{itemize}
    \end{itemize}
    \item The reaction paths and mechanism for parallel reactions.
    \begin{itemize}
        \item Suppose \ce{A} can become both \ce{B} and \ce{C} with respective rate constants $k_B$ and $k_C$.
        \item Then
        \begin{align*}
            \dv{[\ce{A}]}{t} &= -k_B[\ce{A}]-k_C[\ce{A}] = -(k_B+k_C)[\ce{A}]&
            \dv{[\ce{B}]}{t} &= k_B[\ce{A}]&
            \dv{[\ce{C}]}{t} &= k_C[\ce{A}]
        \end{align*}
        \item The integrated rate laws here are
        \begin{align*}
            [\ce{A}] &= [\ce{A}]_0\e[-(k_B+k_C)t]&
            [\ce{B}] &= \frac{k_B}{k_B+k_C}[\ce{A}]_0\left( 1-\e[-(k_B+k_C)t] \right)&
            [\ce{C}] &= \frac{k_C}{k_B+k_C}[\ce{A}]_0\left( 1-\e[-(k_B+k_C)t] \right)
        \end{align*}
        \item The ratio of product concentrations is
        \begin{equation*}
            \frac{[\ce{B}]}{[\ce{C}]} = \frac{k_B}{k_C}
        \end{equation*}
        \item The yield $\Phi_i$ is the probability that a given product $i$ will be formed from the decay of the reactant
        \begin{align*}
            \Phi_i &= \frac{k_i}{\sum_nk_n}&
            \sum_i\Phi_i &= 1
        \end{align*}
    \end{itemize}
    \item Example: If we have parallel reactions satisfying $k_B=2k_C$, then
    \begin{equation*}
        \Phi_C = \frac{k_C}{k_B+k_C}
        = \frac{k_C}{2k_C+k_C}
        = \frac{1}{3}
    \end{equation*}
\end{itemize}



\section{Office Hours (Tian)}
\begin{itemize}
    \item Why does the reduced mass work in the collision frequency derivation?
    \begin{itemize}
        \item We need to start with a simpler case, or the problem will be really hard; thus, we begin by assuming that the particles all are static.
        \item We use the reduced mass to consider the relative speed $u_r$ of the particles with respect to the moving particle as our reference frame. So all the others are the relative speeds to our particle. But this necessitates using the relative mass of the particles to our particle (which is the reduced mass).
    \end{itemize}
\end{itemize}




\end{document}