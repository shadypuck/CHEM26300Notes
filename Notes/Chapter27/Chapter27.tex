\documentclass[../notes.tex]{subfiles}

\pagestyle{main}
\renewcommand{\chaptermark}[1]{\markboth{\chaptername\ \thechapter\ (#1)}{}}
\setcounter{chapter}{26}

\begin{document}




\chapter{Kinetic Theory of Gases}
\section{Background and Ideal Gas Distributions}
\begin{itemize}
    \item \marginnote{3/28:}Learning objectives for CHEM 263.
    \begin{itemize}
        \item The time-dependent phenomena.
        \item Reaction rate and rate laws.
        \item Reaction mechanisms and reaction dynamics.
        \item Surface chemistry and catalysis.
        \item Experimental design and instruments.
    \end{itemize}
    \item Before we move into the content of CHEM 263, a few important notes from CHEM 262.
    \item \textbf{Partition function} (for a system with $N$ states): The following function of temperature. \emph{Denoted by} $\bm{Q(T)}$. \emph{Given by}
    \begin{equation*}
        Q(T) = \sum_{n=1}^N\e[-E_n/\kB T]
    \end{equation*}
    \item \textbf{Observable}: A quantum mechanical operator.
    \item Consider a system described by the partition function $Q$. Let $\ket{i}$ denote the state with energy $E_i$, and let $A$ be an observable. Then the expected value of the observable $A$ is given by
    \begin{equation*}
        \prb{A} = \frac{1}{Q}\sum_{\ket{i}}\ev{A}{i}\e[-E_i/\kB T]
    \end{equation*}
    \begin{itemize}
        \item "This fundamental law is the summit of statistical mechanics, and the entire subject is either the slide-down from this summit, as the principle is applied to various cases, or the climb-up to where the fundamental law is derived and the concepts of thermal equilibrium and temperature $T$ clarified" Richard Feynman, Statistical Mechanics.
    \end{itemize}
    \item Now onto the CHEM 263 content.
    \item Tian duplicates the derivation of the ideal gas law given on \textcite[18-19]{bib:PHYS13300Notes}.
    \begin{itemize}
        \item Note that if $M$ is the molar mass, $m$ is the mass of a single molecule, $\NA$ is Avogadro's number, $N$ is the number of particles present, and $n$ is the number of moles present, then since $N/\NA=n$ and $M/\NA=m$, we have that
        \begin{equation*}
            M = \frac{Nm}{n}
        \end{equation*}
    \end{itemize}
    \item Important values of molecular speed $u$.
    \begin{figure}[h!]
        \centering
        \begin{tikzpicture}[xscale=1.3,yscale=1.2]
            \small
            \draw [stealth-stealth] (0,3) -- node[left]{$F(u)$} (0,0) -- node[below]{$u$} (5,0);
    
            \draw [blx,thick] plot[domain=0:4.9,smooth] (\x,{2*\x*\x*e^(-\x*\x/3)});
            \begin{scope}[on background layer]
                \fill [blt] plot[domain=0:4.9,smooth] (\x,{2*\x*\x*e^(-\x*\x/3)}) -- ++(0,-0.016) -- cycle;
            \end{scope}
    
            \footnotesize
            \draw [semithick]
                (1.732,2.107) -- ++(0,0.2) node[above]{$u_p$}
                (1.954,2.039) -- ++(0,0.2) node[above]{$\bar{u}$}
                (2.121,1.908) -- ++(0,0.2) node[above,xshift=8pt]{$u_\text{rms}$}
            ;
        \end{tikzpicture}
        \caption{Important values of molecular speed.}
        \label{fig:molecularSpeed}
    \end{figure}
    \item \textbf{Maxwell Speed Distribution Function}: The following normalized function, which gives the probability that a particle in an ideal gas will have a given speed. \emph{Denoted by} $\bm{f(u)}$. \emph{Given by}
    \begin{equation*}
        f(u) = 4\pi\left( \frac{M}{2\pi RT} \right)^{3/2}u^2\exp\left( -\frac{Mu^2}{2RT} \right)
    \end{equation*}
    \item \textbf{Most probable speed}: The speed that a particle in an ideal gas is most likely to have. \emph{Denoted by} $\bm{u_p}$. \emph{Given by}
    \begin{equation*}
        u_p = \sqrt{\frac{2RT}{M}}
    \end{equation*}
    \item \textbf{Mean speed}: The average speed of all of the particles in an ideal gas. \emph{Denoted by} $\bm{\bar{u}}$. \emph{Given by}
    \begin{equation*}
        \bar{u} = \sqrt{\frac{8RT}{\pi M}}
    \end{equation*}
    \item \textbf{Root mean squared speed}: The square root of the average of the speeds squared. \emph{Denoted by} $\bm{u_\text{rms}}$. \emph{Given by}
    \begin{equation*}
        u_\text{rms} = \prb{u^2}^{1/2}
        = \sqrt{\frac{3RT}{M}}
    \end{equation*}
    \item The distributions of the molecular speed and velocity components are different.
    \begin{itemize}
        \item While speed follows the Maxwell-Boltzmann distribution, velocity follows (on each Cartesian axis) a Gaussian distribution centered at zero.
        \item At higher temperatures, both distributions "flatten out," but maintain their shape.
    \end{itemize}
    \item Deriving the distribution of the velocity component.
    \begin{itemize}
        \item The velocity components are independent.
        \item Let
        \begin{equation*}
            h(u) = h(u_x,u_y,u_z) = f(u_x)f(u_y)f(u_z)
        \end{equation*}
        be the distribution of speed with velocity components between $u_x,u_x+\dd{u_x}$, $u_y,u_y+\dd{u_y}$, and $u_z,u_z+\dd{u_z}$, where $f(u_i)$ is the probability distribution of components $i$.
        \begin{itemize}
            \item Note that $h(u)$ is \emph{not} the speed distribution with velocity components between $u,u+\dd{u}$.
        \end{itemize}
        \item Clever step: Note that the logarithmic form of the above equation leads to
        \begin{align*}
            \ln h(u) &= \ln f(u_x)+\ln f(u_y)+\ln f(u_z)\\
            \left( \pdv{\ln h}{u_x} \right)_{u_y,u_z} &= \dv{\ln h}{u}\left( \pdv{u}{u_x} \right)_{u_y,u_z}\\
            &= \frac{u_x}{u}\dv{\ln h}{u}
        \end{align*}
        where we evaluate $\pdv*{u}{u_x}$ by using the generalized Pythagorean theorem definition of $u$.
        \item Additionally, we have that
        \begin{equation*}
            \left( \pdv{\ln h}{u_x} \right)_{u_y,u_z} = \dv{\ln f(u_x)}{u_x}
        \end{equation*}
        since the $\ln f(u_i)$ ($i\neq x$) terms are constant with respect to changes in $u_x$.
        \item Thus, combining the last two results, we have that
        \begin{equation*}
            \frac{\dd{\ln h(u)}}{u\dd{u}} = \frac{\dd{\ln f(u_x)}}{u_x\dd{u_x}}
        \end{equation*}
        \item It follows since the gas is isotropic that
        \begin{equation*}
            \frac{\dd{\ln h(u)}}{u\dd{u}} = \frac{\dd{\ln f(u_x)}}{u_x\dd{u_x}}
            = \frac{\dd{\ln f(u_y)}}{u_y\dd{u_y}}
            = \frac{\dd{\ln f(u_z)}}{u_z\dd{u_z}}
        \end{equation*}
        \item But since the three speed components are independent of each other, the above term is constant.
        \item It follows if we call the constant $-2\gamma$, then
        \begin{align*}
            \frac{\dd{\ln f(u_i)}}{u_i\dd{u_i}} &= -2\gamma\\
            f(u_i) &= A\e[-\gamma u_i^2]
        \end{align*}
        for $i=x,y,z$.
        \item We will pick up with solving for $A$ and $\gamma$ in the next lecture.
    \end{itemize}
\end{itemize}



\section{Velocity vs. Speed}
\begin{itemize}
    \item \marginnote{3/30:}Exam preferences.
    \begin{itemize}
        \item Asks for midterm preferences. People prefer a take-home exam.
        \item Asks for final preferences. Probably a 2-hour test?
    \end{itemize}
    \item Continuing with the derivation for the distribution of the velocity component.
    \begin{itemize}
        \item Note that we choose $-2\gamma$ because we know we're gonna have to integrate and we want the final form to be as simple as possible. For instance,
        \begin{equation*}
            \frac{\dd{\ln f(u_i)}}{u_i\dd{u_i}} = \frac{\dd{\ln f(u_i)}}{\frac{1}{2}\dd{u_i^2}}
        \end{equation*}
        should help rationalize the $2$.
        \item Solving for $A$.
        \begin{itemize}
            \item We apply the normalization requirement.
            \begin{align*}
                1 &= \int_{-\infty}^\infty f(u_i)\dd{u_i}\\
                &= A\int_{-\infty}^\infty\e[-\gamma u_i^2]\dd{u_i}\\
                &= 2A\int_0^\infty\e[-\gamma u_i^2]\dd{u_i}\\
                &= 2A\sqrt{\frac{\pi}{4\gamma}}\\
                A &= \sqrt{\frac{\gamma}{\pi}}
            \end{align*}
            \item Thus, for $i=x,y,z$, we have
            \begin{equation*}
                f(u_i) = \sqrt{\frac{\gamma}{\pi}}\e[-\gamma u_i^2]
            \end{equation*}
        \end{itemize}
        \item Solving for $\gamma$.
        \begin{itemize}
            \item We know from the previous lecture that
            \begin{align*}
                \frac{1}{3}m\prb{u^2} &= RT\\
                \prb{u^2} &= \frac{3RT}{M} = \frac{3\kB T}{m}\\
                \prb{u_x^2} &= \frac{RT}{M}
            \end{align*}
            \item But we also have by definition that (taking $u_x$ in particular because $\gamma$ is the same in the equations for $u_x,u_y,u_z$)
            \begin{equation*}
                \prb{u_x^2} = \int_{-\infty}^\infty u_x^2f(u_x)\dd{u_x}
            \end{equation*}
            \item Thus, we have that
            \begin{align*}
                \frac{RT}{M} &= \int_{-\infty}^\infty u_x^2f(u_x)\dd{u_x}\\
                &= \sqrt{\frac{\gamma}{\pi}}\int_{-\infty}^\infty u_x^2\e[-\gamma u_x^2]\dd{u_x}\\
                &= 2\sqrt{\frac{\gamma}{\pi}}\int_0^\infty u_x^2\e[-\gamma u_x^2]\dd{u_x}\\
                &= 2\sqrt{\frac{\gamma}{\pi}}\cdot\frac{1}{4\gamma}\sqrt{\frac{\pi}{\gamma}}\\
                &= \frac{1}{2\gamma}\\
                \gamma &= \frac{M}{2RT}
            \end{align*}
        \end{itemize}
        \item It follows that
        \begin{equation*}
            f(u_i) = \sqrt{\frac{M}{2\pi RT}}\e[-Mu_i^2/2RT]
            = \sqrt{\frac{m}{2\pi\kB T}}\e[-mu_i^2/2\kB T]
        \end{equation*}
    \end{itemize}
    \item Now we can compute other speeds, such as the average velocity $\prb{u_x}$.
    \begin{itemize}
        \item Evaluating the odd integrand gives us $\prb{u_x}=0$, as expected.
    \end{itemize}
    \item As per the Gaussian distribution, if the temperature increases or mass decreases, the distribution of speeds broadens and flattens.
    \item \textbf{Doppler effect}: The change in frequency of a wave in relation to an observer who is moving relative to the wave source. \emph{Also known as} \textbf{Doppler shift}.
    \begin{itemize}
        \item Example: The change of pitch heard when a vehicle sounding a horn approaches and recedes from an observer. Compared to the emitted frequency, the received frequency is higher during the approach, identical at the instant of passing by, and lower during the recession.
    \end{itemize}
    \item An application of the velocity distribution: The Doppler effect and spectral line broadening.
    \begin{itemize}
        \item Radiation emitted from a gas will be spread out due to the motion of the molecules.
        \item The frequency $\nu$ detected by the observer and the frequency $\nu_0$ emitted by the emitter are related by
        \begin{equation*}
            \nu \approx \nu_0\left( 1+\frac{u_x}{c} \right)
        \end{equation*}
        \item Algebraic rearrangement gives us
        \begin{equation*}
            u_x = \frac{c(\nu-\nu_0)}{\nu_0}
        \end{equation*}
        \item Doppler-broadened spectral lineshape.
        \begin{equation*}
            I(\nu) \propto \e[-mc^2(\nu-\nu_0)^2/2\nu_0\kB T]
        \end{equation*}
        \item Thus, the variance of the spectral line is
        \begin{equation*}
            \sigma^2 = \frac{\nu_0^2\kB T}{mc^2}
            = \frac{\nu_0^2RT}{Mc^2}
        \end{equation*}
        \item The result is that if gas particles are at rest, the emission line spectrum will have very narrow lines. If the gas particles are moving, the lines are broadened.
        \begin{itemize}
            \item This is why so much spectroscopy is done at super-low temperatures and with heavier molecules! In particular, because Doppler broadening blurs results.
        \end{itemize}
    \end{itemize}
    \item We know that the average velocity is zero. But we can also consider the average velocity in the positive direction.
    \begin{itemize}
        \item We calculate
        \begin{align*}
            \prb{u_x+} &= \int_0^\infty u_xf(u_x)\dd{u_x}\\
            &= \sqrt{\frac{m}{2\pi\kB T}}\int_0^\infty u_x\e[-mu_x^2/2\kB T]\\
            &= \sqrt{\frac{m}{2\pi\kB T}}\cdot\frac{2\kB T}{2m}\\
            &= \sqrt{\frac{m}{2\pi\kB T}}
        \end{align*}
        \item This will be one-fourth the average speed from Figure \ref{fig:molecularSpeed}, though.
    \end{itemize}
    \item Moving from velocity to speed: Deriving the Maxwell-Boltzmann Speed Distribution.
    \begin{itemize}
        \item Define
        \begin{equation*}
            F(u)\dd{u} \approx f(u_x)\dd{u_x}f(u_y)\dd{u_y}f(u_z)\dd{u_z}
        \end{equation*}
        \begin{itemize}
            \item This function gives us the velocity of each particle in the velocity space. But the speed of each particle is just it's distance from the origin.
        \end{itemize}
        \item We have that
        \begin{align*}
            F(u)\dd{u} &\approx \left( \frac{m}{2\pi\kB T} \right)^{3/2}\e[-m(u_x^2+u_y^2+u_z^2)/2\kB T]\dd{u_x}\dd{u_y}\dd{u_z}
            \intertext{from where we can convert to spherical coordinates using $u^2=u_x^2+u_y^2+u_z^2$ and $4\pi u^2\dd{u}=\dd{u_x}\dd{u_y}\dd{u_z}$ to get our final result.}
            F(u)\dd{u} &= 4\pi\left( \frac{m}{2\pi\kB T} \right)^{3/2}u^2\e[-mu^2/2\kB T]\dd{u}
        \end{align*}
        \item Note that we invoke the equals sign only at the end because speed is inherently spherical in the velocity space; any use of Cartesian infinitesimals must by definition be an approximation at best.
    \end{itemize}
    \item Some important differences.
    \begin{itemize}
        \item $h(u)=h(u_x,u_y,u_z)=f(u_x)f(u_y)f(u_z)$ is the distribution of molecular speeds with velocity components (in Cartesian coordinates) between $u_x,u_x+\dd{u_x}$, $u_y,u_y+\dd{u_y}$, and $u_z,u_z+\dd{u_z}$.
        \item $f(u_x)=\sqrt{M/2\pi RT}\e[-Mu_x^2/2RT]$ is the distribution of molecular speed componentwise in Cartesian coordinates, and has a Gaussian distribution.
        \item $F(u)=4\pi\left( \frac{m}{2\pi\kB T} \right)^{3/2}u^2\e[-mu^2/2\kB T]$ is the distribution of molecular speed, and has a Maxwell-Boltzmann distribution as per spherical coordinates.
    \end{itemize}
    \item Maxwell-Boltzmann speed distribution of noble gases.
    \begin{itemize}
        \item Heavier Noble gases have more "flattened" M-B distributions.
    \end{itemize}
    \item Different metrics of M-B speed distribution.
    \begin{itemize}
        \item We can, from the above formula, calculate the average speed $\prb{u}$, the root mean square speed $\prb{u^2}$, and the most probable speed by taking a derivative and setting it equal to zero.
        \item We get
        \begin{align*}
            \prb{u} &= \sqrt{\frac{8RT}{\pi M}}&
            u_\text{rms} &= \sqrt{\frac{3RT}{M}}&
            u_\text{mp} &= \sqrt{\frac{2\kB T}{m}}
        \end{align*}
    \end{itemize}
\end{itemize}



\section{Energy Distribution and Collision Frequency}
\begin{itemize}
    \item \marginnote{4/1:}The final exam is 50 minutes on the last day of class.
    \item We can also express the M-B distribution in terms of kinetic energy.
    \begin{itemize}
        \item We know that energy $\varepsilon=\frac{1}{2}mu^2$, so $u=\sqrt{2\varepsilon/m}$ and thus $\dd{u}=\dd{\varepsilon}/\sqrt{2m\varepsilon}$.
        \item This allows us to write
        \begin{align*}
            F(\varepsilon)\dd{\varepsilon} &= 4\pi\left( \frac{m}{2\pi\kB T} \right)^{3/2}\cdot\frac{2\varepsilon}{m}\cdot\e[-\varepsilon/\kB T]\frac{\dd{\varepsilon}}{\sqrt{2m\varepsilon}}\\
            &= \frac{2\pi}{(\pi\kB T)^{3/2}}\varepsilon^{1/2}\e[-\varepsilon/\kB T]\dd{\varepsilon}
        \end{align*}
        \item Thus, we can calculate that
        \begin{align*}
            \prb{\varepsilon} &= \int_0^\infty\varepsilon f(\varepsilon)\dd{\varepsilon}\\
            &= \frac{3}{2}\kB T
        \end{align*}
        as expected.
    \end{itemize}
    \item Aside: Understanding the probability distribution $F(u)\dd{u}$ and the relation between $F(u)\dd{u}$ and $F(\varepsilon)\dd{\varepsilon}$.
    \begin{figure}[h!]
        \centering
        \begin{subfigure}[b]{0.49\linewidth}
            \centering
            \begin{tikzpicture}
                \small
                \draw [stealth-stealth] (0,3) -- node[left]{$F(u)$} (0,0) -- node[below]{$u$} (5,0);
        
                \draw [blx,thick] plot[domain=0:4.9,smooth] (\x,{2*\x*\x*e^(-\x*\x/3)});
                \begin{scope}[on background layer]
                    \fill [blt] plot[domain=0:4.9,smooth] (\x,{2*\x*\x*e^(-\x*\x/3)}) -- ++(0,-0.016) -- cycle;
                    \fill [blz]
                        (3.88,0) -- (3.88,0.20) -- (4.08,0.13) -- (4.08,0)
                        (1.63,0) -- (1.63,2.19) -- (1.84,2.19) -- (1.84,0)
                    ;
                    \draw [bly,semithick] plot[ycomb,domain=0:4.9] (\x,{2*\x*\x*e^(-\x*\x/3)});
                \end{scope}
            \end{tikzpicture}
            \caption{Molecular speed distribution.}
            \label{fig:speedStretcha}
        \end{subfigure}
        \begin{subfigure}[b]{0.49\linewidth}
            \centering
            \begin{tikzpicture}[xscale=0.2]
                \small
                \draw [stealth-stealth] (0,3) -- node[left]{$F(\varepsilon)$} (0,0) -- node[below]{$\varepsilon$} (25,0);
        
                \draw [blx,thick] plot[domain=0:4.9,smooth] ({\x*\x},{2*\x*\x*e^(-\x*\x/3)});
                \begin{scope}[on background layer]
                    \shade [left color=blt,right color=blt!80] plot[domain=0:1,smooth] ({\x*\x},{2*\x*\x*e^(-\x*\x/3)}) -- ++(0,-1.43) -- cycle;
                    \shade [left color=blt!80,right color=blt!60] (1,0) -- plot[domain=1:2,smooth] ({\x*\x},{2*\x*\x*e^(-\x*\x/3)}) -- ++(0,-2.11) -- cycle;
                    \shade [left color=blt!60,right color=blt!40] (4,0) -- plot[domain=2:3,smooth] ({\x*\x},{2*\x*\x*e^(-\x*\x/3)}) -- ++(0,-0.9) -- cycle;
                    \shade [left color=blt!40,right color=blt!20] (9,0) -- plot[domain=3:4,smooth] ({\x*\x},{2*\x*\x*e^(-\x*\x/3)}) -- ++(0,-0.15) -- cycle;
                    \shade [left color=blt!20,right color=white] (16,0) -- plot[domain=4:4.9,smooth] ({\x*\x},{2*\x*\x*e^(-\x*\x/3)}) -- ++(0,-0.02) -- cycle;
                    \fill [blz] (2.67,0) -- (2.67,2.19) -- (3.38,2.19) -- (3.38,0);
                    \fill [blz!60] (15.05,0) -- (15.05,0.20) -- (16.67,0.13) -- (16.67,0);
                    \draw [bly,semithick] plot[ycomb,domain=0:4.9] ({\x*\x},{2*\x*\x*e^(-\x*\x/3)});
                \end{scope}
            \end{tikzpicture}
            \caption{Stretching the molecular speed distribution.}
            \label{fig:speedStretchb}
        \end{subfigure}\\[2em]
        \begin{subfigure}[b]{0.49\linewidth}
            \centering
            \begin{tikzpicture}[xscale=0.2]
                \small
                \draw [stealth-stealth] (0,3) -- node[left]{$F(\varepsilon)$} (0,0) -- node[below]{$\varepsilon$} (25,0);
        
                \draw [blx,thick] plot[domain=0:4.9,smooth] ({\x*\x},{2*\x*\x*e^(-\x*\x/3)});
                \draw [rex,thick] plot[domain=0:24,smooth,samples=500] (\x,{2.8*(\x)^(0.5)*e^(-\x/3)});
        
                \footnotesize
                \draw [semithick]
                    (3,2.11) -- ++(0,0.2) node[above]{$\varepsilon_1$}
                    (1.5,1.98) -- ++(0,0.2) node[above]{$\varepsilon_2$}
                    (4.5,1.91) -- ++(0,0.2) node[above,xshift=8pt]{$u_\text{rms}$}
                    (4.5,1.13) node[below]{$\prb{\varepsilon}$} -- ++(0,0.3)
                ;
            \end{tikzpicture}
            \caption{Comparing the energy and speed distributions.}
            \label{fig:speedStretchc}
        \end{subfigure}
        \begin{subfigure}[b]{0.49\linewidth}
            \centering
            \begin{tikzpicture}[xscale=0.2]
                \small
                \draw [stealth-stealth] (0,3) -- node[left]{$F(\varepsilon)$} (0,0) -- node[below]{$\varepsilon$} (25,0);
        
                \draw [rex,thick] plot[domain=0:24,smooth,samples=500] (\x,{2.8*(\x)^(0.5)*e^(-\x/3)});\begin{scope}[on background layer]
                    \fill [ret] plot[domain=0:24,smooth,samples=500] (\x,{2.8*(\x)^(0.5)*e^(-\x/3)}) -- cycle;
                    \draw [rex,semithick] plot[ycomb,domain=0:24] (\x,{2.8*(\x)^(0.5)*e^(-\x/3)});
                \end{scope}
            \end{tikzpicture}
            \caption{Molecular energy distribution.}
            \label{fig:speedStretchd}
        \end{subfigure}
        \caption{Relating molecular speed and molecular energy.}
        \label{fig:speedStretch}
    \end{figure}
    \begin{itemize}
        \item $F(u)$ is a probability distribution. Thus, its graph (see Figure \ref{fig:speedStretcha}) indicates the number density of particles we'd expect to find at a given velocity $u$ by the vertical height of the curve. Importantly, if we imagine filling in the area under the curve with each particle at its $u$-position and evenly spaced in the $F$ direction, eventually we'd get a continuous color under the curve (as in Figure \ref{fig:speedStretcha}; the darkened regions are illustrated as such for the sole purpose of contrast with Figure \ref{fig:speedStretchb}, as discussed below).
        \item We note that $\varepsilon=\frac{1}{2}mu^2$ is a stretching operation. This means that as $u$ increases, $\varepsilon$ increases faster. For example, as $u$ increases $1,2,3,4$, $\varepsilon$ increases proportionally by $1,4,9,16$. Thus, we can approximate $F(\varepsilon)$ by stretching the graph of $F(u)$ horizontally by greater and greater amounts (see Figure \ref{fig:speedStretchb}).
        \item An important consequence of this is that the particles moving within a certain range of velocities have a larger range of energies (compare the darkly shaded blocks of Figures \ref{fig:speedStretcha} and \ref{fig:speedStretchb}, as well as the general increase in spacing of the vertical lines).
        \item However, when we approximate by stretching, we ignore some of the other changes in the equation. For instance, when we sketch the actual energy distribution, its most probable energy $\varepsilon_\text{mp}=\varepsilon_2$ has a lower value than that predicted by just stretching the graph of the speed distribution (which we denote in Figure \ref{fig:speedStretchc} by $\varepsilon_1=\frac{1}{2}mu_\text{mp}^2$). All in one equation,
        \begin{equation*}
            \varepsilon_\text{mp} \neq \frac{1}{2}mu_\text{mp}^2
        \end{equation*}
        \item Additionally, note that the actual curve (Figure \ref{fig:speedStretchd}) has even density beneath it.
    \end{itemize}
    \item Calculating the most probable kinetic energy.
    \begin{align*}
        \dv{F}{\varepsilon} &= \frac{2\pi}{(\pi\kB T)^{3/2}}\left[ \frac{\varepsilon^{-1/2}\e[-\varepsilon/\kB T]}{2}-\frac{\e[-\varepsilon/\kB T]\cdot\varepsilon^{1/2}}{\kB T} \right]\\
        0 &= \frac{2\pi\e[-\varepsilon/\kB T]}{(\pi\kB T)^{3/2}}\left[ \frac{1}{2\sqrt{\varepsilon}}-\frac{\sqrt{\varepsilon}}{\kB T} \right]\\
        \varepsilon_\text{mp} &= \frac{\kB T}{2}
    \end{align*}
    \begin{itemize}
        \item The most probable energy calculated from the most probable speed via $\frac{1}{2}mu_\text{mp}^2$ is $\kB T$, so the actual value is one-half the predicted value (notice how $\varepsilon_2=\frac{1}{2}\varepsilon_1$).
    \end{itemize}
    \item Since $\prb{\varepsilon}=\prb{\frac{1}{2}mu^2}$, $\prb{\varepsilon}$ is related to the root mean square speed.
    \begin{itemize}
        \item This relates $u_\text{rms}^2=3\kB T/m$ to $\prb{\varepsilon}=3\kB T/2$ by a factor of $m/2$.
        \item This linear relation appears in Figure \ref{fig:speedStretchc}, where $u_\text{rms}$ and $\prb{\varepsilon}$ occur in the same place and differ only by a vertical stretch factor ($m/2$).
    \end{itemize}
    \item Calculating the frequency of collisions.
    \begin{figure}[H]
        \centering
        \begin{tikzpicture}
            \footnotesize
            \draw [-stealth] (0,-1,0) -- (0,3,0) node[above]{$z$};
            \draw [-stealth] (-1,0,0) -- (3,0,0) node[right]{$y$};
            \draw [-stealth] (0,0,0) -- (0,0,3) node[below left=-2pt]{$x$};
    
            \begin{scope}[on background layer]
                \filldraw [thick,draw=grx,fill=gry] plot[domain=0:360] ({0.5*cos(\x)},0,{0.5*sin(\x)});
            \end{scope}
            \draw [grx,thick,xshift=1.845cm,yshift=2.23cm] plot[domain=0:360] ({0.5*cos(\x)},0,{0.5*sin(\x)});
            \draw [grx,thick]
                (-0.5,0,0) -- (1.345,2.23,0)
                (0.5,0,0) -- (2.345,2.23,0)
            ;
    
            \draw [very thin] (0,0,0) -- (4,0,4);
            \draw [<->,very thin,shorten <=2pt,shorten >=2pt] (3.5,0,3.5) -- node[fill=white,inner sep=2pt]{$u_x\dd{t}$} (3.5,3.5,3.5);
            \node at (-0.5,0,0.5) {$A$};
            \draw [<->,very thin] (0,0,0) -- node[pos=0.6,fill=white,inner sep=2pt]{$u\dd{t}$} (1.845,2.23,0);
            \draw [very thin,decoration={markings,mark=at position 0.35 with {\node[fill=white,inner sep=2pt]{$\phi$};}},postaction={decorate}] plot[domain=45:90] ({2*cos(\x)},0,{2*sin(\x)});
            \draw [very thin,decoration={markings,mark=at position 0.5 with {\node[fill=white,inner sep=2pt]{$\theta$};}},postaction={decorate}] plot[domain=0:52] ({2/2^(0.5)*sin(\x)},{2*cos(\x)},{2/2^(0.5)*sin(\x)});
        \end{tikzpicture}
        \caption{Collision frequency cylinder.}
        \label{fig:collisionFrequency}
    \end{figure}
    \begin{itemize}
        \item Construct a cylinder to enclose all those molecules that will strike the area $A$ at an angle $\theta$ with speed $u$ in the time interval $\dd{t}$.
        \item Its volume is $V=Au\cos\theta\dd{t}$.
        \item The number of molecules in the cylinder is $\rho V$, where $\rho$ is the number density.
        \item The fraction of molecules that have a speed between $u,u+\dd{u}$ is $F(u)\dd{u}$.
        \item The fraction travelling within a solid angle bounded by $\theta,\theta+\dd{\theta}$ and $\phi,\phi+\dd{\phi}$ is $\sin\theta\dd{\theta}\cdot\dd{\phi}/4\pi$, where $4\pi$ represents a complete solid angle.
        \item The number $\dd{N_\text{coll}}$ of molecules colliding with the area $A$ from the specified direction in the time interval $\dd{t}$ is
        \begin{equation*}
            \dd{N_\text{coll}} = \rho(Au\dd{t})\cos\theta\cdot F(u)\dd{u}\cdot\frac{\sin\theta\dd{\theta\dd{\phi}}}{4\pi}
        \end{equation*}
        \item The number of collisions per unit time per unit area with the wall by molecules whose speeds are in the range $u,u+\dd{u}$ and whose direction lies within the solid angle $\sin\theta\dd{\theta}\dd{\phi}$ is
        \begin{equation*}
            \dd{z_\text{coll}} = \frac{1}{A}\dv{N_\text{coll}}{t}
            = \frac{\rho}{4\pi}uF(u)\dd{u}\cdot\cos\theta\sin\theta\dd{\theta}\dd{\phi}
        \end{equation*}
        \item If we integrate over all possible speeds and directions, then we obtain
        \begin{align*}
            z_\text{coll} &= \frac{\rho}{4\pi}\int_0^\infty uF(u)\dd{u}\int_0^{\pi/2}\cos\theta\sin\theta\dd{\theta}\int_0^{2\pi}\dd{\phi}\\
            &= \frac{\rho\prb{u}}{4}
        \end{align*}
    \end{itemize}
    \item Deriving the pressure through the collision frequency.
    \begin{itemize}
        \item We have
        \begin{align*}
            \dd{P} &= (2mu\cos\theta)\dd{z_\text{coll}}\\
            &= (2mu\cos\theta)\frac{\rho}{4\pi}uF(u)\dd{u}\cos\theta\sin\theta\dd{\theta}\dd{\phi}\\
            &= \rho\left( \frac{m}{2\pi\kB T} \right)^{3/2}(2mu\cos\theta)u^3\e[-mu^2/2\kB T]\dd{u}\cos\theta\sin\theta\dd{\phi}
        \end{align*}
        \item Thus, since
        \begin{align*}
            \int_0^{\pi/2}\cos^2\theta\sin\theta\dd{\theta}\int_0^{2\pi}\dd{\phi} &= \frac{2\pi}{3}&
            4\pi\left( \frac{m}{2\pi\kB T} \right)^{3/2}\int_0^\infty u^4\e[-mu^2/2\kB T]\dd{u} &= \prb{u^2}
        \end{align*}
        we have that
        \begin{equation*}
            P = \frac{1}{3}\rho m\prb{u^2}
            = \frac{1}{3V}Nm\prb{u^2}
        \end{equation*}
    \end{itemize}
\end{itemize}



\section{Office Hours (Tian)}
\begin{itemize}
    \item Can you explain the whole $F(u)\dd{u}$ differential notation for probability?
    \begin{itemize}
        \item $F(u)$ is the probability function. $F(u)$ is the $y$ axis of the individual points. Probability density.
        \item $F(u)\dd{u}$ is the infinitesimal probability at $u$, but only within an infinitely small range. It's an abbreviation/approximation for the tiny infinitesimal rectangle under the curve that we picture as we integrate.
        \item $\int_0^\infty F(u)\dd{u}=1$ (summing all of the tiny probabilities) gets you to 1 for a normalized probability distribution.
    \end{itemize}
    \item What is up with the relation between $u_\text{rms}$ and $\prb{\varepsilon}$?
    \begin{itemize}
        \item We have
        \begin{align*}
            \prb{\varepsilon} &= \prb{\frac{1}{2}mu^2}\\
            &= \frac{1}{2}m\prb{u^2}\\
            &= \frac{1}{2}mu_\text{rms}^2\\
            &= \frac{m}{2}\cdot\frac{3\kB T}{m}\\
            &= \frac{3\kB T}{2}
        \end{align*}
    \end{itemize}
    \item Post lecture notes before class? Write down what's on the lecture slides or listen?
    \begin{itemize}
        \item He has been and will continue to post the slides the night before the lecture.
    \end{itemize}
    \item When will HW 1 be posted?
    \begin{itemize}
        \item No homework this week.
        \item The first homework will be posted next Monday.
        \item He will post a homework every Monday that will be due the next Monday.
    \end{itemize}
    \item When are gases isotropic?
    \begin{itemize}
        \item A gas is isotropic unless there is a driving force.
        \item For example, gas in a closed box is isotropic, but gas in a cylinder with a fan at one end is not isotropic (particles are more likely to move in one direction).
    \end{itemize}
    \item What is the total solid angle geometrically?
    \begin{itemize}
        \item Hard to visualize three dimensionally. You get $4\pi$ by doing the integrals for the components:
        \begin{equation*}
            4\pi = \int_{-\pi/2}^{\pi/2}\sin\theta\dd{\theta}\int_0^{2\pi}\dd{\phi}
        \end{equation*}
    \end{itemize}
    \item We don't need to memorize most of the derivations, but we do need to know the conclusions and the assumptions we need to get them.
    \begin{itemize}
        \item We won't be asked to give a derivation unless we're given the full starting point.
        \item The final can't have much heavy calculation on it because there's not that much time.
        \item The midterm will be a online take-home exam with limited time (probably 2 hours).
    \end{itemize}
    \item Not every topic in the remainder of \textcite{bib:McQuarrieSimon} will be covered; some sections will be skipped.
    \begin{itemize}
        \item He will focus a lot on the practical applications. Once the student understands the basic principle, he wants us to be able to apply it to research and life.
    \end{itemize}
\end{itemize}



\section{Chapter 27: The Kinetic Theory of Gases}
\emph{From \textcite{bib:McQuarrieSimon}.}
\begin{itemize}
    \item \marginnote{3/28:}\textbf{Kinetic theory of gases}: A simple model of gases in which the molecules (pictured as hard spheres) are assumed to be in constant, incessant motion, colliding with each other and with the walls of the container.
    \item \textcite{bib:McQuarrieSimon} does the KMT derivation of the ideal gas law from \textcite{bib:APChemNotes}. Some important notes follow.
    \begin{itemize}
        \item \textcite{bib:McQuarrieSimon} emphasizes the importance of
        \begin{equation*}
            PV = \frac{1}{3}Nm\prb{u^2}
        \end{equation*}
        as a fundamental equation of KMT, as it relates a macroscopic property $PV$ to a microscopic property $m\prb{u^2}$.
        \item In Chapter 17-18, we derived quantum mechanically, and then from the partition function, that the average translational energy $\prb{E_\text{trans}}$ for a single particle of an ideal gas is $\frac{3}{2}k_BT$. From classical mechanics, we also have that $\prb{E_\text{trans}}=\frac{1}{2}m\prb{u^2}$. \emph{This} is why we may let
        \begin{equation*}
            \frac{1}{2}m\prb{u^2} = \frac{3}{2}k_BT
        \end{equation*}
        recovering that the average translational kinetic energy of the molecules in a gas is directly proportional to the Kelvin temperature.
    \end{itemize}
    \item \textbf{Isotropic} (entity): An object or substance that has the same properties in any direction.
    \begin{itemize}
        \item For example, a homogeneous gas is isotropic, and this is what allows us to state that $\prb{u_x^2}=\prb{u_y^2}=\prb{u_z^2}$.
    \end{itemize}
    \item \textcite{bib:McQuarrieSimon} derives
    \begin{equation*}
        u_\text{rms} = \sqrt{\frac{3RT}{M}}
    \end{equation*}
    \begin{itemize}
        \item $u_\text{rms}$ is an estimate of the average speed since $\prb{u^2}\neq\prb{u}^2$ in general.
    \end{itemize}
    \item \textcite{bib:McQuarrieSimon} states without proof that the speed of sound $u_\text{sound}$ in a monatomic ideal gas is given by
    \begin{equation*}
        u_\text{sound} = \sqrt{\frac{5RT}{3M}}
    \end{equation*}
    \item Assumptions of the kinetic theory of gases.
    \begin{itemize}
        \item Particles collide elastically with the wall.
        \begin{itemize}
            \item Justified because although each collision will not be elastic (the particles in the wall are moving too), the average collision will be elastic.
        \end{itemize}
        \item Particles do not collide with each other.
        \begin{itemize}
            \item Justified because "if the gas is in equilibrium, on the average, any collision that deflects the path of a molecule\dots will be balanced by a collision that replaces the molecule" \parencite[1015]{bib:McQuarrieSimon}.
        \end{itemize}
    \end{itemize}
    \item Note that we can do the kinetic derivation at many levels of rigor, but more rigorous derivations offer results that differ only by constant factors on the order of unity.
    \item Deriving a theoretical equation for the distribution of the \emph{components} of molecular velocities.
    \begin{itemize}
        \item Let $h(u_x,u_y,u_z)\dd{u_x}\dd{u_y}\dd{u_z}$ be the fraction of molecules with velocity components between $u_j$ and $u_j+\dd{u_j}$ for $j=x,y,z$.
        \item Assume that the each component of the velocity of a molecule is independent of the values of the other two components\footnote{This can be proven.}. It follows statistically that
        \begin{equation*}
            h(u_x,u_y,u_z) = f(u_x)f(u_y)f(u_z)
        \end{equation*}
        \begin{itemize}
            \item Note that we use just one function $f$ for the probability distribution in each direction because the gas is isotropic.
        \end{itemize}
        \item We can use the isotropic condition to an even greater degree. Indeed, it implies that any information conveyed by $u_x$ is necessarily and sufficiently conveyed by $u_y$, $u_z$, and $u$. Thus, we may take
        \begin{equation*}
            h(u) = h(u_x,u_y,u_z) = f(u_x)f(u_y)f(u_z)
        \end{equation*}
        \item It follows that
        \begin{equation*}
            \pdv{\ln h(u)}{u_x} = \pdv{u_x}(\ln f(u_x)+\text{terms not involving }u_x)
            = \dv{\ln f(u_x)}{u_x}
        \end{equation*}
        \item Since
        \begin{align*}
            u^2 &= u_x^2+u_y^2+u_z^2\\
            \pdv{u_x}(u^2) &= \pdv{u_x}(u_x^2+u_y^2+u_z^2)\\
            2u\pdv{u}{u_x} &= 2u_x\\
            \pdv{u}{u_x} &= \frac{u_x}{u}
        \end{align*}
        we have that
        \begin{align*}
            \pdv{\ln h}{u_x} &= \dv{\ln h}{u}\pdv{u}{u_x} = \frac{u_x}{u}\dv{\ln h}{u}\\
            \frac{\dd{\ln h(u)}}{u\dd{u}} &= \frac{\dd{\ln f(u_x)}}{u_x\dd{u_x}}
        \end{align*}
        which generalizes to
        \begin{equation*}
            \frac{\dd{\ln h(u)}}{u\dd{u}} = \frac{\dd{\ln f(u_x)}}{u_x\dd{u_x}}
            = \frac{\dd{\ln f(u_y)}}{u_y\dd{u_y}}
            = \frac{\dd{\ln f(u_z)}}{u_z\dd{u_z}}
        \end{equation*}
        \item Since $u_x,u_y,u_z$ are independent, we know that the above equation is equal to a constant, which we may call $-\gamma$. It follows that for any $j=x,y,z$, we have that
        \begin{align*}
            \frac{\dd{\ln f(u_j)}}{u_j\dd{u_j}} &= -\gamma\\
            \frac{1}{f}\dv{f}{u_j} &= -\gamma u_j\\
            \int\frac{\dd{f}}{f} &= \int-\gamma u_j\dd{u_j}\\
            \ln f &= -\frac{\gamma}{2}u_j^2+C\\
            f(u_j) &= A\e[-\gamma u_j^2]
        \end{align*}
        where we have incorporated the $1/2$ into $\gamma$.
        \item To determine $A$ and $\gamma$, we let arbitrarily let $j=x$. Since $f$ is a continuous probability distribution, we may apply the normalization requirement.
        \begin{align*}
            1 &= \int_{-\infty}^\infty f(u_x)\dd{u_x}\\
            &= 2A\int_0^\infty\e[-\gamma u_x^2]\dd{u_x}\\
            &= 2A\sqrt{\frac{\pi}{4\gamma}}\\
            A &= \sqrt{\frac{\gamma}{\pi}}
        \end{align*}
        \item Additionally, since we have that $\prb{u_x^2}=\frac{1}{3}\prb{u^2}$ and $\prb{u^2}=3RT/M$, we know that $\prb{u_x^2}=RT/M$. This combined with the definition of $\prb{u_x^2}$ as a continuous probability distribution yields
        \begin{align*}
            \frac{RT}{M} &= \prb{u_x^2}\\
            &= \int_{-\infty}^\infty u_x^2f(u_x)\dd{u_x}\\
            &= 2\sqrt{\frac{\gamma}{\pi}}\int_0^\infty u_x^2\e[-\gamma u_x^2]\dd{u_x}\\
            &= 2\sqrt{\frac{\gamma}{\pi}}\cdot\frac{1}{4\gamma}\sqrt{\frac{\pi}{\gamma}}\\
            &= \frac{1}{2\gamma}\\
            \gamma &= \frac{M}{2RT}
        \end{align*}
        \item Therefore,
        \begin{equation*}
            f(u_x) = \sqrt{\frac{M}{2\pi RT}}\e[-Mu_x^2/2RT]
        \end{equation*}
        \item It is common to rewrite the above in terms of molecular quantities $m$ and $k_B$.
    \end{itemize}
    \item It follows that as temperature increases, more molecules are likely to be found with higher component velocity values.
    \item We can use the above result to show that
    \begin{equation*}
        \prb{u_x} = \int_{-\infty}^\infty u_xf(u_x)\dd{u_x} = 0
    \end{equation*}
    \item We can also calculate that $\prb{u_x^2}=RT/M$ and $m\prb{u_x}^2/2=k_BT/2$ from the above result\footnote{See the equipartition of energy theorem from \textcite{bib:PHYS13300Notes}.}.
    \begin{itemize}
        \item An important consequence is that the total kinetic energy is divided equally into the $x$-, $y$-, and $z$-components.
    \end{itemize}
    \item \textbf{Doppler broadening}: The broadening of spectral lines due to the distribution of molecular velocities.
    \begin{itemize}
        \item Ideally, spectral lines will be very narrow.
        \item However, due to the Doppler effect, if an atom or molecule emits radiation of frequency $\nu_0$ while moving away or toward the observer with speed $u_x$, then the observed frequency will be
        \begin{equation*}
            \nu \approx \nu_0\left( 1+\frac{u_x}{c} \right)
        \end{equation*}
        \item Indeed, "if one observes the radiation emitted from a gas at temperature $T$, then it is found that the spectral line at $\nu_0$ will be spread out by the Maxwell distribution of $u_x$ of the molecule emitting the radiation" \parencite[1021]{bib:McQuarrieSimon}.
        \item It follows by the definition of $f(u_x)$ and the above that
        \begin{equation*}
            I(\nu) \propto \e[-mc^2(\nu-\nu_0)^2/2\nu_0^2k_BT]
        \end{equation*}
        i.e., that $I(\nu)$ is of the form of a Gaussian centered at $\nu_0$ with variance $\sigma^2=\nu_0^2k_BT/mc^2$.
    \end{itemize}
    \item Deriving \textbf{Maxwell-Boltzmann distribution}.
    \begin{itemize}
        \item Let the probability that a molecule has speed between $u$ and $u+\dd{u}$ be defined by a continuous probability distribution $F(u)\dd{u}$. In particular, we have from the above isotropic condition that
        \begin{align*}
            F(u)\dd{u} &= f(u_x)\dd{u_x}f(u_y)\dd{u_y}f(u_z)\dd{u_z}\\
            &= \left( \frac{m}{2\pi k_BT} \right)^{3/2}\e[-m(u_x^2+u_y^2+u_z^2)/2k_BT]\dd{u_x}\dd{u_y}\dd{u_z}
        \end{align*}
        \item Considering $F$ over a \textbf{velocity space}, we realize that we may express the probability distribution $F$ as a function of $u$ via $u^2=u_x^2+u_y^2+u_z^2$ and the differential volume element in every direction over the sphere of equal velocities (a sphere by the isotropic condition) by $4\pi u^2\dd{u}=\dd{u_x}\dd{u_y}\dd{u_z}$.
        \item Thus, the Maxwell-Boltzmann distribution in terms of speed is
        \begin{equation*}
            F(u)\dd{u} = 4\pi\left( \frac{m}{2\pi k_BT} \right)^{3/2}u^2\e[-mu^2/2k_BT]\dd{u}
        \end{equation*}
    \end{itemize}
    \item \textbf{Maxwell-Boltzmann distribution}: The distribution of molecular speeds.
    \item \textbf{Velocity space}: A rectangular coordinate system in which the distances along the axes are $u_x,u_y,u_z$.
    \item We may use the above result to calculate that
    \begin{equation*}
        \prb{u} = \sqrt{\frac{8RT}{\pi m}}
    \end{equation*}
    which only differs from $u_\text{rms}$ by a factor of 0.92.
    \item \textbf{Most probable speed}: The most probable speed of a gas molecule in a sample that obeys the Maxwell-Boltzmann distribution. \emph{Denoted by} $\bm{u_\text{mp}}$. \emph{Given by}
    \begin{equation*}
        u_\text{mp} = \sqrt{\frac{2RT}{M}}
    \end{equation*}
    \begin{itemize}
        \item Derived by setting $\dv*{F}{u}=0$.
    \end{itemize}
    \item We may also express the Maxwell-Boltzmann distribution in terms of energy via $u=\sqrt{2\varepsilon/m}$ and $\dd{u}=\dd{\varepsilon}/\sqrt{2m\varepsilon}$ to give
    \begin{equation*}
        F(\varepsilon)\dd{\varepsilon} = \frac{2\pi}{(\pi k_BT)^{3/2}}\sqrt{\varepsilon}\e[-\varepsilon/k_BT]\dd{\varepsilon}
    \end{equation*}
    \item We can also confirm our previously calculated values for $\prb{u^2}$ and $\prb{\varepsilon}$.
    \item \textcite{bib:McQuarrieSimon} does a higher-level derivation of the ideal gas law that is rather analogous to the one done in class (i.e., via its flux perspective).
    \item \textcite{bib:McQuarrieSimon} discusses a simple and Nobel-prize winning experiment that verified the Maxwell-Boltzmann distribution.
\end{itemize}




\end{document}