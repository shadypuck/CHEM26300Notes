\documentclass[../notes.tex]{subfiles}

\pagestyle{main}
\renewcommand{\chaptermark}[1]{\markboth{\chaptername\ \thechapter\ (#1)}{}}
\setcounter{chapter}{5}

\begin{document}




\chapter{Enhancing Collision Theory}
\section{Threshold Energy and Line-Of-Centers Model}
\begin{itemize}
    \item \marginnote{5/2:}Picking up from the previous lecture\dots
    \begin{itemize}
        \item It follows by plugging in the energy substitutions from last time that
        \begin{equation*}
            u_rf(u_r)\dd{u_r} = \left( \frac{2}{\kB T} \right)^{3/2}\left( \frac{1}{\mu\pi} \right)^{1/2}E_r\e[-E_r/\kB T]\dd{E_r}
        \end{equation*}
        \item Thus,
        \begin{align*}
            k &= \int_0^\infty\dd{u_r}f(u_r)k(u_r)\\
            &= \left( \frac{2}{\kB T} \right)^{3/2}\left( \frac{1}{\mu\pi} \right)^{1/2}\int_0^\infty\dd{E_r}E_r\e[-E_r/\kB T]\sigma_r(E_r)
        \end{align*}
        \item Now assume that only those collisions for which the relative kinetic energy exceeds a threshold energy $E_0$ result in a collision. Thus, define
        \begin{equation*}
            \sigma_r(E_r) =
            \begin{cases}
                0 & E_r<E_0\\
                \pi d_{\ce{A}\ce{B}}^2 & E_r\geq E_0
            \end{cases}
        \end{equation*}
        \item Consequently,
        \begin{align*}
            k &= \left( \frac{2}{\kB T} \right)^{3/2}\left( \frac{1}{\mu\pi} \right)^{1/2}\int_{E_0}^\infty\dd{E_r}E_r\e[-E_r/\kB T]\pi d_{\ce{A}\ce{B}}^2\\
            &= \left( \frac{8\kB T}{\mu\pi} \right)^{1/2}\pi d_{\ce{A}\ce{B}}^2\e[-E_0/\kB T]\left( 1+\frac{E_0}{\kB T} \right)\\
            &= \prb{u_r}\sigma_{\ce{A}\ce{B}}\e[-E_0/\kB T]\left( 1+\frac{E_0}{\kB T} \right)
        \end{align*}
        \item We can use
        \begin{equation*}
            E_a = \kB T^2\dv{\ln k}{T}
        \end{equation*}
        to relate the above to the activation energy.
    \end{itemize}
    \item Another simplification we've made is that the reaction cross section is not constant, but actually depends on relative speed.
    \item Accounting for the collision geometry between the two hard spheres gives rise to the \textbf{line-of-centers model}.
    \item \textbf{Line-of-centers model}: A model for $\sigma_r(E_r)$ in which the cross section depends on the component of the relative kinetic energy that lies along the line that joins the centers of the colliding molecules.
    \begin{figure}[h!]
        \centering
        \begin{tikzpicture}
            \footnotesize
            \draw [dashed]
                (-5,1.3) -- (4,1.3)
                (-5,0) -- (4,0)
            ;
            \draw [very thin,<->,shorten <=1pt,shorten >=1pt] (-4.8,0) -- node[fill=white,inner sep=1.5pt]{$b$} ++(0,1.3);
    
            \filldraw [fill=orx,thick] (-3,1.3) circle (7mm) node[above=7mm]{A};
            \draw [very thick,-latex] (-2.3,1.3) -- ++(1.02,0) node[right,fill=white]{$\mathbf{u}_{\ce{A}}$};
    
            \filldraw [fill=gry,thick] (2,0) circle (1cm) node[below=1cm]{B};
            \draw [very thick,-latex] (1,0) -- ++(-1,0) node[left,fill=white]{$\mathbf{u}_{\ce{B}}$};
    
            \draw (-3,1.3) ++(-135:0.7) -- node[below right=-2pt]{$r_{\ce{A}}$} ++(45:0.7) -- node[pos=0.55,above,sloped]{Line of centers} (2,0) -- node[below right=-2pt]{$r_{\ce{B}}$} ++(45:1);
        \end{tikzpicture}
        \caption{Line-of-centers model.}
        \label{fig:lineOfCenters}
    \end{figure}
    \begin{itemize}
        \item If we denote the relative kinetic energy along the line of centers by $E_{loc}$, then we are assuming that a reaction occurs when $E_{loc}>0$.
        \item The main thrust of this model is that we are redefining $E_r$ instead of $\sigma_r(E_r)$ overall.
    \end{itemize}
    \item The line-of-centers model asserts that two molecules will collide only if the \textbf{impact parameter} is less than the sum of the radii of the colliding molecules.
    \begin{itemize}
        \item In particular, we (re)define
        \begin{equation*}
            \sigma_r(E_r) =
            \begin{cases}
                0 & E_r<E_0\\
                \pi d_{\ce{A}\ce{B}}^2\left( 1-\frac{E_0}{E_r} \right)
            \end{cases}
        \end{equation*}
        \item It follows from math similar to the above that
        \begin{align*}
            k &= \left( \frac{2}{\kB T} \right)^{3/2}\left( \frac{1}{\mu\pi} \right)^{1/2}\int_0^\infty\dd{E_r}E_r\e[-E_r/\kB T]\sigma_r(E_r)\\
            &= \left( \frac{8\kB T}{\mu\pi} \right)^{1/2}\pi d_{\ce{A}\ce{B}}^2\e[-E_0/\kB T]\\
            &= \prb{u_r}\sigma_{\ce{A}\ce{B}}\e[-E_0/\kB T]
        \end{align*}
    \end{itemize}
    \item \textbf{Impact parameter}: The perpendicular distance between the two dashed lines in Figure \ref{fig:lineOfCenters}. \emph{Denoted by} $\bm{b}$.
    \item The cross section exhibits a threshold energy.
    \begin{itemize}
        \item The dependence of the reaction cross section on the relative kinetic energy of the collision is consistent with the line-of-centers model.
    \end{itemize}
    \item Relating $E_0$ to the Arrhenius equation parameters.
    \begin{itemize}
        \item For the activation energy $E_a$, we have
        \begingroup
        \allowdisplaybreaks
        \begin{align*}
            E_a &= \kB T^2\dv{\ln k}{T}\\
            &= \kB T^2\dv{T}\left\{ \ln\left[ \left( \frac{8\kB T}{\pi\mu} \right)^{1/2}\pi d_{\ce{A}\ce{B}}^2 \right]-\frac{E_0}{\kB T} \right\}\\
            &= \kB T^2\dv{T}\left\{ \ln T^{1/2}-\frac{E_0}{\kB T}+\text{terms not involving }T \right\}\\
            &= E_0+\frac{1}{2}\kB T
        \end{align*}
        \endgroup
        \begin{itemize}
            \item Tian wants us to memorize the last line above.
        \end{itemize}
        \item Considering the line-of-centers collision model and the Arrhenius equation yields
        \begin{equation*}
            A = \prb{u_r}\sigma_{\ce{A}\ce{B}}\e[1/2]
        \end{equation*}
    \end{itemize}
    \item Tian goes through a practice problem.
\end{itemize}




\end{document}