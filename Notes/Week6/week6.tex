\documentclass[../notes.tex]{subfiles}

\pagestyle{main}
\renewcommand{\chaptermark}[1]{\markboth{\chaptername\ \thechapter\ (#1)}{}}
\setcounter{chapter}{5}

\begin{document}




\chapter{Enhancing Collision Theory}
\section{Threshold Energy and Line-Of-Centers Model}
\begin{itemize}
    \item \marginnote{5/2:}Picking up from the previous lecture\dots
    \begin{itemize}
        \item It follows by plugging in the energy substitutions from last time that
        \begin{equation*}
            u_rf(u_r)\dd{u_r} = \left( \frac{2}{\kB T} \right)^{3/2}\left( \frac{1}{\mu\pi} \right)^{1/2}E_r\e[-E_r/\kB T]\dd{E_r}
        \end{equation*}
        \item Thus,
        \begin{align*}
            k &= \int_0^\infty\dd{u_r}f(u_r)k(u_r)\\
            &= \left( \frac{2}{\kB T} \right)^{3/2}\left( \frac{1}{\mu\pi} \right)^{1/2}\int_0^\infty\dd{E_r}E_r\e[-E_r/\kB T]\sigma_r(E_r)
        \end{align*}
        \item Now assume that only those collisions for which the relative kinetic energy exceeds a threshold energy $E_0$ result in a collision. Thus, define
        \begin{equation*}
            \sigma_r(E_r) =
            \begin{cases}
                0 & E_r<E_0\\
                \pi d_{\ce{A}\ce{B}}^2 & E_r\geq E_0
            \end{cases}
        \end{equation*}
        \item Consequently,
        \begin{align*}
            k &= \left( \frac{2}{\kB T} \right)^{3/2}\left( \frac{1}{\mu\pi} \right)^{1/2}\int_{E_0}^\infty\dd{E_r}E_r\e[-E_r/\kB T]\pi d_{\ce{A}\ce{B}}^2\\
            &= \left( \frac{8\kB T}{\mu\pi} \right)^{1/2}\pi d_{\ce{A}\ce{B}}^2\e[-E_0/\kB T]\left( 1+\frac{E_0}{\kB T} \right)\\
            &= \prb{u_r}\sigma_{\ce{A}\ce{B}}\e[-E_0/\kB T]\left( 1+\frac{E_0}{\kB T} \right)
        \end{align*}
        \item We can use
        \begin{equation*}
            E_a = \kB T^2\dv{\ln k}{T}
        \end{equation*}
        to relate the above to the activation energy.
    \end{itemize}
    \item Another simplification we've made is that the reaction cross section is not constant, but actually depends on relative speed.
    \item Accounting for the collision geometry between the two hard spheres gives rise to the \textbf{line-of-centers model}.
    \item \textbf{Line-of-centers model}: A model for $\sigma_r(E_r)$ in which the cross section depends on the component of the relative kinetic energy that lies along the line that joins the centers of the colliding molecules.
    \begin{figure}[h!]
        \centering
        \begin{tikzpicture}
            \footnotesize
            \draw [dashed]
                (-5,1.3) -- (4,1.3)
                (-5,0) -- (4,0)
            ;
            \draw [very thin,<->,shorten <=1pt,shorten >=1pt] (-4.8,0) -- node[fill=white,inner sep=1.5pt]{$b$} ++(0,1.3);
    
            \begin{scope}[on background layer]
                \filldraw [fill=orx,thick] (-3,1.3) circle (7mm) node[above=7mm]{A};
                \filldraw [fill=gry,thick] (2,0) circle (1cm) node[below=1cm]{B};
            \end{scope}

            \draw [very thick,-latex] (-2.3,1.3) -- ++(1.02,0) node[right,fill=white]{$\mathbf{u}_{\ce{A}}$};
            \draw [very thick,-latex] (1,0) -- ++(-1,0) node[left,fill=white]{$\mathbf{u}_{\ce{B}}$};
    
            \draw (-3,1.3) ++(-135:0.7) -- node[below right=-2pt]{$r_{\ce{A}}$} ++(45:0.7) -- node[pos=0.55,above,sloped]{Line of centers} (2,0) -- node[below right=-2pt]{$r_{\ce{B}}$} ++(45:1);
        \end{tikzpicture}
        \caption{Line-of-centers model.}
        \label{fig:lineOfCenters}
    \end{figure}
    \begin{itemize}
        \item If we denote the relative kinetic energy along the line of centers by $E_\text{loc}$, then we are assuming that a reaction occurs when $E_\text{loc}>E_0$.
        \item The main thrust of this model is that we are redefining $E_r$ instead of $\sigma_r(E_r)$ overall.
    \end{itemize}
    \item The line-of-centers model asserts that two molecules will collide only if the \textbf{impact parameter} is less than the sum of the radii of the colliding molecules.
    \begin{itemize}
        \item In particular, we (re)define
        \begin{equation*}
            \sigma_r(E_r) =
            \begin{cases}
                0 & E_r<E_0\\
                \pi d_{\ce{A}\ce{B}}^2\left( 1-\frac{E_0}{E_r} \right) & E_r\geq E_0
            \end{cases}
        \end{equation*}
        \item It follows from math similar to the above that
        \begin{align*}
            k &= \left( \frac{2}{\kB T} \right)^{3/2}\left( \frac{1}{\mu\pi} \right)^{1/2}\int_0^\infty\dd{E_r}E_r\e[-E_r/\kB T]\sigma_r(E_r)\\
            &= \left( \frac{8\kB T}{\mu\pi} \right)^{1/2}\pi d_{\ce{A}\ce{B}}^2\e[-E_0/\kB T]\\
            &= \prb{u_r}\sigma_{\ce{A}\ce{B}}\e[-E_0/\kB T]
        \end{align*}
    \end{itemize}
    \item \textbf{Impact parameter}: The perpendicular distance between the two dashed lines in Figure \ref{fig:lineOfCenters}. \emph{Denoted by} $\bm{b}$.
    \item The cross section exhibits a threshold energy.
    \begin{itemize}
        \item The dependence of the reaction cross section on the relative kinetic energy of the collision is consistent with the line-of-centers model.
    \end{itemize}
    \item Relating $E_0$ to the Arrhenius equation parameters.
    \begin{itemize}
        \item For the activation energy $E_a$, we have
        \begingroup
        \allowdisplaybreaks
        \begin{align*}
            E_a &= \kB T^2\dv{\ln k}{T}\\
            &= \kB T^2\dv{T}\left\{ \ln\left[ \left( \frac{8\kB T}{\pi\mu} \right)^{1/2}\pi d_{\ce{A}\ce{B}}^2 \right]-\frac{E_0}{\kB T} \right\}\\
            &= \kB T^2\dv{T}\left\{ \ln T^{1/2}-\frac{E_0}{\kB T}+\text{terms not involving }T \right\}\\
            &= E_0+\frac{1}{2}\kB T
        \end{align*}
        \endgroup
        \begin{itemize}
            \item Tian wants us to memorize the last line above.
        \end{itemize}
        \item Considering the line-of-centers collision model and the Arrhenius equation yields
        \begin{equation*}
            A = \prb{u_r}\sigma_{\ce{A}\ce{B}}\e[1/2]
        \end{equation*}
    \end{itemize}
    \item Tian goes through a practice problem.
\end{itemize}



\section{Isotropy, Internal Energy, and Center of Mass Assumptions}
\begin{itemize}
    \item \marginnote{5/4:}Doing away with the assumption that the spheres are isotropic.
    \begin{figure}[h!]
        \centering
        \begin{tikzpicture}
            \fill [orx]
                (0,0) -- (-1.7,2.91) -- (-1.7,-2.91) -- cycle
                (-1.8,0) ellipse (4mm and 3cm)
            ;
            \draw
                (0,0) -- (-1.7,2.91)
                (0,0) -- (-1.7,-2.91)
                (-1.8,0) ellipse (4mm and 3cm)
            ;
    
            \draw [dash pattern=on 10pt off 2pt on 2pt off 2pt] (-3,0) -- (3,0);
            \draw [dashed] circle (1.8cm);
    
            \draw [thick]
                (0,0) circle (1cm)
                (20:1.8) circle (8mm)
            ;
            \draw [semithick] (-0.13,0)
                to[out=90,in=90,out looseness=2.5,in looseness=2] (1,0)
                to[out=-90,in=-90,out looseness=2,in looseness=2.5] cycle
            ;
            \draw [semithick,looseness=1.3,xshift=-4.2mm,yshift=5mm,rotate=40] (-0.44,0)
                to[out=90,in=90] (0.44,0)
                to[out=-90,in=-90] cycle
            ;
            \draw [semithick,looseness=1.3,xshift=-4.2mm,yshift=-5mm,rotate=-40] (-0.44,0)
                to[out=90,in=90] (0.44,0)
                to[out=-90,in=-90] cycle
            ;
            % \draw [semithick,looseness=1.3,xshift=-6mm,rotate=90] (-0.4,0) to[out=90,in=90] (0.4,0);
            \draw [semithick] (-0.81,0.3) to[bend right=40,looseness=0.85] (-0.81,-0.3);
    
            \draw (0,0) node[circle,fill,inner sep=1.5pt]{} -- (20:1.8) node[circle,fill,inner sep=1.5pt]{};
    
            \small
            \node at (-0.3,0) {\ce{C}};
            \node at (-0.5,0.45) {\ce{H}};
            \node at (-0.5,-0.45) {\ce{H}};
            \node at (0.5,-0.3) {\ce{I}};
            \node at ([xshift=4mm]20:1.8) {\ce{Rb}};
    
            \footnotesize
            \node [align=center] at (-3.5,-1.3) {Cone of\\non-reactivity}
                edge [-latex] (-2.3,-0.5)
            ;
        \end{tikzpicture}
        \caption{Molecules are not isotropic.}
        \label{fig:noIsotropy}
    \end{figure}
    \begin{itemize}
        \item Consider the reaction
        \begin{equation*}
            \ce{Rb(g) + CH3I(g)} \Longrightarrow \ce{RbI(g) + CH3(g)}
        \end{equation*}
        \item The rubidium atom must collide with the iodomethane in the vicinity of the iodine atom for a reaction to occur.
        \item Indeed, many molecules have a \textbf{cone of non-reactivity}.
    \end{itemize}
    \item Additionally, the internal energy of the reactants can affect the cross section of a reaction.
    \begin{itemize}
        \item Consider the reaction
        \begin{equation*}
            \ce{H2+(g) + He(g)} \Longrightarrow \ce{HeH+(g) + H(g)}
        \end{equation*}
        \item As the reactant molecule \ce{H2+} passes through different vibrational states, its reaction cross section changes.
        \item We only need to understand that other types of energy can have an effect qualitatively; we do not need to work with the shape of the curves quantitatively.
    \end{itemize}
    \item A reactive collision can be described in a center-of-mass coordinate system.
    \begin{figure}[h!]
        \centering
        \begin{tikzpicture}
            \small
            \draw [-stealth] (2,-1.5) -- node[fill=white]{Increasing time} ++(6,0);
    
            \footnotesize
            \draw [dash pattern=on 4pt off 3pt]
                (-0.4,2.4) coordinate(A1) -- (5,1.7) coordinate(A3) -- (10.4,2.9) coordinate (A5)
                (0.4,0.6) coordinate(B1) -- (5,1.3) coordinate(B3) -- (9.6,0.1) coordinate (B5)
            ;
    
            \draw [semithick,-latex] (A1) -- ($(A1)!0.28!(A3)$) node[above=1pt]{$\mathbf{u}_{\ce{A}}$};
            \draw [semithick,-latex] (B1) -- ($(B1)!0.15!(B3)$) node[below=2pt]{$\mathbf{u}_{\ce{B}}$};
            \draw [semithick,-latex] (A3) -- ($(A3)!0.21!(A5)$) node[above=1pt]{$\mathbf{u}_{\ce{C}}$};
            \draw [semithick,-latex] (B3) -- ($(B3)!0.24!(B5)$) node[below=2pt]{$\mathbf{u}_{\ce{D}}$};
    
            \begin{scope}
                \draw (0.7,0.4) node[below]{3} -- ++(0,3) node[above]{2} -- ++(-1.4,-0.8) node[above]{1} -- ++(0,-3) node[below]{4} -- cycle;
    
                \draw [densely dashed] (A1) -- (B1);
                % \draw [semithick,-latex] (-0.4,2.4) --
    
                \fill [ball color=blx] (A1) circle (2mm) node[below=2mm]{\ce{A}};
                \fill [ball color=rex] (B1) circle (2mm) node[left=2mm]{\ce{B}};
    
                \draw [semithick,-latex] ($(A1)!0.5!(B1)$) coordinate(COM1) -- ++(1.1,0);
                \filldraw [fill=white] (COM1) circle (2pt);
            \end{scope}
    
            \begin{scope}[xshift=2.5cm]
                \draw (0.7,0.4) -- ++(0,3) -- ++(-1.4,-0.8) -- ++(0,-3) -- cycle;
    
                \draw [densely dashed] ($(A1)!0.5!(A3)$) coordinate(A2) -- ($(B1)!0.5!(B3)$) coordinate(B2);
    
                \fill [ball color=blx] (A2) circle (2mm);
                \fill [ball color=rex] (B2) circle (2mm);
    
                \draw [semithick,-latex] (0,1.5) -- ++(1.1,0);
                \filldraw [fill=white] (0,1.5) circle (2pt);
            \end{scope}
    
            \begin{scope}[xshift=5cm]
                \draw (0.7,0.4) -- ++(0,3) -- ++(-1.4,-0.8) -- ++(0,-3) -- cycle;
    
                \fill [ball color=pux] (A3) circle (2mm);
                \fill [ball color=pux] (B3) circle (2mm);
    
                \draw [semithick,-latex] ($(A3)!0.5!(B3)$) coordinate (COM3) -- ++(1.1,0);
            \end{scope}
    
            \begin{scope}[xshift=7.5cm]
                \draw (0.7,0.4) -- ++(0,3) -- ++(-1.4,-0.8) -- ++(0,-3) -- cycle;
    
                \draw [densely dashed] ($(A3)!0.5!(A5)$) coordinate(A4) -- ($(B3)!0.5!(B5)$) coordinate(B4);
    
                \fill [ball color=grx] (A4) circle (2mm);
                \fill [ball color=orx] (B4) circle (2mm);
    
                \draw [semithick,-latex] (0,1.5) -- ++(1.1,0);
                \filldraw [fill=white] (0,1.5) circle (2pt);
            \end{scope}
    
            \begin{scope}[xshift=10cm]
                \draw (0.7,0.4) -- ++(0,3) -- ++(-1.4,-0.8) -- ++(0,-3) -- cycle;
    
                \draw [densely dashed] (A5) -- (B5);
    
                \fill [ball color=grx] (A5) circle (2mm) node[below=2mm,xshift=1mm]{\ce{C}};
                \fill [ball color=orx] (B5) circle (2mm) node[above right=1mm]{\ce{D}};
    
                \draw [semithick,-latex] ($(A5)!0.5!(B5)$) coordinate(COM5) -- ++(1.1,0) node[right]{$\mathbf{u}_\text{CM}$};
                \filldraw [fill=white] (COM5) circle (2pt);
            \end{scope}
    
            \node [align=center] at (-2,0.5) {Center\\of mass}
                edge [-latex,shorten >=2.5mm] (COM1)
            ;
        \end{tikzpicture}
        \caption{Center-of-mass coordinate system.}
        \label{fig:COMcoordinates}
    \end{figure}
    \begin{itemize}
        \item Consider the collision and subsequent scattering process for the bimolecular reaction
        \begin{equation*}
            \ce{A(g) + B(g)} \Longrightarrow \ce{C(g) + D(g)}
        \end{equation*}
        \item Before the collision, \ce{A} and \ce{B} are traveling with velocities $\mathbf{u}_{\ce{A}}$ and $\mathbf{u}_{\ce{B}}$, respectively.
        \item The collision generates molecules \ce{C} and \ce{D}, which then move away from each other with velocities $\mathbf{u}_{\ce{C}}$ and $\mathbf{u}_{\ce{D}}$, respectively.
        \item $\mathbf{R}$, the location of the center of mass, is given by
        \begin{align*}
            \mathbf{R} &= \frac{m_{\ce{A}}\mathbf{r}_{\ce{A}}+m_{\ce{B}}\mathbf{r}_{\ce{B}}}{M}&
            M &= m_{\ce{A}}+m_{\ce{B}}
        \end{align*}
        \item The velocity $\mathbf{u}_\text{cm}$ of the center of mass is the time derivative of the position vector. Therefore, it is given by
        \begin{equation*}
            \mathbf{u}_\text{cm} = \frac{m_{\ce{A}}\mathbf{u}_{\ce{A}}+m_{\ce{B}}\mathbf{u}_{\ce{B}}}{M}
        \end{equation*}
        \item We assume that this is an elastic collision (thus, energy is conserved).
        \item The total kinetic energy is given by
        \begin{equation*}
            \text{KE}_\text{react} = \frac{1}{2}m_{\ce{A}}u_{\ce{A}}^2+\frac{1}{2}m_{\ce{B}}u_{\ce{B}}^2
        \end{equation*}
        \item Combining the fact that the relative speed of the two molecules\footnote{In particular, the speed of molecule \ce{A} relative to molecule \ce{B}, as it is defined.} is given by $\mathbf{u}_\text{r}=\mathbf{u}_{\ce{A}}-\mathbf{u}_{\ce{B}}$ with the definition of $\mathbf{u}_\text{cm}$ yields
        \begin{align*}
            \mathbf{u}_{\ce{A}} &= \mathbf{u}_\text{cm}+\frac{m_{\ce{B}}}{M}\mathbf{u}_\text{r}&
            \mathbf{u}_{\ce{B}} &= \mathbf{u}_\text{cm}-\frac{m_{\ce{A}}}{M}\mathbf{u}_\text{r}
        \end{align*}
        \begin{itemize}
            \item Note that the change in plus to minus sign between the two above forms hails from our definition of relative speed as \ce{A} minus \ce{B} and not the other way around (as we could also very well define it). In other words, it's just a convention thing, and all that matters is that we're consistent.
        \end{itemize}
        \item It follows that
        \begin{align*}
            \text{KE}_\text{react} &= \frac{m_{\ce{A}}}{2}\left( \mathbf{u}_\text{cm}+\frac{m_{\ce{B}}}{M}\mathbf{u}_\text{r} \right)^2+\frac{m_{\ce{B}}}{2}\left( \mathbf{u}_\text{cm}-\frac{m_{\ce{A}}}{M}\mathbf{u}_\text{r} \right)^2\\
            &= \frac{1}{2}Mu_\text{cm}^2+\frac{1}{2}\mu u_\text{r}^2
        \end{align*}
        \item Thus, the kinetic energy is composed of two contributions: one due to the motion of the center of mass, and one due to the relative motion of the two colliding molecules.
        \item We can do a similar analysis for the products to determine that
        \begin{equation*}
            \text{KE}_\text{prod} = \frac{1}{2}Mu_\text{cm}^2+\frac{1}{2}\mu'u_\text{r}'{}^2
        \end{equation*}
        \item Note that momentum is conserved, i.e.,
        \begin{equation*}
            m_{\ce{A}}\mathbf{u}_{\ce{A}}+m_{\ce{B}}\mathbf{u}_{\ce{B}} = m_{\ce{C}}\mathbf{u}_{\ce{C}}+m_{\ce{D}}\mathbf{u}_{\ce{D}}
        \end{equation*}
        \item This implies that $\mathbf{u}_\text{cm}$ does not change from reactants to products.
        \item The energy associated with the motion of the center of mass is therefore constant, and we will ignore its constant contribution to the total kinetic energy.
        \begin{equation*}
            E_\text{react,int}+\frac{1}{2}\mu u_\text{r}^2 = E_\text{prod,int}+\frac{1}{2}\mu'u_\text{r}'{}^2
        \end{equation*}
        \begin{itemize}
            \item $E_\text{react,int}$ and $E_\text{prod,int}$ are the total internal energies of the reactants and products, respectively.
            \item This internal energy takes into account all the degrees of freedom other than translation.
        \end{itemize}
    \end{itemize}
\end{itemize}



\section{Experimental Techniques and the Simplest Reaction}
\begin{itemize}
    \item \marginnote{5/6:}Reactive collisions can be studied using crossed molecular beam machines.
    \begin{figure}[h!]
        \centering
        \footnotesize
        \begin{subfigure}[b]{0.45\linewidth}
            \centering
            \begin{tikzpicture}
                \fill [rex,opacity=0.5] (-2.5,-0.05) -- ++(0,0.1) -- ++(5,0.15) -- ++(0,-0.4);
                \fill [blx,opacity=0.5] (-0.05,2.5) -- ++(0.1,0) -- ++(0.15,-5) -- ++(-0.4,0);
    
                \draw [semithick]
                    (-3.5,-0.5) rectangle ++(1,1) node[above,align=center,xshift=-5mm]{Molecular\\beam}
                    (-0.5,2.5) node[left,align=center,yshift=5mm]{Molecular\\beam} rectangle ++(1,1)
                ;
                \draw [very thick]
                    (-0.7,2) -- ++(0.5,0)
                    (0.2,2) -- ++(0.5,0)
                    (-2,-0.7) -- ++(0,0.5)
                    (-2,0.2) -- ++(0,0.5)
                ;
    
                \draw [dashed] (40:0.2) -- (40:2.5);
                \draw [latex-latex] (10:2.5) arc[start angle=10,end angle=70,radius=2.5cm];
                \filldraw [semithick,fill=white,rotate around={40:(0,0)}] (2.4,-0.2) rectangle ++(0.2,0.4) node[above right]{Detector};
    
                \node [below right=1cm,align=center] {Collision\\region}
                    edge [->] (0.2,-0.2)
                ;
            \end{tikzpicture}
            \caption{Overall machine.}
            \label{fig:crossedMolBeama}
        \end{subfigure}
        \begin{subfigure}[b]{0.45\linewidth}
            \centering
            \begin{tikzpicture}
                \draw [ultra thick] (0,-1) -- ++(1,0) -- ++(0,0.8) ++(0,0.4) -- ++(0,0.8) -- ++(-1,0);
                \node [align=center] {Reactant/\\inert gas\\mixture};
    
                \draw [rex]
                    (0.9,0.06)   -- ($(0.9,0.06)!0.54!(4.0,0.42)$)
                    (0.9,0.05)   -- ($(0.9,0.05)!0.52!(4.0,0.35)$)
                    (0.9,0.04)   -- ++(3.1,0.24)
                    (0.9,0.03)   -- ++(3.1,0.18)
                    (0.9,0.02)   -- ++(3.1,0.12)  -- ++(0.7,0)
                    (0.9,0.01)   -- ++(3.1,0.06)  -- ++(0.7,0)
                    (0.9,0.00)   -- ++(3.1,0.00)  -- ++(0.7,0)
                    (0.9,-0.01)  -- ++(3.1,-0.06) -- ++(0.7,0)
                    (0.9,-0.02)  -- ++(3.1,-0.12) -- ++(0.7,0)
                    (0.9,-0.03)  -- ++(3.1,-0.18)
                    (0.9,-0.04)  -- ++(3.1,-0.24)
                    (0.9,-0.05)  -- ($(0.9,-0.05)!0.52!(4.0,-0.35)$)
                    (0.9,-0.06)  -- ($(0.9,-0.06)!0.54!(4.0,-0.42)$)
                ;
    
                \filldraw [semithick,fill=white]
                    (2.5,0.2) -- ++(0.4,0.1) -- ++(0,0.2) -- cycle
                    (2.5,-0.2) -- ++(0.4,-0.1) -- ++(0,-0.2) node[below]{Skimmer} -- cycle
                    (4,0.2) -- ++(0,0.5) -- ++(0.1,0) node[above]{Collimator} -- ++(0,-0.4) -- cycle
                    (4,-0.2) -- ++(0,-0.5) -- ++(0.1,0) -- ++(0,0.4) -- cycle
                ;
            \end{tikzpicture}
            \caption{Supersonic molecular beam source.}
            \label{fig:crossedMolBeamb}
        \end{subfigure}
        \caption{Crossed molecular beam machines.}
        \label{fig:crossedMolBeam}
    \end{figure}
    \begin{itemize}
        \item Figure \ref{fig:crossedMolBeama} depicts the overall setup in a crossed molecular beam machine. Each reactant is introduced into the vacuum chamber by a molecular beam source. The two molecular beams collide at the collision region.
        \item Figure \ref{fig:crossedMolBeamb} depicts a supersonic molecular beam source. The reactant is expanded along with an inert gas through a small orifice into the vacuum chamber. A skimmer is used so that a collimated beam of molecules is directed toward the collision region.
        \item The product molecules are detected using a mass spectrometer.
    \end{itemize}
    \item A supersonic molecular beam has several important advantages that make it ideal for crossed-beam studies.
    \begin{itemize}
        \item The supersonic expansion generates a collection of molecules with a high translational energy but a very small spread in molecular speeds.
        \item In addition, molecules can be prepared with low rotational and vibrational energies.
        \item If we measure the number of molecules of a particular reaction product that arrive at the detector as a function of time after the collision, we can resolve the velocity distribution of the product molecules.
        \item If we measure the total number of product molecules as a function of scattering angle, we can determine the angular distribution of the product molecules.
    \end{itemize}
    \item The reaction
    \begin{equation*}
        \ce{F(g) + D2(g)} \Longrightarrow \ce{DF(g) + D(g)}
    \end{equation*}
    \begin{figure}[h!]
        \centering
        \begin{tikzpicture}[
            every node/.style=black
        ]
            \small
            \draw [stealth-] (0,1) -- node[rotate=90,above=8mm]{Energy (\si{\kilo\joule\per\mole})} (0,-5.5);
            \node at (3,-6) {Reaction coordinate};
    
            \footnotesize
            \draw
                (0,0)    -- ++(-0.1,0) node[left]{0}
                (0,-1.7) -- ++(-0.1,0) node[left]{-50}
                (0,-3.4) -- ++(-0.1,0) node[left]{-100}
                (0,-5.1) -- ++(-0.1,0) node[left]{-150}
            ;
    
            \draw [blx,thick]
                (0.2,0) -- node[below]{\ce{F + D2}} ++(1.8,0)
                (4,-4.7) -- node[below]{\ce{DF + D}} ++(1.8,0)
            ;
            \draw [blx,semithick] (2,0)
                to[out=0,in=180,out looseness=1.3] ++(0.4,0.1)
                to[out=0,in=180,out looseness=0.5,in looseness=0.8] ++(1.6,-4.8)
            ;
            \draw [grx,thick]
                (0.8,0.5)  -- ++(0.6,0) node[right]{$v=0$}
                %
                (4.6,0.8)  -- ++(0.6,0) node[right]{$v=5$}
                (4.6,-0.2) -- ++(0.6,0) node[right]{$v=4$}
                (4.6,-1.2) -- ++(0.6,0) node[right]{$v=3$}
                (4.6,-2.2) -- ++(0.6,0) node[right]{$v=2$}
                (4.6,-3.2) -- ++(0.6,0) node[right]{$v=1$}
                (4.6,-4.2) -- ++(0.6,0) node[right]{$v=0$}
            ;
    
            \draw [dashed] (0.2,-4.7) -- ++(3.8,0);
            \draw [<->,shorten <=1pt,shorten >=1pt] (1.9,0) -- node[fill=white]{\SI{-140}{\kilo\joule\per\mole}} ++(0,-4.7);
            \draw [<->,shorten <=1pt,shorten >=1pt] (1.1,0) -- node[left]{$\frac{1}{2}h\nu_{\ce{D2}}$} ++(0,0.5);
            %
            \draw [<->,shorten <=1pt,shorten >=1pt] (4.9,-4.2) -- node[left]{$h\nu_{\ce{DF}}$} ++(0,1);
            \draw [<->,shorten <=1pt,shorten >=1pt] (4.9,-4.7) -- node[left]{$\frac{1}{2}h\nu_{\ce{DF}}$} ++(0,0.5);
        \end{tikzpicture}
        \caption{Energy diagram for \ce{F + D2}.}
        \label{fig:FD2}
    \end{figure}
    \begin{itemize}
        \item Figure \ref{fig:FD2} depicts the energy of the lowest vibrational state of \ce{D2} and the energies of the first six vibrational states of \ce{DF}.
        \begin{itemize}
            \item In drawing these energy states, we have tacitly assumed that the vibrational motion of both \ce{D2} and \ce{DF} is harmonic.
        \end{itemize}
        \item The difference between the ground electronic state energies of \ce{D2} and \ce{DF} is
        \begin{equation*}
            D_e(\ce{D2})-D_e(\ce{DF}) = -\SI{140}{\kilo\joule\per\mole}
        \end{equation*}
        \item The reaction has an activation energy barrier of about \SI{7}{\kilo\joule\per\mole}.
        \item Moreover, we have that
        \begin{equation*}
            E_\text{tot} = E_\text{trans}+E_\text{int}
            = E_\text{trans}'+E_\text{int}'
        \end{equation*}
        where $E_\text{int}$ is the internal energy of the reactants, $E_\text{trans}$ is the relative translational energy of the reactants, $E_\text{int}'$ is the internal energy of the products, and $E_\text{trans}'$ is the relative translational energy of the products.
        \begin{itemize}
            \item Additionally, note that
            \begin{align*}
                E_\text{int} &= E_\text{rot}+E_\text{vib}+E_\text{elec}&
                E_\text{int}' &= E_\text{rot}'+E_\text{vib}'+E_\text{elec}'
            \end{align*}
            i.e., that $E_\text{int},E_\text{int}'$ are the sum of the rotational, vibrational, and electronic energies of the reactants and products, respectively.
            \item Lastly, note that $E_\text{elec}=-D_e(\ce{D2})$ and $E_\text{elec}'=-D_e(\ce{DF})$.
        \end{itemize}
    \end{itemize}
\end{itemize}



\section{Chapter 30: Gas-Phase Reaction Dynamics}
\emph{From \textcite{bib:McQuarrieSimon}.}
\begin{itemize}
    \item \marginnote{5/18:}An underlying assumption of the step model for $\sigma_r(E_r)$.
    \begin{itemize}
        \item All molecular collisions with relative energy $E_r$ yield the same reaction cross section.
        \begin{itemize}
            \item Reality: Not all molecular collisions are head-on; some are more grazing. In a head-on collision, the molecules come to a stop and, in principle, all the relative kinetic energy becomes available for reaction. In contrast, a grazing collision provides almost no energy for reaction.
        \end{itemize}
    \end{itemize}
    \item \textbf{Line-of-centers model} (for $\sigma_r(E_r)$): A model in which the reaction cross section depends on the component of the relative kinetic energy that lies along the line that joins the centers of the colliding molecules. \emph{Given by}
    \begin{equation*}
        \sigma_r(E_r) =
        \begin{cases}
            0 & E_r<E_0\\
            \sigma_{\ce{A}\ce{B}}\left( 1-\frac{E_0}{E_r} \right) & E_r\geq E_0
        \end{cases}
    \end{equation*}
    \item Derivation of the line-of-centers model.
    \begin{figure}[h!]
        \centering
        \begin{tikzpicture}
            \footnotesize
            \draw [dashed]
                (-5,1.3) -- (4,1.3)
                (-5,0) -- (4,0)
            ;
            \draw [very thin,<->,shorten <=1pt,shorten >=1pt] (-4.8,0) -- node[fill=white,inner sep=1.5pt]{$b$} ++(0,1.3);
    
            \begin{scope}[on background layer]
                \filldraw [fill=orx,thick] (-1.1,1.3) coordinate (A) circle (7mm) node[above=7mm]{A};
                \filldraw [fill=gry,thick] (0,0) coordinate (B)      circle (1cm) node[below=1cm]{B};
            \end{scope}
    
            \fill
                (A) circle (2pt)
                (B) circle (2pt)
            ;
            \path (B) -- node[below left=-2pt]{$r_{\ce{B}}$} (130.1:1) -- node[below left=-2pt]{$r_{\ce{A}}$} (130.1:1.7);
            \draw [very thick,-latex] (A) -- ++(2.64,0) coordinate (ur) node[right,fill=white]{$\mathbf{u}_r$};
            \draw [very thick,-latex] (A) -- ($(A)!0.95!(B)$) node[above right,xshift=-1mm]{$\mathbf{u}_\text{loc}$};
    
            \pic [pic text={$\theta$},angle eccentricity=0.83] {angle=B--A--ur};
        \end{tikzpicture}
        \caption{Line-of-centers model derivation.}
        \label{fig:lineOfCentersDeriv}
    \end{figure}
    \begin{itemize}
        \item Qualitative observations and definitions.
        \begin{itemize}
            \item From the reference frame of molecule \ce{B}, molecule \ce{A} approaches with relative speed $\mathbf{u}_r=\mathbf{u}_{\ce{A}}-\mathbf{u}_{\ce{B}}$ and relative kinetic energy $E_r=\mu u_r^2/2$.
            \item Applying the line-of-centers coordinate system, molecule \ce{A} approaches molecule \ce{B} along the loc (line of centers) axis with speed $\mathbf{u}_\text{loc}$.
            \item They will collide if the \textbf{impact parameter} $b$ is less than the sum of the two radii $r_{\ce{A}}+r_{\ce{B}}=d_{\ce{A}\ce{B}}$.
            \item The impact parameter determines the energy of the collision on a scale of zero to $E_r$. In particular, if $b=0$, then $E_\text{loc}=E_r$, and if $b\geq d_{\ce{A}\ce{B}}$, $E_\text{loc}=0$, where $E_\text{loc}$ denotes the relative kinetic energy along the line of centers.
            \item Naturally, a reaction will occur iff $E_\text{loc}\geq E_0$.
        \end{itemize}
        \item Mathematical derivation \parencite{bib:LineOfCenters}.
        \begin{itemize}
            \item We have from basic trigonometry that
            \begin{align*}
                \cos\theta &= \frac{u_\text{loc}}{u_r}&
                \sin\theta &= \frac{b}{d_{\ce{A}\ce{B}}}
            \end{align*}
            \item Thus, we know that
            \begin{equation*}
                u_\text{loc}^2 = u_r^2\cos^2\theta
                = u_r^2(1-\sin^2\theta)
                = u_r^2\left[ 1-\left( \frac{b}{d_{\ce{A}\ce{B}}} \right)^2 \right]
            \end{equation*}
            and hence
            \begin{equation*}
                E_\text{loc} = \frac{1}{2}\mu u_\text{loc}^2
                = \frac{1}{2}\mu u_r^2\left[ 1-\left( \frac{b}{d_{\ce{A}\ce{B}}} \right)^2 \right]
                = E_r\left[ 1-\left( \frac{b}{d_{\ce{A}\ce{B}}} \right)^2 \right]
            \end{equation*}
            \item Now recall that a chemical reaction only takes place if $E_\text{loc}\geq E_0$. By the above, an equivalent requirement for a chemical reaction to take place is that
            \begin{align*}
                E_r\left[ 1-\left( \frac{b}{d_{\ce{A}\ce{B}}} \right)^2 \right] &\geq E_0\\
                \left( \frac{b}{d_{\ce{A}\ce{B}}} \right)^2 &\leq 1-\frac{E_0}{E_r}\\
                \left| \frac{b}{d_{\ce{A}\ce{B}}} \right| &\leq \sqrt{1-\frac{E_0}{E_r}}\\
                \left| b \right| &\leq d_{\ce{A}\ce{B}}\sqrt{1-\frac{E_0}{E_r}}
            \end{align*}
            i.e., that the impact parameter lie within some range, specifically one whose width is determined by the total energy of the reactants $E_r$ and is at most $d_{\ce{A}\ce{B}}$ (in the case of infinite reactant energy).
            \item It is this observation that motivates our definition of the \textbf{reaction diameter}. The physical meaning of this quantity is analogous to that of the collision diameter: Whereas the collision diameter defined a cylinder in which other molecules had to lie to \emph{collide} with one molecule in particular (see Figure \ref{fig:collisionCylinder}), the reaction diameter defines a cylinder (the \textbf{reaction cylinder}) in which other molecules must lie to \emph{react} with one molecule in particular.
            \item One consequence of this physical interpretation is that $\sigma_r(E_r)=\pi b_\text{crit}^2$, i.e., that the reaction cross section is a perpendicular cross section of the reaction cylinder in much the same way that the collision cross section is a perpendicular cross section of the collision cylinder.
            \item Another consequence is that two molecules with impact parameter equal to $b_\text{crit}$ will have \emph{just enough} energy to react, and no more. In other words, they will possess exactly $E_0$ units of energy. Mathematically,
            \begin{equation*}
                E_0 = E_r\left[ 1-\left( \frac{b_\text{crit}}{d_{\ce{A}\ce{B}}} \right)^2 \right]
            \end{equation*}
            \item Solving the above for $b_\text{crit}^2$ and substituting into the definition of $\sigma_r(E_r)$ in terms of $b_\text{crit}$ yields
            \begin{equation*}
                \sigma_r(E_r) = \pi d_{\ce{A}\ce{B}}^2\left( 1-\frac{E_0}{E_r} \right)
                = \sigma_{\ce{A}\ce{B}}\left( 1-\frac{E_0}{E_r} \right)
            \end{equation*}
            for the case where $E_r\geq E_0$, as desired. We will naturally still take $\sigma_r(E_r)=0$ when $E_r<E_0$.
        \end{itemize}
    \end{itemize}
    \item \textbf{Impact parameter}: The perpendicular distance between the paths along which the two molecules travel. \emph{Denoted by} $\bm{b}$.
    \item \textbf{Reaction diameter}: The largest value of the impact parameter at which two molecules of combined energy $E_r$ will still react. \emph{Denoted by} $\bm{b_\textbf{crit}}$. \emph{Also known as} \textbf{critical impact parameter}.
    \item \textcite{bib:McQuarrieSimon} gives an example of a bimolecular gas phase reaction for which the relationship between the reaction cross section and the the collision energy is as described by the line-of-centers model.
    \item Substituting the reaction cross section as defined by the line-of-centers model into the integral defining $k$ yields
    \begin{equation*}
        k = \prb{u_r}\sigma_{\ce{A}\ce{B}}\e[-E_0/\kB T]
    \end{equation*}
    \item \textcite{bib:McQuarrieSimon} relates $E_0$ to the Arrhenius equation parameters as in class.
    \item Upon reviewing experimentally determined collision cross section functions for a number of reactions, we can conclude that even the line-of-centers model is insufficient, and we need to move beyond simple hard-sphere collision theories.
    \item An underlying assumption of all hard-sphere models.
    \begin{itemize}
        \item Every collision with sufficient energy is reactive, regardless of orientation.
        \begin{itemize}
            \item Reality: A specific orientation of components is typically required for molecules that are not spherically symmetric.
        \end{itemize}
    \end{itemize}
    \item \textcite{bib:McQuarrieSimon} discusses the \ce{Rb(g) + CH3I(g)} reaction from class and Figure \ref{fig:noIsotropy}.
    \item An underlying assumption of all hard-sphere models.
    \begin{itemize}
        \item Only the translational energy of reacting molecules affects the reaction cross section.
        \begin{itemize}
            \item Reality: The presence or lack thereof of internal energy, as well as how it's distributed between the rotational, vibrational, and electronic states, plays an integral role in determining the reaction cross section. Additionally, internal energy changes can occur during the course of a reaction. 
        \end{itemize}
    \end{itemize}
    \item An analysis of the following reaction with the hydrogen molecular ion in different vibrational states.
    \begin{equation*}
        \ce{H2+(g) + He(g)} \Longrightarrow \ce{HeH+(g) + H(g)}
    \end{equation*}
    \begin{itemize}
        \item For the vibrational states $v=0,\dots,3$, there is a threshold energy of about \SI{70}{\kilo\joule\per\mole}.
        \item For vibrational states $v=4,5$, there is no threshold energy.
        \item This is because these higher vibrational states can lend more than the threshold energy worth of energy to the reaction, so no supplementary translational motion is needed.
    \end{itemize}
    \item \textcite{bib:McQuarrieSimon} discusses the center-of-mass coordinate system.
    \begin{itemize}
        \item The energy and momentum conservation laws allow us to define the velocity of the products.
        \item They do not, however, allow us to define the angle at which the products scatter.
        \item We will soon see that the scattering angles are often highly \textbf{anisotropic}.
    \end{itemize}
    \item \textbf{Anisotropic} (entity): An object or substance the properties of which vary by the direction in which they're measured.
    \item \textbf{Crossed molecular beam method}: An experimental setup designed to cross a beam of \ce{A} molecules with a beam of \ce{B} molecules at a specific location inside a large vacuum chamber before detecting the product molecules with a mass spectrometer.
    \begin{itemize}
        \item "In some crossed molecular beam machines, the detector can be rotated in the plane defined by the two molecular beams, thereby allowing the measurement of the angular distribution of the scattered products" \parencite[1244]{bib:McQuarrieSimon}.
        \item "A supersonic molecular beam can be generated by taking a high-pressure, dilute mixture of the reactant molecule of interest in an inert carrier gas (\ce{He} and \ce{Ne} are commonly used) and pulsing the mixture through a small nozzle into the vacuum chamber" \parencite[1244]{bib:McQuarrieSimon}.
    \end{itemize}
    \item Crossed molecular beam experiments can be used to measure reaction cross sections: We alter the relative velocities (and hence collision energies) of the reactants and record the product yield.
    \item \marginnote{5/19:}\textbf{Potential energy diagram}: A diagram that indicates how the potential energy changes as the reaction proceeds along the reaction coordinate.
    \item \textcite{bib:McQuarrieSimon} discusses the potential energy diagram for the \ce{F + D2} reaction (Figure \ref{fig:FD2}).
    \begin{itemize}
        \item Using the conservation of energy, we can calculate that $v\leq 4$ for the product \ce{DF}.
        \item Additionally, we can experimentally measure the vibrational state of \ce{DF} because molecules with a higher vibrational state must have more of there energy internal instead of as translational, meaning that they move slower.
    \end{itemize}
    \item Example: Consider the reaction
    \begin{equation*}
        \ce{F(g) + D2}(v=0) \Longrightarrow \ce{DF($v$) + D(g)}
    \end{equation*}
    where the relative translational energy of the reactants is \SI{7.62}{\kilo\joule\per\mole}. Assume the reactants and products are in their ground electronic [$D_e(\ce{D2})-D_e(\ce{DF})=\SI{-140}{\kilo\joule\per\mole}$] and rotational states. Treat the vibrational motion of \ce{D2} and \ce{DF} as harmonic with $\tilde{\nu}_{\ce{D2}}=\SI{2990}{\per\centi\meter}$ and $\tilde{\nu}_{\ce{DF}}=\SI{2907}{\per\centi\meter}$. Determine the range of possible vibrational states of the product.
    \begin{itemize}
        \item We apply conservation of energy.
        \begin{align*}
            E_\text{reactants} &= E_\text{products}\\
            E_\text{trans}+E_\text{vib}+E_\text{elec} &= E_\text{trans}'+E_\text{vib}'+E_\text{elec}'\\
            E_\text{trans}+E_\text{vib}-D_e(\ce{D2}) &= E_\text{trans}'+E_\text{vib}'-D_e(\ce{DF})
        \end{align*}
        \item We are given in the problem statement that $E_\text{trans}=\SI{7.62}{\kilo\joule\per\mole}$, $E_\text{vib}=\frac{1}{2}h\nu_{\ce{D2}}$ ($v=0$ for \ce{D2}), and $D_e(\ce{D2})-D_e(\ce{DF})=\SI{-140}{\kilo\joule\per\mole}$. Moreover, since
        \begin{equation*}
            \nu_{\ce{D2}} = \frac{c}{\lambda}
            = \frac{c}{1/\tilde{\nu}}
            = \frac{\SI{2.998e8}{\meter\per\second}}{1/\SI{2.99e5}{\per\meter}}
            = \SI{8.96e13}{\per\second}
        \end{equation*}
        we have that
        \begin{equation*}
            E_\text{vib} = \frac{1}{2}h\nu_{\ce{D2}}
            = \frac{1}{2}\left( \frac{\SI{1}{\kilo\joule}}{\SI{e3}{\joule}} \right)(\NA\,\si{\per\mole})(\SI{6.626e-34}{\joule\second})(\SI{8.96e13}{\per\second})
            = \SI{17.9}{\kilo\joule\per\mole}
        \end{equation*}
        \item Thus, we know that
        \begin{align*}
            E_\text{vib}' &= E_\text{trans}+E_\text{vib}-[D_e(\ce{D2})-D_e(\ce{DF})]-E_\text{trans}'\\
            &= \SI{7.62}{\kilo\joule\per\mole}+\SI{17.9}{\kilo\joule\per\mole}+\SI{140}{\kilo\joule\per\mole}-E_\text{trans}'\\
            &= \SI{166}{\kilo\joule\per\mole}-E_\text{trans}'
        \end{align*}
        \item Since $E_\text{trans}'\geq 0$, we must have $E_\text{vib}'\leq\SI{166}{\kilo\joule\per\mole}$. This combined with the fact that
        \begin{equation*}
            E_\text{vib}' = \left( v+\frac{1}{2} \right)h\nu_{\ce{DF}}
            = \left( v+\frac{1}{2} \right)(\SI{34.8}{\kilo\joule\per\mole})
        \end{equation*}
        reveals that $v\leq 4$, where $h\nu_{\ce{DF}}$ is evaluated analogously to how we computed $h\nu_{\ce{D2}}$.
        \item Note that this result is in agreement with Figure \ref{fig:FD2}.
    \end{itemize}
\end{itemize}




\end{document}